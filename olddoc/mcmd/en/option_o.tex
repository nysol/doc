%\documentclass[a4paper]{book}
%\usepackage{mcmd}
%\begin{document}

\subsection{o= Output file name\label{sect:option_o}}
Specify the name of output file.
Most commands only allow specification of a single file name, with the exception of \verb|mtee| command where multiple files can be specified. There is also the command that does not require output data, for example, \verb|msep|.

When this parameter is not defined, data is read from standard input through pipeline. 
In the following example \verb|o=| is not specified in \verb|msortf| because the output data is sent to standard output through pipeline.
 
\begin{Verbatim}[baselinestretch=0.7,frame=single]
$ msortf f=a i=dat.csv | msum k=a f=b o=rsl.csv
\end{Verbatim}

The example below is similar to the above. The difference is that o= parameter is specified for the \verb|msortf| and the result of \verb|msortf| is saved to \verb|tmp.csv|. Even though the two commands are connected with pipeline, there is no data stream from standard output to \verb|msum|, the receiving process could not read data from pipeline and stays idle. 
 
\begin{Verbatim}[baselinestretch=0.7,frame=single]
$ msortf f=a i=dat.csv o=tmp.csv | msum k=a f=b o=rsl.csv
\end{Verbatim}

Below is a more complicated example by using \verb|mtee| to connect the data streams between the two commands.

\begin{Verbatim}[baselinestretch=0.7,frame=single]
$ msortf f=a i=dat.csv | mtee o=tmp.csv | msum k=a f=b o=rsl.csv
\end{Verbatim}

The \verb|mtee| command writes to a standard input file specified at \verb|o=| and send the data to standard output concurrently. The results of \verb|msortf| is written to \verb|tmp.csv|, at the same time, \verb|msum| receives the data stream through pipeline from \verb|mtee|. The final result is saved to \verb|rsl.csv|.

\subsection*{Examples}
\subsubsection*{Example 1: Basic Example}

The result of \verb|mcut| is saved to \verb|rsl1.csv| as specified in \verb|o=| parameter.


\begin{Verbatim}[baselinestretch=0.7,frame=single]
$ more dat1.csv
customer,quantity,amount
A,1,10
A,2,20
$ mcut f=customer,amount i=dat1.csv o=rsl1.csv
#END# kgcut f=customer,amount i=dat1.csv o=rsl1.csv
$ more rsl1.csv
customer,amount
A,10
A,20
\end{Verbatim}
\subsubsection*{Example 2: Redirect}

Write to standard input using redirection (\verb|">"|).


\begin{Verbatim}[baselinestretch=0.7,frame=single]
$ mcut f=customer,amount i=dat1.csv >rsl2.csv
#END# kgcut f=customer,amount i=dat1.csv
$ more rsl2.csv
customer,amount
A,10
A,20
\end{Verbatim}


\subsubsection*{Related commands}
This parameter can be used in all commands except for certain commands such as \verb|sep|.

%\end{document}

