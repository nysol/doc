
%\begin{document}

\section{msep2 - Separate Records And Return Fields Table \label{sect:msep2}}
\index{msep2@msep2}
Separate the data according to the value of the field(s) specified in \verb|k=|. Partitioned data is automatically stored in numbered file name sequence. A table is created with the list of keys specified at \verb|k=|  and the corresponding file name for each key. 


\subsection*{Format}
\verb|msep2 k= O= a= [-p]|
\hyperref[sect:option_i]{[i=]}
\hyperref[sect:option_o]{[o=]}
\hyperref[sect:option_assert_nullkey]{[-assert\_nullkey]}
\hyperref[sect:option_nfn]{[-nfn]} 
\hyperref[sect:option_nfno]{[-nfno]}  
\hyperref[sect:option_x]{[-x]}
\hyperref[sect:option_q]{[-q]}
\hyperref[sect:option_option_tmppath]{[tmpPath=]}
\verb|[--help]|
\verb|[--helpl]|
\verb|[--version]|\\

\subsection*{Parameters}
\begin{table}[htbp]
%\begin{center}
{\small
\begin{tabular}{ll}
\verb|k=|    & List of field name(s) as unit of division. \\
\verb|O=|    & Create list of sequentially numbered file (serial number starting from 0) in the specified directory.\\
\verb|o=|    & Correspondence table with sequentially numbered file names and  values specified as key at \verb|k=| is output as CSV file. \\
		& Output is printed to standard output if this parameter is not specified. \\
\verb|a=|    & Field name of the path of output specified at \verb|o=|.\\
\verb|-p|    & Force create directory specified by pathname at  \verb|O=|.\\

\end{tabular} 
}
\end{table} 


\subsection*{Examples}
\subsubsection*{例1: 基本例}

\verb|item|項目別にデータを分割する。
出力ファイル名は0から始まる連番であり、どの番号がどのキーに対応しているかが\verb|table.csv|に出力される。


\begin{Verbatim}[baselinestretch=0.7,frame=single]
$ more dat1.csv
item,no
A,1
A,1
A,2
B,1
B,2
$ msep2 k=item O=./output a=fileName o=table.csv i=dat1.csv
#END# kgsep2 O=./output a=fileName i=dat1.csv k=item o=table.csv
$ ls ./output
0
1
$ more table.csv
item%0,fileName
A,./output/0
B,./output/1
$ more output/0
item%0,no
A,1
A,1
A,2
$ more output/1
item%0,no
B,1
B,2
\end{Verbatim}
\subsubsection*{例2: 複数キー項目}

複数のキー項目\verb|item,no|を設定しても同様に各ファイル名は連番で作成される。
\verb|table.csv|に複数のキー項目と番号の対応表が出力されている。


\begin{Verbatim}[baselinestretch=0.7,frame=single]
$ more dat1.csv
item,no
A,1
A,1
A,2
B,1
B,2
$ msep2 k=item,no O=./output2 a=fileName o=table.csv i=dat1.csv
#END# kgsep2 O=./output2 a=fileName i=dat1.csv k=item,no o=table.csv
$ ls ./output2
0
1
2
3
$ more table.csv
item%0,no%1,fileName
A,1,./output2/0
A,2,./output2/1
B,1,./output2/2
B,2,./output2/3
$ more output/0
item%0,no
A,1
A,1
A,2
\end{Verbatim}

\subsection*{Related Command}
\hyperref[sect:msep] {msep} : Use this command to include the field header name in file name.

%\end{document}
