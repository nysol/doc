%\documentclass[a4paper]{book}
%\usepackage{mcmd}
%\begin{document}
\section{mtraflg - Convert Cross (pivot) Table to Transaction Fields\label{sect:mtraflg}}
\index{mtraflg@mtraflg}
Check whether the field(s) specified in the f= parameter contains NULL value. Fields with non-NULL values are connected as one item and saved as a new vector. \\

\subsection*{Format}
\verb|mtraflg a= f= [delim=] [-r] | 
\hyperref[sect:option_i]{[i=]}
\hyperref[sect:option_o]{[o=]}
\hyperref[sect:option_assert_diffSize]{[-assert\_diffSize]}
\hyperref[sect:option_assert_nullin]{[-assert\_nullin]}
\hyperref[sect:option_assert_nullout]{[-assert\_nullout]}
\hyperref[sect:option_nfn]{[-nfn]} 
\hyperref[sect:option_nfno]{[-nfno]}  
\hyperref[sect:option_x]{[-x]}
\hyperref[sect:option_q]{[-q]}
\hyperref[sect:option_option_tmppath]{[tmpPath=]}
\verb|[--help]|
\verb|[--helpl]|
\verb|[--version]|\\

\subsection*{Parameters}
\begin{table}[htbp]
%\begin{center}
{\small
\begin{tabular}{ll}
\verb|a=|      & Specify the transaction field name.\\
\verb|f=|      & Check the values in the specified field name(s) (multiple fields can be specified)  to create transaction data.\\
               & (\verb|-r| option is specified, extract list of values as the field name of the transaction data)\\
\verb|delim=|  & Specify the character to separate each transaction field item (Default character is space if this parameter is omitted).\\
               & Character string should not be used. 1 byte character can be specified. \\
\verb|-r|      & Reverse conversion\\
               & Convert transaction based data to vertically structured data. \\
\end{tabular} 
}
\end{table} 

\subsection*{Examples}
\subsubsection*{Example 1: Basic Example}

Create a string of vector from the list of non-null values in column \verb|egg| and \verb|milk|.


\begin{Verbatim}[baselinestretch=0.7,frame=single]
$ more dat1.csv
customer,egg,milk
A,1,1
B,,1
C,1,
D,1,1
$ mtraflg f=egg,milk a=transaction i=dat1.csv o=rsl1.csv
#END# kgtraflg a=transaction f=egg,milk i=dat1.csv o=rsl1.csv
$ more rsl1.csv
customer,transaction
A,egg milk
B,milk
C,egg
D,egg milk
\end{Verbatim}
\subsubsection*{Example 2: Basic Example 2}

Use \verb|-r| option to revert the output results back to the original data.


\begin{Verbatim}[baselinestretch=0.7,frame=single]
$ mtraflg -r f=egg,milk a=transaction i=rsl1.csv o=rsl2.csv
#END# kgtraflg -r a=transaction f=egg,milk i=rsl1.csv o=rsl2.csv
$ more rsl2.csv
customer,egg,milk
A,1,1
B,,1
C,1,
D,1,1
\end{Verbatim}
\subsubsection*{Example 3: Specify the delimiter}

Combine items using the “-” (hyphen) as delimiter. Save output in column named \verb|transaction|.


\begin{Verbatim}[baselinestretch=0.7,frame=single]
$ mtraflg f=egg,milk a=transaction delim=- i=dat1.csv o=rsl3.csv
#END# kgtraflg a=transaction delim=- f=egg,milk i=dat1.csv o=rsl3.csv
$ more rsl3.csv
customer,transaction
A,egg-milk
B,milk
C,egg
D,egg-milk
\end{Verbatim}

\subsection*{Related Commands}
\hyperref[sect:mvsort] {mvsort} : Vector based transaction data can be processed by a set of commands  (with \verb|mv| as prefix) which handles vector data.

\hyperref[sect:mcross] {mcross} : Rather than converting as transaction data, every item is saved separately as individual field in the output.

\hyperref[sect:mtra] {mtra} : Create transaction data using values in the field.

\hyperref[sect:mtrafld] {mtrafld} :  Create transaction data with the format “field name=value”.
%\end{document}
