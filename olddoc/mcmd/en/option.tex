%\documentclass[a4paper]{book}
%\usepackage{mcmd}
%\begin{document}

\section{Specify Parameters}
The format of the parameters used in M-Command is slightly different than UNIX commands. 
The keyword and specified value is separated by an equal sign i.e.  "\verb|keyword=value|". 
Option type parameters precedes with a minus sign e.g. "\verb|-keyword|" and do not require specified value.

Many parameters share common functions in M-Command. The parameters are explained below. However, in some command, it works as a completely different function. 

\begin{table}[!htbp]
{\small
\begin{center}
\begin{tabular}{l|l}
\hline
Keyword&Description \\
\hline
\hyperref[sect:option_i]{i=}                & Input file name \\
\hyperref[sect:option_o]{o=}                & Output file name \\
\hyperref[sect:option_f]{f=}                & Input and output field name \\
\hyperref[sect:option_k]{k=}                & Key field name \\
\hyperref[sect:option_s]{s=}                & Sort field name  \\
\hyperref[sect:option_a]{a=}                & Add item name \\
\hyperref[sect:option_nfn]{-nfn}            & CSV without field name \\
\hyperref[sect:option_nfno]{-nfno}          & Output without field name \\
\hyperref[sect:option_x]{-x}                & Specify the field number \\
\hyperref[sect:option_q]{-q}                & Disable automatic sorting \\
\hyperref[sect:option_assert_diffSize]{[-assert\_diffSize]} & Compare numbers of inputs and outputs \\ % added from ver.3.0
\hyperref[sect:option_assert_nullkey]{[-assert\_nullkey]}   & Check whether a key field contains a NULL value \\ % added from ver.3.0
\hyperref[sect:option_assert_nullin]{[-assert\_nullin]}     & Check NULL value in the input field specified by f= or vf= \\ % added from ver.3.0
\hyperref[sect:option_assert_nullout]{[-assert\_nullout]}   & Check NULL value in output fields \\ % added from ver.3.0
\hyperref[sect:option_precision]{precision=}& Number of significant figures \\
\hyperref[sect:option_tmpPath]{tmpPath=}    & Work file storage path name \\
\hyperref[sect:option_delim]{delim=}        & Delimiter of vector data \\
\hyperref[sect:option_bufcount]{bufcount=}  & Number of buffers \\
                         \verb|--help|      & Display help \\
\hline
\end{tabular} 
\end{center}
}
\end{table} 

%\documentclass[a4paper]{book}
%\usepackage{mcmd}
%\begin{document}

\subsection{i= Input file name\label{sect:option_i}}
Specify the name of input file.
Most commands only allow a single file to be specified, with the exception of \verb|mcat| command where multiple files can be specified separated with a comma.
Yet, certain commands such as \verb|mnewnumber| and \verb|mnewrand| do not require input data. 

When this parameter is not defined, data is read from standard input by using pipeline. In the example below, \verb|i=| parameter is not specified for \verb|msum| command because the input data is the result of \verb|msortf|, which is read from standard input through the pipeline.

\begin{Verbatim}[baselinestretch=0.7,frame=single]
$ msortf f=a i=dat.csv | msum k=a f=b o=rsl.csv
\end{Verbatim}

However, it is difficult to identify errors when results are piped directly from one command to the next. 
In the following example, \verb|i=| parameter is also specified for \verb|msum|. 
The results of \verb|msortf| is sent to standard output, and msum reads input data from \verb|dat.csv|. 
Since \verb|msortf| did not add meaning to the input for \verb|msum|, the results from this example is different from the above. 

\begin{Verbatim}[baselinestretch=0.7,frame=single]
$ msortf f=a i=dat.csv | msum k=a f=b i=dat.csv o=rsl.csv
\end{Verbatim}

\subsection*{Examples}
\subsubsection*{例1: 基本例}

\verb|dat1.csv|を入力データとして\verb|mcut|は実行される。


\begin{Verbatim}[baselinestretch=0.7,frame=single]
$ more dat1.csv
顧客,数量,金額
A,1,10
A,2,20
$ mcut f=顧客,金額 i=dat1.csv o=rsl1.csv
#END# kgcut f=顧客,金額 i=dat1.csv o=rsl1.csv
$ more rsl1.csv
顧客,金額
A,10
A,20
\end{Verbatim}
\subsubsection*{例2: 出力項目名の指定}

標準入力をリダイレクト(\verb|"<"|記号)して読み込む。


\begin{Verbatim}[baselinestretch=0.7,frame=single]
$ mcut f=顧客,金額 o=rsl2.csv <dat1.csv
#END# kgcut f=顧客,金額 o=rsl2.csv
$ more rsl2.csv
顧客,金額
A,10
A,20
\end{Verbatim}


\subsubsection*{Related commands}
The parameter can be used in all commands except for commands such as \verb|mnewnumber| and \verb|mnewrand|.

%\end{document}



%\begin{document}

\subsection{o= 出力ファイル名\label{sect:option_o}}
出力ファイル名を指定する。
単一のファイルのみ指定可能である。
ただし、例外として、\verb|mtee|コマンドは複数の出力ファイルを指定でき、
また出力データを必要としないコマンド、例えば、\verb|msep|などもある。

このパラメータが省略された時には標準出力にデータを書き込む。
この機能があるために、パイプラインによる接続が可能となる。
例えば、以下の例では、\verb|msortf|で\verb|o=|を指定していないが、
これは\verb|msortf|の結果が標準出力を通じてパイプラインに出力されるためである。

\begin{Verbatim}[baselinestretch=0.7,frame=single]
$ msortf f=a i=dat.csv | msum k=a f=b o=rsl.csv
\end{Verbatim}

また、上記の似たような処理ではあるが、以下に示した例はうまく動作しない。
上記との違いは\verb|msortf|に\verb|o=|パラメータが指定されている点である。
\verb|msortf|の結果は\verb|tmp.csv|に出力されるが、標準出力に出力するデータがなく、
パイプ接続された\verb|msum|はいつまでも入力データが来るのを待つこととなり、一見動いているように見えて、いつまでも終了しない。

\begin{Verbatim}[baselinestretch=0.7,frame=single]
$ msortf f=a i=dat.csv o=tmp.csv | msum k=a f=b o=rsl.csv
\end{Verbatim}

少し複雑な例であるが、上記の例は以下のように\verb|mtee|コマンドを利用することでうまく動作するようになる。

\begin{Verbatim}[baselinestretch=0.7,frame=single]
$ msortf f=a i=dat.csv | mtee o=tmp.csv | msum k=a f=b o=rsl.csv
\end{Verbatim}

\verb|mtee|コマンドは、標準入力を\verb|o=|で指定されたファイルおよび標準出力に書き出す。
結果として、\verb|msortf|の結果は\verb|tmp.csv|に書きこまれ、\verb|msum|も\verb|mtee|よりパイプラインを通じてデータ供給を受け、
最終結果を\verb|rsl.csv|に書き出す。


\subsection*{利用例}
\subsubsection*{Example 1: Basic Example}

The result of \verb|mcut| is saved to \verb|rsl1.csv| as specified in \verb|o=| parameter.


\begin{Verbatim}[baselinestretch=0.7,frame=single]
$ more dat1.csv
customer,quantity,amount
A,1,10
A,2,20
$ mcut f=customer,amount i=dat1.csv o=rsl1.csv
#END# kgcut f=customer,amount i=dat1.csv o=rsl1.csv
$ more rsl1.csv
customer,amount
A,10
A,20
\end{Verbatim}
\subsubsection*{Example 2: Redirect}

Write to standard input using redirection (\verb|">"|).


\begin{Verbatim}[baselinestretch=0.7,frame=single]
$ mcut f=customer,amount i=dat1.csv >rsl2.csv
#END# kgcut f=customer,amount i=dat1.csv
$ more rsl2.csv
customer,amount
A,10
A,20
\end{Verbatim}


\subsubsection*{対応コマンド}
\verb|sep|など一部のコマンドを除いて全てのコマンドで利用できる。

%\end{document}


%\documentclass[a4paper]{book}
%\usepackage{mcmd}
%\begin{document}

\subsection{f= Input and output field name\label{sect:option_f}}

Specify the input and output field name for processing.
For example, this parameter specifies the "field name to select" in mcut, "field name to aggregate" for magg, and "field name to merge" for mjoin.
In addition, multiple field names can be specified separated by a comma in between such as \verb|f=a,b,c|.

The output field name for every specified item from the input file can be renamed in MCMD. This can be done by defining the input field name and output field name separated by a colon in between e.g. \verb|f=a:A,b:B|. The field name in the output remains the same if the output field name is not specified. 

\subsection*{Examples}
\subsubsection*{Example 1: Basic Example}

Extract fields \verb|val1| and \verb|val2|.


\begin{Verbatim}[baselinestretch=0.7,frame=single]
$ more dat1.csv
id,val1,val2
A,1,2
B,2,3
C,3,4
$ mcut f=val1,val2 i=dat1.csv o=rsl1.csv
#END# kgcut f=val1,val2 i=dat1.csv o=rsl1.csv
$ more rsl1.csv
val1,val2
1,2
2,3
3,4
\end{Verbatim}
\subsubsection*{Example 2: Specify name of output field}

Aggregate \verb|val1,val2|, and rename the fields in the output as \verb|sum1,sum2| respectively.


\begin{Verbatim}[baselinestretch=0.7,frame=single]
$ msum f=val1:sum1,val2:sum2 i=dat1.csv o=rsl2.csv
#END# kgsum f=val1:sum1,val2:sum2 i=dat1.csv o=rsl2.csv
$ more rsl2.csv
id,sum1,sum2
C,6,9
\end{Verbatim}


\subsubsection*{Related commands}
\hyperref[sect:mcut]{mcut},
\hyperref[sect:msum]{msum},
\hyperref[sect:mcat]{mcat},
\hyperref[sect:mjoin]{mjoin}, etc. 

%\end{document}



%\begin{document}

\subsection{k= キー項目名\label{sect:option_k}}



キー項目を指定する(複数項目指定可)。
キー項目とは、集計の単位として指定したり、
またファイルの結合時に2ファイル間の共通項目として指定したりする項目である。

たとえば\verb|msum|コマンドでは、同一キーごとに合計処理を実施する(集計キーブレイク処理)。
またmjoinコマンドでは、2つのデータファイルについて、キー項目の大小を見比べて
結合処理を実施する(結合キーブレイク処理)。

\verb|k=|パラメータが指定されたとき、コマンドはまずその項目を文字列昇順で並べ替えた上で、
それぞれの処理を実行する(ただし、\hyperref[sect:mhashsum]{mhashsum}コマンドのような例外もある)。

なおキーブレイク処理の詳細は、\hyperref[sect:keybreak]{キーブレイク処理}を参照のこと。
項目の並べ替えが頻繁に発生するとパフォーマンスの低下を招くため、
キーブレイク処理の内容と必要性を理解した上で、並べ替えの回数を少なくする
スクリプトを記述することが望ましい。

\subsection*{利用例}
\subsubsection*{Example 1: Basic Example}

Compute sum on \verb|val| column by \verb|id|.


\begin{Verbatim}[baselinestretch=0.7,frame=single]
$ more dat1.csv
id,val
A,1
B,1
B,2
A,2
B,3
$ msum i=dat1.csv k=id f=val o=rsl1.csv
#END# kgsum f=val i=dat1.csv k=id o=rsl1.csv
$ more rsl1.csv
id%0,val
A,3
B,6
\end{Verbatim}
\subsubsection*{Example 2: Join Process}

Use the join key “id” from \verb|dat1.csv|, and join the field “name” from \verb|ref1.csv|.


\begin{Verbatim}[baselinestretch=0.7,frame=single]
$ more dat1.csv
id,val
A,1
B,1
B,2
A,2
B,3
$ more ref1.csv
id,name
A,nysol
B,mcmd
$ mjoin k=id i=dat1.csv m=ref1.csv f=name o=rsl4.csv
#END# kgjoin f=name i=dat1.csv k=id m=ref1.csv o=rsl4.csv
$ more rsl4.csv
id%0,val,name
A,1,nysol
A,2,nysol
B,1,mcmd
B,2,mcmd
B,3,mcmd
\end{Verbatim}


\subsubsection*{対応コマンド}
\hyperref[sect:msum]{msum},
\hyperref[sect:mslide]{mslide},
\hyperref[sect:mjoin]{mjoin},
\hyperref[sect:mrjoin]{mrjoin},
\hyperref[sect:mcommon]{mcommon}など

%\end{document}



%\begin{document}

\subsection{s= Sort Field Name\label{sect:option_s}}

Specify the field name for sorting (multiple fields can be specified). 

The order of records affects the process results for some commands such as \verb|maccum|. 
When \verb|s=| parameter is specified, sorting is carried out on the specified fields before the processing command. 

There are four combinations of sorting methods (order), including numeric / string, and ascending / descending order. 
The sorting methods can be specified by appending \verb|%| followed by \verb|n| or \verb|r| after the column name. The examples are as follows. 

Character string ascending order: \verb|field|  (\verb|%| not required), character string descending order: \verb|f=field%r|, numeric ascending order: \verb|f=field%n|, numeric descending order:\verb|f=field%nr|.


\subsection*{Example}
\subsubsection*{例1: 基本例}

\verb|id|項目で並べ替えた後、\verb|val|項目の累計を計算する。


\begin{Verbatim}[baselinestretch=0.7,frame=single]
$ more dat1.csv
id,val
A,1
B,1
B,2
A,2
B,3
$ maccum s=id k=id f=val:val_accum i=dat1.csv o=rsl1.csv
#END# kgaccum f=val:val_accum i=dat1.csv k=id o=rsl1.csv s=id
$ more rsl1.csv
id,val,val_accum
A,1,1
A,2,3
B,1,1
B,2,3
B,3,6
\end{Verbatim}
\subsubsection*{例2: 並べ替え方法を指定する}

\verb|val|項目を数値降順で並べ替えた後、\verb|val|項目の累計を計算する。


\begin{Verbatim}[baselinestretch=0.7,frame=single]
$ more dat1.csv
id,val
A,1
B,1
B,2
A,2
B,3
$ maccum s=id,val%nr k=id f=val:val_accum i=dat1.csv o=rsl1.csv
#END# kgaccum f=val:val_accum i=dat1.csv k=id o=rsl1.csv s=id,val%nr
$ more rsl1.csv
id,val,val_accum
A,2,2
A,1,3
B,3,3
B,2,5
B,1,6
\end{Verbatim}


\subsubsection*{Corresponding Commands}
\hyperref[sect:maccum]{maccum},
\hyperref[sect:mbest]{mbest},
\hyperref[sect:mmvavg]{mmvavg},
\hyperref[sect:mnumber]{mnumber},
\hyperref[sect:mslide]{mslide}, etc. 

%\end{document}



%\begin{document}

\subsection{a= 追加項目名\label{sect:option_a}}

新たに項目を追加するようなコマンドにおいて、その項目名を指定する。
多くのコマンドは、追加する項目は一つであるため、ここで指定する項目名も一つであることが多い。
中には、\verb|mcombi|コマンドのように複数の項目を出力するものもあるが、その際はカンマで区切って複数の項目名を指定する。

\subsection*{利用例}
\subsubsection*{Example 1: Basic Example}

Add a new field as “payday”.


\begin{Verbatim}[baselinestretch=0.7,frame=single]
$ more dat1.csv
id
A
B
C
$ msetstr v=20070101 a=payday i=dat1.csv o=rsl1.csv
#END# kgsetstr a=payday i=dat1.csv o=rsl1.csv v=20070101
$ more rsl1.csv
id,payday
A,20070101
B,20070101
C,20070101
\end{Verbatim}
\subsubsection*{Example 2: Add multiple fields}

Enumerate the two combination of each item \verb|A,B,C| in the column “id”.


\begin{Verbatim}[baselinestretch=0.7,frame=single]
$ mcombi f=id n=2 a=id1,id2 i=dat1.csv o=rsl2.csv
#END# kgcombi a=id1,id2 f=id i=dat1.csv n=2 o=rsl2.csv
$ more rsl2.csv
id,id1,id2
C,A,B
C,A,C
C,B,C
\end{Verbatim}


\subsubsection*{対応コマンド}
\hyperref[sect:mcal]{mcal},
\hyperref[sect:mcombi]{mcombi},
\hyperref[sect:mrand]{mrand},
\hyperref[sect:msetstr]{msetstr}など

%\end{document}



%\begin{document}

\subsection{-nfn 項目名行のないCSV(No Field Namesの略)\label{sect:option_nfn}}

このオプションを指定すると入力データの1行目を項目名行とみなさない。
主に1行目に項目名がないデータの場合に利用される。
このフラグを指定すると項目指定のときに項目名は利用できないので項目番号指定をすることになる。
項目番号は0から始まる整数で指定することに注意する。
\verb|-nfn|オプションを指定すると、出力ファイルにも項目名は出力されない。

\subsection*{利用例}
\subsubsection*{Example 1: Basic Example}

Extract column0 and 2.


\begin{Verbatim}[baselinestretch=0.7,frame=single]
$ more dat1.csv
A,1,10
A,2,20
B,1,15
B,3,10
B,1,20
$ mcut -nfn f=0,2 i=dat1.csv o=rsl1.csv
#END# kgcut -nfn f=0,2 i=dat1.csv o=rsl1.csv
$ more rsl1.csv
A,10
A,20
B,15
B,10
B,20
\end{Verbatim}


\subsubsection*{対応コマンド}
mchkcsv以外全てのコマンドで利用できる。

%\end{document}



%\begin{document}

\subsection{-nfno 項目名行を出力しない(No Field Names for Outputの略)\label{sect:option_nfno}}
このオプションを指定すると出力データに項目名行を出力しない。
\verb|-nfn|とは違い、\verb|i=|や\verb|m=|で指定される入力データは項目名行を伴うデータを前提とする。

\subsection*{利用例}
\subsubsection*{Example 1: Basic Example}

Extract quantity and amount,but the field names is removed from the output data.


\begin{Verbatim}[baselinestretch=0.7,frame=single]
$ more dat1.csv
顧客,数量,金額
A,1,10
A,2,20
B,1,15
B,3,10
B,1,20
$ mcut -nfno f= quantity, amount i=dat1.csv o=rsl1.csv
#ERROR# invalid argument: quantity, (kgcut)
$ more rsl1.csv
rsl1.csv: No such file or directory
\end{Verbatim}


\subsubsection*{対応コマンド}
mchkcsv以外全てのコマンドで利用できる。

%\end{document}


%\documentclass[a4paper]{book}
%\usepackage{mcmd}
%\begin{document}

\subsection{-x Specify by item number\label{sect:option_x}}

This option allows user to specify a column with corresponding field number where input data includes field names. Users can specify the output field name(s) by adding colon right after input field, followed by the output field name. 

\subsection*{Examples}
\subsubsection*{例1: 基本例}

0番目項目を集計キーとして1番目と2番目の項目を合計する。


\begin{Verbatim}[baselinestretch=0.7,frame=single]
$ more dat1.csv
顧客,数量,金額
A,1,10
A,2,20
B,1,15
B,3,10
B,1,20
$ msum -x k=0 f=1,2 i=dat1.csv o=rsl1.csv
#END# kgsum -x f=1,2 i=dat1.csv k=0 o=rsl1.csv
$ more rsl1.csv
顧客%0,数量,金額
A,3,30
B,5,45
\end{Verbatim}
\subsubsection*{例2: 出力項目名も利用可能}

1番目と2番目の項目は、\verb|a,b|という名前で出力する。


\begin{Verbatim}[baselinestretch=0.7,frame=single]
$ msum -x k=0 f=1:a,2:b i=dat1.csv o=rsl2.csv
#END# kgsum -x f=1:a,2:b i=dat1.csv k=0 o=rsl2.csv
$ more rsl2.csv
顧客%0,a,b
A,3,30
B,5,45
\end{Verbatim}
\subsubsection*{例3: -nfnではうまくいかない}

\verb|-nfn|は、最初の行をデータ行としてみなすので、「数量」「金額」というデータを合計しようとしてしまい、うまくいかない。
\verb|-x|は、あくまでも最初の行は項目名行とみなす点が\verb|-nfn|と異なる。


\begin{Verbatim}[baselinestretch=0.7,frame=single]
$ msum -nfn k=0 f=1,2 i=dat1.csv o=rsl3.csv
#END# kgsum -nfn f=1,2 i=dat1.csv k=0 o=rsl3.csv
$ more rsl3.csv
顧客,0,0
A,3,30
B,5,45
\end{Verbatim}


\subsubsection*{Related commands}
This option can be used in all commands except mchkcsv.

%\end{document}



%\begin{document}

\subsection{-q 自動並べ替えの無効化\label{sect:option_q}}

\verb|k=|パラメータで指定した項目による自動並べ替えを無効にしたい場合にこのオプションを用いる。
\verb|s=|オプションも省略可能となり、各コマンドはMCMD Ver. 1.0と同等の動作をするようになる。

\subsection*{利用例}
\subsubsection*{Example 1: Basic Example }

Find out the cumulative value by \verb|id| field. When \verb|-q| option is specified, sorting by field specified at \verb|k=| parameter will be disabled.


\begin{Verbatim}[baselinestretch=0.7,frame=single]
$ more dat1.csv
id,val
A,1
B,1
B,2
A,2
B,3
$ maccum -q k=id f=val:val_accum i=dat1.csv o=rsl1.csv
#END# kgaccum -q f=val:val_accum i=dat1.csv k=id o=rsl1.csv
$ more rsl1.csv
id,val,val_accum
A,1,1
B,1,1
B,2,3
A,2,2
B,3,3
\end{Verbatim}


\subsubsection*{対応コマンド}
\verb|k=|パラメータを持つすべてのコマンドで利用できる。

%\end{document}




%\documentclass[a4paper]{jsbook}
%\usepackage{mcmd_jp}
%\begin{document}

\subsection{-assert\_diffSize Compare numbers of inputs and outputs\label{sect:option_assert_diffSize}}

Specify this option to compare the numbers of the input and output files. 
When the numbers do not match, the \verb|“#WARNING# ; the number of lines is different”| message is shown.

\subsection*{Example}
\subsubsection*{(基本例) }
例えば、mjoin(参照ファイルの項目結合)コマンドを利用する際に、入力ファイルのキー項目(k=パラメータで指定する項目)と参照ファイルのキー項目(K=パラメータで指定する項目)が完全に一致しているかどうかを確認したい場合を想定する。mjoinコマンドでNULL値を出力する-n、-Nパラメータを指定しない場合は、入力ファイルと参照ファイルで共通のキー項目のみが結合され、一致しないキー項目の値は除外される為、入力データと出力データの件数が異なる。その際、-assert\_diffSizeパラメータを指定しておくと、入力ファイルと出力ファイルの件数の比較を行い、入力ファイルと出力ファイルの件数が異なる場合に\verb|「#WARNING# ; the number of lines is different」|というメッセージを表示するので入力ファイルと参照ファイルのキー項目が完全に一致していないことを確認することができる。

\begin{Verbatim}[baselinestretch=0.7,frame=single]
$ more dat1.csv
item,date,price
A,20081201,100
A,20081213,98
B,20081002,400
B,20081209,450
C,20081201,100

$ more ref1.csv
item,cost
A,50
B,300
E,200

$ mjoin k=item f=cost m=ref1.csv -assert_diffSize i=dat1.csv o=rsl1.csv
#WARNING# ; the number of lines is different
#END# kgjoin -assert_diffSize f=cost i=dat1.csv k=item m=ref1.csv o=rsl1.csv; IN=5 OUT=4

$ more rsl1.csv
item%0,date,price,cost
A,20081201,100,50
A,20081213,98,50
B,20081002,400,300
B,20081209,450,300
\end{Verbatim}


\subsubsection*{Related commands}
This option can be used for all commands except for the following: \\
\hyperref[sect:marff2csv]{marff2csv},
\hyperref[sect:mchkcsv]{mchkcsv},
\hyperref[sect:mcsv2arff]{mcsv2arff},
\hyperref[sect:mnewnumber]{mnewnumber},
\hyperref[sect:mnewrand]{mnewrand},
\hyperref[sect:mnewstr]{mnewstr},
\hyperref[sect:msep]{msep},
\hyperref[sect:msep2]{msep2},
\hyperref[sect:mtee]{mtee},
\hyperref[sect:mxml2csv]{mxml2csv}\\

%\end{document}



%\documentclass[a4paper]{jsbook}
%\usepackage{mcmd_jp}
%\begin{document}

\subsection{-assert\_nullkey Check whether a key field contains a NULL value \label{sect:option_assert_nullkey}}

Specify this option to check whether a key field (a field specified by the k= or K= parameter) contains a NULL value. When a NULL value is contained, the \verb|“#WARNING# ; exist NULL in key filed”| message is shown.

\subsection*{Example}
\subsubsection*{(Basic example) }
Assume, for instance, the msum command (sum the field values) is used. When the values in the fields specified by the f= parameter is summed for the rows that have the same value in the fields specified by the k= parameter, the value of the key field specified by the k= parameter may contain a NULL value. Specify the -assert\_nullkey option to check whether a key field contains a NULL value. When a NULL value is contained, the \verb|“#WARNING# ; exist NULL in key filed”| message is shown.

\begin{Verbatim}[baselinestretch=0.7,frame=single]
$ more dat1.csv
customer,quantity,金額
A,1,10
,1,10
B,1,15
A,2,20
B,3,10
B,1,20

$ msum k=customer f=quantity:quantityT,Amount:AmountT -assert_nullkey i=dat1.csv o=rsl1.csv
#WARNING# ; exist NULL in key filed
#END# kgsum -assert_nullkey f=quantity:quantityT,Amount:AmountT i=dat1.csv k=customer o=rsl1.csv

$ more rsl1.csv
customer%0,quantityT,AmountT
,1,10
A,3,30
B,5,45
\end{Verbatim}


\subsubsection*{Related commands}
This option can be used for the following commands: \\
\hyperref[sect:maccum]{maccum},
\hyperref[sect:mavg]{mavg},
\hyperref[sect:mbest]{mbest},
\hyperref[sect:mbucket]{mbucket},
\hyperref[sect:mcal]{mcal},
\hyperref[sect:mcommon]{mcommon},
\hyperref[sect:mcount]{mcount},
\hyperref[sect:mcross]{mcross},
\hyperref[sect:mdelnull]{mdelnull},
\hyperref[sect:mhashavg]{mhashavg},
\hyperref[sect:mhashsum]{mhashsum},
\hyperref[sect:mjoin]{mjoin},
\hyperref[sect:mkeybreak]{mkeybreak},
\hyperref[sect:mmbucket]{mmbucket},
\hyperref[sect:mmvavg]{mmvavg},
\hyperref[sect:mmvsim]{mmvsim},
\hyperref[sect:mstats]{mstats},
\hyperref[sect:mnjoin]{mnjoin},
\hyperref[sect:mnormalize]{mnormalize},
\hyperref[sect:mnrcommon]{mnrcommon},
\hyperref[sect:mnrjoin]{mnrjoin},
\hyperref[sect:mnumber]{mnumber},
\hyperref[sect:mpadding]{mpadding},
\hyperref[sect:mrand]{mrand},
\hyperref[sect:mrjoin]{mrjoin},
\hyperref[sect:mselnum]{mselnum},
\hyperref[sect:mselrand]{mselrand},
\hyperref[sect:mselstr]{mselstr},
\hyperref[sect:msep2]{msep2},
\hyperref[sect:mshare]{mshare},
\hyperref[sect:msim]{msim},
\hyperref[sect:mslide]{mslide},
\hyperref[sect:mstats]{mstats},
\hyperref[sect:msum]{msum},
\hyperref[sect:msummary]{msummary},
\hyperref[sect:mtra]{mtra},
\hyperref[sect:muniq]{muniq},
\hyperref[sect:mwindow]{mwindow}\\

%\end{document}




%\documentclass[a4paper]{jsbook}
%\usepackage{mcmd_jp}
%\begin{document}

\subsection{-assert\_nullin -assert nullin Check NULL value in the input field specified by f= or vf= \label{sect:option_assert_nullin}}
Specify this option to check whether the input field specified by f= or vf= contains a NULL value. When a NULL value is contained, the \verb|“#WARNING# ; exist NULL in input data”| message is shown.

\subsection*{Example}
\subsubsection*{(Basic example) }
Assume, for instance, the maccum command (accumulation) is used. The maccum command ignores the fields specified by the f= parameter if they contain a NULL value. Specify the -assert\_nullin option to check whether a key field specified by the f= parameter contains a NULL value. When a NULL value is contained, the \verb|“#WARNING# ; exist NULL in input data”| message is shown.

\begin{Verbatim}[baselinestretch=0.7,frame=single]
$ more dat1.csv
customer,quantity,amount
A,1,
A,2,20
B,1,15
B,3,10
B,,20

$ maccum s=customer f=quantity:quantityC,amount:amountC -assert_nullin i=dat1.csv o=rsl1.csv
#WARNING# ; exist NULL in input data
#END# kgaccum -assert_nullin f=quantity:quantityC,amount:amountC i=dat1.csv o=rsl1.csv s=customer

$ more rsl1.csv
customer%0,quantity,amount,quantityC,amountC
A,1,,1,
A,2,20,3,20
B,1,15,4,35
B,3,10,7,45
B,,20,,65
\end{Verbatim}


\subsubsection*{Related commands}
This option can be used for all commands except for the following: \\
\hyperref[sect:marff2csv]{marff2csv},
\hyperref[sect:mbest]{mbest},
\hyperref[sect:mbucket]{mbucket},
\hyperref[sect:mchkcsv]{mchkcsv},
\hyperref[sect:mcommon]{mcommon},
\hyperref[sect:mcount]{mcount},
\hyperref[sect:mdelnull]{mdelnull},
\hyperref[sect:mfldname]{mfldname},
\hyperref[sect:mkeybreak]{mkeybreak},
\hyperref[sect:mnewnumber]{mnewnumber},
\hyperref[sect:mnewrand]{mnewrand},
\hyperref[sect:mnewstr]{mnewstr},
\hyperref[sect:mnrcommon]{mnrcommon},
\hyperref[sect:mnullto]{mnullto},
\hyperref[sect:mnumber]{mnumber},
\hyperref[sect:mrand]{mrand},
\hyperref[sect:msel]{msel},
\hyperref[sect:mselrand]{mselrand},
\hyperref[sect:msep2]{msep2},
\hyperref[sect:msetstr]{msetstr},
\hyperref[sect:msortf]{msortf},
\hyperref[sect:mtee]{mtee},
\hyperref[sect:muniq]{muniq},
\hyperref[sect:mwindow]{mwindow},
\hyperref[sect:mxml2csv]{mxml2csv}\\

%\end{document}




%\documentclass[a4paper]{jsbook}
%\usepackage{mcmd_jp}
%\begin{document}

\subsection{-assert\_nullout Check NULL value in output fields\label{sect:option_assert_nullout}}

Specify this option to check whether an output field contains a NULL value. When a NULL value is contained, the \verb|“#WARNING# ; exist NULL in output data”| message is shown. This option does not work on fields in which the input data is output as is, such as the calculation field.

\subsection*{Example}
\subsubsection*{(Basic example)}
Assume, for instance, the mslide command (slide rows) is used. When the values of the fields specified by the f= parameter are slid for the number of rows specified by the t= parameter for each field specified by the k= parameter, the number of slides specified by the t= parameter can be greater than the number of rows of the fields specified by the f= parameter depending on the data. If that is the case, the -n option can be specified to output a field with a NULL value in it in the absence of the next (or previous) row. Specify the -assert\_nullout option to check whether an output field contains a NULL value. When a NULL value is contained, the \verb|“#WARNING# ; exist NULL in output data”| message is shown.
\\In the example below, the option is used to check whether the syo\_1 and syo\_2 output fields contain a NULL value. Since the two fields contain a NULL value, the \verb|“#WARNING# ; exist NULL in output data”| message is shown.

\begin{Verbatim}[baselinestretch=0.7,frame=single]
$ more dat1.csv
customer,date,product,quantity
A,20130406,a,1
A,20130408,b,1
A,20130416,c,1
B,20130407,k,2
C,20130408,d,1
C,20130409,e,4

$ mslide s=customer,date k=customer f=product:syo_ t=2 -n -assert_nullout i=dat1.csv o=rsl1.csv
#WARNING# ; exist NULL in output data
#END# kgslide -assert_nullout -n f=product:syo_ i=dat1.csv k=customer o=rsl1.csv s=customer,date t=2

$ more rsl1.csv
customer,date,product,quantity,syo_1,syo_2
A,20130406,a,1,b,c
A,20130408,b,1,c,
A,20130416,c,1,,
B,20130407,k,2,,
C,20130408,d,1,e,
C,20130409,e,4,,
\end{Verbatim}


\subsubsection*{Related commands}
This option can be used for all commands except for the following: \\
\hyperref[sect:mbest]{mbest},
\hyperref[sect:mcat]{mcat},
\hyperref[sect:mcombi]{mcombi},
\hyperref[sect:mcommon]{mcommon},
\hyperref[sect:mcount]{mcount},
\hyperref[sect:mcsv2arff]{mcsv2arff},
\hyperref[sect:mcut]{mcut},
\hyperref[sect:mdelnull]{mdelnull},
\hyperref[sect:mduprec]{mduprec},
\hyperref[sect:mfldname]{mfldname},
\hyperref[sect:mfsort]{mfsort},
\hyperref[sect:mnewnumber]{mnewnumber},
\hyperref[sect:mnewrand]{mnewrand},
\hyperref[sect:mnewstr]{mnewstr},
\hyperref[sect:mnrcommon]{mnrcommon},
\hyperref[sect:mnullto]{mnullto},
\hyperref[sect:mnumber]{mnumber},
\hyperref[sect:mproduct]{mproduct},
\hyperref[sect:mrand]{mrand},
\hyperref[sect:msel]{msel},
\hyperref[sect:mselnum]{mselnum},
\hyperref[sect:mselrand]{mselrand},
\hyperref[sect:mselstr]{mselstr},
\hyperref[sect:msep]{msep},
\hyperref[sect:msep2]{msep2},
\hyperref[sect:msetstr]{msetstr},
\hyperref[sect:msortf]{msortf},
\hyperref[sect:mtee]{mtee},
\hyperref[sect:mtonull]{mtonull},
\hyperref[sect:muniq]{muniq},
\hyperref[sect:mvcount]{mvcount},
\hyperref[sect:mwindow]{mwindow},
\hyperref[sect:mxml2csv]{mxml2csv}\\

%\end{document}



%\begin{document}

\subsection{precision= 有効桁数\label{sect:option_precision}}
内部的にはC言語におけるsprintfの書式「\verb|"%.|$n$\verb|g"|」を用いている。
この書式は、データの桁数と指定した有効桁数によって、標準標記(整数部.小数部: ex. \verb|123.456|)と、
指数表記(仮数部e$\pm$指数部: ex. \verb|1.23456e+02|)を切り替える。
切り替えの基準であるが、データを指数表記で表したときに、指数部が指定の有効桁数を超えるか、
もしくは-5以下の場合(すなわち、小数点以下に0が4つ以上続く場合)に指数表記を採用する。

$n$は1〜16の整数が指定可能で、デフォルトは10である。
$n<1$の場合は$n=1$にセットされ、$n>16$の場合は$n=16$にセットされる。

また、環境変数\verb|KG_Precision|を設定することでも有効桁数を変更できる。
ただし、環境変数を変更すると、それ以降に実行するコマンド全てに反映されることに注意する。

\subsection*{利用例}
\subsubsection*{例1: 基本例}

id=1は指数表現で1.2345678e+08であり、指数部が有効桁数6を超えているので指数表記となり、仮数部の有効桁数が6となっている。
id=2は指数表現で1.23456789e+03であり、指数部が有効桁数7を超えていないので標準標記となり、整数部+小数部の桁数が6となっている。
id=4は指数表現で1.23456789e-04であり、指数部が-4未満ではないので標準標記となり、有効桁数が6となっている。
id=5は指数表現で1.23456789e-05であり、指数部が-4未満となるため指数表記となり、仮数部の有効桁数が6となっている。


\begin{Verbatim}[baselinestretch=0.7,frame=single]
$ more dat1.csv
id,val
1,123456789
2,1234.56789
3,0.123456789
4,0.000123456789
5,0.0000123456789
$ mcal c='${val}' a=result precision=6 i=dat1.csv o=rsl1.csv
#END# kgcal a=result c=${val} i=dat1.csv o=rsl1.csv precision=6
$ more rsl1.csv
id,val,result
1,123456789,1.23457e+08
2,1234.56789,1234.57
3,0.123456789,0.123457
4,0.000123456789,0.000123457
5,0.0000123456789,1.23457e-05
\end{Verbatim}
\subsubsection*{例2: presicion=2の場合}



\begin{Verbatim}[baselinestretch=0.7,frame=single]
$ mcal c='${val}' a=result precision=2 i=dat1.csv o=rsl2.csv
#END# kgcal a=result c=${val} i=dat1.csv o=rsl2.csv precision=2
$ more rsl2.csv
id,val,result
1,123456789,1.2e+08
2,1234.56789,1.2e+03
3,0.123456789,0.12
4,0.000123456789,0.00012
5,0.0000123456789,1.2e-05
\end{Verbatim}
\subsubsection*{例3: 環境変数による指定}

環境変数によって設定すると、それ以降全てのコマンドがその設定値を使う。


\begin{Verbatim}[baselinestretch=0.7,frame=single]
$ export KG_Precision=4
$ mcal c='${val}' a=result i=dat1.csv o=rsl3.csv
#END# kgcal a=result c=${val} i=dat1.csv o=rsl3.csv
$ more rsl3.csv
id,val,result
1,123456789,1.235e+08
2,1234.56789,1235
3,0.123456789,0.1235
4,0.000123456789,0.0001235
5,0.0000123456789,1.235e-05
\end{Verbatim}


\subsubsection*{対応コマンド}
\hyperref[sect:msum]{msum},
\hyperref[sect:mcal]{mcal}などの実数値の演算を伴うコマンド全てで利用できる。

%\end{document}


%\documentclass[a4paper]{book}
%\usepackage{mcmd}
%\begin{document}

\subsection{tmpPath= Path name of temporary file\label{sect:option_tmpPath}}

Specify the name of the directory which stores the temporary files for use by the command.
For example, the results from \verb|msortf| is saved as a temporary file during partitioned sort. If the path is not specified, the file is saved in \verb|/tmp|. The name of temporary files begins with \verb|__KGTMP|.

The temporary files are deleted if the command terminates normally (includes termination by exit signal, or termination by signal from MCMD signal). Temporary files will be retained in the directory when the program is terminated unexpectedly by power outage or bug.

Depending on the amount of data, enormous amount of temporary data may be generated (more than 1 million files). This will significantly slow down the execution of commands, therefore, clean out the files in the temporary path on a regular basis. Currently there is no plans to implement functions for garbage collection to remove objects no longer used by the program.

The temporary directory can be changed by setting the environment variable \verb|KG_Tmp_Path|, however, the same variable applies to the execution of all commands. 


\subsection*{Examples}
\subsubsection*{Example 1: Basic Example}

Set the \verb|tmp| directory under the current directory for temporary files.


\begin{Verbatim}[baselinestretch=0.7,frame=single]
$ msortf f=val tmpPath=./tmp i=dat1.csv o=rsl1.csv
#ERROR# internal error: cannot create temp file (kgsortf)
\end{Verbatim}
\subsubsection*{Example 2: Specify the environment variable}

The settings of the environment variable will be applied to subsequent commands.


\begin{Verbatim}[baselinestretch=0.7,frame=single]
$ export KG_TmpPath=~/tmp
$ msortf f=val i=dat1.csv o=rsl1.csv
#END# kgsortf f=val i=dat1.csv o=rsl1.csv
\end{Verbatim}


\subsubsection*{Related commands}
This applies to commands such as \hyperref[sect:msortf]{msortf} and \hyperref[sect:mselstr]{mdelnull} which select records by key field,  and commands such as \hyperref[mbucket]{mbucket}, \hyperref[mnjoin]{mnjoin}, and \hyperref[mshare]{mshare} that require multiple pass scanning based on key field.

%\end{document}



%\begin{document}

\subsection{delim= ベクトル要素の区切り文字\label{sect:option_delim}}
ベクトル型データについて、要素の区切り文字を指定する。
デフォルトは半角スペースである。
CSVの区切り文字であるカンマを指定することもできるが、
ベクトルの区切り文字と混同しないよう、ベクトル全体がダブルクオーテーションで囲われる。

\subsection*{利用例}
\subsubsection*{例1: 基本例}

コロンを区切り文字として、ベクトル項目\verb|vec|の要素を並べ替える。


\begin{Verbatim}[baselinestretch=0.7,frame=single]
$ more dat1.csv
vec
b:a:c
x:p
$ mvsort vf=vec delim=: i=dat1.csv o=rsl1.csv
#END# kgvsort delim=: i=dat1.csv o=rsl1.csv vf=vec
$ more rsl1.csv
vec
a:b:c
p:x
\end{Verbatim}
\subsubsection*{例2: delimを指定しないと}

delimを指定していないので\verb|b:a:c|や\verb|x:p|は一つの要素として解釈される。


\begin{Verbatim}[baselinestretch=0.7,frame=single]
$ mvsort vf=vec i=dat1.csv o=rsl2.csv
#END# kgvsort i=dat1.csv o=rsl2.csv vf=vec
$ more rsl2.csv
vec
b:a:c
x:p
\end{Verbatim}
\subsubsection*{例3: カンマを区切り文字にする}

区切り文字をカンマにした場合は、ベクトル全体がダブルクオーテーションで囲われることで
CSVの区切り文字との区別がつけられる。


\begin{Verbatim}[baselinestretch=0.7,frame=single]
$ more dat2.csv
id,vec1,vec2
1,a,b
2,p,q
$ mvcat vf=vec1,vec2 a=vec3 delim=, i=dat2.csv o=rsl3.csv
#END# kgvcat a=vec3 delim=, i=dat2.csv o=rsl3.csv vf=vec1,vec2
$ more rsl3.csv
id,vec3
1,"a,b"
2,"p,q"
\end{Verbatim}


\subsubsection*{対応コマンド}
\hyperref[sect:mvcat]{mvcat},
\hyperref[sect:mvsort]{mvsort}などベクトル型項目を扱うコマンド全てに適用できる。

%\end{document}



%\begin{document}

\subsection{bufcount= バッファの数\label{sect:option_bufcount}}
mbucket,mnjoin,mshareなど、キー単位の処理において、データを複数パス走査する必要のあるコマンドにおいて
利用する内部バッファの数(ブロック数)を指定する。
一つのバッファは4MBで、デフォルトでは10ブロック(40MB)である。
データがバッファに収まらない場合は一時ファイルに書き出されるため、
キーのサイズが非常に大きい場合は、メモリに余裕があれば、このパラメータを調整することで処理速度の向上が期待できる。

\subsection*{利用例}
\subsubsection*{例1: 基本例}

参照ファイルのキーサイズが80MB(4MB×20)以内であれば、一時ファイルは使われない。


\begin{Verbatim}[baselinestretch=0.7,frame=single]
$ mnjoin k=id m=ref.csv f=name i=dat.csv o=rsl.csv bufcount=20
#END# kgnjoin bufcount=20 f=name i=dat.csv k=id m=ref.csv o=rsl.csv
\end{Verbatim}


\subsubsection*{対応コマンド}
\hyperref[sect:mbucket]{mbucket},
\hyperref[sect:mnjoin]{mnjoin},
\hyperref[sect:mshare]{mshare}など、キー単位の処理において、データを複数パス走査する必要のあるコマンド。

%\end{document}



%\end{document}
