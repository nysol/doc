
%\begin{document}

\section{mnrjoin - Natural Join within Multiple Ranges with Reference File\label{sect:mnrjoin}}
\index{mnrjoin@mnrjoin}
Join columns according to the range of values in the column from reference file. The field specified at \verb|r=| parameter is matched with the the range of values defined as two arguments at the \verb|R=| parameter in the reference file defined at the \verb|m=| parameter. the field(s) specified at\verb|f=| parameter are joined for records with the same value. 

If there are more than one match for each record, natural join returns output for all rows. The range of values is compared as character strings by default. Attach \%n after the field name at the \verb|r=| parameter to process as numerical values.


\subsection*{Format}
\verb/mnrjoin  R= r= [k=] [K=] [f=] [-n] [-N] m=|/ 
\hyperref[sect:option_i]{i=}
\hyperref[sect:option_o]{[o=]}
\hyperref[sect:option_assert_diffSize]{[-assert\_diffSize]}
\hyperref[sect:option_assert_nullkey]{[-assert\_nullkey]}
\hyperref[sect:option_assert_nullin]{[-assert\_nullin]}
\hyperref[sect:option_assert_nullout]{[-assert\_nullout]}
\hyperref[sect:option_nfn]{[-nfn]} 
\hyperref[sect:option_nfno]{[-nfno]}  
\hyperref[sect:option_x]{[-x]}
\hyperref[sect:option_q]{[-q]}
\hyperref[sect:option_option_tmppath]{[tmpPath=]}
\verb|[--help]|
\verb|[--helpl]|
\verb|[--version]|\\

\subsection*{Parameters}
\begin{table}[htbp]
%\begin{center}
{\small
\begin{tabular}{ll}
\verb|f=|    & The field name(s) (multiple fields can be specified) to join from the reference file. \\
             & When this is not defined, the all fields except the key specified at \verb|K=| will be joined. \\
\verb|m=|    & Reference file name.\\
             & Read from standard input if this parameter is not set (when \verb|i=| is specified). \\
\verb|R=|    & Field names of the range (limit to 2 fields). \\
             & Field names (start,end) of the range in reference file.  \\
             & If the first field is NULL, the range is any number less than the ending value of the range.\\
             & If the second field is NULL, the range is any number greater than the starting value of the range. \\
\verb|r=|    & Compare the values in this field [\%{n}] against the range.\\
             & Field name in the input file. \\
             & Add \%n after the field name in the \verb|r=| parameter to process as numerical values. \\
\verb|k=|    & Key field name(s) (multiple fields can be specified) from the input data for comparison  \\
             &  Join records with same key fields in the input data \verb|k=| and reference data \verb|K=|. \\
\verb|K=|    & Key field name(s) (multiple fields can be specified) from the reference data for comparison \\
             & This key field(s) from reference data is compared with the key field(s) from the input data \\
             & specified at \verb|k=| parameter, fields where records with common key are joined. \\
             & This parameter is not required if the field name name is the same as the one defined at the \verb|k=| parameter. \\
\verb|-n|    & Output NULL values when reference data does not consist of input data. \\
\verb|-N|    & Output NULL values when input data does not consist of reference data.\\
\end{tabular} 
}
\end{table} 

%\subsection*{Sort criteria}
%The fields defined at \verb|r=| and \verb|R=| must be sorted beforehand. However, when the numeric value is compared with the numeric range, the values in the columns defined at \verb|r=| and \verb|R=| parameter must sorted by numerical ascending order. When \verb|k=| and \verb|K=| is specified, the values in the columns defined at \verb|k=| and \verb|K=| parameter must be sorted in ascending alphabetical order. 

For example, given the parameters \verb|k=key K=Key r=val%n R=range i=dat.csv m=ref.csv|, if the sort criteria for the input data \verb|dat.csv| is carried out by \verb|msortf f=key,val%n|, the sort criteria for \verb|ref.csv| should follow accordingly as \verb|msortf f=Key,range%n|. 



\subsection*{Examples}
\subsubsection*{例1: 基本例}

日付項目の値が\verb|20080203|で、「金額」項目の値が\verb|5|以上\verb|15|未満の入力データ行には\verb|avg=150|を、
\verb|40|以上\verb|50|未満の行には\verb|avg=200|を結合する。


\begin{Verbatim}[baselinestretch=0.7,frame=single]
$ more dat1.csv
date,price
20080123,10
20080123,20
20080203,10
20080203,35
200804l0,50
$ more ref1.csv
date,priceF,priceT,avg
20080203,5,15,150
20080203,40,50,200
$ mnrjoin k=date f=avg m=ref1.csv R=priceF,priceT r=price%n i=dat1.csv o=rsl1.csv
#END# kgnrjoin R=priceF,priceT f=avg i=dat1.csv k=date m=ref1.csv o=rsl1.csv r=price%n
$ more rsl1.csv
date%0,price,avg
20080203,10,150
\end{Verbatim}
\subsubsection*{例2: 未結合データ出力}

\verb|-n|を指定することで、参照ファイルにマッチしない入力ファイルの行(\verb|avg=|がNULL値の行)も出力し、
\verb|-N|を指定することで、入力ファイルにマッチしない参照ファイルの行(\verb|price=|がNULL値の行)も出力する。
いわゆる外部結合である。


\begin{Verbatim}[baselinestretch=0.7,frame=single]
$ mnrjoin k=date f=avg m=ref1.csv R=priceF,priceT r=price%n -n -N i=dat1.csv o=rsl2.csv
#END# kgnrjoin -N -n R=priceF,priceT f=avg i=dat1.csv k=date m=ref1.csv o=rsl2.csv r=price%n
$ more rsl2.csv
date%0,price,avg
20080123,10,
20080123,20,
20080203,10,150
20080203,35,
20080203,,200
200804l0,50,
\end{Verbatim}

\subsection*{Related Command}
\hyperref[sect:mrjoin] {mrjoin} : Use \verb|mrjoin| if there are repeated values in join key (\verb|K=| field) from the reference data. 

%\end{document}
