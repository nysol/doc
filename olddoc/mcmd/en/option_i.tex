%\documentclass[a4paper]{book}
%\usepackage{mcmd}
%\begin{document}

\subsection{i= Input file name\label{sect:option_i}}
Specify the name of input file.
Most commands only allow a single file to be specified, with the exception of \verb|mcat| command where multiple files can be specified separated with a comma.
Yet, certain commands such as \verb|mnewnumber| and \verb|mnewrand| do not require input data. 

When this parameter is not defined, data is read from standard input by using pipeline. In the example below, \verb|i=| parameter is not specified for \verb|msum| command because the input data is the result of \verb|msortf|, which is read from standard input through the pipeline.

\begin{Verbatim}[baselinestretch=0.7,frame=single]
$ msortf f=a i=dat.csv | msum k=a f=b o=rsl.csv
\end{Verbatim}

However, it is difficult to identify errors when results are piped directly from one command to the next. 
In the following example, \verb|i=| parameter is also specified for \verb|msum|. 
The results of \verb|msortf| is sent to standard output, and msum reads input data from \verb|dat.csv|. 
Since \verb|msortf| did not add meaning to the input for \verb|msum|, the results from this example is different from the above. 

\begin{Verbatim}[baselinestretch=0.7,frame=single]
$ msortf f=a i=dat.csv | msum k=a f=b i=dat.csv o=rsl.csv
\end{Verbatim}

\subsection*{Examples}
\subsubsection*{Example 1: Basic Example}

Run \verb|mcut| using \verb|dat1.csv| as input data.


\begin{Verbatim}[baselinestretch=0.7,frame=single]
$ more dat1.csv
customer,quantity,amount
A,1,10
A,2,20
$ mcut f=customer,amount i=dat1.csv o=rsl1.csv
#END# kgcut f=customer,amount i=dat1.csv o=rsl1.csv
$ more rsl1.csv
customer,amount
A,10
A,20
\end{Verbatim}
\subsubsection*{Example 2: Specify output field name}

Read standard input using redirection ("\verb|"<"|").


\begin{Verbatim}[baselinestretch=0.7,frame=single]
$ mcut f= customer, amount o=rsl2.csv <dat1.csv
#ERROR# invalid argument: customer, (kgcut)
$ more rsl2.csv
rsl2.csv: No such file or directory
\end{Verbatim}


\subsubsection*{Related commands}
The parameter can be used in all commands except for commands such as \verb|mnewnumber| and \verb|mnewrand|.

%\end{document}

