\subsubsection*{Example 1: Basic Example}

Generate a dataset with 5 sequential numbers starting from 1 incremented by 1. Name the sequence as \verb|No.|.


\begin{Verbatim}[baselinestretch=0.7,frame=single]
$ mnewnumber a=No. I=1 S=1 l=5 o=rsl1.csv
#END# kgNewnumber I=1 S=1 a=No. l=5 o=rsl1.csv
$ more rsl1.csv
No.
1
2
3
4
5
\end{Verbatim}
\subsubsection*{Example 2: Change the starting number and interval }

Generate a dataset consisting of 5 sequential numbers starting from 10 with an incremental interval of 5. Name the sequence as \verb|No.|.


\begin{Verbatim}[baselinestretch=0.7,frame=single]
$ mnewnumber a=No. I=5 S=10 l=5 o=rsl2.csv
#END# kgNewnumber I=5 S=10 a=No. l=5 o=rsl2.csv
$ more rsl2.csv
No.
10
15
20
25
30
\end{Verbatim}
\subsubsection*{Example 3: Generate series of alphabet}

Generate a dataset consisting of 5 alphabet sequence starting from A with 1 alphabet in between. Name the sequence as \verb|No.|.


\begin{Verbatim}[baselinestretch=0.7,frame=single]
$ mnewnumber a=No. I=1 S=A l=5 o=rsl3.csv
#END# kgNewnumber I=1 S=A a=No. l=5 o=rsl3.csv
$ more rsl3.csv
No.
A
B
C
D
E
\end{Verbatim}
\subsubsection*{Example 4: Generate data without header}

Generate a dataset consisting of 11 alphabet sequence starting from B with 3 alphabets in between. Exclude the header from the output.


\begin{Verbatim}[baselinestretch=0.7,frame=single]
$ mnewnumber  -nfn  I=3 l=11 S=B o=rsl4.csv
#END# kgNewnumber -nfn I=3 S=B l=11 o=rsl4.csv
$ more rsl4.csv
B
E
H
K
N
Q
T
W
Z
AC
AF
\end{Verbatim}
