\subsubsection*{Example 1: Example 1: Basic example}

The example below is illustrated in the summary above. Output the 5 CSV fields with /a/b set as the key elements.


\begin{Verbatim}[baselinestretch=0.7,frame=single]
$ more dat1.xml
<a att="aa">
  <b att="bb1">
    <c p="pp1" q="qq1"/>
    <d>text1</d>
  </b>
  <b att="bb2">
    <c q="qq2"></c>
    <d>text2</d>
  </b>
  <b>
    <c p="pp3"/>
    <d>text3</d>
  </b>
</a>
$ mxml2csv k=/a/b f=@att:b_att,c@p:c_p,c@p%f:c_p_f,d:d,/a:a i=dat1.xml  o=rsl1.csv
#END# kgxml2csv f=@att:b_att,c@p:c_p,c@p%f:c_p_f,d:d,/a:a i=dat1.xml k=/a/b o=rsl1.csv
$ more rsl1.csv
b_att,c_p,c_p_f,d,a
bb1,pp1,1,text1,text1
bb2,,,text2,text1text2
,pp3,1,text3,text1text2text3
\end{Verbatim}
\subsubsection*{Example 2: Example 2: Absolute path}

Specification of same element as in the basic example with an absolute path. Output the 5 CSV fields with /a/b as the key elements.


\begin{Verbatim}[baselinestretch=0.7,frame=single]
$ mxml2csv k=/a/b f=/a/b@att:b_att,/a/b/c@p:c_p,/a/b/c@p%f:c_p_f,/a/b/d:d,/a:a i=dat1.xml  o=rsl2.csv
#END# kgxml2csv f=/a/b@att:b_att,/a/b/c@p:c_p,/a/b/c@p%f:c_p_f,/a/b/d:d,/a:a i=dat1.xml k=/a/b o=rsl2.csv
$ more rsl2.csv
b_att,c_p,c_p_f,d,a
bb1,pp1,1,text1,text1
bb2,,,text2,text1text2
,pp3,1,text3,text1text2text3
\end{Verbatim}
\subsubsection*{Example 3: Example 3: Changing key elements}

Example of changing a key element to a using an absolute path. Since there is only one end tag a, one row of record will be returned as output. /a/b@att specified at f= appeared twice, the last value is returned as output.


\begin{Verbatim}[baselinestretch=0.7,frame=single]
$ mxml2csv k=/a f=/a/b@att:b_att,/a/b/c@p:c_p,/a/b/c@p%f:c_p_f,/a/b/d:d,/a:a i=dat1.xml o=rsl3.csv
#END# kgxml2csv f=/a/b@att:b_att,/a/b/c@p:c_p,/a/b/c@p%f:c_p_f,/a/b/d:d,/a:a i=dat1.xml k=/a o=rsl3.csv
$ more rsl3.csv
b_att,c_p,c_p_f,d,a
bb2,pp3,1,text3,text1text2text3
\end{Verbatim}
