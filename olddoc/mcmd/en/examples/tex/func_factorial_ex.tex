\subsubsection*{Example 1: Basic Example}



\begin{Verbatim}[baselinestretch=0.7,frame=single]
$ more dat1.csv
id,val
1,1
2,5
3,
4,10000
$ mcal c='factorial(${val})' a=rsl i=dat1.csv o=rsl1.csv
#END# kgcal a=rsl c=factorial(${val}) i=dat1.csv o=rsl1.csv
$ more rsl1.csv
id,val,rsl
1,1,1
2,5,120
3,,
4,10000,
\end{Verbatim}
\subsubsection*{Example 2: Example of using constants}

Calculate the factorial of 5. When constants is used as an argument, all rows will return the same result.


\begin{Verbatim}[baselinestretch=0.7,frame=single]
$ mcal c='factorial(5)' a=rsl i=dat1.csv o=rsl2.csv
#END# kgcal a=rsl c=factorial(5) i=dat1.csv o=rsl2.csv
$ more rsl2.csv
id,val,rsl
1,1,120
2,5,120
3,,120
4,10000,120
\end{Verbatim}
