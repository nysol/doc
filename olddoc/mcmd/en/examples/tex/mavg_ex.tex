\subsubsection*{Example 1: Basic Example}

Calculate the average values of "Quantity" and "Amount" fields for each "Customer", save the computed output in new columns named "AverageVolume" and "AverageAmount".


\begin{Verbatim}[baselinestretch=0.7,frame=single]
$ more dat1.csv
Customer,Quantity,Amount
A,1,5
A,2,20
B,1,15
B,,10
B,5,20
$ mavg k=Customer f=Quantity:AvgQuantity,Amount:AvgAmount i=dat1.csv o=rsl1.csv
#END# kgavg f=Quantity:AvgQuantity,Amount:AvgAmount i=dat1.csv k=Customer o=rsl1.csv
$ more rsl1.csv
Customer%0,AvgQuantity,AvgAmount
A,1.5,12.5
B,3,15
\end{Verbatim}
\subsubsection*{Example 2: Output consisting of NULL values}

Calculate the average values of "Quantity" and "Amount" fields for each "Customer", save output in a new columns named "AverageVolume" and "AverageAmount".
When specifying the \verb|-n| option, if a NULL value is included in the input, the result will return NULL value.



\begin{Verbatim}[baselinestretch=0.7,frame=single]
$ mavg k=Customer f=Quantity:AvgQuantity,Amount:AvgAmount -n i=dat1.csv o=rsl2.csv
#END# kgavg -n f=Quantity:AvgQuantity,Amount:AvgAmount i=dat1.csv k=Customer o=rsl2.csv
$ more rsl2.csv
Customer%0,AvgQuantity,AvgAmount
A,1.5,12.5
B,,15
\end{Verbatim}
\subsubsection*{Example 3: Calculate sum without key field}

Calculate the average values of "Quantity" and "Amount" fields, and save the outputs in columns "AvgQuantity" and "AvgAmount".


\begin{Verbatim}[baselinestretch=0.7,frame=single]
$ mavg f=Quantity:AvgQuantity,Amount:AvgAmount i=dat1.csv o=rsl3.csv
#END# kgavg f=Quantity:AvgQuantity,Amount:AvgAmount i=dat1.csv o=rsl3.csv
$ more rsl3.csv
Customer,AvgQuantity,AvgAmount
B,2.25,14
\end{Verbatim}
