\begin{figure}[htbp]
\begin{Verbatim}[baselinestretch=0.7,frame=single]
$ mcut --help

MCUT - SELECT COLUMN

Extract the specified column(s). The specified column is removed with -r
option.

Format

$ mcut f= [-r] [-nfni] [i=] [o=] [-assert_diffSize] [-assert_nullin]
[-nfn] [-nfno] [-x] [-q] [tmpPath=] [--help] [--helpl] [--version]

Parameters

  f=      Define name of column to be extracted
          New column name for can be specified by defining the field name, followed by colon and the new field name.
          ex. f=a:A,b:B
  -r      Field removal switch
          Remove all columns defined in the f= parameter.
  -nfni   When header is not present in first row of the input data, position number of column is used to identify corresponding field(s).
          New column name(s) for each column can be specified in the output file as follows.
          Example f=0:date,2:store,3:quantity

Examples

Example 1: Basic Example

Extract customer and amount information from the data file dat1.csv
Rename the column “amount ” to “sales” in the output.

    $ more dat1.csv
    customer,quantity,amount
    A,1,10
    A,2,20
    B,1,15
    B,3,10
    B,1,20
    $ mcut f=customer,amount:sales i=dat1.csv o=rsl1.csv
    #END# kgcut f=customer,amount:sales i=dat1.csv o=rsl1.csv
    $ more rsl1.csv
    customer,sales
    A,10
    A,20
    B,15
    B,10
    B,20

Example 2: Remove columns

Remove columns customer and amount specified at -r.

    $ mcut f=customer,amount -r i=dat1.csv o=rsl2.csv
    #END# kgcut -r f=customer,amount i=dat1.csv o=rsl2.csv
    $ more rsl2.csv
    quantity
    1
    2
    1
    3
    1

Example 3: Data without field names

Select columns 0, 2 from an input file without field header, add
customer and amount as field names in the output file.

    $ mcut f=0:customer,2:amount -nfni i=dat1.csv o=rsl3.csv
    #END# kgcut -nfni f=0:customer,2:amount i=dat1.csv o=rsl3.csv
    $ more rsl3.csv
    customer,amount
    customer,amount
    A,10
    A,20
    B,15
    B,10
    B,20

mcut --version
lib Version 2:1:0:0: mod Version 0
\end{Verbatim}
\caption{Display help information\label{fig:abstract3_1}}
\end{figure}
