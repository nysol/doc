\subsubsection*{Example 1: Basic Example}

Find out the position of the longest substring that matches the regular expression \verb|c.*a|. Note that the first character starts at position 0. Since the same input data and substring are used in the \verb|regexstr| function, it is easy to compare the results and understand the differences.


\begin{Verbatim}[baselinestretch=0.7,frame=single]
$ more dat1.csv
id,str
1,xcbbbayy
2,xxcbaay
3,
4,bacabbca
$ mcal c='regexpos($s{str},"c.*a")' a=rsl i=dat1.csv o=rsl1.csv
#END# kgcal a=rsl c=regexpos($s{str},"c.*a") i=dat1.csv o=rsl1.csv
$ more rsl1.csv
id,str,rsl
1,xcbbbayy,1
2,xxcbaay,2
3,,
4,bacabbca,2
\end{Verbatim}
\subsubsection*{Example 2: Multibyte character}

Find out the longest substring that matches the regular expression \verb|"い.*あ"|. However, since regexpos does not support multibyte characters, the function returns the position of bytes instead of characters.


\begin{Verbatim}[baselinestretch=0.7,frame=single]
$ more dat2.csv
id,str
1,漢漢あbbbいyy
2,漢あbいいy
3,
4,bあいあbbいあ
$ mcal c='regexpos($s{str},"あ.*い")' a=rsl i=dat2.csv o=rsl2.csv
#END# kgcal a=rsl c=regexpos($s{str},"あ.*い") i=dat2.csv o=rsl2.csv
$ more rsl2.csv
id,str,rsl
1,漢漢あbbbいyy,6
2,漢あbいいy,3
3,,
4,bあいあbbいあ,1
\end{Verbatim}
\subsubsection*{Example 3: Multibyte character 2}

Find out the longest substring that matches the regular expression \verb|"い.*あ"|. The regexposw function is able to process multibyte characters accurately.


\begin{Verbatim}[baselinestretch=0.7,frame=single]
$ mcal c='regexposw($s{str},"あ.*い")' a=rsl i=dat2.csv o=rsl3.csv
#END# kgcal a=rsl c=regexposw($s{str},"あ.*い") i=dat2.csv o=rsl3.csv
$ more rsl3.csv
id,str,rsl
1,漢漢あbbbいyy,2
2,漢あbいいy,1
3,,
4,bあいあbbいあ,1
\end{Verbatim}
