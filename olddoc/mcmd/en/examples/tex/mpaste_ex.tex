\subsubsection*{Example 1: Basic Example}



\begin{Verbatim}[baselinestretch=0.7,frame=single]
$ more dat1.csv
id1
1
2
3
4
$ more ref1.csv
id2
a
b
c
d
$ mpaste m=ref1.csv i=dat1.csv o=rsl1.csv
#END# kgpaste i=dat1.csv m=ref1.csv o=rsl1.csv
$ more rsl1.csv
id1,id2
1,a
2,b
3,c
4,d
\end{Verbatim}
\subsubsection*{Example 2: Example of merging data of different sizes}

If the number of rows in the input file is different from the reference file , merge records according to the smaller file.


\begin{Verbatim}[baselinestretch=0.7,frame=single]
$ more ref2.csv
id2
a
b
$ mpaste m=ref2.csv i=dat1.csv o=rsl2.csv
#END# kgpaste i=dat1.csv m=ref2.csv o=rsl2.csv
$ more rsl2.csv
id1,id2
1,a
2,b
\end{Verbatim}
\subsubsection*{Example 3: Outer join}

If there are less number of rows in the reference file, NULL values will be assigned to records that did not match with the input file when \verb|-n| option is specified.


\begin{Verbatim}[baselinestretch=0.7,frame=single]
$ mpaste m=ref2.csv -n i=dat1.csv o=rsl3.csv
#END# kgpaste -n i=dat1.csv m=ref2.csv o=rsl3.csv
$ more rsl3.csv
id1,id2
1,a
2,b
3,
4,
\end{Verbatim}
\subsubsection*{Example 4: Define fields to join}



\begin{Verbatim}[baselinestretch=0.7,frame=single]
$ more ref3.csv
id2,val
a,R0
b,R1
c,R2
d,R3
$ mpaste f=val m=ref3.csv i=dat1.csv o=rsl4.csv
#END# kgpaste f=val i=dat1.csv m=ref3.csv o=rsl4.csv
$ more rsl4.csv
id1,val
1,R0
2,R1
3,R2
4,R3
\end{Verbatim}
