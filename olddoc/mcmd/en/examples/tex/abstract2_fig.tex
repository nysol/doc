\begin{figure}[htbp]
\begin{Verbatim}[baselinestretch=0.7,frame=single]
$ mdata man1 >man1.csv
#END# kgData -man1
$ more man1.csv
顧客,日付,商品
A,20130916,a
A,20130916,c
A,20130917,a
A,20130917,e
B,20130916,d
B,20130917,a
B,20130917,d
B,20130917,f
$ mcut f=顧客:customer,日付:date,商品:product i=man1.csv |
$ mcut f=customer,product o=xxa
#END# kgcut f=顧客:customer,日付:date,商品:product i=man1.csv
#END# kgcut f=customer,product o=xxa
$ more xxa
customer,product
A,a
A,c
A,a
A,e
B,d
B,a
B,d
B,f
$ msortf f=customer,product i=xxa o=xxb
#END# kgsortf f=customer,product i=xxa o=xxb
$ more xxb
customer%0,product%1
A,a
A,a
A,c
A,e
B,a
B,d
B,d
B,f
# Count the number of rows by customer and product.
$ mcount k=customer,product a="number of items" i=xxb o=xxc
#END# kgcount a=number of items i=xxb k=customer,product o=xxc
$ more xxc
customer%0,product%1,number of items
A,a,2
A,c,1
A,e,1
B,a,1
B,d,2
B,f,1
# Perform a cross tabulation by the item of product. The number of the product that is not purchased gives 0.
$ mcross k=customer s=product f="number of items" v=0 i=xxc o=xxd
#END# kgcross f=number of items i=xxc k=customer o=xxd s=product v=0
$ more xxd
customer%0,fld,a,c,d,e,f
A,number of items,2,1,0,1,0
B,number of items,1,0,2,0,1
# remove extra item "fld".
$ mcut f=fld -r i=xxd o=output.csv
#END# kgcut -r f=fld i=xxd o=output.csv
$ more output.csv
customer%0,a,c,d,e,f
A,2,1,0,1,0
B,1,0,2,0,1
\end{Verbatim}
\caption{Customer-based product-quantity matrix\label{fig:abstract2_1}}
\end{figure}
