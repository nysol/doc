\subsubsection*{Example 1: Basic Example}



\begin{Verbatim}[baselinestretch=0.7,frame=single]
$ more dat1.csv
id,date
1,20000101
2,20121021
3,
4,19770812
$ mcal c='leapyear($d{date})' a=rsl i=dat1.csv o=rsl1.csv
#END# kgcal a=rsl c=leapyear($d{date}) i=dat1.csv o=rsl1.csv
$ more rsl1.csv
id,date,rsl
1,20000101,1
2,20121021,1
3,,
4,19770812,0
\end{Verbatim}
\subsubsection*{Example 2: Determine leap year from time formatted data}



\begin{Verbatim}[baselinestretch=0.7,frame=single]
$ more dat2.csv
id,time
1,20000101000000
2,20121021111213
3,
4,19770812122212
$ mcal c='leapyear($t{time})' a=rsl i=dat2.csv o=rsl2.csv
#END# kgcal a=rsl c=leapyear($t{time}) i=dat2.csv o=rsl2.csv
$ more rsl2.csv
id,time,rsl
1,20000101000000,1
2,20121021111213,1
3,,
4,19770812122212,0
\end{Verbatim}
