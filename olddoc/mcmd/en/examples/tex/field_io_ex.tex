\subsubsection*{Example 1: Basic Example}

By specifying "quantity:unit sales", the field name is converted from “quantity” to “unit sales” in the output.


\begin{Verbatim}[baselinestretch=0.7,frame=single]
$ more dat1.csv
brand,quantity
A,10
B,20
C,30
D,40
$ mcut f=brand,quantity:salesquantity i=dat1.csv o=rsl1.csv
#END# kgcut f=brand,quantity:salesquantity i=dat1.csv o=rsl1.csv
$ more rsl1.csv
brand,salesquantity
A,10
B,20
C,30
D,40
\end{Verbatim}
\subsubsection*{Example 2: Add field name}

The maccum command accumulates the values in the "quantity" field, and add the field name "cumulative quantity" in the output results. If the parameter is specified as "f=quantity", the field name of the cumulative result will remain as "quantity", thus results in error because the same field name “quantity” exists in the output. 


\begin{Verbatim}[baselinestretch=0.7,frame=single]
$ maccum f=quantity:accumulationquantity i=dat1.csv o=rsl2.csv
#ERROR# parameter s= is mandatory without -q,-nfn (kgaccum)
$ more rsl2.csv
$ maccum f=quantity i=dat1.csv o=rsl2.csv
#ERROR# parameter s= is mandatory without -q,-nfn (kgaccum)
\end{Verbatim}
\subsubsection*{Example 3: Mixing field name and field number}

The field name and field number can be specified at the same time.


\begin{Verbatim}[baselinestretch=0.7,frame=single]
$ mcut f=0,1:salesquantity -x i=dat1.csv o=rsl3.csv
#END# kgcut -x f=0,1:salesquantity i=dat1.csv o=rsl3.csv
$ more rsl3.csv
brand,salesquantity
A,10
B,20
C,30
D,40
\end{Verbatim}
