\subsubsection*{Example 1: Basic Example}

Calculate the average \verb|Quantity| and average \verb|Amount| for each \verb|Customer|.


\begin{Verbatim}[baselinestretch=0.7,frame=single]
$ more dat1.csv
Customer,Quantity,Amount
A,1,
B,,15
A,2,20
B,3,10
B,1,20
$ mhashavg k=Customer f=Quantity,Amount i=dat1.csv o=rsl1.csv
#END# kghashavg f=Quantity,Amount i=dat1.csv k=Customer o=rsl1.csv
$ more rsl1.csv
Customer,Quantity,Amount
A,1.5,20
B,2,15
\end{Verbatim}
\subsubsection*{Example 2: NULL value in output}

The output returns NULL if there NULL value is present in \verb|Quantity| and \verb|Amount|. Use \verb|-n| option to print the null value.


\begin{Verbatim}[baselinestretch=0.7,frame=single]
$ mhashavg k=Customer f=Quantity,Amount -n i=dat1.csv o=rsl2.csv
#END# kghashavg -n f=Quantity,Amount i=dat1.csv k=Customer o=rsl2.csv
$ more rsl2.csv
Customer,Quantity,Amount
A,1.5,
B,,15
\end{Verbatim}
