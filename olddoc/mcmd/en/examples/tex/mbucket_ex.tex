\subsubsection*{Example 1: Basic Example}

Partition x and y into two subsets of equal extent and save the output file as rng.csv


\begin{Verbatim}[baselinestretch=0.7,frame=single]
$ more dat1.csv
id,x,y
A,2,7
B,6,7
C,5,6
D,7,5
E,6,4
F,1,3
G,3,3
H,4,2
I,7,2
J,2,1
$ mbucket f=x:xb,y:yb n=2 O=rng1.csv i=dat1.csv o=rsl1.csv
#END# kgbucket O=rng1.csv f=x:xb,y:yb i=dat1.csv n=2 o=rsl1.csv
$ more rsl1.csv
id,x,y,xb,yb
A,2,7,1,2
B,6,7,2,2
C,5,6,2,2
D,7,5,2,2
E,6,4,2,2
F,1,3,1,1
G,3,3,1,1
H,4,2,1,1
I,7,2,2,1
J,2,1,1,1
$ more rng1.csv
fieldName,bucketNo,rangeFrom,rangeTo
x,1,,4.5
x,2,4.5,
y,1,,3.5
y,2,3.5,
\end{Verbatim}
\subsubsection*{Example 2: Partition by equal range}

Use \verb|-rng| option to partition the data by uniform value ranges.


\begin{Verbatim}[baselinestretch=0.7,frame=single]
$ mbucket f=x:xb,y:yb n=2 -rng O=rng2.csv i=dat1.csv o=rsl2.csv
#END# kgbucket -rng O=rng2.csv f=x:xb,y:yb i=dat1.csv n=2 o=rsl2.csv
$ more rsl2.csv
id,x,y,xb,yb
A,2,7,1,2
B,6,7,2,2
C,5,6,2,2
D,7,5,2,2
E,6,4,2,2
F,1,3,1,1
G,3,3,1,1
H,4,2,2,1
I,7,2,2,1
J,2,1,1,1
$ more rng2.csv
fieldName,bucketNo,rangeFrom,rangeTo
x,1,,4
x,2,4,
y,1,,4
y,2,4,
\end{Verbatim}
\subsubsection*{Example 3: Example using key field}

Partition x and y into two subsets of equal extent using "id" as the key parameter.
By specifying \verb|n=2,3|, field \verb|x| is divided into 2 buckets, and field
\verb|y| is divided into 3 buckets.
Include bucket numbers and value range of buckets in the output file (\verb|F=2|).


\begin{Verbatim}[baselinestretch=0.7,frame=single]
$ more dat2.csv
id,x,y
A,2,7
A,6,7
A,5,6
B,7,5
B,6,4
B,1,3
C,3,3
C,4,2
C,7,2
C,2,1
$ mbucket k=id f=x:xb,y:yb n=2,3 F=2 i=dat2.csv o=rsl3.csv
#END# kgbucket F=2 f=x:xb,y:yb i=dat2.csv k=id n=2,3 o=rsl3.csv
$ more rsl3.csv
id%0,x,y,xb,yb
A,2,7,1:_3.5,2:6.5_
A,6,7,2:3.5_,2:6.5_
A,5,6,2:3.5_,1:_6.5
B,7,5,2:3.5_,3:4.5_
B,6,4,2:3.5_,2:3.5_4.5
B,1,3,1:_3.5,1:_3.5
C,3,3,1:_3.5,3:2.5_
C,4,2,2:3.5_,2:1.5_2.5
C,7,2,2:3.5_,2:1.5_2.5
C,2,1,1:_3.5,1:_1.5
\end{Verbatim}
