\subsubsection*{Example 1: Basic Example}

Extract customer and amount information from the data file \verb|dat1.csv|
Rename the column "amount " to "sales" in the output.


\begin{Verbatim}[baselinestretch=0.7,frame=single]
$ more dat1.csv
customer,quantity,amount
A,1,10
A,2,20
B,1,15
B,3,10
B,1,20
$ mcut f=customer,amount:sales i=dat1.csv o=rsl1.csv
#END# kgcut f=customer,amount:sales i=dat1.csv o=rsl1.csv
$ more rsl1.csv
customer,sales
A,10
A,20
B,15
B,10
B,20
\end{Verbatim}
\subsubsection*{Example 2: Remove columns}

Remove columns customer and amount specified at \verb|-r|.


\begin{Verbatim}[baselinestretch=0.7,frame=single]
$ mcut f=customer,amount -r i=dat1.csv o=rsl2.csv
#END# kgcut -r f=customer,amount i=dat1.csv o=rsl2.csv
$ more rsl2.csv
quantity
1
2
1
3
1
\end{Verbatim}
\subsubsection*{Example 3: Data without field names}

Select columns 0, 2 from an input file without field header, add \verb|customer| and \verb|amount| as field names in the output file.


\begin{Verbatim}[baselinestretch=0.7,frame=single]
$ mcut f=0:customer,2:amount -nfni i=dat1.csv o=rsl3.csv
#END# kgcut -nfni f=0:customer,2:amount i=dat1.csv o=rsl3.csv
$ more rsl3.csv
customer,amount
customer,amount
A,10
A,20
B,15
B,10
B,20
\end{Verbatim}
