\subsubsection*{Example 1: Example 1: Basic Example}

The \verb|item| field is used as the key, and the \verb|date| field is horizontally transposed and the \verb|quantity| field is output.


\begin{Verbatim}[baselinestretch=0.7,frame=single]
$ more dat1.csv
item,date,quantity
A,20081201,1
A,20081202,2
A,20081203,3
B,20081201,4
B,20081203,5
$ m2cross k=item f=quantity s=date i=dat1.csv o=rsl1.csv
#END# kg2cross f=quantity i=dat1.csv k=item o=rsl1.csv s=date
$ more rsl1.csv
item%0,20081201,20081202,20081203
A,1,2,3
B,4,,5
\end{Verbatim}
\subsubsection*{Example 2: Example 2: Restoring the original input data}

The \verb|m2cross| command is used to restore the original input data for the output results of Example 1.


\begin{Verbatim}[baselinestretch=0.7,frame=single]
$ more rsl1.csv
item%0,20081201,20081202,20081203
A,1,2,3
B,4,,5
$ m2cross f=2008* a=date,quantity i=rsl1.csv o=rsl2.csv
#END# kg2cross a=date,quantity f=2008* i=rsl1.csv o=rsl2.csv
$ more rsl2.csv
item%0,date,quantity
A,20081201,1
A,20081202,2
A,20081203,3
B,20081201,4
B,20081202,
B,20081203,5
\end{Verbatim}
\subsubsection*{Example 3: Example 3: Reversing data sequence}

The sequence of the horizontally transposed fieldnames is reversed.


\begin{Verbatim}[baselinestretch=0.7,frame=single]
$ m2cross k=item f=quantity s=date%r i=dat1.csv o=rsl4.csv
#END# kg2cross f=quantity i=dat1.csv k=item o=rsl4.csv s=date%r
$ more rsl4.csv
item%0,20081203,20081202,20081201
A,3,2,1
B,5,,4
\end{Verbatim}
