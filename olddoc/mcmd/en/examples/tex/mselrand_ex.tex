\subsubsection*{Example 1: Basic Example}

Randomly select 1 transaction for each customer.


\begin{Verbatim}[baselinestretch=0.7,frame=single]
$ more dat1.csv
Customer,Date,Amount
A,20081201,10
A,20081207,20
A,20081213,30
B,20081002,40
B,20081209,50
$ mselrand k=Customer c=1 S=1 i=dat1.csv o=rsl1.csv
#END# kgselrand S=1 c=1 i=dat1.csv k=Customer o=rsl1.csv
$ more rsl1.csv
Customer%0,Date,Amount
A,20081201,10
B,20081002,40
\end{Verbatim}
\subsubsection*{Example 2: Randomly select a percentage of records}

Select 50\% of each customers' records at random. Save other records to a separate file \verb|oth.csv|.


\begin{Verbatim}[baselinestretch=0.7,frame=single]
$ mselrand k=Customer p=50 S=1 u=oth2.csv i=dat1.csv o=rsl2.csv
#END# kgselrand S=1 i=dat1.csv k=Customer o=rsl2.csv p=50 u=oth2.csv
$ more rsl2.csv
Customer%0,Date,Amount
A,20081201,10
B,20081002,40
$ more oth2.csv
Customer%0,Date,Amount
A,20081207,20
A,20081213,30
B,20081209,50
\end{Verbatim}
\subsubsection*{Example 3: Select records by same key}

In the following example, select two out of the four customers \verb|A,B,C,D| at random.
Customer \verb|C,D| is selected, and all records of customer \verb|C,D| is printed to the output.



\begin{Verbatim}[baselinestretch=0.7,frame=single]
$ more dat2.csv
Customer,Date,Amount
A,20081201,10
A,20081207,20
A,20081213,30
B,20081002,40
B,20081209,50
C,20081210,60
D,20081201,70
D,20081205,80
D,20081209,90
$ mselrand k=Customer c=2 S=1 -B i=dat2.csv o=rsl3.csv
#END# kgselrand -B S=1 c=2 i=dat2.csv k=Customer o=rsl3.csv
$ more rsl3.csv
Customer%0,Date,Amount
C,20081210,60
D,20081201,70
D,20081205,80
D,20081209,90
\end{Verbatim}
