\subsubsection*{Example 1: Basic Example}



\begin{Verbatim}[baselinestretch=0.7,frame=single]
$ more dat1.csv
date,val
20130406,1
20130407,2
20130408,3
20130409,4
$ mwindow wk=date:win t=2 i=dat1.csv o=rsl1.csv
#END# kgwindow i=dat1.csv o=rsl1.csv t=2 wk=date:win
$ more rsl1.csv
win%0,date,val
20130407,20130406,1
20130407,20130407,2
20130408,20130407,2
20130408,20130408,3
20130409,20130408,3
20130409,20130409,4
\end{Verbatim}
\subsubsection*{Example 2: Use first row as baseline data}



\begin{Verbatim}[baselinestretch=0.7,frame=single]
$ mwindow wk=date:win t=3 -r i=dat1.csv o=rsl2.csv
#END# kgwindow -r i=dat1.csv o=rsl2.csv t=3 wk=date:win
$ more rsl2.csv
win%0,date,val
20130406,20130406,1
20130406,20130407,2
20130406,20130408,3
20130407,20130407,2
20130407,20130408,3
20130407,20130409,4
\end{Verbatim}
\subsubsection*{Example 3: Print all window intervals even if the window size is less than the defined parameter}



\begin{Verbatim}[baselinestretch=0.7,frame=single]
$ mwindow wk=date:win t=3 -r -n i=dat1.csv o=rsl3.csv
#END# kgwindow -n -r i=dat1.csv o=rsl3.csv t=3 wk=date:win
$ more rsl3.csv
win%0,date,val
20130406,20130406,1
20130406,20130407,2
20130406,20130408,3
20130407,20130407,2
20130407,20130408,3
20130407,20130409,4
20130408,20130408,3
20130408,20130409,4
20130409,20130409,4
\end{Verbatim}
\subsubsection*{Example 4: Example of specifying key field}



\begin{Verbatim}[baselinestretch=0.7,frame=single]
$ more dat2.csv
store,date,val
a,20130406,1
a,20130407,2
a,20130408,3
a,20130409,4
b,20130406,11
b,20130407,12
b,20130408,13
b,20130409,14
$ mwindow k=store wk=date:win t=2 i=dat2.csv o=rsl4.csv
#END# kgwindow i=dat2.csv k=store o=rsl4.csv t=2 wk=date:win
$ more rsl4.csv
win%1,store%0,date,val
20130407,a,20130406,1
20130407,a,20130407,2
20130408,a,20130407,2
20130408,a,20130408,3
20130409,a,20130408,3
20130409,a,20130409,4
20130407,b,20130406,11
20130407,b,20130407,12
20130408,b,20130407,12
20130408,b,20130408,13
20130409,b,20130408,13
20130409,b,20130409,14
\end{Verbatim}
\subsubsection*{Example 5: Find out the moving averages between current day and previous day}

In the above example, moving average is calculated based on the last day of the window.  \verb|mslide| can be used for instances to calculate the moving averages of current day and previous day. The example is as follows:


\begin{Verbatim}[baselinestretch=0.7,frame=single]
$ mslide f=date:date2 -q i=dat1.csv o=rsl5.csv
#END# kgslide -q f=date:date2 i=dat1.csv o=rsl5.csv
$ more rsl5.csv
date,val,date2
20130406,1,20130407
20130407,2,20130408
20130408,3,20130409
\end{Verbatim}
\subsubsection*{Example 6: Find out the moving averages from the previous day}



\begin{Verbatim}[baselinestretch=0.7,frame=single]
$ mwindow wk=date2:win t=2 i=rsl5.csv o=rsl6.csv
#END# kgwindow i=rsl5.csv o=rsl6.csv t=2 wk=date2:win
$ more rsl6.csv
win%0,date,val,date2
20130408,20130406,1,20130407
20130408,20130407,2,20130408
20130409,20130407,2,20130408
20130409,20130408,3,20130409
\end{Verbatim}
