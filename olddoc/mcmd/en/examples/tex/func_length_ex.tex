\subsubsection*{Example 1: Basic Example}



\begin{Verbatim}[baselinestretch=0.7,frame=single]
$ more dat1.csv
id,str
1,abc
2,3.1415
3,
4,hello world!
$ mcal c='length($s{str})' a=rsl i=dat1.csv o=rsl1.csv
#END# kgcal a=rsl c=length($s{str}) i=dat1.csv o=rsl1.csv
$ more rsl1.csv
id,str,rsl
1,abc,3
2,3.1415,6
3,,0
4,hello world!,12
\end{Verbatim}
\subsubsection*{Example 2: Include multibyte character}

The following example uses Japanese in UTF-8 encoding. Each UTF-8 Japanese character is encoded in 3 bytes, thus, the length function returns the number of bytes rather than the number of Japanese characters.


\begin{Verbatim}[baselinestretch=0.7,frame=single]
$ more dat2.csv
id,str
1,こんにちは
2,大阪
$ mcal c='length($s{str})' a=rsl i=dat2.csv o=rsl2.csv
#END# kgcal a=rsl c=length($s{str}) i=dat2.csv o=rsl2.csv
$ more rsl2.csv
id,str,rsl
1,こんにちは,15
2,大阪,6
\end{Verbatim}
\subsubsection*{Example 3: Initialize wide character}

The \verb|lengthw| function converts each wide characters internally into multibyte character for computation.


\begin{Verbatim}[baselinestretch=0.7,frame=single]
$ mcal c='lengthw($s{str})' a=rsl i=dat2.csv o=rsl3.csv
#END# kgcal a=rsl c=lengthw($s{str}) i=dat2.csv o=rsl3.csv
$ more rsl3.csv
id,str,rsl
1,こんにちは,5
2,大阪,2
\end{Verbatim}
