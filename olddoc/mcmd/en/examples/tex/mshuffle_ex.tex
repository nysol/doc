\subsubsection*{Example 1: Example 1: Basic example}

The input file is partitioned into two, so that the rows are output to the same file as long as the value of the specified field (customer) is the same.


\begin{Verbatim}[baselinestretch=0.7,frame=single]
$ more dat2.csv
customer,date,amount
A,20081201,10
A,20081207,20
A,20081213,30
B,20081002,40
B,20081209,50
C,20081003,60
C,20081219,20
$ mshuffle f=customer d=./dat/d n=2 i=dat2.csv
#END# kgshuffle d=./dat/d f=customer i=dat2.csv n=2
$ ls ./dat
d_0
d_1
$ more ./dat/d_0
customer,date,amount
B,20081002,40
B,20081209,50
$ more ./dat/d_1
customer,date,amount
A,20081201,10
A,20081207,20
A,20081213,30
C,20081003,60
C,20081219,20
\end{Verbatim}
\subsubsection*{Example 2: Example 2: Not specifying f=}

The f= parameter is not specified and the input file is partitioned into two. As row number hash values are used, the two files will have almost the same numbers of rows.


\begin{Verbatim}[baselinestretch=0.7,frame=single]
$ more dat2.csv
customer,date,amount
A,20081201,10
A,20081207,20
A,20081213,30
B,20081002,40
B,20081209,50
C,20081003,60
C,20081219,20
$ mshuffle d=./dat/d n=2 i=dat2.csv
#END# kgshuffle d=./dat/d i=dat2.csv n=2
$ ls ./dat
d_0
d_1
$ more ./dat/d_0
customer,date,amount
A,20081207,20
B,20081002,40
C,20081003,60
$ more ./dat/d_1
customer,date,amount
A,20081201,10
A,20081213,30
B,20081209,50
C,20081219,20
\end{Verbatim}
\subsubsection*{Example 3: Example 3: Specifying v=,f=}

With v=2,1 specified, the input file is partitioned into two. Two hash values are allocated to file 0 (d\_0) and one hash value is allocated to file 1 (d\_1).


\begin{Verbatim}[baselinestretch=0.7,frame=single]
$ more dat2.csv
customer,date,amount
A,20081201,10
A,20081207,20
A,20081213,30
B,20081002,40
B,20081209,50
C,20081003,60
C,20081219,20
$ mshuffle f=customer d=./dat/d v=2,1 i=dat2.csv
#END# kgshuffle d=./dat/d f=customer i=dat2.csv v=2,1
$ ls ./dat
d_0
d_1
$ more ./dat/d_0
customer,date,amount
B,20081002,40
B,20081209,50
C,20081003,60
C,20081219,20
$ more ./dat/d_1
customer,date,amount
A,20081201,10
A,20081207,20
A,20081213,30
\end{Verbatim}
\subsubsection*{Example 4: Example 4: Specifying v=}

The script of Example 3 is run without the f= parameter specified. As row number hash values are used, the ratio of the numbers of output rows will be almost the same as the ratio of the weights.


\begin{Verbatim}[baselinestretch=0.7,frame=single]
$ more dat2.csv
customer,date,amount
A,20081201,10
A,20081207,20
A,20081213,30
B,20081002,40
B,20081209,50
C,20081003,60
C,20081219,20
$ mshuffle d=./dat/d v=2,1 i=dat2.csv
#END# kgshuffle d=./dat/d i=dat2.csv v=2,1
$ ls ./dat
d_0
d_1
$ more ./dat/d_0
customer,date,amount
A,20081201,10
A,20081213,30
B,20081002,40
C,20081003,60
C,20081219,20
$ more ./dat/d_1
customer,date,amount
A,20081207,20
B,20081209,50
\end{Verbatim}
