\subsubsection*{Example 1: Specify -nfn}

When \verb|-nfn| (no field name) is specified, the first row in the data will not be considered as field names. Thus, each field is specified as a number (note that the number starts from 0). 


\begin{Verbatim}[baselinestretch=0.7,frame=single]
$ more dat2.csv
a,2
b,5
b,4
$ msum -nfn k=0 f=1 i=dat2.csv o=rsl1.csv
#END# kgsum -nfn f=1 i=dat2.csv k=0 o=rsl1.csv
$ more rsl1.csv
a,2
b,9
\end{Verbatim}
\subsubsection*{Example 2: Specify -nfno}

When \verb|-nfno| (no field name for output) is specified, the first row of input data is initialized as field names, but the field names is removed from the output data.


\begin{Verbatim}[baselinestretch=0.7,frame=single]
$ more dat1.csv
key,val
a,2
b,5
b,4
$ msum k=key f=val -nfno i=dat1.csv o=rsl2.csv
#END# kgsum -nfno f=val i=dat1.csv k=key o=rsl2.csv
$ more rsl2.csv
a,2
b,9
\end{Verbatim}
\subsubsection*{Example 3: Specify -nfni}

The option \verb|-nfni| (no field names or input) is only available in the mcut command. This option does the opposite of - nfno; the first row of input data is not treated as a fieldname row, but the fieldnames will be shown in the output data. Thus, you need to specify an output fieldname after a colon (:) following an input field number.


\begin{Verbatim}[baselinestretch=0.7,frame=single]
$ mcut f=0:key,1:val -nfni i=dat2.csv o=rsl3.csv
#END# kgcut -nfni f=0:key,1:val i=dat2.csv o=rsl3.csv
$ more rsl3.csv
key,val
a,2
b,5
b,4
\end{Verbatim}
\subsubsection*{Example 4: Specify -x}

For CSV data with a field names, use the \verb|-x| option to specify the field number.


\begin{Verbatim}[baselinestretch=0.7,frame=single]
$ msum -x k=0 f=1 i=dat1.csv o=rsl4.csv
#END# kgsum -x f=1 i=dat1.csv k=0 o=rsl4.csv
$ more rsl4.csv
key%0,val
a,2
b,9
\end{Verbatim}
