\subsubsection*{Example 1: Basic Example}

Replace NULL values in the ¥verb|birthday| field with the string \verb|“no value”|.


\begin{Verbatim}[baselinestretch=0.7,frame=single]
$ more dat1.csv
customer,birthday
A,19690103
B,
C,19500501
D,
E,
$ mnullto f=birthday v="no value" i=dat1.csv o=rsl1.csv
#END# kgnullto f=birthday i=dat1.csv o=rsl1.csv v=no value
$ more rsl1.csv
customer,birthday
A,19690103
B,no value
C,19500501
D,no value
E,no value
\end{Verbatim}
\subsubsection*{Example 2: Replace non-NULL values}

Replace Null values in the \verb|birthday| field with the string \verb|"no value"| and change non-null values to the string ¥verb|"value"|, and rename the output column as \verb|entry|.


\begin{Verbatim}[baselinestretch=0.7,frame=single]
$ mnullto f=birthday:entry v="no value" O="value" i=dat1.csv o=rsl2.csv
#END# kgnullto O=value f=birthday:entry i=dat1.csv o=rsl2.csv v=no value
$ more rsl2.csv
customer,entry
A,value
B,no value
C,value
D,no value
E,no value
\end{Verbatim}
\subsubsection*{Example 3: Add new column}

Replace Null values in the \verb|birthday| field with the string \verb|"no value"| and change non-null values to the string \verb|"value"|. Output the replacement strings in a new column named \verb|entry|.


\begin{Verbatim}[baselinestretch=0.7,frame=single]
$ mnullto f=birthday:entry v="no value" O="value" -A i=dat1.csv o=rsl3.csv
#END# kgnullto -A O=value f=birthday:entry i=dat1.csv o=rsl3.csv v=no value
$ more rsl3.csv
customer,birthday,entry
A,19690103,value
B,,no value
C,19500501,value
D,,no value
E,,no value
\end{Verbatim}
\subsubsection*{Example 4: Replace values in previous row}



\begin{Verbatim}[baselinestretch=0.7,frame=single]
$ more dat2.csv
id,date
A,19690103
B,
C,19500501
D,
E,
$ mnullto f=date -p i=dat2.csv o=rsl4.csv
#END# kgnullto -p f=date i=dat2.csv o=rsl4.csv
$ more rsl4.csv
id,date
A,19690103
B,19690103
C,19500501
D,19500501
E,19500501
\end{Verbatim}
