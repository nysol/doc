\subsubsection*{Example 1: Basic Example }

The \verb|item| field in the input file is compared with the \verb|item| field from the reference file, add \verb|cost| field for records with the same value. There are two records where \verb|item=A| in both input file and reference file, therefore, 2$\times$2=4 rows of \verb|item=A| is written to the output file.


\begin{Verbatim}[baselinestretch=0.7,frame=single]
$ more dat1.csv
item,date,price
A,20081201,100
A,20081213,98
B,20081002,400
B,20081209,450
C,20081201,100
$ more ref1.csv
item,cost
A,50
A,70
B,300
E,200
$ mnjoin k=item f=cost m=ref1.csv i=dat1.csv o=rsl1.csv
#END# kgnjoin f=cost i=dat1.csv k=item m=ref1.csv o=rsl1.csv
$ more rsl1.csv
item%0,date,price,cost
A,20081201,100,50
A,20081201,100,70
A,20081213,98,50
A,20081213,98,70
B,20081002,400,300
B,20081209,450,300
\end{Verbatim}
\subsubsection*{Example 2: Ouput unmatched data}

Use \verb|-n| to print records in the input data that do not match with those in the reference file (row where \verb|item="C"|), and use -N to print records in the reference file that do not match with those in the input file (row where \verb|item="E"|).


\begin{Verbatim}[baselinestretch=0.7,frame=single]
$ more ref2.csv
item,cost
A,50
B,300
E,200
$ mnjoin k=item f=cost m=ref2.csv -n -N i=dat1.csv o=rsl2.csv
#END# kgnjoin -N -n f=cost i=dat1.csv k=item m=ref2.csv o=rsl2.csv
$ more rsl2.csv
item%0,date,price,cost
A,20081201,100,50
A,20081213,98,50
B,20081002,400,300
B,20081209,450,300
C,20081201,100,
E,,,200
\end{Verbatim}
