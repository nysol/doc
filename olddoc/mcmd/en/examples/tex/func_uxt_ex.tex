\subsubsection*{Example 1: Basic Example}

Convert the \verb|date| formatted strings in the date column to UNIX time using \verb|d2uxt| function, and convert back to original date string using \verb|uxt2d| function.


\begin{Verbatim}[baselinestretch=0.7,frame=single]
$ more dat1.csv
id,date
1,20000101
2,20121021
3,
4,19700101
$ mcal c='uxt($d{date})' a=uxt i=dat1.csv o=rsl1.csv
#END# kgcal a=uxt c=uxt($d{date}) i=dat1.csv o=rsl1.csv
$ more rsl1.csv
id,date,uxt
1,20000101,946684800
2,20121021,1350777600
3,,
4,19700101,0
$ mcal c='uxt2d(${uxt})' a=date2 i=rsl1.csv o=rsl2.csv
#END# kgcal a=date2 c=uxt2d(${uxt}) i=rsl1.csv o=rsl2.csv
$ more rsl2.csv
id,date,uxt,date2
1,20000101,946684800,20000101
2,20121021,1350777600,20121021
3,,,
4,19700101,0,19700101
\end{Verbatim}
\subsubsection*{Example 2: Example of using time formatted data}



\begin{Verbatim}[baselinestretch=0.7,frame=single]
$ more dat2.csv
id,time
1,20000101000000
2,20121021111213
3,
4,19700101000100
$ mcal c='uxt($t{time})' a=uxt i=dat2.csv o=rsl3.csv
#END# kgcal a=uxt c=uxt($t{time}) i=dat2.csv o=rsl3.csv
$ more rsl3.csv
id,time,uxt
1,20000101000000,946684800
2,20121021111213,1350817933
3,,
4,19700101000100,60
$ mcal c='uxt2t(${uxt})' a=time2 i=rsl3.csv o=rsl4.csv
#END# kgcal a=time2 c=uxt2t(${uxt}) i=rsl3.csv o=rsl4.csv
$ more rsl4.csv
id,time,uxt,time2
1,20000101000000,946684800,20000101000000
2,20121021111213,1350817933,20121021111213
3,,,
4,19700101000100,60,19700101000100
\end{Verbatim}
