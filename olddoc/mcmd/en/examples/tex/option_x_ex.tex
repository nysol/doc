\subsubsection*{Example 1: Basic Example}

Compute the sum of all items in column 1 and 2 of the same key. 


\begin{Verbatim}[baselinestretch=0.7,frame=single]
$ more dat1.csv
customer,quantity,amount
A,1,10
A,2,20
B,1,15
B,3,10
B,1,20
$ msum -x k=0 f=1,2 i=dat1.csv o=rsl1.csv
#END# kgsum -x f=1,2 i=dat1.csv k=0 o=rsl1.csv
$ more rsl1.csv
customer%0,quantity,amount
A,3,30
B,5,45
\end{Verbatim}
\subsubsection*{Example 2: Output column names}

Rename column 1 and 2 as \verb|a,b| respectively. 


\begin{Verbatim}[baselinestretch=0.7,frame=single]
$ msum -x k=0 f=1:a,2:b i=dat1.csv o=rsl2.csv
#END# kgsum -x f=1:a,2:b i=dat1.csv k=0 o=rsl2.csv
$ more rsl2.csv
customer%0,a,b
A,3,30
B,5,45
\end{Verbatim}
\subsubsection*{Example 3: Error when using -nfn}

The \verb|-nfn| option assumes data starts from the first row when computing the sum of "quantity" and "amount". However, the result will not be computed as expected since the position of first row of data is defined differently when using \verb|-x| and \verb|-nfn|. 


\begin{Verbatim}[baselinestretch=0.7,frame=single]
$ msum -nfn k=0 f=1,2 i=dat1.csv o=rsl3.csv
#END# kgsum -nfn f=1,2 i=dat1.csv k=0 o=rsl3.csv
$ more rsl3.csv
customer,0,0
A,3,30
B,5,45
\end{Verbatim}
