\subsubsection*{Example 1: Basic Example}

Calculate the cosine and Pearson's product-moment correlation coefficient for the combination of two items among \verb|x, y, z| fields.


\begin{Verbatim}[baselinestretch=0.7,frame=single]
$ more dat1.csv
x,y,z
14,0.17,-14
11,0.2,-1
32,0.15,-2
13,0.33,-2
$ msim c=pearson,cosine f=x,y,z i=dat1.csv o=rsl1.csv
#END# kgsim c=pearson,cosine f=x,y,z i=dat1.csv o=rsl1.csv
$ more rsl1.csv
fld1,fld2,pearson,cosine
x,y,-0.5088704666,0.7860308044
x,z,0.1963041929,-0.5338153343
y,z,0.3311001423,-0.5524409416
\end{Verbatim}
\subsubsection*{Example 2: Output diagonal matrix, the upper triangular matrix}

Calculate the cosine and Pearson's product-moment correlation coefficient for the combination of two items between \verb|x, y, z|  fields (with d option).


\begin{Verbatim}[baselinestretch=0.7,frame=single]
$ msim c=pearson,cosine f=x,y,z -d i=dat1.csv o=rsl2.csv
#END# kgsim -d c=pearson,cosine f=x,y,z i=dat1.csv o=rsl2.csv
$ more rsl2.csv
fld1,fld2,pearson,cosine
x,x,1,1
x,y,-0.5088704666,0.7860308044
x,z,0.1963041929,-0.5338153343
y,x,-0.5088704666,0.7860308044
y,y,1,1
y,z,0.3311001423,-0.5524409416
z,x,0.1963041929,-0.5338153343
z,y,0.3311001423,-0.5524409416
z,z,1,1
\end{Verbatim}
\subsubsection*{Example 3: Calculation based on key unit}

Calculate using \verb|key| field as unit.


\begin{Verbatim}[baselinestretch=0.7,frame=single]
$ more dat2.csv
key,x,y,z
A,14,0.17,-14
A,11,0.2,-1
A,32,0.15,-2
B,13,0.33,-2
B,10,0.8,-5
B,15,0.45,-9
$ msim k=key c=pearson,cosine f=x,y,z i=dat2.csv o=rsl3.csv
#END# kgsim c=pearson,cosine f=x,y,z i=dat2.csv k=key o=rsl3.csv
$ more rsl3.csv
key%0,fld1,fld2,pearson,cosine
A,x,y,-0.8746392857,0.8472573627
A,x,z,0.3164384831,-0.521983618
A,y,z,0.1830936883,-0.6719258683
B,x,y,-0.7919009884,0.8782575583
B,x,z,-0.471446429,-0.9051543403
B,y,z,-0.1651896746,-0.8514129252
\end{Verbatim}
\subsubsection*{Example 4: Specify the type of degree of similarity}

Using the data with 01 values, compute the phi coefficient and Hamming distance.


\begin{Verbatim}[baselinestretch=0.7,frame=single]
$ more dat3.csv
x,y,z
1,1,0
1,0,1
1,0,1
0,1,1
$ msim c=hamming,phi f=x,y,z i=dat3.csv o=rsl4.csv
#END# kgsim c=hamming,phi f=x,y,z i=dat3.csv o=rsl4.csv
$ more rsl4.csv
fld1,fld2,hamming,phi
x,y,0.75,-0.5773502692
x,z,0.5,-0.3333333333
y,z,0.75,-0.5773502692
\end{Verbatim}
\subsubsection*{Example 5: Rename the column containing degree of similarity}

Using the data with 01 values, compute the phi coefficient and Hamming distance and change the output field name.


\begin{Verbatim}[baselinestretch=0.7,frame=single]
$ msim c=hamming:HammingDist,phi:PhiCoeff a=variable1,variable2 f=x,y,z i=dat3.csv o=rsl5.csv
#END# kgsim a=variable1,variable2 c=hamming:HammingDist,phi:PhiCoeff f=x,y,z i=dat3.csv o=rsl5.csv
$ more rsl5.csv
variable1,variable2,HammingDist,PhiCoeff
x,y,0.75,-0.5773502692
x,z,0.5,-0.3333333333
y,z,0.75,-0.5773502692
\end{Verbatim}
