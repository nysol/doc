\subsubsection*{Example 1: Basic Example}

Select records with common customers in both input file and reference file. Save unmatched records to a separate file \verb|oth.csv|.


\begin{Verbatim}[baselinestretch=0.7,frame=single]
$ more dat1.csv
Customer,Quantity
A,1
B,2
C,1
D,3
E,1
$ more ref1.csv
Customer,Gender
A,Female
B,Male
E,Female
$ mcommon k=Customer m=ref1.csv u=oth.csv i=dat1.csv o=rsl1.csv
#END# kgcommon i=dat1.csv k=Customer m=ref1.csv o=rsl1.csv u=oth.csv
$ more rsl1.csv
Customer%0,Quantity
A,1
B,2
E,1
$ more oth.csv
Customer%0,Quantity
C,1
D,3
\end{Verbatim}
\subsubsection*{Example 2: Select unmatched records from the input file}

Reverse selection criteria by using the \verb|-r| option, the "Customer" not in the reference file are selected.


\begin{Verbatim}[baselinestretch=0.7,frame=single]
$ mcommon k=Customer m=ref1.csv -r i=dat1.csv o=rsl2.csv
#END# kgcommon -r i=dat1.csv k=Customer m=ref1.csv o=rsl2.csv
$ more rsl2.csv
Customer%0,Quantity
C,1
D,3
\end{Verbatim}
\subsubsection*{Example 3: Different names of join key}

If the join key field name in the reference file is different, specify the field name at ¥verb|K=|.


\begin{Verbatim}[baselinestretch=0.7,frame=single]
$ more ref2.csv
CustomerID,Gender
A,Female
B,Male
E,Female
$ mcommon k=Customer K=CustomerID i=dat1.csv m=ref2.csv o=rsl3.csv
#END# kgcommon K=CustomerID i=dat1.csv k=Customer m=ref2.csv o=rsl3.csv
$ more rsl3.csv
Customer%0,Quantity
A,1
B,2
E,1
\end{Verbatim}
\subsubsection*{Example 4: Example with duplicate key fields}

Record selection with duplicate records in both input file and reference file.


\begin{Verbatim}[baselinestretch=0.7,frame=single]
$ more dat3.csv
Customer,Quantity 
A,1
A,2
A,3
B,1
D,1
D,2
$ more ref3.csv
Customer
A
A
D
$ mcommon k=Customer m=ref3.csv -r i=dat3.csv o=rsl4.csv
#END# kgcommon -r i=dat3.csv k=Customer m=ref3.csv o=rsl4.csv
$ more rsl4.csv
Customer%0,Quantity 
B,1
\end{Verbatim}
