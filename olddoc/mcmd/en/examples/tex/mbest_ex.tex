\subsubsection*{Example 1: Basic Example}

This example assumed that the "quantity" and "amount" fields are sorted from the largest value (descending order).
Records are selected from the first row (line 0) by default if \verb|from=|,\verb|to=|,\verb|size=| parameters are not specified.


\begin{Verbatim}[baselinestretch=0.7,frame=single]
$ more dat1.csv
Customer,Quantity,Amount
A,20,5200 
B,18,4000   
C,15,3500 
D,10,2000 
E,3,800 
$ mbest s=Quantity%nr,Amount%nr i=dat1.csv o=rsl1.csv
#END# kgbest i=dat1.csv o=rsl1.csv s=Quantity%nr,Amount%nr
$ more rsl1.csv
Customer,Quantity%0nr,Amount%1nr
A,20,5200 
\end{Verbatim}
\subsubsection*{Example 2: Basic Example 2}

After sorting by "customers", select 3 rows from the first row (line 0).


\begin{Verbatim}[baselinestretch=0.7,frame=single]
$ mbest s=Customer from=0 size=3 i=dat1.csv o=rsl2.csv
#END# kgbest from=0 i=dat1.csv o=rsl2.csv s=Customer size=3
$ more rsl2.csv
Customer%0,Quantity,Amount
A,20,5200 
B,18,4000   
C,15,3500 
\end{Verbatim}
\subsubsection*{Example 3: Basic Example 3}

Without sorting (in the original order), select from line 0 to line 1.


\begin{Verbatim}[baselinestretch=0.7,frame=single]
$ mbest -q from=0 to=1 i=dat1.csv o=rsl3.csv
#END# kgbest -q from=0 i=dat1.csv o=rsl3.csv to=1
$ more rsl3.csv
Customer,Quantity,Amount
A,20,5200 
B,18,4000   
\end{Verbatim}
\subsubsection*{Example 4: Reverse Selection}

Select records other than customers' first visit to store.
Save the records of customers' first visit to the file \verb|fvd.csv|.


\begin{Verbatim}[baselinestretch=0.7,frame=single]
$ more dat2.csv
Customer,Date,Amount
A,20081201,10
A,20081207,20
A,20081213,30
B,20081002,40
B,20081209,50
$ mbest s=Customer,Date k=Customer -r u=fvd.csv i=dat2.csv o=rsl4.csv
#END# kgbest -r i=dat2.csv k=Customer o=rsl4.csv s=Customer,Date u=fvd.csv
$ more rsl4.csv
Customer,Date,Amount
A,20081207,20
A,20081213,30
B,20081209,50
$ more fvd.csv
Customer,Date,Amount
A,20081201,10
B,20081002,40
\end{Verbatim}
