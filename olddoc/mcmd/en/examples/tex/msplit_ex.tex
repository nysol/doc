\subsubsection*{Example 1: Example 1: Basic example}

Fields are partitioned by en spaces.


\begin{Verbatim}[baselinestretch=0.7,frame=single]
$ more dat1.csv
id,data
A,1 10 2
A,2 20 3
B,1 15 5
B,3 10 4
B,1 20 6
$ msplit f=data a=d1,d2,d3 i=dat1.csv o=rsl1.csv
#END# kgsplit a=d1,d2,d3 f=data i=dat1.csv o=rsl1.csv
$ more rsl1.csv
id,data,d1,d2,d3
A,1 10 2,1,10,2
A,2 20 3,2,20,3
B,1 15 5,1,15,5
B,3 10 4,3,10,4
B,1 20 6,1,20,6
\end{Verbatim}
\subsubsection*{Example 2: Example 2: Using -r}

The \verb|-r| option is specified to delete the fields specified by the \verb|f=| parameter.


\begin{Verbatim}[baselinestretch=0.7,frame=single]
$ msplit f=data a=d1,d2,d3 -r i=dat1.csv o=rsl2.csv
#END# kgsplit -r a=d1,d2,d3 f=data i=dat1.csv o=rsl2.csv
$ more rsl2.csv
id,d1,d2,d3
A,1,10,2
A,2,20,3
B,1,15,5
B,3,10,4
B,1,20,6
\end{Verbatim}
\subsubsection*{Example 3: Example 3: Unmatched partition counts}

If the partitionable fields are fewer than the number of fields specified by the \verb|a=| parameter, NULL will be added. If they are more than the number of fields specified by the \verb|a=| parameter, they are output from the beginning until the specified number of partitions is reached.


\begin{Verbatim}[baselinestretch=0.7,frame=single]
$ more dat2.csv
id,data
A,1 10 2
A,2 20 3
B,1 15 5
B,3 4
B,1
$ msplit f=data a=d1,d2 i=dat2.csv o=rsl3.csv
#END# kgsplit a=d1,d2 f=data i=dat2.csv o=rsl3.csv
$ more rsl3.csv
id,data,d1,d2
A,1 10 2,1,10
A,2 20 3,2,20
B,1 15 5,1,15
B,3 4,3,4
B,1,1,
\end{Verbatim}
\subsubsection*{Example 4: Example 4: Specifying delim}

The \verb|delim=| parameter is used to partition fields by a character other than the en space.


\begin{Verbatim}[baselinestretch=0.7,frame=single]
$ more dat3.csv
id,data
A,1_10_3
A,2_20_5
B,1_15_6
B,3_10_7
B,1_20_8
$ msplit f=data a=d1,d2,d3 delim=_ i=dat3.csv o=rsl4.csv
#END# kgsplit a=d1,d2,d3 delim=_ f=data i=dat3.csv o=rsl4.csv
$ more rsl4.csv
id,data,d1,d2,d3
A,1_10_3,1,10,3
A,2_20_5,2,20,5
B,1_15_6,1,15,6
B,3_10_7,3,10,7
B,1_20_8,1,20,8
\end{Verbatim}
