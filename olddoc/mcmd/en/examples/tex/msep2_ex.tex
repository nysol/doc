\subsubsection*{Example 1: Basic Example}

Split the data by corresponding values in \verb|item| field. Output file names are sequential numbers starting from 0. The key and corresponding number is printed to \verb|table.csv|.


\begin{Verbatim}[baselinestretch=0.7,frame=single]
$ more dat1.csv
item,no
A,1
A,1
A,2
B,1
B,2
$ msep2 k=item O=./output a=fileName o=table.csv i=dat1.csv
#END# kgsep2 O=./output a=fileName i=dat1.csv k=item o=table.csv
$ ls ./output
0
1
$ more table.csv
item%0,fileName
A,./output/0
B,./output/1
$ more output/0
item%0,no
A,1
A,1
A,2
$ more output/1
item%0,no
B,1
B,2
\end{Verbatim}
\subsubsection*{Example 2: Multiple key fields}

Each file name is created according to the sequential number using \verb|item,no| as the composite key field. The key field and its corresponding sequential file names are printed to \verb|table.csv|.


\begin{Verbatim}[baselinestretch=0.7,frame=single]
$ more dat1.csv
item,no
A,1
A,1
A,2
B,1
B,2
$ msep2 k=item,no O=./output2 a=fileName o=table.csv i=dat1.csv
#END# kgsep2 O=./output2 a=fileName i=dat1.csv k=item,no o=table.csv
$ ls ./output2
0
1
2
3
$ more table.csv
item%0,no%1,fileName
A,1,./output2/0
A,2,./output2/1
B,1,./output2/2
B,2,./output2/3
$ more output/0
item%0,no
A,1
A,1
A,2
\end{Verbatim}
