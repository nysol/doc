\subsubsection*{Example 1: Basic Example}

Returns the highest integer by rounding up and truncating decimals.


\begin{Verbatim}[baselinestretch=0.7,frame=single]
$ more dat1.csv
id,val
1,3.28
2,3.82
3,
4,-0.6
$ mcal c='floor(${val})' a=rsl i=dat1.csv o=rsl1.csv
#END# kgcal a=rsl c=floor(${val}) i=dat1.csv o=rsl1.csv
$ more rsl1.csv
id,val,rsl
1,3.28,3
2,3.82,3
3,,
4,-0.6,-1
\end{Verbatim}
\subsubsection*{Example 2: Basic Example}

Round up to the nearest decimal place.


\begin{Verbatim}[baselinestretch=0.7,frame=single]
$ mcal c='floor(${val},0.1)' a=rsl i=dat1.csv o=rsl2.csv
#END# kgcal a=rsl c=floor(${val},0.1) i=dat1.csv o=rsl2.csv
$ more rsl2.csv
id,val,rsl
1,3.28,3.2
2,3.82,3.8
3,,
4,-0.6,-0.6
\end{Verbatim}
\subsubsection*{Example 3: Example of using 0.5 as base}

Convert value to base 0.5.


\begin{Verbatim}[baselinestretch=0.7,frame=single]
$ mcal c='floor(${val},0.5)' a=rsl i=dat1.csv o=rsl3.csv
#END# kgcal a=rsl c=floor(${val},0.5) i=dat1.csv o=rsl3.csv
$ more rsl3.csv
id,val,rsl
1,3.28,3
2,3.82,3.5
3,,
4,-0.6,-1
\end{Verbatim}
\subsubsection*{Example 4: Example of using 10 as base}

一桁目を切り上げる。


\begin{Verbatim}[baselinestretch=0.7,frame=single]
$ more dat2.csv
id,val
1,1341.28
2,188
3,1.235E+3
4,-1.235E+3
$ mcal c='round(${val},10)' a=rsl i=dat2.csv o=rsl4.csv
#END# kgcal a=rsl c=round(${val},10) i=dat2.csv o=rsl4.csv
$ more rsl4.csv
id,val,rsl
1,1341.28,1340
2,188,190
3,1.235E+3,1240
4,-1.235E+3,-1230
\end{Verbatim}
