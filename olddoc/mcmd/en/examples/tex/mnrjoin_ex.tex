\subsubsection*{Example 1: Basic Example}

For records where the value of date field is \verb|20080203|, select those records in the input data where \verb|amount| field is more than \verb|5| but less than \verb|15| and join field where \verb|avg=150|. For records where \verb|amount| field is more than \verb|40| but less than \verb|50|, join field \verb|avg=200|.


\begin{Verbatim}[baselinestretch=0.7,frame=single]
$ more dat1.csv
date,price
20080123,10
20080123,20
20080203,10
20080203,35
200804l0,50
$ more ref1.csv
date,priceF,priceT,avg
20080203,5,15,150
20080203,40,50,200
$ mnrjoin k=date f=avg m=ref1.csv R=priceF,priceT r=price%n i=dat1.csv o=rsl1.csv
#END# kgnrjoin R=priceF,priceT f=avg i=dat1.csv k=date m=ref1.csv o=rsl1.csv r=price%n
$ more rsl1.csv
date%0,price,avg
20080203,10,150
\end{Verbatim}
\subsubsection*{Example 2: Output unmatched data}

Use \verb|-n| to return all records in the input data even if they do not match with those in the reference file (row where \verb|avg=| Null), and use \verb|-N| to return records in the reference file even if they do not match with those in the input file (rows where \verb|price=| null). This is known as outer-join.


\begin{Verbatim}[baselinestretch=0.7,frame=single]
$ mnrjoin k=date f=avg m=ref1.csv R=priceF,priceT r=price%n -n -N i=dat1.csv o=rsl2.csv
#END# kgnrjoin -N -n R=priceF,priceT f=avg i=dat1.csv k=date m=ref1.csv o=rsl2.csv r=price%n
$ more rsl2.csv
date%0,price,avg
20080123,10,
20080123,20,
20080203,10,150
20080203,35,
20080203,,200
200804l0,50,
\end{Verbatim}
