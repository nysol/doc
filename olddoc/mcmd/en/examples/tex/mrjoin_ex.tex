\subsubsection*{Example 1: Basic Example}

Join category field \verb|low, middle,high| to corresponding \verb|price| range.


\begin{Verbatim}[baselinestretch=0.7,frame=single]
$ more dat1.csv
price
8
15
35
50
90
200
$ more ref1.csv
range,category
10,low
35,middle
80,high
100,
$ mrjoin r=price%n m=ref1.csv R=range f=category i=dat1.csv o=rsl1.csv
#END# kgrjoin R=range f=category i=dat1.csv m=ref1.csv o=rsl1.csv r=price%n
$ more rsl1.csv
price%0n,category
15,low
35,middle
50,middle
90,high
\end{Verbatim}
\subsubsection*{Example 2: Basic Example 2}



\begin{Verbatim}[baselinestretch=0.7,frame=single]
$ mrjoin -lo r=price%n m=ref1.csv R=range f=category i=dat1.csv o=rsl2.csv
#END# kgrjoin -lo R=range f=category i=dat1.csv m=ref1.csv o=rsl2.csv r=price%n
$ more rsl2.csv
price%0n,category
15,low
35,low
50,middle
90,high
\end{Verbatim}
\subsubsection*{Example 3: Basic Example 3}



\begin{Verbatim}[baselinestretch=0.7,frame=single]
$ mrjoin -n r=price%n m=ref1.csv R=range f=category i=dat1.csv o=rsl3.csv
#END# kgrjoin -n R=range f=category i=dat1.csv m=ref1.csv o=rsl3.csv r=price%n
$ more rsl3.csv
price%0n,category
8,
15,low
35,middle
50,middle
90,high
200,
\end{Verbatim}
\subsubsection*{Example 4: Join with different ranges for corresponding products}



\begin{Verbatim}[baselinestretch=0.7,frame=single]
$ more dat2.csv
item,price
A,10
A,20
B,10
B,20
$ more ref2.csv
item,price,category
A,10,low
A,15,high
A,100,
B,10,low
B,35,high
B,100,
$ mrjoin k=item r=price%n m=ref2.csv f=category i=dat2.csv o=rsl4.csv
#END# kgrjoin f=category i=dat2.csv k=item m=ref2.csv o=rsl4.csv r=price%n
$ more rsl4.csv
item%0,price%1n,category
A,10,low
A,20,high
B,10,low
B,20,low
\end{Verbatim}
