\subsubsection*{Example 1: Basic Example}

Print NULL values to column \verb|rsl|. 


\begin{Verbatim}[baselinestretch=0.7,frame=single]
$ more dat1.csv
id
1
2
3
$ mcal c='nulls()' a=rsl i=dat1.csv o=rsl1.csv
#END# kgcal a=rsl c=nulls() i=dat1.csv o=rsl1.csv
$ more rsl1.csv
id,rsl
1,
2,
3,
\end{Verbatim}
\subsubsection*{Example 2: Use of if statement}

Use nulln() function to match the value specified in the second parameter.


\begin{Verbatim}[baselinestretch=0.7,frame=single]
$ mcal c='if(${id}==1,1,nulln())' a=rsl i=dat1.csv o=rsl2.csv
#END# kgcal a=rsl c=if(${id}==1,1,nulln()) i=dat1.csv o=rsl2.csv
$ more rsl2.csv
id,rsl
1,1
2,
3,
\end{Verbatim}
\subsubsection*{Example 3: Equivalent to isnull function}



\begin{Verbatim}[baselinestretch=0.7,frame=single]
$ mcal c='if(${val}==nulln(),"null","notNull")' a=rsl i=dat2.csv o=rsl3.csv
#END# kgcal a=rsl c=if(${val}==nulln(),"null","notNull") i=dat2.csv o=rsl3.csv
$ more rsl3.csv
id,val,rsl
1,a,
2,,
3,b,
\end{Verbatim}
