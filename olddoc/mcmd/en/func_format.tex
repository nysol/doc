
%\begin{document}

\section{format - Formatted Output\label{sect:format}}
\index{format@format}

Format: format($num$, output format)

mcal internally manages floating point numbers. This function changes the output format of the number by specifying the output format available in the C language printf function. The 3 available formats are as follows. 

\begin{itemize}
\item \%f: Decimal format
\item \%e,\%E: Exponent format (Specify \%E to use uppercase E as exponent notation)
\item \%g,\%G: Automatically select output format of f or e (Specify \%G for uppercase notation)
\end{itemize}

For example, the decimal format is expressed as \verb|0.00726| can be expressed as \verb|7.260000e-03| in exponent format, similarly, the exponent format of \verb|1265| is represented as \verb|1.265000e+03|. 

By specifying the \verb|%| or plus sign after the number, the function formats the decimal digits and characters. The number format can be specified in the format of “\verb|whole number.decimal|”. For example, the number \verb|123.456789| with the format specifier of \verb|%5.2f| becomes \verb|123.46|, and the format specifier of \verb|%8.3f| print number as \verb|123.458| with a floating point of 8 characters and 3 characters after decimal. 

Add a plus sign before the number to display the plus sign before the number. For example, \verb|123.456789| with the format specifier of \verb|%+5.2f| format the number as \verb|+123.46|. 

Besides the format highlighted above, it is possible to include any character string in the format. For example, the number \verb|250| can be formatted as \verb|“Total 250 Yen”| using the format specifier of “Total %g Yen”. Use the format specifier of “%%” to display the symbol \%. The number \verb|15| can be formatted as \verb|15%| using the format specifier of \verb|"%g%%"|.    



\subsection*{Examples}
\subsubsection*{例1: 基本例}

\verb|val|を実数として小数点以下2桁に変換する。


\begin{Verbatim}[baselinestretch=0.7,frame=single]
$ more dat1.csv
id,val
1,0.00726
2,123.456789
3,
4,-0.335
$ mcal c='format(${val},"%8.3f")' a=rsl i=dat1.csv o=rsl1.csv
#END# kgcal a=rsl c=format(${val},"%8.3f") i=dat1.csv o=rsl1.csv
$ more rsl1.csv
id,val,rsl
1,0.00726,   0.007
2,123.456789, 123.457
3,,
4,-0.335,  -0.335
\end{Verbatim}
\subsubsection*{例2: 指数表現}

\verb|val|を指数表現で出力。


\begin{Verbatim}[baselinestretch=0.7,frame=single]
$ mcal c='format(${val},"%e")' a=rsl i=dat1.csv o=rsl2.csv
#END# kgcal a=rsl c=format(${val},"%e") i=dat1.csv o=rsl2.csv
$ more rsl2.csv
id,val,rsl
1,0.00726,7.260000e-03
2,123.456789,1.234568e+02
3,,
4,-0.335,-3.350000e-01
\end{Verbatim}


%\end{document}

