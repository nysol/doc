
%\begin{document}

\section*{mvcal operations on vector }
Performs various operations on vector .

\begin{table}[htbp]
\begin{center}
\caption{list of vector operation\label{tbl:ope}}
\begin{tabular}{ccc}
\hline
operator&signification \\
\hline
1&a b c \\
2&a d \\
3&b f e f \\
4&f c d \\
\hline
\end{tabular}
\end{center}
\end{table}

Table \ref{tbl:input} shows the string sequence under the items column. Each element in the string is separated by a space. 

Some examples are shown in Table \ref{tbl:input}〜\ref{tbl:out3}.

\begin{table}[htbp]
\begin{center}
\begin{tabular}{ccc}

\begin{minipage}{0.3\hsize}
\begin{center}
\caption{Input data\label{tbl:input}}
\verb|in.csv| \\
{\small
\begin{tabular}{cl}
\hline
no&items \\
\hline
1&a b c \\
2&a d \\
3&b f e f \\
4&f c d \\
\hline

\end{tabular}
}
\end{center}
\end{minipage}

\begin{minipage}{0.3\hsize}
\begin{center}
\caption{Reference file\label{tbl:ref}}
\verb|ref.csv| \\
{\small
\begin{tabular}{cc}
\hline
item&taxo \\
\hline
a&X \\
b&Y \\
c&Z \\
e&X \\
f&Z \\
\hline
\end{tabular}
}
\end{center}
\end{minipage}
\end{tabular}
\end{center}
\end{table}

\begin{table}[htbp]
\begin{center}
\begin{tabular}{ccc}

\begin{minipage}{0.5\hsize}
\begin{center}
\caption{Basic example\label{tbl:out2}}
\verb|vf=items m=ref.csv K=item f=taxo|
{\small
\begin{tabular}{ll}
\hline
no&items \\
\hline
1&a b c X Y Z \\
2&a d X \\
3&b f e f Y Z X Z \\
4&f c d Z Z \\
\hline

\end{tabular}
}
\end{center}
\end{minipage}

\begin{minipage}{0.50\hsize}
\begin{center}
\caption{the example of specification of an unmatched item.\label{tbl:out3}}
\verb|vf=items m=ref.csv K=item f=taxo n=* | \\
{\small
\begin{tabular}{ll}
\hline
no&items \\
\hline
1&a b c X Y Z \\
2&a d X * \\
3&b f e f Y Z X Z \\
4&f c d Z Z * \\
\hline
\end{tabular}
}
\end{center}
\end{minipage}

\end{tabular}
\end{center}
\end{table}

Since the \verb|mvjoin| command once sets reference file data to a memory altogether, when a huge reference file is specified, it is cautious of a memory being used up. 

\subsection*{format}
\verb|mvjoin vf= [K=] m= f= [n=]|
[\href{run:delim.pdf}{delim=}]
[\href{run:input.pdf}{i=}]
[\href{run:output.pdf}{o=}]
[\href{run:nfn.pdf}{-nfn}]
[\href{run:nfno.pdf}{-nfno}]
[\href{run:x.pdf}{-x}] [--help]

\begin{table}[htbp]
%\begin{center}
{\small
\begin{tabular}{ll}
\verb|vf=| & The item name of the item set used as a joint key (\verb|i=|on a file) is specified. 
This parameter is mandatory.\\
           & Multiple items can be specified. Sorting of the item does not have to be carried out.  \\
\verb|m=|  & A reference file is specified. \\
\verb|K=|  & The item name of the item used as the joint key on reference file(\verb|m=|) is specified. 
This parameter is mandatory.\\
\verb|f=|  & The item (set) item name to combine is specified. 
This parameter is mandatory.\\
\verb|n=|  & The character string combined when the item of vf= and K= does not match is specified.  \\
           & When it omits, combination of the target item is not performed. 
 \\
\end{tabular}
}
\end{table} 

\subsection*{Usage examples}
\subsubsection*{Example 1 the example combined to multiple items}
\begin{verbatim}
------------------------------------------------
# dat1.csv
items1,items2
b a c,b b
c c,a d
e a a,a a

# ref.csv
item,taxo
a,X
b,X
c,X
d,Y
e,Y

$ mvjoin vf=items1,items2 m=ref.csv f=taxo i=dat1.csv o=rsl1.csv

# rsl1.csv
items1,items2
b a c X X X,b b X X
c c X X,a d X Y
e a a Y X X,a a X X
------------------------------------------------
\end{verbatim}

\subsection*{related command}
\begin{itemize}
\item \href{run:mvcommon.pdf}{mvcommon}: selection of the vector element by a reference file 

\end{itemize}

\end{document}
