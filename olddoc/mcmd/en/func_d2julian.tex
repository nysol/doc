
%\begin{document}

\section{d2julian: date-Change date to a Julian day\label{sect:d2julian}}
\index{d2julian@d2julian}

Format: d2julian($date$)
Use this function to obtain a Julian day from the date.
\begin{verbatim}
A Julian day is the number of days from noon of January 1, 4713 B.C. (universal time).
The effective range of the Gregorian calendar, however, is 1400-1-1 to 9999-12-31, so the corresponding range of the Julius days is 2232300 to 5373484.
NULL is output if the result is outside the range.
\end{verbatim}

\subsection*{Usage}
\begin{verbatim}
c=d2julian(0d20080822)
\end{verbatim}

\subsection*{Example}

\begin{table}[hbt]
\begin{center}
 \caption{Input data}
  \begin{tabular}{|c|c|c|} \hline
Date&Time&Number\\ \hline\hline
20020824&20020824145408&10660\\ \hline
20020622&20020622173449&22740\\ \hline
20020824&20020824145408&14800\\ \hline
20021009&20021009095743&54510\\ \hline
20020121&20020121173449&18750\\ \hline
  \end{tabular}
  \end{center}
\end{table}

Examples based on the above data are shown below.

\subsubsection*{Execution example1)}
The value in the “Date” field is input and converted into a Julian day. A “Julian day” field is newly created, and the result is output.
\begin{verbatim}
------------------------------------------------
mcal c='d2julian($d{Date})' a="Julian day" i=date.csv o=od2julian.csv
------------------------------------------------
\end{verbatim}

\begin{table}[hbt]
\begin{center}
 \caption{Output file(od2julian.csv)}
  \begin{tabular}{|c|c|c|c|} \hline
Date&Time&Number&Julian day\\ \hline\hline
20020824&20020824145408&10660&2452511\\ \hline
20020622&20020622173449&22740&2452448\\ \hline
20020824&20020824145408&14800&2452511\\ \hline
20021009&20021009095743&54510&2452557\\ \hline
20020121&20020121173449&18750&2452296\\ \hline
  \end{tabular}
  \end{center}
\end{table}

\href{run:hizuke.pdf}{Return to mcal - date related}\\
%\end{document}

