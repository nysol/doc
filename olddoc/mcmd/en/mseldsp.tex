
%\documentclass[a4paper]{jsbook}
%\usepackage{mcmd_jp}
%\begin{document}

\section{mseldsp Display option selection screen\label{sect:mseldsp}}
\index{mseldsp@mseldsp}
\underline{Note: This command is a beta. Its specifications may be changed.}

This command displays a list of character strings to choose from at the position on the terminal specified by coordinate \verb|x=,y=|. The list is specified by the \verb|i=| or \verb|seldata=| parameter. 
The user can choose only one from the list. To allow multiple choice, use the \hyperref[sect:mmseldsp]{mmseldsp} command. 
The output CSV file contains one row and one field. When the command ends with nothing entered in the input frame, null is output (that is, only a line feed is output). To output a fieldname, use the \verb|f=| parameter. With \verb|f=| omitted, no fieldname header is output. 
If there are too many choices to fit into the screen, use the \verb|height=| parameter to specify the number of rows shown in a scroll window.
When the Enter key is pressed on the selection screen, the command returns end status 0 and ends. When the ESC key is pressed on the screen, the command returns end status 1 and ends. In either case, the choice made on the selection screen is output to a file.
In the coordinate system, the top left is \verb|x=1,y=1| (escape sequence specifications). If the value specified for \verb|x=| or \verb|y=| is smaller than 1, the command assumes 1 and operates. The operation is indefinite if the specified coordinate is outside the scope of the terminal.

\subsection*{Format}
\verb/mseldsp x= y= [height=] i=|seldata=/
\hyperref[sect:option_o]{o=}
\hyperref[sect:option_nfn]{[-nfn]} 
\hyperref[sect:option_nfno]{[-nfno]}  
\hyperref[sect:option_x]{[-x]}
\verb|[--help]|
\verb|[--helpl]|
\verb|[--version]|\\

\subsection*{Parameters}
\begin{table}[htbp]
%\begin{center}
{\small
\begin{tabular}{ll}
\verb|x=|   & Specify the display start position (1 or greater) on the x axis (horizontal, left to right)\\
\verb|y=|   & Specify the display start position (1 or greater) on the y axis (vertical, top to bottom)\\
\verb|height=| & Specify the number of rows in which the choices are shown.\\
\verb|i=|   & Specify the name of the CSV file containing the choice character strings as fields.\\
\verb|f=|   & Specify the name of the CSV file containing the choice character strings as fields.\\
\verb|seldata=| & Specify the list of character strings to choose from, using commas ad delimiters.\\
\end{tabular} 
}
\end{table} 

\subsection*{Examples}

\subsubsection*{Example 1: Basic example}
The contents of \verb|sel.txt| are displayed at x=10,y=3 on the terminal, and the character string selected by the user is output to \verb|rsl1.txt|.

%\begin{Verbatim}[baselinestretch=0.7,frame=single,commandchars=\\\{\}]
\begin{Verbatim}[baselinestretch=0.7,frame=single]
$ more sel.txt
1:I agree.
2:I do not know.
3:I disagree.
$ mseldsp x=10 y=3 i=sel.txt o=rsl1.txt
# Assume that the user has selected the first row.
$ mose rsl1.txt
1:I agree.

The following will be displayed:
+--------------------------------------
|
|
|          \textColor{red}{black}{1:I agree.}
|          \textColor{white}{black}{2:I do not know.}
|          \textColor{white}{black}{3:I disagree.}
|
\end{Verbatim}

\subsubsection*{Example 2: Using arguments to specify the character strings}
This script works in the same way as Example 1, but the option character strings are specified by \verb|seldata=|.

%\begin{Verbatim}[baselinestretch=0.7,frame=single,commandchars=\\\{\}]
\begin{Verbatim}[baselinestretch=0.7,frame=single]
$ mseldsp x=10 y=3 seldata=1:I agree.,2:I do not know.,3:I disagree. o=rsl2.txt
# Assume that the user has selected the second row.
$ mose rsl2.txt
2:I do not know.

The following will be displayed:
+--------------------------------------
|
|
|          \textColor{white}{black}{1:I agree.}
|          \textColor{red}{black}{2:I do not know.}
|          \textColor{white}{black}{3:I disagree.}
|
\end{Verbatim}

\subsubsection*{Example 3: Judging end status}
The parameters are the same as in Example 1. This script judges the end status and performs different operations accordingly.

\begin{Verbatim}[baselinestretch=0.7,frame=single]
$ more scp.sh
rm -f rsl3.csv
clear
mseldsp x=10 y=3 seldata=aaaa,bbbb,cccc o=rsl3.csv
if [ $? = 0 ] ; then
  clear ; echo "end by enter key"
else
  clear ; echo "end by escape key"
fi

# Result of selecting aaaa and pressing Enter
$ bash scp.sh
end by enter key
$ more rsl3.csv
aaaa

# Result of selecting bbbb and pressing ESC
$ bash scp.sh
end by escape key
$ more rsl3.csv
bbbb
\end{Verbatim}

\subsection*{Related Commands}
\hyperref[sect:minput] {minput} : Displays the input screen.
\hyperref[sect:mminput] {mminput} :Displays the input screen with multiple input frames.
\hyperref[sect:mdsp] {mdsp} : Displays a character string at a specified position on the screen.
\hyperref[sect:mmseldsp] {mmseldsp} : Displays a multiple-choice input window on the screen.

%\end{document}
