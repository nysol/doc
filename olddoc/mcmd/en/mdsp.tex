

%\documentclass[a4paper]{jsbook}
%\usepackage{color}
%\usepackage{mcmd_jp}
%\begin{document}

\section{mdsp Display character strings on screen\label{sect:mdsp}}
\index{mdsp@mdsp}
\underline{Note: This command is a beta. Its specifications may be changed.}

This command displays the character string specified by the \verb|str=| parameter at the position on the terminal specified by coordinate \verb|x=|,\verb|y=|. Use the \verb|i=| parameter to specify a filename, and its contents will be displayed. When both \verb|str=| and \verb|i=| are specified, \verb|i=| will prevail. If multiple rows are specified for the \verb|i=| parameter, all rows will be displayed from the coordinate specified by \verb|x=|.
Use the \verb|fc=| parameter to specify the color of the character string. Use the \verb|bg=| parameter to specify the background color of the character string. You can choose from eight colors: black, red, green, yellow, blue, magenta, cyan, and white. By default, \verb|fc=black| and \verb|bg=white| are assumed. Specify the \verb|-bold| option to display the character string in bold letters.

\subsection*{Format}
\verb/mdsp x= y= str=|i= [fc=] [bg=] [-bold] /
\verb|[--help]|
\verb|[--helpl]|
\verb|[--version]|\\

\subsection*{Parameters}
\begin{table}[htbp]
%\begin{center}
{\small
\begin{tabular}{ll}
\verb|x=|   & Specify the display start position (1 or greater) on the x axis (horizontal, left to right).\\
\verb|y=|   & Specify the display start position (1 or greater) on the y axis (vertical, top to bottom).\\
\verb|str=| & Character string to be displayed\\
\verb|i=|   & Filename of the contents to be displayed\\
\verb|fc=|  & Character color\\
\verb|bg=|  & Background color\\
\verb|-bold|& Bold letter\\
\end{tabular} 
}
\end{table} 


\subsection*{Examples}

\subsubsection*{Example 1:  Basic example}
Character string \verb|abcd| is displayed at \verb|x=10,y=5| on the terminal.

\begin{Verbatim}[baselinestretch=0.7,frame=single]
$ mdsp x=10 y=5 str=abcd

The following will be displayed:
+--------------------------------------
|
|
|
|
|          abcd
|
|
\end{Verbatim}

\subsubsection*{Example 2: Using a filename to specify what to display}
The contents of \verb|dat.txt| are displayed at \verb|x=10,y=5| on the terminal.

\begin{Verbatim}[baselinestretch=0.7,frame=single]
$ more dat.txt
abcd
efg
$ mdsp x=10 y=5 i=dat.txt

The following will be displayed:
+--------------------------------------
|
|
|
|
|          abcd
|          efg
|
\end{Verbatim}

\subsubsection*{Example 3: Using colors}
Character string \verb|abcd| is displayed at \verb|x=10,y=5| on the terminal, with the characters inbeing red and the background inbeing blue.

%\begin{Verbatim}[baselinestretch=0.7,frame=single,commandchars=\\\{\}]
\begin{Verbatim}[baselinestretch=0.7,frame=single]
$ mdsp x=10 y=5 str=abcd fc=red bc=blue

The following will be displayed:
+--------------------------------------
|
|
|
|
|          \textColor{blue}{red}{abcd}
|
|
\end{Verbatim}

\subsection*{Related Commands}
\hyperref[sect:minput] {minput} : Displays the input screen.
\hyperref[sect:mminput] {mminput} : Displays the input screen consisting of multiple input frames.
\hyperref[sect:mseldsp] {mseldsp} : Displays a single-choice input window on the screen.
\hyperref[sect:mmseldsp] {mmseldsp} : Displays a multiple-choice input window on the screen.

%\end{document}

