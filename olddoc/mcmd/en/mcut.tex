
%\begin{document}

\section{mcut - Select Column\label{sect:mcut}}
\index{mcut@mcut}
Extract the specified column(s). 
The specified column is removed with \verb|-r| option. 

\subsection*{Format}
\verb|mcut f= [-r] [-nfni]|
\hyperref[sect:option_i]{[i=]}
\hyperref[sect:option_o]{[o=]}
\hyperref[sect:option_assert_diffSize]{[-assert\_diffSize]}
\hyperref[sect:option_assert_nullin]{[-assert\_nullin]}
\hyperref[sect:option_nfn]{[-nfn]} 
\hyperref[sect:option_nfno]{[-nfno]}  
\hyperref[sect:option_x]{[-x]}
\hyperref[sect:option_q]{[-q]}
\hyperref[sect:option_option_tmppath]{[tmpPath=]}
\verb|[--help]|
\verb|[--helpl]|
\verb|[--version]|\\

\subsection*{Parameters}
\begin{table}[htbp]
%\begin{center}
{\small
\begin{tabular}{ll}
\verb|f=|    & Define name of column to be extracted  \\
             & New column name for can be specified by defining the field name, followed by colon and the new field name. \\
             & ex. \verb|f=a:A,b:B| \\
\verb|-r|    & Field removal switch \\
             & Remove all columns defined in the \verb|f=| parameter.\\
\verb|-nfni| & When header is not present in first row of the input data, position number of column is used to identify corresponding field(s). \\
             & New column name(s) for each column can be specified in the output file as follows. \\
             & Example f=0:date,2:store,3:quantity \\
\end{tabular} 
}
\end{table} 


\subsection*{Examples}
\subsubsection*{例1: 基本例}

「顧客」と「金額」項目を選択する。ただし、「金額」項目は「売上」と名前を変更して出力している。


\begin{Verbatim}[baselinestretch=0.7,frame=single]
$ more dat1.csv
顧客,数量,金額
A,1,10
A,2,20
B,1,15
B,3,10
B,1,20
$ mcut f=顧客,金額:売上 i=dat1.csv o=rsl1.csv
#END# kgcut f=顧客,金額:売上 i=dat1.csv o=rsl1.csv
$ more rsl1.csv
顧客,売上
A,10
A,20
B,15
B,10
B,20
\end{Verbatim}
\subsubsection*{例2: 項目削除}

\verb|-r|を指定することで、項目を削除できる。


\begin{Verbatim}[baselinestretch=0.7,frame=single]
$ mcut f=顧客,金額 -r i=dat1.csv o=rsl2.csv
#END# kgcut -r f=顧客,金額 i=dat1.csv o=rsl2.csv
$ more rsl2.csv
数量
1
2
1
3
1
\end{Verbatim}
\subsubsection*{例3: 項目名なしデータ}

ヘッダなし入力ファイルから、0,2番目の項目を選択し、
「顧客」と「金額」という名前で出力する。


\begin{Verbatim}[baselinestretch=0.7,frame=single]
$ more dat2.csv
A,1,10
A,2,20
B,1,15
B,3,10
B,1,20
$ mcut f=0:顧客,2:金額 -nfni i=dat2.csv o=rsl3.csv
#END# kgcut -nfni f=0:顧客,2:金額 i=dat2.csv o=rsl3.csv
$ more rsl3.csv
顧客,金額
A,10
A,20
B,15
B,10
B,20
\end{Verbatim}


\subsection*{related command}
\hyperref[sect:mfldname]{mfldname} : Use \verb|mfldname| to change the field names. 

%\href{run:msubstr.pdf}{msubstr}:extraction of substring
%\end{document}
