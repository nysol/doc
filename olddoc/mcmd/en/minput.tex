
%\documentclass[a4paper]{jsbook}
%\usepackage{mcmd_jp}
%\begin{document}

\section{minput Enable on-screen input\label{sect:minput}}
\index{minput@minput}
\underline{Note: This command is a beta. Its specifications may be changed.}

This command displays an input frame of the length specified by the \verb|len=| parameter at a position on the terminal specified by coordinate \verb|x=,y=|, and outputs the input contents to the file specified by the \verb|o=| parameter. When the Enter key is pressed on the input screen, the command returns end status 0 and ends. When the ESC key is pressed on the screen, the command returns end status 1 and ends. In either case, the contents that have been entered that far are output to the file.
The output CSV file contains one row and one field. When the command ends with nothing entered in the input frame, null is output (that is, only a line feed is output). To output a fieldname, use the \verb|f=| parameter. With \verb|f=| omitted, no fieldname header is output. 
In the coordinate system, the top left is \verb|x=1,y=1| (escape sequence specifications). If the value specified for \verb|x=| or \verb|y=| is smaller than 1, the command assumes 1. The operation is indefinite if the specified coordinate is outside the scope of the terminal.

\subsection*{Format}
\verb/minput x= y= len= [f=] /
\hyperref[sect:option_o]{o=}
\hyperref[sect:option_nfn]{[-nfn]} 
\hyperref[sect:option_nfno]{[-nfno]}  
\hyperref[sect:option_x]{[-x]}
\verb|[--help]|
\verb|[--helpl]|
\verb|[--version]|\\

\subsection*{Parameters}
\begin{table}[htbp]
%\begin{center}
{\small
\begin{tabular}{ll}
\verb|x=|   & Specify the display start position (1 or greater) on the x axis (horizontal, left to right).\\
\verb|y=|   & Specify the display start position (1 or greater) on the y axis (vertical, top to bottom).\\
\verb|len=| & Specify the length by the number of single-byte characters in the input area.\\
\verb|f=|   & Specify the output fieldname.\\
\end{tabular} 
}
\end{table} 

\subsection*{Examples}

\subsubsection*{Example 1: Basic example}
A 12-character input area is displayed at x=10,y=5 on the terminal. The user types abcd in the input frame and presses Enter. The input results are saved as rsl1.csv.

\begin{Verbatim}[baselinestretch=0.7,frame=single]
$ minput x=10 y=5 len=12 o=rsl1.csv
$ more rsl1.csv
abcd

The following will be displayed:
+--------------------------------------
|
|
|
|
|          [abcd        ]
|
|
\end{Verbatim}

\subsubsection*{Example 2: Basic example, with a fieldname specified}
Operation is the same as in Example 1, with an addition of the f= parameter to specify a fieldname.

\begin{Verbatim}[baselinestretch=0.7,frame=single]
$ minput x=10 y=5 len=12 f=name o=rsl1.csv
$ more rsl1.csv
name
abcd
\end{Verbatim}


\subsubsection*{Example 3: Judging end status}
The parameters are the same as in Example 1. This script judges the end status and performs different operations accordingly.

\begin{Verbatim}[baselinestretch=0.7,frame=single]
$ more scp.sh
rm -f rsl2.csv
clear
minput x=10 y=5 len=12 o=rsl2.csv
if [ $? = 0 ] ; then
  clear ; echo "end by enter key"
else
  clear ; echo "end by escape key"
fi

# Result of typing abcd and pressing Enter
$ bash scp.sh
end by enter key
$ more rsl2.csv
abcd

# Result of typing abcd and pressing ESC key
$ bash scp.sh
end by escape key
$ more rsl2.csv
abcd
\end{Verbatim}

\subsection*{Related Commands}
\hyperref[sect:mminput] {mminput} : Displays the input screen consisting of multiple input frames.
\hyperref[sect:mdsp] {mdsp} : Displays a character string at the specified position on the screen.
\hyperref[sect:mseldsp] {mseldsp} : Displays a single-choice input window on the screen.
\hyperref[sect:mmseldsp] {mmseldsp} : Displays a multiple-choice input window on the screen.

%\end{document}
