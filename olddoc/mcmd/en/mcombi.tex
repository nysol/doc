
%\begin{document}

\section{mcombi - Compute Combination\label{sect:mcombi}}
\index{mcombi@mcombi}
\verb|f=| parameter specifys the set of fields, \verb|n=| parameter defines the combination, and \verb|a=|parameter specifies the output field name. Permutation output is possible by specifying  \verb|-p| option. 

\subsection*{Format}
\verb/mcombi a= f= n= [s=] [k=] [-p] [-dup]/
\hyperref[sect:option_i]{[i=]}
\hyperref[sect:option_o]{[o=]}
\hyperref[sect:option_assert_diffSize]{[-assert\_diffSize]}
\hyperref[sect:option_assert_nullin]{[-assert\_nullin]}
\hyperref[sect:option_nfn]{[-nfn]} 
\hyperref[sect:option_nfno]{[-nfno]}  
\hyperref[sect:option_x]{[-x]}
\hyperref[sect:option_q]{[-q]}
\hyperref[sect:option_option_tmppath]{[tmpPath=]}
\verb|[--help]|
\verb|[--helpl]|
\verb|[--version]|\\

\subsection*{Parameters}
\begin{table}[htbp]
%\begin{center}
{\small
\begin{tabular}{ll}
\verb|a=|    & Name of the field to be added.\\
\verb|f=|    & Compute the combinations for the set of field name(s) (multiple fields can be specified) specified .\\
             & Enumerate all combinations of the array of values in the fields specified.\\
\verb|n=|    & Number of combinations.\\
             & When you increase the number of combinations, note that the number of output records will increase exponentially.\\
         \verb|s=|    & After sorted by specified field  (multiple fields can be specified), combinations of items specified in field \verb|f=| are enumerated. \\
\verb|k=|    & List of key field name(s) (multiple fields can be specified)\\
             & Compute  combinations based on the list of key field name(s). \\
\verb|-p|    & Compute the permutations. \\
\verb|-dup|  & Output combinations with the same value. \\
\end{tabular} 
}
\end{table} 

\subsection*{Examples}
\subsubsection*{Example 1: Basic Example}

Enumerate all combinations of two items in the \verb|item| field for each \verb|customer|, and save the output in \verb|item1,item2|.  Fields not specified at \verb|k=,f=| (\verb|item| field in this case) remains after the key field column.


\begin{Verbatim}[baselinestretch=0.7,frame=single]
$ more dat1.csv
customer,item
A,a1
A,a2
A,a3
B,a4
B,a5
$ mcombi k=customer f=item n=2 a=item1,item2 i=dat1.csv o=rsl1.csv
#END# kgcombi a=item1,item2 f=item i=dat1.csv k=customer n=2 o=rsl1.csv
$ more rsl1.csv
customer%0,item,item1,item2
A,a3,a1,a2
A,a3,a1,a3
A,a3,a2,a3
B,a5,a4,a5
\end{Verbatim}
\subsubsection*{Example 2: Basic Example 2}

When you specify the \verb|-dup| option, the output includes combination of the same field.


\begin{Verbatim}[baselinestretch=0.7,frame=single]
$ mcombi k=customer f=item n=2 a=item1,item2 i=dat1.csv o=rsl2.csv -dup
#END# kgcombi -dup a=item1,item2 f=item i=dat1.csv k=customer n=2 o=rsl2.csv
$ more rsl2.csv
customer%0,item,item1,item2
A,a3,a1,a1
A,a3,a1,a2
A,a3,a1,a3
A,a3,a2,a2
A,a3,a2,a3
A,a3,a3,a3
B,a5,a4,a4
B,a5,a4,a5
B,a5,a5,a5
\end{Verbatim}
\subsubsection*{Example 3: Compute permutation}

Enumerate permutation of two items in the \verb|item| field for each \verb|customer|, and save the output in column \verb|item1,item2|.


\begin{Verbatim}[baselinestretch=0.7,frame=single]
$ mcombi k=customer f=item n=2 a=item1,item2 -p i=dat1.csv o=rsl3.csv
#END# kgcombi -p a=item1,item2 f=item i=dat1.csv k=customer n=2 o=rsl3.csv
$ more rsl3.csv
customer%0,item,item1,item2
A,a3,a1,a2
A,a3,a2,a1
A,a3,a1,a3
A,a3,a3,a1
A,a3,a2,a3
A,a3,a3,a2
B,a5,a4,a5
B,a5,a5,a4
\end{Verbatim}

\subsection*{Related Command}
%\end{document}
