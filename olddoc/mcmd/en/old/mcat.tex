
% how to compile: platex xxx.tex ; dvipdfmx xxx.dvi

\documentclass[a4paper]{jarticle}

%--余白の設定
\setlength{\topmargin}{20mm}
\addtolength{\topmargin}{-1in}
\setlength{\oddsidemargin}{20mm}
\addtolength{\oddsidemargin}{-1in}
\setlength{\evensidemargin}{15mm}
\addtolength{\evensidemargin}{-1in}
\setlength{\textwidth}{170mm}
\setlength{\textheight}{254mm}
\setlength{\headsep}{0mm}
\setlength{\headheight}{0mm}
\setlength{\topskip}{0mm}

%--ハイバーリンクを可能にするパッケージ
\usepackage[dvipdfmx,%
 bookmarks=true,%
 bookmarksnumbered=true,%
 colorlinks=true,%
 setpagesize=false,%
 pdftitle={mcat},%
 pdfauthor={BMRC},%
 pdfkeywords={TeX; dvipdfmx; hyperref; color;}]{hyperref}


\begin{document}

\setlength{\baselineskip}{4mm}

\section*{mcat (conCATenate) command}
Concatenate CSV files specified.

\subsection*{Format}
mcat [f=] [-skip\_fnf] [-nostop/-skip/-force] 
[\href{run:option.pdf}{-nfn}] 
[\href{run:option.pdf}{-nfno}] 
[\href{run:option.pdf}{-x}] 
[\href{run:option.pdf}{i=}]
[\href{run:option.pdf}{o=}] 
[--help]\\

\subsection*{Parameters}
\begin{table}[htbp]
%\begin{center}
{\small
\begin{tabular}{ll}
\verb [i= ] 	& Input file name \\
& Read multiple CSV files separated by comma delimiter. Wild card characters can be used in the file name. \\
\verb [f= ] 	& Specify the field name(s) to concatenate. \\
& If \emph{f=} is not specified, the field names defaults to the first file defined in the \emph{i=} parameter. \\
\end{tabular}
}
\end{table}

\subsection*{Options}
\begin{table}[htbp]
%\begin{center}
{\small
\begin{tabular}{l p{15cm} l}
\verb [-skip_fnf] 	& If the specified file in  the \emph{i=} parameter does not exist, the program will bypass the error. The program returns an error if all files cannot be found. \\
\verb [-nostop] 	  & \emph{-nostop,-skip,-force} are parameters for controlling exceptions when header is not present. \\
& \emph{-nostop} flag returns null if field name is not specified. When \emph{-nfn} flag is used with \emph{stop} flag, the program terminates if the number of items in the data is different than the parameter defined. \\
\verb [-skip] 		& Files are not concatenated if field name(s) is not specified. \\
& When \emph{-nfn} flag is used with \emph{-skip} flag, the files are not concatenated if the number of data items are different. \\
\verb [-force] 		& Force concatenation of files using location of fields when header is not present. Print output to null if item number is not available. \\
\end{tabular}
}
\end{table}


\subsection*{Note}
\begin{itemize}
\item This command cannot be connected with pipe if file input stream is not available. 
\item The files are concatenated according to the order defined in the \emph{i=} parameter. 
\item Wild card characters ("?" and "*") can be used to read multiple directory and file names. 
\item An input file with 0 bytes returns an error and return bytes with value 0. 
\end{itemize}

\subsection*{Examples}
\subsubsection*{Example 1 Concatenate files with the same header}

\hrule width 8cm
\begin{verbatim}

# dat1.csv
customer,date,amount
A,20081201,10
B,20081002,40

# dat2.csv
customer,date,amount
A,20081207,20
A,20081213,30
B,20081209,50

$ mcat i=dat1.csv,dat2.csv o=rsl.csv

# rsl.csv
customer,date,amount
A,20081201,10
B,20081002,40
A,20081207,20
A,20081213,30
B,20081209,50
\end{verbatim}
\hrule width 8cm

\subsubsection*{Example 2 Concatenate files with different header}

\hrule width 8cm
\begin{verbatim}

# dat1.csv
customer,date,sum total
A,20081201,10
B,20081002,40

# dat2.csv
customer,date,amount
A,20081207,20
A,20081213,30
B,20081209,50

$ mcat -nostop i=dat1.csv,dat2.csv o=rsl.csv

# rsl.csv
customer,date,sum total
A,20081201,10
B,20081002,40
A,20081207,
A,20081213,
B,20081209,
\end{verbatim}
\hrule width 8cm

\subsubsection*{Example 3 Concatenate specific field names from input files}

\hrule width 8cm
\begin{verbatim}

# dat1.csv
customer,date,quantity
A,20081201,3
B,20081002,1

# dat2.csv
customer,date,amount
A,20081207,20
A,20081213,30
B,20081209,50

$ mcat f=customer,date i=dat1.csv,dat2.csv o=rsl1.csv

# rsl1.csv
customer,date
A,20081201
B,20081002
A,20081207
A,20081213
B,20081209

$ mcat f=customer,date,amount i=dat1.csv,dat2.csv o=rsl2.csv

# rsl.csv
customer,date,amount
A,20081201,
B,20081002,
A,20081207,20
A,20081213,30
B,20081209,50

\end{verbatim}
\hrule width 8cm

\subsubsection*{Example 4 Wild card}
This example concatenates all the files with file names starting with "abc". Wild card characters should be enclosed in single quotes to avoid misinterpretation in shell. \\

\hrule width 8cm
\begin{verbatim}
$ mcat i='abc*'
\end{verbatim}
\hrule width 8cm
\leavevmode \\ \\

Concatenates all the files that end it by "abc" or " xyz'' in the following examples.\\

\hrule width 8cm
\begin{verbatim}
$ mcat i='abc*,*xyz'
\end{verbatim}
\hrule width 8cm

\subsection*{Related command}

\hyperlink{msep.pdf}{msep} : Separate data files
%\hypertarget{ht}
%\hyperlink{ht}{bbb}

\end{document}

