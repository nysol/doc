
% how to compile: platex xxx.tex ; dvipdfmx xxx.dvi

\documentclass[fleqn,a4paper]{jarticle}

%--余白の設定
\setlength{\topmargin}{20mm}
\addtolength{\topmargin}{-1in}
\setlength{\oddsidemargin}{20mm}
\addtolength{\oddsidemargin}{-1in}
\setlength{\evensidemargin}{15mm}
\addtolength{\evensidemargin}{-1in}
\setlength{\textwidth}{170mm}
\setlength{\textheight}{254mm}
\setlength{\headsep}{0mm}
\setlength{\headheight}{0mm}
\setlength{\topskip}{0mm}

%--ハイバーリンクを可能にするパッケージ
\usepackage[dvipdfmx,%
 bookmarks=true,%
 bookmarksnumbered=true,%
 colorlinks=true,%
 setpagesize=false,%
 pdftitle={mcal},%
 pdfauthor={BMRC},%
 pdfkeywords={TeX; dvipdfmx; hyperref; color;}]{hyperref}

\begin{document}

\setlength{\baselineskip}{4mm}

\section*{mcal (Calculation of between the items or between the inter-record) command}
Calculate by the formula that is specified in the "c=" parameter, and outputs it to the item name that you specified in the "a =" parameter. The functions include date-time-related, number-related, character string-related, regular expression-related, logic-related and line item-related.

\subsection*{Format}
mcal a= c= [\href{run:option.pdf}{-nfn}] [\href{run:option.pdf}{-nfno}]  [\href{run:option.pdf}{-x}] [\href{run:option.pdf}{precision=}] [\href{run:option.pdf}{i=}] [\href{run:option.pdf}{o=}] [--help]\\

\subsection*{Parameter}
\begin{table}[htbp]
%\begin{center}
{\small
\begin{tabular}{ll}
\verb|a=|    & New item name. This parameter is mandatory.\\
& Specify the name of the item to be added as an output of the calculation results.\\
& At the time of the x,nfn option use, it can be specified it with an item number ( 0... ). \\
\verb|c=|    & Formula. None required.\\
& Specify a formula to be calculated combining of functions arranged.\\
\end{tabular} 
}
\end{table} 

\subsection*{Option}
\begin{table}[htbp]
%\begin{center}
{\small
\begin{tabular}{ll}

\end{tabular} 
}
\end{table} 

\subsection*{Data form}

mcal is a command that can calculate flexibly by combining functions arranged.\\
At that time, it is necessary to be careful on the type of data input and output.\\
For example, to the date value which represents  "August 15, 2008" as "20080815", if 17 days are added, a result will be set to "20080901", but it is because a result will be set to "20080832" if 17 is added as a numerical value.\\
mcal has five types : number, character, date(for example, "20080815"), time (for example, "200808155959") and authenticity
("True" or "False").\\

\subsection*{Constant}
\noindent
Number:20 0.55 1.5*e10\\
Character:"abc" "Japanese"\\
Date:0d20080923 (0dyyyymmdd)\\
Time:0t20080923121115 (0tyyyymmddhhmmss)\\
 (Note) It is possible to be specified by omitting the date. It is complemented by today's date in that case.\\
  0t121115 -complementary-> dtyyyymmddhh20080923121115(yyyymmdd is today's date)\\
Authenticity: 0b1(True) , 0b0(False) \\

\subsection*{Item}
\noindent
Number: \$\{Item name\}\\
Character: \$s\{Item name\}\\
Date: \$d\{Item name\}\\
Time: \$t\{Item name\}\\
 (Note) It is used today's date, in case the value on the data is hhmmss.\\
Authenticity: \$b\{Item name\}\\
 (Note) It is interpreted as the first letter of the item value is "1" to True, "0" to False and others to null.\\

\subsection*{Item of the previous line}
\noindent
It is possible to use\#\{Item name\} in order to obtain the item value of the previous line. It has the following five different types by the data type as well as \$\{Item name\}.\\
Number: \#\{Item name\}\\
Character: \#s\{Item name\}\\
Date: \#d\{Item name\}\\
Time: \#t\{Item name\}\\
Authenticity: \#b\{Item name\}\\
As a specialty item, it is possible to specify a result item of the previous line by omitting an item name. With combining the if function and it is available to cumulative computation.\\
The following example is the cumulative computation of the amount item.\\

mcal c='if(top(),\$\{Amount\},\$\{Amount\}+\#\{\})' a=Cumulative amount
 
\subsection*{Operator}
\noindent
It exists the four arithmetic operations (only addition and subtraction except numerical value), the comparison operation (including the if function) and the logical operation in common on the five data type.\\

\noindent
\href{run:sisoku.pdf}{Four arithmetic operations}\\
\href{run:hikaku.pdf}{Comparison operation}\\
\href{run:ronri.pdf}{Logical operation}\\

\subsection*{Data type conversion}
\noindent
To convert it into a data type for input, it has conversion function as follows :\\
For example, to convert the data of the date type into the character type, it is used as d2s(0d20080815).\\

\noindent
\href{run:katahenkan.pdf}{Type conversion function}\\

\subsection*{Function}
\noindent
The functions for mcal, it is necessary to make a conscious use the type of input and output values.\\
The function is divided into the five items:number-related, character string-related, regular expression-related, date-time-related and line item-related as follows.\\

\noindent
\href{run:sutikanren.pdf}{mcal Number-related}(The function that mainly performs the calculation of regular numerical value  such as sum().)\\
\href{run:sankakukansu.pdf}{mcal Trigonometric function-related}(The function that mainly performs the calculation related trigonometric function such as sin()).\\
\href{run:mojiretu.pdf}{mcal Character string-related}(The functions that mainly perform calculations related to the character-string manipulation such as extraction of a substring.)\\
\href{run:seikihyougen.pdf}{mcal Regular expression-related}(The function that perform calculations related to match with a regular expression.)\\
\href{run:hizuke.pdf}{mcal Date-time-related}(The function that perform calculations of the type of date-time such as among the dates.)\\
\href{run:gyoukoumoku.pdf}{mcal Line item-related}(The function that perform calculations related to the three different lines such as top(),line(),fldcnt())\\

\subsection*{Note}
\noindent
In case of using a shell such as bash the OS of UNIX system, it is essential to be enclosed in single quotation. Otherwise, those symbols are being interpreted to the shell.\\
In case of specifying an item, it is required to be enclosed the item name in curly brackets with the symbol of "\$d" which specifies the data type to the top.(Example:\$d\{date\})\\
To specify it by a regular expression, it is necessary to use an asterisk such as \$s\{date*\}.\\

\subsection*{Error message}
\noindent
  \begin{itemize}
   \item unknown function or operator\\
    
  \end{itemize}
When this error message comes out, there is a mistake in the specification of designation of the operator or function.
For example, consider the error message for associative function(cat) of character string as follows.
\begin{verbatim}   
 ./mcal c='cat("-",1,2)'
ERROR : unknown function or operator: cat_SNN(cat_SN) (kgcal)
\end{verbatim}
\noindent
cat in front of the underscore of "cat\_SNN" indicates the function name, subsequent SNN shows the type of the argument.
As for S is a type of strings. N is of the number. D is of date. T is of time. B is of boolean. Since three arguments are specified, they are three characters (SNN).The error message means "the function called cat that takes the SNN as arguments " has not been registered. At that time, the second and third arguments turn to a character string type as follows. \begin{verbatim}   
 ./mcal c='cat("-","1","2")'
 \end{verbatim}   
\noindent
ex)
\begin{verbatim}   
$ mcal c='cat(${productID},${price},"-")' a=productID-price i=dat.csv o=rsl1.csv
\end{verbatim}


\subsection*{related command}


\end{document}
