
% how to compile: platex xxx.tex ; dvipdfmx xxx.dvi

\documentclass[a4paper]{jarticle}

%--余白の設定
\setlength{\topmargin}{20mm}
\addtolength{\topmargin}{-1in}
\setlength{\oddsidemargin}{20mm}
\addtolength{\oddsidemargin}{-1in}
\setlength{\evensidemargin}{15mm}
\addtolength{\evensidemargin}{-1in}
\setlength{\textwidth}{170mm}
\setlength{\textheight}{254mm}
\setlength{\headsep}{0mm}
\setlength{\headheight}{0mm}
\setlength{\topskip}{0mm}

%--ハイバーリンクを可能にするパッケージ
\usepackage[dvipdfmx,%
 bookmarks=true,%
 bookmarksnumbered=true,%
 colorlinks=true,%
 setpagesize=false,%
 pdftitle={mcat},%
 pdfauthor={BMRC},%
 pdfkeywords={TeX; dvipdfmx; hyperref; color;}]{hyperref}

\begin{document}

\renewcommand{\tablename}{Table }

\setlength{\baselineskip}{4mm}

\section*{mvcommon select common elements of vector}
Select common elements in reference file from vector. 
See examples in Table \ref{tbl:input}〜\ref{tbl:out3}.

\begin{table}[htbp]
\begin{center}
\begin{tabular}{ccc}

\begin{minipage}{0.3\hsize}
\begin{center}
\caption{Input data\label{tbl:input}}
\verb|in.csv| \\
{\small
\begin{tabular}{cl}
\hline
no&items \\
\hline
1&a b c \\
2&a d \\
3&b f e f \\
4&f c d \\
\hline

\end{tabular}
}
\end{center}
\end{minipage}

\begin{minipage}{0.3\hsize}
\begin{center}
\caption{Reference file\label{tbl:ref}}
\verb|ref.csv| \\
{\small
\begin{tabular}{c}
\hline
item \\
\hline
a \\
c \\
e \\
\hline
\end{tabular}
}
\end{center}
\end{minipage}
\end{tabular}
\end{center}
\end{table}

\begin{table}[htbp]
\begin{center}
\begin{tabular}{ccc}

\begin{minipage}{0.5\hsize}
\begin{center}
\caption{Basic example\label{tbl:out2}}
\verb|vf=items m=ref.csv K=item| \\
{\small
\begin{tabular}{ll}
\hline
no&items \\
\hline
1&a c \\
2&a \\
3&e \\
4&c \\
\hline

\end{tabular}
}
\end{center}
\end{minipage}

\begin{minipage}{0.50\hsize}
\begin{center}
\caption{Selection of unmatched items. \label{tbl:out3}}
\verb|vf=items m=ref.csv K=item -r| \\
{\small
\begin{tabular}{ll}
\hline
no&items \\
\hline
1&b \\
2&d \\
3&b f f \\
4&f d \\
\hline
\end{tabular}
}
\end{center}
\end{minipage}

\end{tabular}
\end{center}
\end{table}

 Take note that huge reference file may use massive amount of memory, since the \verb|mvcommon| command reads the whole reference file at once in memory.


\subsection*{Format}
\verb|mvjoin vf= [K=] m= [-r]|
[\href{run:delim.pdf}{delim=}]
[\href{run:input.pdf}{i=}]
[\href{run:output.pdf}{o=}]
[\href{run:nfn.pdf}{-nfn}]
[\href{run:nfno.pdf}{-nfno}]
[\href{run:x.pdf}{-x}] [--help]

\begin{table}[htbp]
%\begin{center}
{\small
\begin{tabular}{ll}

\verb|vf=| & Specify the field name of vector (from \emph{i=} input file) as join key. 【required parameter】\\
           & Multiple fields can be specified. Sorting on vectors is not required. \\
           & Output of vector series can be defined with ':' followed after the vector name. \\
\verb|m=|  & Reference file. 【required parameter】 \\
\verb|K=|  &  Join key in reference file (\verb|m=|) where corresponding taxonomy elements are joined to the vector. 【required parameter】\\
\verb|-r|  & Select records where key elements that do no match in \emph{vf=} and \emph{K=} .   \\
\end{tabular}
}
\end{table} 

\subsection*{Examples}
\subsubsection*{Example 1 Match common elements in multiple vectors }
\begin{verbatim}
------------------------------------------------
# dat1.csv
items1,items2
b a c,b b
c c,a d
e a a,a a

# ref.csv
item
a
c
e

$ mvcommon vf=items1,items2 K=item m=ref.csv i=dat1.csv o=ref.csv

# rsl1.csv
items1,items2
a c,
c c,a
e a a,a a
------------------------------------------------
\end{verbatim}

\subsubsection*{Example 2 Print outputs in new column }
Define new column name  \verb|new2| for \verb|item2|. 
\begin{verbatim}
------------------------------------------------
# dat1.csv
items1,items2
b a c,b b
c c,a d
e a a,a a

# ref.csv
item
a
c
e

$ mvcommon vf=items1,items2:new2 K=item m=ref.csv i=dat1.csv o=rsl1.csv

# rsl2.csv
items1,items2,new2
a c,b b,
c c,a d,a
e a a,a a,a a
------------------------------------------------
\end{verbatim}


\subsection*{related command}
\begin{itemize}
\item \href{run:mvjoin.pdf}{mvjoin}: Join reference element to vector
\end{itemize}

\end{document}
