
% how to compile: platex xxx.tex ; dvipdfmx xxx.dvi

\documentclass[a4paper]{jarticle}

%--余白の設定
\setlength{\topmargin}{20mm}
\addtolength{\topmargin}{-1in}
\setlength{\oddsidemargin}{20mm}
\addtolength{\oddsidemargin}{-1in}
\setlength{\evensidemargin}{15mm}
\addtolength{\evensidemargin}{-1in}
\setlength{\textwidth}{170mm}
\setlength{\textheight}{254mm}
\setlength{\headsep}{0mm}
\setlength{\headheight}{0mm}
\setlength{\topskip}{0mm}

%--ハイバーリンクを可能にするパッケージ
\usepackage[dvipdfmx,%
 bookmarks=true,%
 bookmarksnumbered=true,%
 colorlinks=true,%
 setpagesize=false,%
 pdftitle={mcat},%
 pdfauthor={BMRC},%
 pdfkeywords={TeX; dvipdfmx; hyperref; color;}]{hyperref}

\begin{document}

\renewcommand{\tablename}{Table }

\setlength{\baselineskip}{4mm}

\section*{mvreplace replace an element in the vector }
Replace vector data with corresponding taxonomy character in the reference file joined by key. Table \ref{tbl:input} shows items in a vector with sequential elements separated by a space. The examples are highlighted in Table \ref{tbl:input}〜\ref{tbl:out3}.

\begin{table}[htbp]
\begin{center}
\begin{tabular}{ccc}

\begin{minipage}{0.3\hsize}
\begin{center}
\caption{Input data\label{tbl:input}}
\verb|in.csv| \\
{\small
\begin{tabular}{cl}
\hline
no&items \\
\hline
1&a b c \\
2&a d \\
3&b f e f \\
4&f c d \\
\hline

\end{tabular}
}
\end{center}
\end{minipage}

\begin{minipage}{0.3\hsize}
\begin{center}
\caption{Reference file\label{tbl:ref}}
\verb|ref.csv| \\
{\small
\begin{tabular}{cc}
\hline
item&taxo \\
\hline
a&X \\
b&Y \\
c&Z \\
e&X \\
f&Z \\
\hline
\end{tabular}
}
\end{center}
\end{minipage}
\end{tabular}
\end{center}
\end{table}

\begin{table}[htbp]
\begin{center}
\begin{tabular}{ccc}

\begin{minipage}{0.5\hsize}
\begin{center}
\caption{Basic example\label{tbl:out2}}
\verb|vf=items m=ref.csv K=item f=taxo| \\
{\small
\begin{tabular}{ll}
\hline
no&items \\
\hline
1&X Y Z \\
2&X d \\
3&Y Z X Z \\
4&Z Z d \\
\hline

\end{tabular}
}
\end{center}
\end{minipage}

\begin{minipage}{0.50\hsize}
\begin{center}
\caption{Define unmatched elements 
\label{tbl:out3}}
\verb|vf=items m=ref.csv K=item f=taxo n=* | \\
{\small
\begin{tabular}{ll}
\hline
no&items \\
\hline
1&X Y Z \\
2&X * \\
3&Y Z X Z \\
4&Z Z * \\
\hline
\end{tabular}
}
\end{center}
\end{minipage}

\end{tabular}
\end{center}
\end{table}

 Take note that huge reference file may use massive amount of memory, since the \verb|mvreplace| command reads the whole reference file at once in memory.


\subsection*{Format}
\verb|mvjoin vf= [K=] m= f= [n=]|
[\href{run:delim.pdf}{delim=}]
[\href{run:input.pdf}{i=}]
[\href{run:output.pdf}{o=}]
[\href{run:nfn.pdf}{-nfn}]
[\href{run:nfno.pdf}{-nfno}]
[\href{run:x.pdf}{-x}] [--help]

\begin{table}[htbp]
%\begin{center}
{\small
\begin{tabular}{ll}
\verb|vf=| & Specify the field name of vector as join key (from the input file specified in the \emph{i=} parameter) is specified. 【required parameter】\\
           & Multiple fields can be specified. Sorting on vectors is not required. \\
\verb|m=|  & Reference file  \\
\verb|K=|  &  Key field in reference file where corresponding taxonomy elements are joined with vector items in defined in \emph{vf=}. 【required parameter】\\
           & The sequence of the vector is unique, sorting is not required. \\
           & The output may differ if the string sequence is not unique.  \\ 
\verb|f=|  & Field name of vector to join with reference file. 【required parameter】\\
\verb|n=|  & Specify the replacement character when the elements that do not match in \emph{vf=} and \emph{K=} . \\
& The element will not be joined with the reference file when this option not specified.  \\
\end{tabular}
}
\end{table} 

\subsection*{Examples}
\subsubsection*{Example 1 Replace elements in a vector}
\begin{verbatim}
------------------------------------------------
# dat1.csv
items
b a c
c c
e a a

# ref.csv
item,taxo
a,X Y
b,X
c,Z Z

$ mvreplace vf=items m=ref.csv K=item f=taxo i=dat1.csv o=rsl1.csv 

# rsl1.csv
items
X X Y Z Z
Z Z Z Z
e X Y X Y
------------------------------------------------
\end{verbatim}

\subsubsection*{Example 2 Replace character on multiple elements}
\begin{verbatim}
------------------------------------------------
# dat1.csv
items1,items2
b a c,b b
c c,a d
e a a,a a

# ref.csv
item,taxo
a,X
b,X
c,Y
d,Y

$ mvreplace vf=items1,items2 m=ref.csv K=item f=taxo i=dat1.csv o=rsl1.csv

# rsl1.csv
items1,items2
X X Y,X X
Y Y,X Y
e X X,X X
------------------------------------------------
\end{verbatim}

\subsection*{related command}
\begin{itemize}
\item \href{run:mvjoin.pdf}{mvjoin}: join reference of vector element
\end{itemize}

\end{document}
