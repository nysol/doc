
% how to compile: platex xxx.tex ; dvipdfmx xxx.dvi

\documentclass[a4paper]{jarticle}

%--余白の設定
\setlength{\topmargin}{20mm}
\addtolength{\topmargin}{-1in}
\setlength{\oddsidemargin}{20mm}
\addtolength{\oddsidemargin}{-1in}
\setlength{\evensidemargin}{15mm}
\addtolength{\evensidemargin}{-1in}
\setlength{\textwidth}{170mm}
\setlength{\textheight}{254mm}
\setlength{\headsep}{0mm}
\setlength{\headheight}{0mm}
\setlength{\topskip}{0mm}

%--ハイバーリンクを可能にするパッケージ
\usepackage[dvipdfmx,%
 bookmarks=true,%
 bookmarksnumbered=true,%
 colorlinks=true,%
 setpagesize=false,%
 pdftitle={mcat},%
 pdfauthor={BMRC},%
 pdfkeywords={TeX; dvipdfmx; hyperref; color;}]{hyperref}

\begin{document}

\setlength{\baselineskip}{4mm}

\section*{mcut (select column) command}
Extract specified record(s) from the CSV file. 

\subsection*{Format}
mcut [\href{run:i.pdf}{i=}] f= [-r] [\href{run:nfn.pdf}{-nfn}] [-nfni] [\href{run:nfno.pdf}{-nfno}] [\href{run:x.pdf}{-x}] [\href{run:o.pdf}{o=}] [--help]\\

\subsection*{Parameters}
\begin{table}[htbp]
%\begin{center}
{\small
\begin{tabular}{ll}
\verb|f=|    & Define name of column to be extracted from file  【required parameter】 \\
\end{tabular} 
}
\end{table} 

\subsection*{Options}
\begin{table}[htbp]
%\begin{center}
{\small
\begin{tabular}{l p{15cm} l}
\verb|-r|    & Reverse selection of field(s) \\
& Remove all columns defined in the \emph{f=} parameter from the data.\\
\verb|-nfn|  & Data without header \\
&  Use when header is not present in the first row of data.\\
&(Primarily used when the data do not have header / field names(s) in the first row). \\
& Position number of column is used to identify specific field(s) since field name is not available, and the output file will not include header.\\
& The same applies to both options of -nfni and -nfno shown below.\\
& (nfn stands for no field names)\\
\verb|-nfni| &  Data without header \\
& Use when header is not present in the first row of data. As such, position number of column is used to identify specific field(s). \\
& New column name(s) for each column can be specified in the output file. \\
& Example f=0:date,2:store,3:quantity \\
\verb|-nfno| & Data without header\\
& Header in the input file will be removed in the output file. \\
\end{tabular} 
}
\end{table} 

\subsection*{Examples}
\subsubsection*{Example 1 Extract customer and amount information from the data file dat1.csv}

\begin{verbatim}
------------------------------------------------
# input file(dat1.csv)
customer,quantity,amount
A,1,10
A,2,20
B,1,15
B,3,10
B,1,20

$ mcut f=customer,amount i=dat1.csv o=rsl1.csv

# output file(rsl1.csv)
customer,amount
A,10
A,20
B,15
B,10
B,20
------------------------------------------------
\end{verbatim}

\subsubsection*{Example 2 Remove customer and amount information from the data - dat1.csv}

\begin{verbatim}
------------------------------------------------
# input file(dat1.csv)
customer,quantity,amount
A,1,10
A,2,20
B,1,15
B,3,10
B,1,20

$ mcut f=customer,amount -r i=dat1.csv o=rsl2.csv

# output file(rsl2.csv)
quantity
1
2
1
3
1
\end{verbatim}


\subsubsection*{Example 3 Select columns at 0,2 from an input file without header (Add customer and amount in the header of the output file) }

\begin{verbatim}
------------------------------------------------
# input file(dat2.csv)
A,1,10
A,2,20
B,1,15
B,3,10
B,1,20

$ mcut f=0:new customer,2:new amount -nfni i=dat2.csv o=rsl3.csv

# output file(rsl3.csv)
new customer,new amount
A,10
A,20
B,15
B,10
B,20
------------------------------------------------
\end{verbatim}

\subsubsection*{Example 4 Select columns at 0,2 from an input file without header (Exclude header in the output file)}

\begin{verbatim}
------------------------------------------------
# input file(dat2.csv)
A,1,10
A,2,20
B,1,15
B,3,10
B,1,20

All synonymous below
$ mcut f=0,2 -x          i=dat2.csv o=rslr4.csv
$ mcut f=0,2 -nfn        i=dat2.csv o=rslr4.csv
$ mcut f=0,2 -nfni -nfno i=dat2.csv o=rslr4.csv

# output file(rsl4.csv)
A,10
A,20
B,15
B,10
B,20
------------------------------------------------
\end{verbatim}

\subsection*{related command}

\href{run:msubstr.pdf}{msubstr}:extraction of substring
\end{document}
