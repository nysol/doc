

% how to compile: platex xxx.tex ; dvipdfmx xxx.dvi

\documentclass[a4paper]{jarticle}

%--余白の設定
\setlength{\topmargin}{20mm}
\addtolength{\topmargin}{-1in}
\setlength{\oddsidemargin}{20mm}
\addtolength{\oddsidemargin}{-1in}
\setlength{\evensidemargin}{15mm}
\addtolength{\evensidemargin}{-1in}
\setlength{\textwidth}{170mm}
\setlength{\textheight}{254mm}
\setlength{\headsep}{0mm}
\setlength{\headheight}{0mm}
\setlength{\topskip}{0mm}

%--ハイバーリンクを可能にするパッケージ
\usepackage[dvipdfmx,%
 bookmarks=true,%
 bookmarksnumbered=true,%
 colorlinks=true,%
 setpagesize=false,%
 pdftitle={mhashavg},%
 pdfauthor={BMRC},%
 pdfkeywords={TeX; dvipdfmx; hyperref; color;}]{hyperref}

\begin{document}

\setlength{\baselineskip}{4mm}

\section*{mhashavg (compute average with hash function) command}
Use hash function to calculate the average of data series specified at \emph{f=} parameter based on the key defined at \emph{k=}. 

The processing speed of this command is faster than \verb|mavg| without having to sort by key fields before the command. However, variation in length of keys (different length of strings in field) will slow down the processing speed. 

\subsection*{Format}
mhashavg f= [hs=] [k=] [-n] 
[\href{run:option.pdf}{-nfn}] 
[\href{run:option.pdf}{-nfno}] 
[\href{run:option.pdf}{-x}] 
[\href{run:option.pdf}{precision=}] 
[\href{run:option.pdf}{i=}] 
[\href{run:option.pdf}{o=}] 
[--help]\\

\subsection*{Parameters}
\begin{table}[htbp]
%\begin{center}
{\small
\begin{tabular}{ll}
\verb|f=|    & Field name (Multiple fields can be specified.)【required parameter】\\
& Calculate average of the column specified. \\
& Rename the field header by specifying the new field name after ":". Example f=Quantity:AverageQuantity. \\
\verb|k=|    & Key item (Multiple keys can be specified.)\\
& Calculate the average on the data series based on the key field(s) defined. \\
& ※Note:Unlike \href{run:mavg.pdf} {mavg}, \verb|mhashavg| does not require sorting on the key field \emph{k=} in advance.  \\
& Use x,nfn option to specify an item number ( 0... ).\\
\verb|hs=|    & Hash size \\
& Specify the hash size (Default: 199999).\\
& Refer to \href{run:mhashsum.pdf}{mhashsum} for more information.
\end{tabular} 
}
\end{table} 

\subsection*{Option}
\begin{table}[htbp]
%\begin{center}
{\small
\begin{tabular}{ll}
\verb|-n|    & Print output as null if there are null values in \emph{f=}. \\
\end{tabular} 
}
\end{table} 

\subsection*{Examples}
\subsubsection*{Example 1 Calculate the average quantity and average amount for each customer.}

\begin{verbatim}
------------------------------------------------
# Input file (dat.csv)
Customer,Quantity,Amount
A,1,
B,,15
A,2,20
B,3,10
B,1,20

$ mhashavg k="Customer" f="Quantity","Amount" i=dat.csv o=rsl.csv

# OUtput file (rsl.csv)
Customer,Quantity,Amount
A,1.5,20
B,2,15
------------------------------------------------
\end{verbatim}

\subsubsection*{Example 2 Calculate the average quantity and average amount for each customer. Calculated output is null if there is a null value in quantity and amount. Use \emph{-n} option to print the null value. 
}
\begin{verbatim}
------------------------------------------------
# Input file (dat.csv)
Customer,Quantity,Amount
A,1,
B,,15
A,2,20
B,3,10
B,1,20

$ mhashavg k="Customer" f="Quantity","Amount" -n i=dat.csv o=rsl.csv

# Output file (rsl.csv)
Customer, Quantity, Amount
A,1.5,
B,,15
------------------------------------------------
\end{verbatim}

\subsection*{Remarks}
Refer to the section on benchmark at \href{run mhashsum.pdf} {mhashsum} to find out more on processing speed. 

\subsection*{Related commands}
\noindent
\href{run:mavg.pdf}{mavg}: Compute average \\
\href{run:mhashsum.pdf}{mhashsum}: Compute hash total value  \\
\href{run:mstat.pdf}{mstat}: Compute statistics of one variable\\

\end{document}
