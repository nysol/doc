
% how to compile: platex xxx.tex ; dvipdfmx xxx.dvi

\documentclass[a4paper]{jarticle}

%--余白の設定
\setlength{\topmargin}{20mm}
\addtolength{\topmargin}{-1in}
\setlength{\oddsidemargin}{20mm}
\addtolength{\oddsidemargin}{-1in}
\setlength{\evensidemargin}{15mm}
\addtolength{\evensidemargin}{-1in}
\setlength{\textwidth}{170mm}
\setlength{\textheight}{254mm}
\setlength{\headsep}{0mm}
\setlength{\headheight}{0mm}
\setlength{\topskip}{0mm}

%--ハイバーリンクを可能にするパッケージ
\usepackage[dvipdfmx,%
 bookmarks=true,%
 bookmarksnumbered=true,%
 colorlinks=true,%
 setpagesize=false,%
 pdftitle={mcross},%
 pdfauthor={BMRC},%
 pdfkeywords={TeX; dvipdfmx; hyperref; color;}]{hyperref}

\begin{document}

\setlength{\baselineskip}{4mm}

\section*{mcross (transpose columns to rows) command}
Transpose fields names specified by \emph{s=} parameter horizontally and itemize the array of data specified at the \emph{f=} parameter accordingly. 

\subsection*{Format}
mcross f= s= [a=] [k=] [v=] [-n]    
[\href{run:option.pdf}{-nfn}] 
[\href{run:option.pdf}{-nfno}] 
[\href{run:option.pdf}{-x}] 
[\href{run:option.pdf}{i=}] 
[\href{run:option.pdf}{o=}] 
[--help]\\

\subsection*{Parameter}
\begin{table}[htbp]
%\begin{center}
{\small
\begin{tabular}{ll}
\verb|f=|    & Field name. 【required parameter】\\
& The data series in this column will be transposed horizontally to the corresponding key. \\
& Specify the fields names of the new rows transposed at the \emph{s=} parameter. \\
& Use \emph{a=} parameter to rename the field. \\
\verb|s=|    & Field names of transposed data. 【required parameter】 \\
& The data series in the field becomes column name of transposed rows.\\
& Thus, it is important that the value of the field specified here is unique.\\
& Use \emph{r} option to specify multiple field names. \\
& \emph{fld} will be printed before the transposed rows of field names. \\
& \emph{fld} can be renamed using a ":" followed by the item name. \\
& Example  s=2008*:Date"\&"\\
\verb|a=|    & New field name. 【required parameter】\\
& Rename field name.  The field name is set to \emph{fld} by default if this parameter is not used. \\
\verb|k=|    & Key field name. 【required parameter】\\
& Key to transpose data into horizontal rows. \\
\verb|v=|    & Replace null character.  【optional parameter】\\
& Replace null character with a character defined at \emph{v=} parameter. \\
\end{tabular} 
}
\end{table} 

\subsection*{Option}
\begin{table}[htbp]
%\begin{center}
{\small
\begin{tabular}{ll}
\end{tabular} 
}
\end{table} 

\subsection*{Condition for Sorting }
When \emph{k=} is specified, the array of data specified should be sorted beforehand. Sorting can be done in numerical or reverse order, there is no particular order on how the the array is sorted. 

\subsection*{Examples}
\subsubsection*{Example 1 Transpose the array of \emph{date} horizontally and itemize quantity and price to the corresponding date.
}
(Note) The field to be transposed into rows is sorted by product \emph{item}. 

\begin{verbatim}
------------------------------------------------
# Input file (dat.csv)
item,date,quantity,price
A,20081201,1,10
A,20081202,2,20
A,20081203,3,30
B,20081201,4,40
B,20081203,5,50

$ mcross k=item f=quantity,price s=date i=dat.csv o=rsl.csv

# Output file (rsl.csv)
item,fld,20081201,20081202,20081203
A,quantity,1,2,3
A,price,10,20,30
B,quantity,4,,5
B,price,40,,50
------------------------------------------------
\end{verbatim}

\subsubsection*{Example 2  Restore the output from Example 1 to the original input data with  \emph{mcross}. Use \emph{a=} parameter to change the field name. 
The transaction record at row 2008/12/02 for product b do not exist, does null values are included when the data is restored. Use \emph{mdelnull} to delete row with null values. 
}

\begin{verbatim}
------------------------------------------------
# Input file (dat.csv)
item,fld,20081201,20081202,20081203
A,quantity,1,2,3
A,price,10,20,30
B,quantity,4,,5
B,price,40,,50

$ mcross k=item f=2008* s=fld a=date i=rsl.csv o=dat.csv

# Output file (rsl.csv)
item,date,quantity,price
A,20081201,1,10
A,20081202,2,20
A,20081203,3,30
B,20081201,4,40
B,20081202,,
B,20081203,5,50
------------------------------------------------
\end{verbatim}

\subsubsection*{Example 3 Transpose the rows according to date in reverse order starting with end date. 
}

\begin{verbatim}
------------------------------------------------
# Input file (dat.csv)
item,date,quantity,price
A,20081201,1,10
A,20081202,2,20
A,20081203,3,30
B,20081201,4,40
B,20081203,5,50

$ mcross k=item f=quantity,price s=date%r i=dat.csv o=rsl.csv

# Output file (rsl.csv)
item,fld,20081203,20081202,20081201
A,quantity,3,2,1
A,price,30,20,10
B,quantity,5,,4
B,price,50,,40
------------------------------------------------
\end{verbatim}

\subsection*{Related command}
\noindent
\href{run:mtra.pdf}{mtra}: Convert vertical data to a transaction item\\
\href{run:mtrafld.pdf}{mtrafld}: Convert cross tabulation to a transaction item(Convert item and value pairs to an item)\\
\href{run:mtraflg.pdf}{mtraflg}: Convert cross tabulation to a transaction item(Convert item name to a data value)\\
\end{document}
