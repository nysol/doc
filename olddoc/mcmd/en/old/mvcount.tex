
% how to compile: platex xxx.tex ; dvipdfmx xxx.dvi

\documentclass[a4paper]{jarticle}

%--余白の設定
\setlength{\topmargin}{20mm}
\addtolength{\topmargin}{-1in}
\setlength{\oddsidemargin}{20mm}
\addtolength{\oddsidemargin}{-1in}
\setlength{\evensidemargin}{15mm}
\addtolength{\evensidemargin}{-1in}
\setlength{\textwidth}{170mm}
\setlength{\textheight}{254mm}
\setlength{\headsep}{0mm}
\setlength{\headheight}{0mm}
\setlength{\topskip}{0mm}

%--ハイバーリンクを可能にするパッケージ
\usepackage[dvipdfmx,%
 bookmarks=true,%
 bookmarksnumbered=true,%
 colorlinks=true,%
 setpagesize=false,%
 pdftitle={mcat},%
 pdfauthor={BMRC},%
 pdfkeywords={TeX; dvipdfmx; hyperref; color;}]{hyperref}

\begin{document}

\renewcommand{\tablename}{table }

\setlength{\baselineskip}{4mm}

\section*{mvcount calculate vector size}
Calculate the size of vector (number of elements in a vector) . 
An example is shown in Table \ref{tbl:input}〜\ref{tbl:out1}.

\begin{table}[htbp]
\begin{center}
\begin{tabular}{ccc}

\begin{minipage}{0.3\hsize}
\begin{center}
\caption{Input data\label{tbl:input}}
\verb|in.csv| \\
{\small
\begin{tabular}{cl}
\hline
no&items \\
\hline
1&a b c \\
2&a d \\
3&b f e f \\
4& \\
\hline

\end{tabular}
}
\end{center}
\end{minipage}

\begin{minipage}{0.5\hsize}
\begin{center}
\caption{Basic example\label{tbl:out1}}
\verb|mvcount vf=items:size i=in.csv| \\
{\small
\begin{tabular}{lll}
\hline
no&items&size \\
\hline
1&a b c&3 \\
2&a d&2 \\
3&b f e f&4 \\
4& &0\\
\hline

\end{tabular}
}
\end{center}
\end{minipage}

\end{tabular}
\end{center}
\end{table}

\subsection*{Format}
\verb|mvcount vf=|
[\href{run:delim.pdf}{delim=}]
[\href{run:input.pdf}{i=}]
[\href{run:output.pdf}{o=}]
[\href{run:nfn.pdf}{-nfn}]
[\href{run:nfno.pdf}{-nfno}]
[\href{run:x.pdf}{-x}] [--help]

\begin{table}[htbp]
%\begin{center}
{\small
\begin{tabular}{ll}
\verb|vf=| & Specify the field names (from the input file i=) of vectors for calculation  【required parameter】\\
           & Field name(s) of result(s) can be defined with ':' followed after the vector name. \\
           & Multiple vectors can be specified.\\
\end{tabular}
}
\end{table} 

\subsection*{Example}
\subsubsection*{Example 1 Counting multiple vectors }
\begin{verbatim}
------------------------------------------------
# dat1.csv
items1,items2
b a c,b
c c,
e a a,a a a

$ mvcount vf=items1:size1,items2:size2 i=dat1.csv o=rsl1.csv

# rsl1.csv
items1,items2,size1,size2
b a c,b,3,1
c c,,2,0
e a a,a a a,3,3
------------------------------------------------
\end{verbatim}

\subsection*{Related command}
%\begin{itemize}
%\item \href{run:mvjoin.pdf}{mvjoin}:reference combination of a vector element
%\end{itemize}

\end{document}
