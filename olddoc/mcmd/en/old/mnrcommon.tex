
% how to compile: platex xxx.tex ; dvipdfmx xxx.dvi

\documentclass[a4paper]{jarticle}

%--余白の設定
\setlength{\topmargin}{20mm}
\addtolength{\topmargin}{-1in}
\setlength{\oddsidemargin}{20mm}
\addtolength{\oddsidemargin}{-1in}
\setlength{\evensidemargin}{15mm}
\addtolength{\evensidemargin}{-1in}
\setlength{\textwidth}{170mm}
\setlength{\textheight}{254mm}
\setlength{\headsep}{0mm}
\setlength{\headheight}{0mm}
\setlength{\topskip}{0mm}

%--ハイバーリンクを可能にするパッケージ
\usepackage[dvipdfmx,%
 bookmarks=true,%
 bookmarksnumbered=true,%
 colorlinks=true,%
 setpagesize=false,%
 pdftitle={mnrcommon},%
 pdfauthor={BMRC},%
 pdfkeywords={TeX; dvipdfmx; hyperref; color;}]{hyperref}

\begin{document}

\setlength{\baselineskip}{4mm}

\section*{mnrcommon (select records within specified range(s) from reference file) command}
Select the record in the input file that matches the records within the defined range(s) defined from the reference file.
 \emph{k=} parameter specifies the key field name from the input file to match with the key defined in \emph{K=} in the reference file. The selection criteria is based on the data series from the input file defined in \emph{v=} parameter for records that falls within the data range in the reference file defined in the \emph{R= } parameter. 
Add \verb|%n| after the item name if the field defined at \emph{v=} parameter is a numerical value.  

\subsection*{Format}
mnrcommon  k= m= R= v= [K=] [u=] [-r] 
[\href{run:option.pdf}{-nfn}] 
[\href{run:option.pdf}{-nfno}] 
[\href{run:option.pdf}{-x}] 
[\href{run:option.pdf}{i=}] 
[\href{run:option.pdf}{o=}] 
[--help]\\

\subsection*{Parameters}
\begin{table}[htbp]
%\begin{center}
{\small
\begin{tabular}{ll}
\verb|k=|    & Key item(s) to match in the input file. (Multiple keys can be specified)【required parameter】\\
& The key(s) specified will be matched with the key field(s) defined in \emph{K=} parameter from the reference file. \\
& Sort data by key field(s) as specified in the "k =" parameter prior to the command.  \\
& Use the x,nfn option to pecify item number (0〜). \\

\verb|m=|    & Reference file【required parameter】\\
& Specify name of reference file. \\
\verb|R=|    & Values of data range 【required parameter】\\
& Define start and end values of data range in reference file. \\
& If the start value of the data range is null, the end value will be treated as null. \\
\verb|v=|    & Field name of data series for matching [\%{n}] 【required parameter】\\
&  Records in the input file that matches the key field specified in the \emph{k=} parameter in the reference data is selected.  \\
& This parameter specifies the field name for matching based on the key field.  \\
\verb|K=|    &  Key field(s) in the reference data (Multiple keys can be specified). 【optional parameter】\\
& The key specified will be matched with the key field defined in \emph{k=} parameter from the input file.  \\
& Records in the input file that matches the key field specified in the \emph{k=} parameter in the reference data is selected. \\
& ※Note: Sort data by key field(s) specified in the "K=" parameter prior to the command. \\
& Use x,nfn option to specify the item number (0...). \\
\verb|u=|    & Output file containing unmatched records. 【optional parameter】\\
&  Write the unmatched records to this output file.  \\
\end{tabular} 
}
\end{table} 

\subsection*{Option}
\begin{table}[htbp]
%\begin{center}
{\small
\begin{tabular}{ll}
\verb|-r|  & Reverse selection\\
& Select records that is not within the data range defined in the \emph{R=} parameter. \\
\end{tabular} 
}
\end{table} 

\subsection*{Examples}
\subsubsection*{Example 1 Select records where the transaction date is 20080203 with transaction amount [greater than 10 and less than 35] or [greater than 5 and less than 40].   \\Print all records that fall outside the defined range to a separate file oth.csv. }
\noindent
※Note: Input file and reference file should be sorted according to date beforehand. 

\begin{verbatim}
------------------------------------------------
# Input file (dat.csv)
Date,Quantity,Amount
20080123,1,10
20080123,2,20
20080203,1,10
20080203,3,35
200804l0,1,50

# Reference file (ref.csv)
Date,AmountF,AmountT,AverageAmount
20080203,10,35,25
20080203,5,40,35

$ mnrcommon k=Date m=ref.csv R=AmountF,AmountT v=Amount%n u=oth.csv i=dat.csv o=rsl.csv

# Output file(rsl.csv)
Date,Quantity,Amount
20080203,1,10
20080203,3,35

#Output file2 (oth.csv)
Date,Quantity,Amount
20080123,1,10
20080123,2,20
200804l0,1,50

------------------------------------------------
\end{verbatim}

\subsubsection*{Example 2 Select records where the transaction date is 20080203 with transaction amount greater than 10 and less than 35, or records outside the range that is above 5 and below 40. }
\noindent
※Note:Input file and reference file should be sorted according to date beforehand. 

\begin{verbatim}
------------------------------------------------
# Input file (dat.csv)
Date,Quantity,Amount
20080123,1,10
20080123,2,20
20080203,1,10
20080203,3,35
200804l0,1,50

# Reference file(ref.csv)
Date,AmountF,AmountT,AverageAmount
20080203,10,35,25
20080203,5,40,35

$ mnrcommon k=Date m=ref.csv R=AmountF,AmountT v=Amount%n -r i=dat.csv o=rsl2.csv

# Output file1(rsl2.csv)
Date,Quantity,Amount
20080123,1,10
20080123,2,20
200804l0,1,50
------------------------------------------------
\end{verbatim}

\subsection*{Related commands}
\noindent
\href{run:mcommon.pdf}{mcommon}:Select common records in reference file\\
\href{run:mnrjoin.pdf}{mnrjoin}:Natural join data from the reference file with multiple ranges. 参照ファイルの複数範囲条件による自然結合\\
\href{run:mrjoin.pdf}{mrjoin}:Join data from reference file within data range \\


\end{document}
