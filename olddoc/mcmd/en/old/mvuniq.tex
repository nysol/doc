
% how to compile: platex xxx.tex ; dvipdfmx xxx.dvi

\documentclass[a4paper]{jarticle}

%--余白の設定
\setlength{\topmargin}{20mm}
\addtolength{\topmargin}{-1in}
\setlength{\oddsidemargin}{20mm}
\addtolength{\oddsidemargin}{-1in}
\setlength{\evensidemargin}{15mm}
\addtolength{\evensidemargin}{-1in}
\setlength{\textwidth}{170mm}
\setlength{\textheight}{254mm}
\setlength{\headsep}{0mm}
\setlength{\headheight}{0mm}
\setlength{\topskip}{0mm}

%--ハイバーリンクを可能にするパッケージ
\usepackage[dvipdfmx,%
 bookmarks=true,%
 bookmarksnumbered=true,%
 colorlinks=true,%
 setpagesize=false,%
 pdftitle={mcat},%
 pdfauthor={BMRC},%
 pdfkeywords={TeX; dvipdfmx; hyperref; color;}]{hyperref}

\begin{document}

\renewcommand{\tablename}{Table }

\setlength{\baselineskip}{4mm}

\section*{mvuniq merge vector elements}

【The output is dependent on application of the command, the results from this example may differ from the actual implementation. 】

This command merges the elements in a vector. However, since the merging process uses hash structure, the sequence of output maybe not be in order. \verb|mvsort| command can be used to sort output elements in sequence.    

 \verb|-n| option reads the vector as a sequential series. The vector series is scanned from the beginning of the string, and prints unique character strings in the vector. 

The examples are highlighted in Table \ref{tbl:out1}, \ref{tbl:out2}.
Table \ref{tbl:out1} shows all merged elements in the data series. When \verb|-n| option is specified, the same elements next to each other are merged in sequential order (see Table \ref{tbl:out2}). 


\begin{table}[htbp]
\begin{center}
\begin{tabular}{lll}

\begin{minipage}{0.3\hsize}
\begin{center}
\caption{Input data\label{tbl:input}}
\verb|in.csv| \\
{\small
\begin{tabular}{cll}
\hline
no&items \\
\hline
1&b a a \\
2&a a b b b \\
3&a b b a \\
4&a b c \\
\hline

\end{tabular}
}
\end{center}
\end{minipage}

\begin{minipage}{0.35\hsize}
\begin{center}
\caption{Basic example\label{tbl:out1}}
\verb|vf=items i=in.csv| \\
{\small
\begin{tabular}{ll}
\hline
no&items \\
\hline
1&a b \\
2&a b \\
3&a b \\
4&a b c \\
\hline
\end{tabular}
}
\end{center}
\end{minipage}

\begin{minipage}{0.35\hsize}
\begin{center}
\caption{Merge same elements adjacent to each other in a vector series \label{tbl:out2}}
\verb|vf=items -n i=in.csv| \\
{\small
\begin{tabular}{ll}
\hline
no&items \\
\hline
1&b a \\
2&a b \\
3&a b a \\
4&a b c \\
\hline
\end{tabular}
}
\end{center}
\end{minipage}

\end{tabular}
\end{center}
\end{table}

\subsection*{Format}
\verb|mvuniq vf= |
[\href{run:delim.pdf}{delim=}]
[\href{run:input.pdf}{i=}]
[\href{run:output.pdf}{o=}]
[\href{run:nfn.pdf}{-nfn}]
[\href{run:nfno.pdf}{-nfno}]
[\href{run:x.pdf}{-x}] [--help]

\begin{table}[htbp]
%\begin{center}
{\small
\begin{tabular}{ll}
\verb|vf=| & Specify the field name(s) of vectors to merge elements. \\
& Multiple field name(s) of vectors can be specified. 【required parameter】 \\
\end{tabular}
}
\end{table} 

\subsection*{Example}
\subsubsection*{Example 1 Merges all elements in a vector}
\begin{verbatim}
------------------------------------------------
# dat1.csv
items1,items2
b a c,1 1
c c,2 2 3
e a a,3 1

$ mvuniq vf=items1,items2 i=dat1.csv o=rsl1.csv

# rsl1.csv
items1,items2
c b a,1
c,2 3
e a,3 1
------------------------------------------------
\end{verbatim}


\subsection*{Related command}
\begin{itemize}
\item \href{run:mvsort.pdf}{mvsort} : Sorting vectors 
\end{itemize}

\end{document}
