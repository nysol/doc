
% how to compile: platex xxx.tex ; dvipdfmx xxx.dvi

\documentclass[a4paper]{jarticle}

%--余白の設定
\setlength{\topmargin}{20mm}
\addtolength{\topmargin}{-1in}
\setlength{\oddsidemargin}{20mm}
\addtolength{\oddsidemargin}{-1in}
\setlength{\evensidemargin}{15mm}
\addtolength{\evensidemargin}{-1in}
\setlength{\textwidth}{170mm}
\setlength{\textheight}{254mm}
\setlength{\headsep}{0mm}
\setlength{\headheight}{0mm}
\setlength{\topskip}{0mm}

%--ハイバーリンクを可能にするパッケージ
\usepackage[dvipdfmx,%
 bookmarks=true,%
 bookmarksnumbered=true,%
 colorlinks=true,%
 setpagesize=false,%
 pdftitle={mselrand},%
 pdfauthor={BMRC},%
 pdfkeywords={TeX; dvipdfmx; hyperref; color;}]{hyperref}

\begin{document}

\setlength{\baselineskip}{4mm}

\section*{mselrand (random sampling) command}
Random selection of records based on the variables set at \emph{c=} and \emph{p=} parameters. 

\subsection*{Format}
mselrand [c=]  [u=] [k=] [p=] [S=] [u=] [\href{run:option.pdf}{-nfn}] [\href{run:option.pdf}{-nfno}]  [\href{run:option.pdf}{-x}]  [\href{run:option.pdf}{i=}] [\href{run:option.pdf}{o=}] [--help]\\

\subsection*{Parameters}
\begin{table}[htbp]
%\begin{center}
{\small
\begin{tabular}{ll}
\verb|c=|    & Number of records\\
& Select row(s) based on the number of keys and field specified. \\
& \emph{p=} should always be specified with this parameter. \\
\verb|k=|    & Key field name(Allow multiple values) \\
& Random selection of records based on the defined key.  \\
& Note: Sort data by key field(s) as specified in the \emph{k =} parameter prior to the command. \\
\verb|p=|    & Percentage of records \\
& Define the ratio of records based on each key value in percentage.  \\
& \emph{c=} parameter should always be specified with this parameter.  \\
\verb|S=|    & Random seed 【default = current time】\\
& The same random seed generates the same row selection sequence. \\
& The default setting of random seed is set to the current time if this parameter is not specified.  \\
& Range of random seed value is between -2147483648〜2147483647. \\
\verb|u=|    & Output file of unmatched records. \\
& Print unmatched records to this output file. \\
\end{tabular} 
}
\end{table} 

\subsection*{Option}
\begin{table}[htbp]
%\begin{center}
{\small
\begin{tabular}{ll}
\end{tabular} 
}
\end{table} 

\subsection*{Examples}
\subsubsection*{Example 1 Random Selection}
Randomly select a transaction based on each customer.  Save other records to a separate file oth.csv. \\
(Note) Sort the input file by customer in ascending order beforehand. 

\begin{verbatim}
------------------------------------------------
# Input file (dat1.csv)
Customer,Date,Amount
A,20081201,10
A,20081207,20
A,20081213,30
B,20081002,40
B,20081209,50

$ mselrand k=Customer c=1 S=1 u=oth.csv i=dat.csv o=rsl.csv

# Output file 1(rsl.csv)
Customer,Date,Amount
A,20081201,10
B,20081002,40

#Output file 2(oth.csv)
Customer,Date,Amount
A,20081207,20
A,20081213,30
B,20081209,50

------------------------------------------------
\end{verbatim}

\subsubsection*{Example 2 Select 50\% of the records at random}
Select 50\% of each customers' records at random. Save other records to a separate file oth.csv. \\
(Note)Sort the input file by customer in ascending order beforehand. 

\begin{verbatim}
------------------------------------------------
# dat.csv
Customer,Date,Amount
A,20081201,10
A,20081207,20
A,20081213,30
B,20081002,40
B,20081209,50

$ mselrand k=Customer p=50 S=1 u=oth.csv i=dat.csv o=rsl.csv

# Output file 1(dat.csv)
Customer,Date,Amount
A,20081201,10
B,20081002,40


# Output file 2(oth.csv)
Customer,Date,Amount
A,20081207,20
A,20081213,30
B,20081209,50

------------------------------------------------
\end{verbatim}

\subsection*{Remarks}
This command used Mersenne twister (developed in 1937) as pseudo random number generator.

\subsection*{Related Commands}
\noindent
\href{run:msel.pdf}{msel}: Select records with conditions \\
\href{run:mrand.pdf}{mrand}: Generate random number 

\begin{thebibliography}{8}
boostライブラリ: http://www.boost.org/doc/libs/1\_37\_0/libs/random/random-generators.html\\
原作者のページ: http://www.math.sci.hiroshima-u.ac.jp/~m-mat/MT/MT2002/mt19937ar.html
\end{thebibliography}

\end{document}
