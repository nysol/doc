
% how to compile: platex xxx.tex ; dvipdfmx xxx.dvi

\documentclass[a4paper]{jarticle}

%--余白の設定
\setlength{\topmargin}{20mm}
\addtolength{\topmargin}{-1in}
\setlength{\oddsidemargin}{20mm}
\addtolength{\oddsidemargin}{-1in}
\setlength{\evensidemargin}{15mm}
\addtolength{\evensidemargin}{-1in}
\setlength{\textwidth}{170mm}
\setlength{\textheight}{254mm}
\setlength{\headsep}{0mm}
\setlength{\headheight}{0mm}
\setlength{\topskip}{0mm}

%--ハイバーリンクを可能にするパッケージ
\usepackage[dvipdfmx,%
 bookmarks=true,%
 bookmarksnumbered=true,%
 colorlinks=true,%
 setpagesize=false,%
 pdftitle={mcat},%
 pdfauthor={BMRC},%
 pdfkeywords={TeX; dvipdfmx; hyperref; color;}]{hyperref}

\begin{document}

\renewcommand{\tablename}{Table }

\setlength{\baselineskip}{4mm}

\section*{mvcat combine vectors}
Merge multiple vectors into one vector. 

Examples are shown in Table \ref{tbl:input}〜\ref{tbl:out2}. 

\begin{table}[htbp]
\begin{center}
\begin{tabular}{ccc}

\begin{minipage}{0.3\hsize}
\begin{center}
\caption{Input data\label{tbl:input}}
\verb|in.csv| \\
{\small
\begin{tabular}{cll}
\hline
no&items1,items2 \\
\hline
1&a c,b \\
2&a d,a e \\
3&b f, \\
4&e,e \\
\hline

\end{tabular}
}
\end{center}
\end{minipage}

\begin{minipage}{0.5\hsize}
\begin{center}
\caption{Basic example\label{tbl:out1}}
\verb|mvcat vf=item1,items2 a=catItems i=in.csv| \\
{\small
\begin{tabular}{lll}
\hline
no&catItems \\
\hline
1&a c b \\
2&a d a e \\
3&b f \\
4&e e \\
\hline
\end{tabular}
}
\end{center}
\end{minipage}

\end{tabular}
\end{center}
\end{table}

\begin{table}[htbp]
\begin{center}
\begin{tabular}{ccc}

\begin{minipage}{0.5\hsize}
\begin{center}
\caption{Retain original vectors in the input file with the merged vectors shown in Table \label{tbl:out2}}
\verb|mvcat vf=item1,items2 -A i=in.csv| \\
{\small
\begin{tabular}{lll}
\hline
no&items1,items2,new \\
\hline
1&a c,b,a c b \\
2&a d,a e,a d a e \\
3&b f,,b f \\
4&e,e,e e \\
\hline
\end{tabular}
}
\end{center}
\end{minipage}

\end{tabular}
\end{center}
\end{table}

\subsection*{Format}
\verb|mvcat vf= [-A]|
[\href{run:delim.pdf}{delim=}]
[\href{run:input.pdf}{i=}]
[\href{run:output.pdf}{o=}]
[\href{run:nfn.pdf}{-nfn}]
[\href{run:nfno.pdf}{-nfno}]
[\href{run:x.pdf}{-x}] [--help]

\begin{table}[htbp]
%\begin{center}
{\small
\begin{tabular}{ll}
\verb|vf=| & Merge specified field names of vectors from the input file (i=)【required parameter】\\
           & Wildcard can be substituted for a character within the field name.\\
\end{tabular}
}
\end{table} 

\subsection*{Example}
\subsubsection*{Example 1 Merge vectors using a wildcard character}
\begin{verbatim}
------------------------------------------------
# dat1.csv
items1,items2,items3,items4
b a c,b,x,y
c c,,x,y
e a a,a a a,x,y

$ mvcat vf=items* a=items i=dat1.csv o=rsl1.csv

# rsl1.csv
items
b a c b x y
c c x y
e a a a a a x y
------------------------------------------------
\end{verbatim}

\subsection*{Related command}
%\begin{itemize}
%\item \href{run:mvjoin.pdf}{mvjoin}:ベクトル要素の参照結合
%\end{itemize}

\end{document}
