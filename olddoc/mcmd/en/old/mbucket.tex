
% how to compile: platex xxx.tex ; dvipdfmx xxx.dvi

\documentclass[a4paper]{jarticle}

%--?????
\setlength{\topmargin}{20mm}
\addtolength{\topmargin}{-1in}
\setlength{\oddsidemargin}{20mm}
\addtolength{\oddsidemargin}{-1in}
\setlength{\evensidemargin}{15mm}
\addtolength{\evensidemargin}{-1in}
\setlength{\textwidth}{170mm}
\setlength{\textheight}{254mm}
\setlength{\headsep}{0mm}
\setlength{\headheight}{0mm}
\setlength{\topskip}{0mm}

%--??????????????????
\usepackage[dvipdfmx,%
 bookmarks=true,%
 bookmarksnumbered=true,%
 colorlinks=true,%
 setpagesize=false,%
 pdftitle={mcat},%s
 pdfauthor={BMRC},%
 pdfkeywords={TeX; dvipdfmx; hyperref; color;}]{hyperref}

\begin{document}

\setlength{\baselineskip}{4mm}

\section*{mbucket command (Partition data into uniform buckets) }
Data item(s) specified in \emph{f=} parameter will be determined and distributed into number of buckets defined in\emph{n=} parameter.  

\subsection*{Format}
mbucket f= n= [-rng] [F=] [k=] [O=]  [\href{run:nfn.pdf}{-nfn}] [\href{run:nfno.pdf}{-nfno}]  [\href{run:x.pdf}{-x}][\href{run:i.pdf}{i=}] [\href{run:o.pdf}{o=}] [--help]\\

\subsection*{Parameters}
\begin{table}[htbp]
%\begin{center}
{\small
\begin{tabular}{ll}
\verb|f=|    & Numerical data item for partitioning (multiple fields can be specified) 【required parameter】\\
& Data partitioning is based on the field name(s) specified in the parameter. \\
& Target field name for partition:New field name \\
& x,nfn option can be used to specify item number (0 ...).\\
\verb|n=|    & Number of buckets 【required parameter】\\
& Specify the number of buckets to be partitioned for the data item(s) defined in \emph{f=} parameter(s). \\
& If 1 is defined, the command will partition by the number of items specified in \emph{f=} parameter.
\\
\verb|F=|    & Output format 【default=0】\\
& Output format of bucket value. If parameter is not specified, F defaults to 0.\\
& 0:Display bucket numbers\\
& 1:Display value range of buckets\\
& 2:Display both bucket numbers and value range of buckets. \\
\verb|k=|    & Key item(s) (multiple keys can be specified in priority order) 【optional parameter】\\
& Unique key to retrieve rows of data incrementally for partitioning into buckets.\\
& (Note) Data should be sorted by the key field name specified in the \emph{k=} parameter before partitioning into buckets.\\
\verb|O=|    & Output file with values range of bucket 【optional parameter】\\
& Specify output file name with values range of bucket on the field name(s) defined in parameter \emph{f=}. \\

\end{tabular} 
}
\end{table} 

\subsection*{Options}
\begin{table}[htbp]
%\begin{center}
{\small
\begin{tabular}{ll}
\verb|-rng|  & Value range of buckets \\
& This parameter defines a specific range of each bucket\\
\end{tabular} 
}
\end{table} 

\subsection*{Examples}
\subsubsection*{Example 1 Partition x and y into two subsets of equal extent and save the output file as rng.csv} 
\noindent
※Note: The input file is sorted by customer id beforehand.

\begin{verbatim}
------------------------------------------------
# input file(dat.csv)
id,x,y
A,2,7
B,6,7
C,5,6
D,7,5
E,6,4
F,1,3
G,3,3
H,4,2
I,7,2
J,2,1

$ mbucket f=x:xb,y:yb n=2,2 O=rng.csv i=dat.csv o=rsl.csv

# output file(rsl.csv)
id,x,y,xb,yb
A,2,7,1,2
B,6,7,2,2
C,5,6,2,2
D,7,5,2,2
E,6,4,2,2
F,1,3,1,1
G,3,3,1,1
H,4,2,1,1
I,7,2,2,1
J,2,1,1,1

#output file2(rng.csv)
fieldName,bucketNo,rangeFrom,rangeTo
x,1,,4.5
x,2,4.5,
y,1,,3.5
y,2,3.5,

------------------------------------------------
\end{verbatim}

\subsubsection*{Example 2 Partition x and y into two subsets of equal extent using "id" as the key parameter. \\Include bucket numbers and value range of buckets in the output file.}
\noindent
※Note: The input file is sorted by customer id beforehand. 

\begin{verbatim}
------------------------------------------------
# input data(dat.csv)
id,x,y
A,2,7
A,6,7
A,5,6
B,7,5
B,6,4
B,1,3
C,3,3
C,4,2
C,7,2
C,2,1

$ mbucket k=id f=x:xb,y:yb n=2,2 F=2 i=dat.csv o=rsl2.csv

# output file1(rsl2.csv)
id,x,y,xb,yb
A,2,7,1:_3.5,2:6.5_
A,6,7,2:3.5_,2:6.5_
A,5,6,2:3.5_,1:_6.5
B,7,5,2:3.5_,2:3.5_
B,6,4,2:3.5_,2:3.5_
B,1,3,1:_3.5,1:_3.5
C,3,3,1:_3.5,2:1.5_
C,4,2,2:3.5_,2:1.5_
C,7,2,2:3.5_,2:1.5_
C,2,1,1:_3.5,1:_1.5
------------------------------------------------
\end{verbatim}

\subsection*{Theorem of Arithmetic}

Assume $D$ is a data set of n elements ($x_1,x_2,\cdots,x_n$).
Partition $D$ uniformly into $k$ number of groups (referred to as buckets) and ensure that the data is partitioned uniformly in each bucket. Dispersion is used as the fundamental evaluation criteria for uniformity.

\[
X=\{x_i~|~1~\leq~i~\leq~n~,~x_i~\in~D\}
\]

Arrange each value within $X$ in  ascending order $v_1,v_2,\dots,v_n$.
Partition of the buckets $X$ is represented by partition of intervals  $\{v_1,v_2,\dots,v_n\}$.
This interval is defined as $I_1,I_2,\dots,I_k$. Based on these elements,$D_j$?$n_j$, the following is defined.

\[
 D_j=\{x_i~|~1~\leq~i~\leq~n~,~x_i~\in~I_j\}
  \]
  \[
 n_j=|D_j|
 \]
The basis of uniform dispersion is described as follows. \\
\[
Var=\sum_{j=1}^{k}~(n_j-\bar{n})
\]
If $\bar{n}=n/k$, the equation becomes\\
\[
Var~=~\sum_{j=1}^{k}~(n_j^2-2n_j\bar{n}+\bar{n}^2)~~~~=~\sum_{j=1}^{k}~n_j^2~-~k\bar{n}^2
\]

$k\bar{n}^2$ remains a constant regardless of how the data is partitioned. Therefore, the first term from the above equation:  
\[
Var'=~\sum_{j=1}^{k}~n_j^2
\]
minimizes $Var$ and splits the data into intervals. 


\subsection*{Algorithm}

Recursive equation for dynamic programming is used for optimization as illustrated in the following. 
The minimum value of $\displaystyle \sum_{j=1}^{h}~n_j^2$
is divided into $h$ intervals of $I_1,I_2,\dots,I_h$  of $v_1,v_2,\dots,v_m$, to determine  $DP(n,k)$. 
 $DP(m,h)$ is substituted in the following function.

\[
DP(m,~h)~=~\min_{g=h-1,~\dots,m-1}~\{~DP(g,h-1)~+~|\{~x_i~|~v_{g+1}~\leq~x_i~\leq~v_m~\}|^2\}
\]
The above function recursively defines a sequence. Given the initial term below: 
\[
DP(m,~1)~=~|\{~x_i~|~v_1~\leq~x_i~\leq~v_m~\}|^2~,~m=1,...,n
\]
the next term of the sequence is defined as a function of the preceding term and iterated as follows:
$DP(m,2)(m=1,\dots,n),DP(m,3)(m=1,\dots,n),\dots,DP(m,k-1)(m=1,\dots,n)$.

The function ends until it searches for the term $DP(n,k)$
\[
DP(m,~k)~=~\min_{g=k-1,~\dots,n-1}~\{~DP(g,k-1)~+~|\{~x_i~|~v_{g+1}~\leq~x_i~\leq~v_n~\}|^2\}
\]

\subsection*{Related command}
\noindent
\href{run:mmbucket.pdf}{mmbucket}:Partition multidimensional data into uniform buckets \\

\end{document}
