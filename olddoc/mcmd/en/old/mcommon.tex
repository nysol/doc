
% how to compile: platex xxx.tex ; dvipdfmx xxx.dvi

\documentclass[a4paper]{jarticle}

%--余白の設定
\setlength{\topmargin}{20mm}
\addtolength{\topmargin}{-1in}
\setlength{\oddsidemargin}{20mm}
\addtolength{\oddsidemargin}{-1in}
\setlength{\evensidemargin}{15mm}
\addtolength{\evensidemargin}{-1in}
\setlength{\textwidth}{170mm}
\setlength{\textheight}{254mm}
\setlength{\headsep}{0mm}
\setlength{\headheight}{0mm}
\setlength{\topskip}{0mm}

%--ハイバーリンクを可能にするパッケージ
\usepackage[dvipdfmx,%
 bookmarks=true,%
 bookmarksnumbered=true,%
 colorlinks=true,%
 setpagesize=false,%
 pdftitle={mcommon},%
 pdfauthor={BMRC},%
 pdfkeywords={TeX; dvipdfmx; hyperref; color;}]{hyperref}

\begin{document}

\setlength{\baselineskip}{4mm}

\section*{mcommon (select common records in reference file) command}
Compare the records in the input file with the records in the reference file specified by the \emph{m=} parameter. Select the records that are common in both files based on the key specified in the \emph{k=} parameter. 

\subsection*{Format}
mcommon  k= m= [K=] [u=] [-r] 
[\href{run:option.pdf}{-nfn}] 
[\href{run:option.pdf}{-nfno}] 
[\href{run:option.pdf}{-x}] 
[\href{run:option.pdf}{i=}] 
[\href{run:option.pdf}{o=}] 
[--help]\\

\subsection*{Parameters}
\begin{table}[htbp]
%\begin{center}
{\small
\begin{tabular}{ll}
\verb|k=|    & Key field(s) in input file  (Multiple keys can be specified)【required parameter】\\
& Records in the reference file that matches the key field specified in the \emph{k=} parameter in the input data is selected.   \\
& Note: Sort data by key field(s) as specified in the "k =" parameter prior to the command. \\
& Records with same key values will therefore appear in consecutive rows.  \\
& Specify in the item number (0〜) when using the x,nfn option. \\
\verb|m=|    & Reference file【required parameter】\\
& Name of reference file \\
& Standard input specified by \emph{i=} option is used when reference file is not defined. \\
\verb|K=|    & Key field(s) in the reference data (Multiple keys can be specified). 【optional parameter】\\
& Records in the input file that matches the key field specified in the \emph{k=} parameter in the reference data is selected.   \\
& The parameter is not required if the key field is the same on both input data and reference file. \\
&  ※Note: Sort data by key field(s) as specified in the "K=" parameter prior to the command. \\
& Records with same key values will therefore appear in consecutive rows. \\
& Use x,nfn option to specify the item number (0...). \\
\verb|u=|    & Output file containing unmatched records. 【optional parameter】\\
& Write the unmatched records to this output file. \\
\end{tabular} 
}
\end{table} 

\subsection*{Option}
\begin{table}[htbp]
%\begin{center}
{\small
\begin{tabular}{ll}
\verb|-r|  & Reverse selection\\
& The field(s) specified in \emph{k=} parameter from the input file is compared with the same field(s) specified in the \emph{m=} paramater. \\
& Select unmatched records defined at specified fields in both input and output files. \\
\end{tabular} 
}
\end{table} 

\subsection*{Remarks}
Key field in reference file

\subsection*{Examples}
\subsubsection*{Example 1 Select records with common customers in both input file and reference file. Save the records that did not exist in the reference file to a separate file oth.csv.}
\noindent
※Note: Data is sorted by customer in ascending order in both input file and reference file beforehand.

\begin{verbatim}
------------------------------------------------
# Input file (dat.csv)
Customer, Quantity, Amount
A,1,10
B,2,20
C,1,15
D,3,10
E,1,20

# Reference file(ref.csv)
Customer, Sex
A,Female
B,Male
E,Female

$ mcommon k=Customer m=ref.csv u=oth.csv i=dat.csv o=rsl.csv

# Output file (rsl.csv)
Customer, Quantity, Amount
A,1,10
B,2,20
E,1,20

#Output file2(oth.csv)
Customer, Quantity, Amount
C,1,15
D,3,10

------------------------------------------------
\end{verbatim}

\subsubsection*{Example 2 Select the customers in the input file that are not in the reference file.}
\noindent
※Note: Data is sorted by customer in ascending order in both input file and reference file beforehand.

\begin{verbatim}
------------------------------------------------
# Input file (dat.csv)
Customer, Quantity, Amount
A,1,10
B,2,20
C,1,15
D,3,10
E,1,20

# Reference file(ref.csv)
Customer, Sex
A,Female
B,Male
E,Female

$ mcommon k=Customer m=ref.csv -r i=dat.csv o=rsl2.csv

# Output file1(rsl2.csv)
Customer, Quantity, Amount
C,1,15
D,3,10
------------------------------------------------
\end{verbatim}

\subsubsection*{Example 3 Select records with the same customer and date in both input file and reference file. }
\noindent
※Note: Data is sorted by customer in ascending order in both input file and reference file beforehand. 

\begin{verbatim}
------------------------------------------------
# Input file (dat2.csv)
Customer, Quantity, Amount, Date
A,1,10,2012/01/01
A,1,10,2012/01/01
A,1,10,2012/01/02
A,1,10,2012/01/03
B,2,20,2012/01/01
B,2,20,2012/01/01
C,1,15,2012/01/01
C,1,15,2012/01/03
C,1,15,2012/01/04
C,1,15,2012/01/05
D,3,10,2012/01/01
D,3,10,2012/01/01
E,1,20,2012/01/02

# Reference file(ref2.csv)
Customer ID, Sex, Date of coming
A,Female,2012/01/01
A,Female,2012/01/02
B,Male,2012/01/01
B,Male,2012/01/02
B,Male,2012/01/03
E,Female,2012/01/01
E,Female,2012/01/02

$ mcommon k=Customer,Date K=Customer ID, Date of coming i=dat2.csv m=ref2.csv  o=rsl3.csv

# Output file(rsl3.csv)
Customer, Quantity, Amount, Date
A,1,10,2012/01/01
A,1,10,2012/01/01
A,1,10,2012/01/02
B,2,20,2012/01/01
B,2,20,2012/01/01
E,1,20,2012/01/02
------------------------------------------------
\end{verbatim}

\subsubsection*{Example 4 Select records with duplicate common key fields from input file and reference file.  }
\noindent
※Note: Data is sorted by customer in ascending order in both input file and reference file beforehand. 

\begin{verbatim}
------------------------------------------------
# Input file (dat4.csv)
Customer,Quantity 
A,1
A,2
A,3
B,1
D,1
D,2

# Reference file (ref4.csv)
Customer
A
A
D

$ mcommon k=Customer m=ref4.csv i=dat4.csv o=rsl4.csv

# Output file 4(rsl4.csv)
Customer,Quantity
A,1
A,2
A,3
D,1
D,2
------------------------------------------------
\end{verbatim}


\subsection*{Related commands}
\noindent
\href{run:mselstr.pdf}{mselstr}: Select records by character string\\
\href{run:msel.pdf}{msel}:Select records with condition\
\href{run:mjoin.pdf}{mjoin}:Join data from the reference file \\
\href{run:mnjoin.pdf}{mnjoin}:Natural join with data from reference file \\

\end{document}
