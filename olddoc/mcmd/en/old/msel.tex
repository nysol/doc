
% how to compile: platex xxx.tex ; dvipdfmx xxx.dvi

\documentclass[a4paper]{jarticle}

%--余白の設定
\setlength{\topmargin}{20mm}
\addtolength{\topmargin}{-1in}
\setlength{\oddsidemargin}{20mm}
\addtolength{\oddsidemargin}{-1in}
\setlength{\evensidemargin}{15mm}
\addtolength{\evensidemargin}{-1in}
\setlength{\textwidth}{170mm}
\setlength{\textheight}{254mm}
\setlength{\headsep}{0mm}
\setlength{\headheight}{0mm}
\setlength{\topskip}{0mm}

%--ハイバーリンクを可能にするパッケージ
\usepackage[dvipdfmx,%
 bookmarks=true,%
 bookmarksnumbered=true,%
 colorlinks=true,%
 setpagesize=false,%
 pdftitle={mcat},%
 pdfauthor={BMRC},%
 pdfkeywords={TeX; dvipdfmx; hyperref; color;}]{hyperref}

\begin{document}

\setlength{\baselineskip}{4mm}

\section*{msel (select records with conditions) command}

Use comparison search conditions to select records that match the criteria specified in \emph{c=} parameter. The row of record is selected if condition is true (1). 
All operators and functions available in mcal command can be used in the conditional functions. For more details, please refer to \href{run:mcal.pdf}{mcal}.

\subsection*{Format}
msel c=  [u=] [-r]  [\href{run:option.pdf}{-nfn}] [\href{run:option.pdf}{-nfno}]  [\href{run:option.pdf}{-x}]  [\href{run:option.pdf}{i=}] [\href{run:option.pdf}{o=}] [--help]\\


\subsection*{Parameters}
\begin{table}[htbp]
%\begin{center}
{\small
\begin{tabular}{ll}
\verb|c=|    & Condition【required parameter】\\
& Select records with conditions using a set of operators and functions. \\
& Please refer to \href{run:mcal.pdf}{mcal} for more details. \\

\verb|o=|    & Output file name of matching records  \\
& Records matching the condition will be printed to this output file. \\
\verb|u=|    & Output file with unmatched records \\
& Records that do not match the condition will be printed to this output file. \\
\end{tabular} 
}
\end{table} 

\subsection*{Option}
\begin{table}[htbp]
%\begin{center}
{\small
\begin{tabular}{ll}
\verb|-r|    & Reverse selection\\
& Select records excluded in the condition defined in \emph{c-}\\
\end{tabular} 
}
\end{table} 

\subsection*{Examples}
\subsubsection*{Example 1 Select records if the amount is greater than 40. Write the unmatched records to a different output file file oth.csv. }

\begin{verbatim}
------------------------------------------------
# dat1.csv
customer, quantity, amount 
A,1,10
A,2,20
B,1,30
B,3,40
B,1,50

$ msel c='${amount}>40' u=oth.csv i=dat.csv o=rsl.csv

# output file 1(rsl.csv)
customer, quantity, amount 
B,1,50

# output file 2(oth.csv)
customer, quantity, amount 

A,1,10
A,2,20
B,1,30
B,3,40
------------------------------------------------
\end{verbatim}

\subsubsection*{Example 2 Selecting records with null value(s)}
No records will be selected when the condition defined \emph{c=} returned a null value. Records that do not match the condition will be written to a separate file defined in \emph{u=}. 
 
In the following example, the first three rows of data from column b are -1, null, and 1. When selecting records where b is greater than 0, the query record with a null value will be treated as an exception saved in the unmatched records file. 


\begin{verbatim}
------------------------------------------------
# dat1.csv
a,b
A,-1
B,
C,1

$  msel c='${b}>0' i=dat.csv o=match.csv u=unmatch.csv

# matched output data (match.csv)
a,b
C,1

# unmatched output data (unmatch.csv)
a,b
A,-1
B,
------------------------------------------------
\end{verbatim}

\subsubsection*{Example 3 Specifying Program Options}
Null value is always evaluated as a unknown value regardless of the condition. Thus, the record with null value is not selected. 
 
In the following example, the reverse selection parameter \emph{-r} is used with the same condition in the previous example. Even though the selection criteria is inverted, the record with the query record with a null value will be treated as an exception saved in the unmatched records file as in previous example. 

\begin{verbatim}
------------------------------------------------
# dat1.csv
a,b
A,-1
B,
C,1

$ msel -r c='${b}>0' i=dat.csv o=match.csv u=unmatch.csv

# matched output data(match.csv)
a,b
A,-1

# unmatched output data(unmatch.csv)
a,b
B,
C,1
------------------------------------------------
\end{verbatim}

\subsection*{related command}
\noindent
\href{run:mdelnul.pdf}{mdelnul}: delete records with null value(s)\\
\href{run:mselstr.pdf}{mselstr}: select records matching query string

\end{document}
