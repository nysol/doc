
% how to compile: platex xxx.tex ; dvipdfmx xxx.dvi

\documentclass[a4paper]{jarticle}

%--余白の設定
\setlength{\topmargin}{20mm}
\addtolength{\topmargin}{-1in}
\setlength{\oddsidemargin}{20mm}
\addtolength{\oddsidemargin}{-1in}
\setlength{\evensidemargin}{15mm}
\addtolength{\evensidemargin}{-1in}
\setlength{\textwidth}{170mm}
\setlength{\textheight}{254mm}
\setlength{\headsep}{0mm}
\setlength{\headheight}{0mm}
\setlength{\topskip}{0mm}

%--ハイバーリンクを可能にするパッケージ
\usepackage[dvipdfmx,%
 bookmarks=true,%
 bookmarksnumbered=true,%
 colorlinks=true,%
 setpagesize=false,%
 pdftitle={mcount},%
 pdfauthor={BMRC},%
 pdfkeywords={TeX; dvipdfmx; hyperref; color;}]{hyperref}

\begin{document}

\setlength{\baselineskip}{4mm}

\section*{mcount (Return the number of rows) command}
Count the number of rows and store the data in a new column defined in "a =" parameter.

\subsection*{Format}
mcount a= [k=] [\href{run:option.pdf}{-nfn}] [\href{run:option.pdf}{-nfno}]  [\href{run:option.pdf}{-x}]  [\href{run:option.pdf}{i=}] [\href{run:option.pdf}{o=}] [--help]\\

\subsection*{Parameter}
\begin{table}[htbp]
%\begin{center}
{\small
\begin{tabular}{ll}
\verb|a=|    & New column name  【required parameter】 \\
& Specify the new field name to store the count data.\\
& This parameter is not required when x,nfn option is specified.\\
\verb|k=|    & Key item(s) (Multiple keys can be specified.)\\
& Count the number of instances for incremental rows based on the key field(s) defined.  \\
& ※Note: Sort the data by key field(s) as the unit of counting specified in the "k =" parameter prior to the command. \\
& Use the x,nfn option to specify the item number (0〜).\\
\end{tabular} 
}
\end{table} 

\subsection*{Options}
\begin{table}[htbp]
%\begin{center}
{\small
\begin{tabular}{ll}
\end{tabular} 
}
\end{table} 

\subsection*{Examples}
\subsubsection*{Example 1 Count the number of customers by date. Store the count data in a new new column as "number of customers".}
(Note) The data is sorted by the key field \emph{date} beforehand. Field name(s) with spaces should be enclosed with double quotes. 

\begin{verbatim}
------------------------------------------------
# input file(dat.csv)
date, customer
20090109,A
20090109,B
20090109,C
20090110,D
20090110,E

$ mcount k="date" a="number of customer" i=dat.csv o=rsl.csv

# output file(rsl.csv)
date, customer,number of customer
20090109,C,3
20090110,E,2
------------------------------------------------
\end{verbatim}

\subsection*{Related command}
\noindent


\end{document}
