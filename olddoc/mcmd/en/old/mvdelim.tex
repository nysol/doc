
% how to compile: platex xxx.tex ; dvipdfmx xxx.dvi

\documentclass[a4paper]{jarticle}

%--余白の設定
\setlength{\topmargin}{20mm}
\addtolength{\topmargin}{-1in}
\setlength{\oddsidemargin}{20mm}
\addtolength{\oddsidemargin}{-1in}
\setlength{\evensidemargin}{15mm}
\addtolength{\evensidemargin}{-1in}
\setlength{\textwidth}{170mm}
\setlength{\textheight}{254mm}
\setlength{\headsep}{0mm}
\setlength{\headheight}{0mm}
\setlength{\topskip}{0mm}

%--ハイバーリンクを可能にするパッケージ
\usepackage[dvipdfmx,%
 bookmarks=true,%
 bookmarksnumbered=true,%
 colorlinks=true,%
 setpagesize=false,%
 pdftitle={mcat},%
 pdfauthor={BMRC},%
 pdfkeywords={TeX; dvipdfmx; hyperref; color;}]{hyperref}

\begin{document}

\renewcommand{\tablename}{Table }

\setlength{\baselineskip}{4mm}

\section*{mvdelim  change vector delimiter}

Change delimiter used to separate between string of characters in a vector. 
Some examples are shown in Table \ref{tbl:input}〜\ref{tbl:out4}. When comma is used as delimiter, a pair of double quotation marks is added to the vector as seen in Table \ref{tbl:out2}.  The delimiter between characters will be removed if a null character is specified as delimiter at \emph{v=} as seen in Table \ref{tbl:out3}.   

Alphabet and chinese characters can be used as delimiter as shown in Table \ref{tbl:out4}. Since the character type delimiter is read as character string as part of the vector by other commands such as mvsort, character type delimiter can be specified in \verb|delim=|. 


\begin{table}[htbp]
\begin{center}
\begin{tabular}{ll}

\begin{minipage}{0.20\hsize}
\begin{center}
\caption{input data\label{tbl:input}}
\verb|in.csv|\\
{\small
\begin{tabular}{cll}
\hline
no&items \\
\hline
1&b a a \\
2&a a b b \\
3&a b b a \\
4&a b c \\
\hline

\end{tabular}
}
\end{center}
\end{minipage}

\begin{minipage}{0.40\hsize}
\begin{center}
\caption{Basic example : Replace space delimiter with minus character. \label{tbl:out1}}

\verb|vf=items v=- i=in.csv|\\
{\small
\begin{tabular}{ll}
\hline
no&items \\
\hline
1&b-a-a \\
2&a-a-b-b \\
3&a-b-b-a \\
4&a-b-c \\
\hline
\end{tabular}
}
\end{center}
\end{minipage}

\begin{minipage}{0.40\hsize}
\begin{center}
\caption{Use comma as a delimiter. \label{tbl:out2}}
\verb|vf=items v=, i=in.csv|\\
{\small
\begin{tabular}{ll}
\hline
no&items \\
\hline
1&"b,a,a" \\
2&"a,a,b,b" \\
3&"a,b,b,a" \\
4&"a,b,c" \\
\hline
\end{tabular}
}
\end{center}
\end{minipage}

\end{tabular}
\end{center}
\end{table}


\begin{table}[htbp]
\begin{center}
\begin{tabular}{ll}

\begin{minipage}{0.4\hsize}
\begin{center}
\caption{Remove delimiter between characters in a vector. 
\label{tbl:out3}}
\verb|vf=items v= i=in.csv| \\
{\small
\begin{tabular}{ll}
\hline
no&items \\
\hline
1&baa \\
2&aabb \\
3&abba \\
4&abc \\
\hline
\end{tabular}
}
\end{center}
\end{minipage}

\begin{minipage}{0.4\hsize}
\begin{center}
\caption{Use character as a delimiter.
\label{tbl:out4}}
\verb|vf=items v=xx i=in.csv| \\
{\small
\begin{tabular}{ll}
\hline
no&items \\
\hline
1&bxxaxxa \\
2&axxaxxbxxb \\
3&axxbxxbxxa \\
4&axxbxxc \\
\hline
\end{tabular}
}
\end{center}
\end{minipage}


\end{tabular}
\end{center}
\end{table}

\subsection*{Format}
\verb|midelim vf= v= |
[\href{run:delim.pdf}{delim=}]
[\href{run:input.pdf}{i=}]
[\href{run:output.pdf}{o=}]
[\href{run:nfn.pdf}{-nfn}]
[\href{run:nfno.pdf}{-nfno}]
[\href{run:x.pdf}{-x}] [--help]

\begin{table}[htbp]
%\begin{center}
{\small
\begin{tabular}{ll}
\verb|vf=|   & Field name of vector. Multiple fields can be specified. 【required parameter】\\
& Change delimiters in vectors at the specified field. \\
\verb|v=|    & Define new delimiter.   【required parameter】
If the parameter is not defined, the delimiter is treated as a null character.  \\
\end{tabular}
}
\end{table} 

\subsection*{Example}
\subsection*{Related command}
%\begin{itemize}
%\item \href{run:miselstr.pdf}{miselstr}:アイテムの選択
%\end{itemize}

\end{document}
