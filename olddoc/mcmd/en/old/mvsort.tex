
% how to compile: platex xxx.tex ; dvipdfmx xxx.dvi

\documentclass[a4paper]{jarticle}

%--余白の設定
\setlength{\topmargin}{20mm}
\addtolength{\topmargin}{-1in}
\setlength{\oddsidemargin}{20mm}
\addtolength{\oddsidemargin}{-1in}
\setlength{\evensidemargin}{15mm}
\addtolength{\evensidemargin}{-1in}
\setlength{\textwidth}{170mm}
\setlength{\textheight}{254mm}
\setlength{\headsep}{0mm}
\setlength{\headheight}{0mm}
\setlength{\topskip}{0mm}

%--ハイバーリンクを可能にするパッケージ
\usepackage[dvipdfmx,%
 bookmarks=true,%
 bookmarksnumbered=true,%
 colorlinks=true,%
 setpagesize=false,%
 pdftitle={mcat},%
 pdfauthor={BMRC},%
 pdfkeywords={TeX; dvipdfmx; hyperref; color;}]{hyperref}

\begin{document}

\renewcommand{\tablename}{Table }

\setlength{\baselineskip}{4mm}

\section*{mvsort sorting vectors }
This command sort series of vectors. 
The items in Table \ref{tbl:input} shows multiple character strings separated by space. 
Table \ref{tbl:input}〜\ref{tbl:out3} highlight examples sorting vectors. In Table  \ref{tbl:out1}, character strings are arranged in ascending order by default. The character strings can be sorted in ascending order by including \verb|n| after \verb|%| behind an item name (see Table \ref{tbl:out2}), and in reverse order by specifying \verb|r| behind an item name (Table \ref{tbl:out3}).  

\begin{table}[htbp]
\begin{center}
\begin{tabular}{lrrr}

\begin{minipage}{0.20\hsize}
\begin{center}
\caption{Input data\label{tbl:input}}
\verb|in.csv| \\
{\small
\begin{tabular}{cll}
\hline
no&items \\
\hline
1&2 1 13 \\
2&4 5 2 5 \\
3&112 14 \\
4&5 31 \\
\hline

\end{tabular}
}
\end{center}
\end{minipage}

\begin{minipage}{0.25\hsize}
\begin{center}
\caption{Basic usage: Sort character string elements in vector in ascending order. 
\label{tbl:out1}}
\verb|vf=items| \\
{\small
\begin{tabular}{ll}
\hline
no&items \\
\hline
1&1 13 2 \\
2&2 4 5 5 \\
3&112 14 \\
4&31 5 \\
\hline
\end{tabular}
}
\end{center}
\end{minipage}

\begin{minipage}{0.25\hsize}
\begin{center}
\caption{Sort numerical in ascending order. 
\label{tbl:out2}}
\verb|vf=items%n| \\
{\small
\begin{tabular}{ll}
\hline
no&items \\
\hline
1&1 2 13 \\
2&2 4 5 5 \\
3&14 112 \\
4&5 31 \\
\hline
\end{tabular}
}
\end{center}
\end{minipage}

\begin{minipage}{0.25\hsize}
\begin{center}
\caption{Sort numerical in descending order. 
\label{tbl:out3}}
\verb|vf=items%nr| \\
{\small
\begin{tabular}{ll}
\hline
no&items \\
\hline
1&13 2 \\
2&5 5 4 2 \\
3&112 14 \\
4&31 5 \\
\hline
\end{tabular}
}
\end{center}
\end{minipage}

\end{tabular}
\end{center}
\end{table}

\subsection*{format}
\verb|mvsort vf= [-r] [-n]|
[\href{run:delim.pdf}{delim=}]
[\href{run:input.pdf}{i=}]
[\href{run:output.pdf}{o=}]
[\href{run:nfn.pdf}{-nfn}]
[\href{run:nfno.pdf}{-nfno}]
[\href{run:x.pdf}{-x}] [--help]

\begin{table}[htbp]
%\begin{center}
{\small
\begin{tabular}{ll}
\verb|vf=|   &  Specify the field name(s) of vectors for sorting. Multiple fields can be specified.【required parameter】\\
             & Add \verb|n%| after field name to sort in ascending numerical order. \\
             & Add \verb|r%| after field name to sort in reverse order. \\
             & Add both \verb|`n' and `r'| after field name to sort in descending numerical order. \\
\end{tabular}
}
\end{table} 

\subsection*{Examples}
\subsubsection*{Example 1 Sort multiple vectors  }
Sort item1 data series in ascending order and item2 in numerical ascending order.   

\begin{verbatim}
------------------------------------------------
# dat1.csv
items1,items2
b a c,10 2
c c,2 5 3
e a a,1

$ mvsort vf=items1%r,items2%n i=dat1.csv o=rsl1.csv

# rsl1.csv
items1,items2
c b a,2 10
c c,2 3 5
e a a,1
------------------------------------------------
\end{verbatim}


\subsection*{related command}
\begin{itemize}
\item \href{run:mvselstr.pdf}{mvselstr} : select a vector element 
\end{itemize}

\end{document}
