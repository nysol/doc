
\section{mbucket - Partition Data into Uniform Buckets\label{sect:mbucket}}
\index{mbucket@mbucket}
Partition numerical data field(s) specified at \verb|f=| into a number of segments specified by \verb|n=|. There are two ways to compute the bucket intervals. The first method is to compute an uniform spread of data points for each partition (referred to as partition of uniform buckets). The second method is to compute uniform interval ranges across partitions (referred to as partition of uniform ranges). Data is partitioned into equal interval ranges when the \verb|-rng| option is specified. Otherwise, data will be partitioned uniformly across buckets if this option is not specified. When multiple fields are defined at \verb|f=|, the data buckets are generated separately for each field. 

\subsection*{Format}
\verb|mbucket f= n= [-rng] [-r] [F=] [k=] [O=]| 
\hyperref[sect:option_i]{[i=]}
\hyperref[sect:option_o]{[o=]}
\hyperref[sect:option_bufcount]{[bufcount=]} 
\hyperref[sect:option_assert_diffSize]{[-assert\_diffSize]}
\hyperref[sect:option_assert_nullkey]{[-assert\_nullkey]}
\hyperref[sect:option_assert_nullout]{[-assert\_nullout]}
\hyperref[sect:option_nfn]{[-nfn]} 
\hyperref[sect:option_nfno]{[-nfno]}  
\hyperref[sect:option_x]{[-x]}
\hyperref[sect:option_q]{[-q]}
\hyperref[sect:option_option_tmppath]{[tmpPath=]}
\hyperref[sect:option_precision]{[precision=]}
\verb|[--help]|
\verb|[--helpl]|
\verb|[--version]|\\

\subsection*{Parameters}
\begin{table}[htbp]
%\begin{center}
{\small
\begin{tabular}{ll}
\verb|f=|    & Partitioning is based on the value specified in this field (multiple fields can be specified). \\
             & Target partition field name:new field name \\
 \verb|n=|    & Number of buckets \\
             & Specify the number of buckets to be partitioned for the field(s) defined in \verb|f=| parameter(s). \\
             & If 1 is defined, the command will partition by the number of items specified in \verb|f=| parameter.\\

\verb|F=|    & Output format [default value: 0]\\
             & Output format of bucket label. \\
             & 0:Display bucket numbers\\
             & 1:Display value range of buckets\\
             & 2:Display both bucket numbers and value range of buckets. \\
\verb|k=|    & Key field(s) to retrieve rows of data incrementally for bucket partitions (multiple keys can be specified). \\
\verb|O=|    & Output file with values range of bucket \\
             & Specify output file name with values range of bucket on the field name(s) defined in parameter \verb|f=|. \\
\verb|-rng|  & Define equal value of bucket range \\
             & Divide buckets by the specified value range. \\
\verb|-r|    & Print bucket numbers in reverse order.  \\

\end{tabular} 
}
\end{table} 

\subsection*{Examples}
\subsubsection*{例1: 基本例}

\verb|x,y|項目それぞれで、件数ができるだけ均等になるように2分割する。
その際、各バケットの数値範囲を\verb|rng1.csv|に出力する。


\begin{Verbatim}[baselinestretch=0.7,frame=single]
$ more dat1.csv
id,x,y
A,2,7
B,6,7
C,5,6
D,7,5
E,6,4
F,1,3
G,3,3
H,4,2
I,7,2
J,2,1
$ mbucket f=x:xb,y:yb n=2 O=rng1.csv i=dat1.csv o=rsl1.csv
#END# kgbucket O=rng1.csv f=x:xb,y:yb i=dat1.csv n=2 o=rsl1.csv
$ more rsl1.csv
id,x,y,xb,yb
A,2,7,1,2
B,6,7,2,2
C,5,6,2,2
D,7,5,2,2
E,6,4,2,2
F,1,3,1,1
G,3,3,1,1
H,4,2,1,1
I,7,2,2,1
J,2,1,1,1
$ more rng1.csv
fieldName,bucketNo,rangeFrom,rangeTo
x,1,,4.5
x,2,4.5,
y,1,,3.5
y,2,3.5,
\end{Verbatim}
\subsubsection*{例2: 範囲均等化分割}

\verb|-rng|オプションを指定すると範囲均等化分割となる。


\begin{Verbatim}[baselinestretch=0.7,frame=single]
$ mbucket f=x:xb,y:yb n=2 -rng O=rng2.csv i=dat1.csv o=rsl2.csv
#END# kgbucket -rng O=rng2.csv f=x:xb,y:yb i=dat1.csv n=2 o=rsl2.csv
$ more rsl2.csv
id,x,y,xb,yb
A,2,7,1,2
B,6,7,2,2
C,5,6,2,2
D,7,5,2,2
E,6,4,2,2
F,1,3,1,1
G,3,3,1,1
H,4,2,2,1
I,7,2,2,1
J,2,1,1,1
$ more rng2.csv
fieldName,bucketNo,rangeFrom,rangeTo
x,1,,4
x,2,4,
y,1,,4
y,2,4,
\end{Verbatim}
\subsubsection*{例3: キー項目を指定した例}

id項目を集計キーとして、\verb|x,y|項目それぞれを件数均等化バケット分割する。
\verb|n=2,3|と指定することで、\verb|x|項目の分割数は2に、
\verb|y|項目の分割数は3となる。
出力形式はバケット番号とバケット範囲の両方を表示する(\verb|F=2|)。


\begin{Verbatim}[baselinestretch=0.7,frame=single]
$ more dat2.csv
id,x,y
A,2,7
A,6,7
A,5,6
B,7,5
B,6,4
B,1,3
C,3,3
C,4,2
C,7,2
C,2,1
$ mbucket k=id f=x:xb,y:yb n=2,3 F=2 i=dat2.csv o=rsl3.csv
#END# kgbucket F=2 f=x:xb,y:yb i=dat2.csv k=id n=2,3 o=rsl3.csv
$ more rsl3.csv
id%0,x,y,xb,yb
A,2,7,1:_3.5,2:6.5_
A,6,7,2:3.5_,2:6.5_
A,5,6,2:3.5_,1:_6.5
B,7,5,2:3.5_,3:4.5_
B,6,4,2:3.5_,2:3.5_4.5
B,1,3,1:_3.5,1:_3.5
C,3,3,1:_3.5,3:2.5_
C,4,2,2:3.5_,2:1.5_2.5
C,7,2,2:3.5_,2:1.5_2.5
C,2,1,1:_3.5,1:_1.5
\end{Verbatim}


\subsection*{Theorem of Arithmetic}

Assume $D$ is a data set of n elements ($x_1,x_2,\cdots,x_n$).
Partition $D$ uniformly into $k$ number of groups (referred to as buckets) and ensure that the data is partitioned uniformly in each bucket. Dispersion is used as the fundamental evaluation criteria for uniformity.

\[
X=\{x_i~|~1~\leq~i~\leq~n~,~x_i~\in~D\}
\]

Arrange each value within $X$ in  ascending order $v_1,v_2,\dots,v_n$.
Partition of the buckets $X$ is represented by partition of intervals  $\{v_1,v_2,\dots,v_n\}$.
This interval is defined as $I_1,I_2,\dots,I_k$. Based on these elements,$D_j$?$n_j$, the following is defined.

\[
 D_j=\{x_i~|~1~\leq~i~\leq~n~,~x_i~\in~I_j\}
  \]
  \[
 n_j=|D_j|
 \]
The basis of uniform dispersion is described as follows. \\
\[
Var=\sum_{j=1}^{k}~(n_j-\bar{n})
\]
If $\bar{n}=n/k$, the equation becomes\\
\[
Var~=~\sum_{j=1}^{k}~(n_j^2-2n_j\bar{n}+\bar{n}^2)~~~~=~\sum_{j=1}^{k}~n_j^2~-~k\bar{n}^2
\]

$k\bar{n}^2$ remains a constant regardless of how the data is partitioned. Therefore, the first term from the above equation:  
\[
Var'=~\sum_{j=1}^{k}~n_j^2
\]
minimizes $Var$ and splits the data into intervals. 


\subsection*{Algorithm}

Recursive equation for dynamic programming is used for optimization as illustrated in the following. 
The minimum value of $\displaystyle \sum_{j=1}^{h}~n_j^2$
is divided into $h$ intervals of $I_1,I_2,\dots,I_h$  of $v_1,v_2,\dots,v_m$, to determine  $DP(n,k)$. 
 $DP(m,h)$ is substituted in the following function.

\[
DP(m,~h)~=~\min_{g=h-1,~\dots,m-1}~\{~DP(g,h-1)~+~|\{~x_i~|~v_{g+1}~\leq~x_i~\leq~v_m~\}|^2\}
\]
The above function recursively defines a sequence. Given the initial term below: 
\[
DP(m,~1)~=~|\{~x_i~|~v_1~\leq~x_i~\leq~v_m~\}|^2~,~m=1,...,n
\]
the next term of the sequence is defined as a function of the preceding term and iterated as follows:
$DP(m,2)(m=1,\dots,n),DP(m,3)(m=1,\dots,n),\dots,DP(m,k-1)(m=1,\dots,n)$.

The function ends until it searches for the term $DP(n,k)$
\[
DP(m,~k)~=~\min_{g=k-1,~\dots,n-1}~\{~DP(g,k-1)~+~|\{~x_i~|~v_{g+1}~\leq~x_i~\leq~v_n~\}|^2\}
\]

\subsection*{Related command}
\hyperref[sect:mmbucket]{mmbucket} : Partition multidimensional data into uniform buckets \\


