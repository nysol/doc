
%\begin{document}

\section{mcat - ConCATenate\label{sect:mcat}}
\index{mcat@mcat}
Merge all records in the files specified at \verb|i=| parameter according to the order of files. If a wild card is used to specify file names, the files will be merged in alphabetical order of the file name. 

\subsection*{Format}
\verb/mcat [f=] [-skip_fnf] [-nostop|-skip|-force]/
\hyperref[sect:option_i]{[i=]}
\hyperref[sect:option_o]{[o=]}
\hyperref[sect:option_add_fname]{[-add\_fname]}
\hyperref[sect:option_-stdin]{[-stdin]}
\hyperref[sect:option_assert_diffSize]{[-assert\_diffSize]}
\hyperref[sect:option_assert_nullin]{[-assert\_nullin]}
\hyperref[sect:option_nfn]{[-nfn]} 
\hyperref[sect:option_nfno]{[-nfno]}  
\hyperref[sect:option_x]{[-x]}
\hyperref[sect:option_q]{[-q]}
\hyperref[sect:option_option_tmppath]{[tmpPath=]}
\verb|[--help]|
\verb|[--helpl]|
\verb|[--version]|\\

\subsection*{Parameters}
\begin{table}[htbp]
%\begin{center}
{\small
\begin{tabular}{ll}
\verb|i=|        & Specify list of input file names. \\
                 & Read multiple CSV files separated by comma delimiter. Wild card characters can be used in file name. \\
\verb|f=|        & Specify the field name(s) to concatenate. \\
                 & If \verb|f=| is not specified, the field names defaults to the first file defined in the \verb|i=| parameter. \\
\verb|-skip_fnf| & If a specified file in  the \verb|i=| parameter does not exist, the program will bypass the error. \\
		& However, the program returns an error if all files cannot be found. \\
\verb|-nostop|   & \verb|-nostop,-skip,-force| are parameters for controlling exceptions when header is not present. \\
                 & \verb|-nostop| flag returns null if field name is not specified. When \verb|-nfn| flag is used with \verb|stop| flag, \\
                 & the program terminates if the number of items in the data is different than the parameter defined. \\
\verb|-skip|     & Files are not concatenated if field name(s) is not specified. \\
                 & When \verb|-nfn| flag is used with \verb|-skip| flag, files are not concatenated if the number of data items are different. \\
\verb|-force|    & Force concatenation of files using location of fields when header is not present. \\
		& Print output to null if item number is not available. \\
\verb|-stdin|    & Merge from standard input. \\
\verb|-add_fname|& Add  file name in the last column. \\
                 & Standard input will be named as \verb|/dev/stdin|.\\
                 & The field name for this option is fixed as \verb|"fileName"|, \\
                 & error will be returned if input data contains the same field name. \\
\end{tabular}
}
\end{table}

\subsection*{Note}
\begin{itemize}
\item Wild card characters ("?" and "*") can be used to specify multiple directory and file names. 
\item The symbol \verb|~/|can be used to indicate home directory. 
\item The files are concatenated according according to the order specified in the \verb|i=| parameter. If a wild card is used, files will be merged in alphabetical order. Standard input takes precedence when merging files. 
\end{itemize}


\subsection*{Examples}
\subsubsection*{Example 1: Concatenate files with the same header}



\begin{Verbatim}[baselinestretch=0.7,frame=single]
$ more dat1.csv
customer,date,amount
A,20081201,10
B,20081002,40
$ more dat2.csv
customer,date,amount
A,20081207,20
A,20081213,30
B,20081209,50
$ mcat i=dat1.csv,dat2.csv o=rsl1.csv
#END# kgcat i=dat1.csv,dat2.csv o=rsl1.csv
$ more rsl1.csv
customer,date,amount
A,20081201,10
B,20081002,40
A,20081207,20
A,20081213,30
B,20081209,50
\end{Verbatim}
\subsubsection*{Example 2: Concatenate files with different header}

The first file \verb|dat1.csv| defined at \verb|i=| contains columns "customer,date,amount". However, since "amount" is not present in \verb|dat3.csv|, it will return an error. Nevertheless, the contents in the first file \verb|dat1.csv| is merged and saved in the output.


\begin{Verbatim}[baselinestretch=0.7,frame=single]
$ more dat3.csv
customer,date,quantity
A,20081201,3
B,20081002,1
$ mcat i=dat1.csv,dat3.csv o=rsl2.csv
#ERROR# field name [amount] not found on file [dat3.csv] (kgcat)
$ more rsl2.csv
customer,date,amount
A,20081201,10
B,20081002,40
\end{Verbatim}
\subsubsection*{Example 3: Concatenate files with different header2}

When previous example is attached with \verb|-nostop| option, the command will continue processing and return NULL value for the data item not found. Other options such as \verb|skip,force| handle conditions when the field name is not found. For details, refer to the description of parameters.


\begin{Verbatim}[baselinestretch=0.7,frame=single]
$ more dat3.csv
customer,date,quantity
A,20081201,3
B,20081002,1
$ mcat -nostop i=dat1.csv,dat3.csv o=rsl3.csv
#END# kgcat -nostop i=dat1.csv,dat3.csv o=rsl3.csv
$ more rsl3.csv
customer,date,amount
A,20081201,10
B,20081002,40
A,20081201,
B,20081002,
\end{Verbatim}
\subsubsection*{Example 4: Concatenate specific field names from input files}

Merge field names specified at \verb|f=|.


\begin{Verbatim}[baselinestretch=0.7,frame=single]
$ mcat f=customer,date i=dat2.csv,dat3.csv o=rsl4.csv
#END# kgcat f=customer,date i=dat2.csv,dat3.csv o=rsl4.csv
$ more rsl4.csv
customer,date
A,20081207
A,20081213
B,20081209
A,20081201
B,20081002
\end{Verbatim}
\subsubsection*{Example 5: Merge from standard input}

Read file \verb|dat2.csv| from standard input by specifying \verb|-stdin| option.



\begin{Verbatim}[baselinestretch=0.7,frame=single]
$ mcat -stdin i=dat1.csv o=rsl5.csv <dat2.csv
#END# kgcat -stdin i=dat1.csv o=rsl5.csv
$ more rsl5.csv
customer,date,amount
A,20081207,20
A,20081213,30
B,20081209,50
A,20081201,10
B,20081002,40
\end{Verbatim}
\subsubsection*{Example 6: Add file name as new column}

When \verb|-add_fname| is specified, the original file name \verb|fileName| is added as a new column.
File name of standard input is \verb|/dev/stdin|.


\begin{Verbatim}[baselinestretch=0.7,frame=single]
$ mcat -add_fname -stdin i=dat1.csv o=rsl6.csv <dat2.csv
#END# kgcat -add_fname -stdin i=dat1.csv o=rsl6.csv
$ more rsl6.csv
customer,date,amount,fileName
A,20081207,20,/dev/stdin
A,20081213,30,/dev/stdin
B,20081209,50,/dev/stdin
A,20081201,10,dat1.csv
B,20081002,40,dat1.csv
\end{Verbatim}
\subsubsection*{Example 7: Specify wild card}

Specifying wild card \verb|dat*.csv| to concatenate the three CSV files \verb|dat1.csv,dat2.csv,dat3.csv| in the current directory.


\begin{Verbatim}[baselinestretch=0.7,frame=single]
$ more dat1.csv
customer,date,amount
A,20081201,10
B,20081002,40
$ more dat2.csv
customer,date,amount
A,20081207,20
A,20081213,30
B,20081209,50
$ more dat3.csv
customer,date,quantity
A,20081201,3
B,20081002,1
$ mcat -force i=dat*.csv o=rsl7.csv
#END# kgcat -force i=dat*.csv o=rsl7.csv
$ more rsl7.csv
customer,date,amount
A,20081201,10
B,20081002,40
A,20081207,20
A,20081213,30
B,20081209,50
A,20081201,3
B,20081002,1
\end{Verbatim}
\subsubsection*{Example 8: Concatenate the same file multiple times}

Same file can be specified more than one time.


\begin{Verbatim}[baselinestretch=0.7,frame=single]
$ mcat i=dat1.csv,dat1.csv,dat1.csv o=rsl8.csv
#END# kgcat i=dat1.csv,dat1.csv,dat1.csv o=rsl8.csv
$ more rsl8.csv
customer,date,amount
A,20081201,10
B,20081002,40
A,20081201,10
B,20081002,40
A,20081201,10
B,20081002,40
\end{Verbatim}


\subsection*{Related command}
\hyperref[sect:msep]{msep} : Reverse the operation mentioned above and separate data files. 

%\end{document}
