
%\begin{document}

\section{msel - Select Records with Conditions\label{sect:msel}}
\index{msel@msel}
Define the computation criteria at \verb|c=| parameter, the record is selected if condition returns true. 
All operators and functions available in mcal command can be used in the conditional function. For more details, please refer to \hyperref[sect:mcal]{mcal}.

\subsection*{Format}
\verb|msel c=  [u=] [-r]|
\hyperref[sect:option_i]{[i=]}
\hyperref[sect:option_o]{[o=]}
\hyperref[sect:option_assert_diffSize]{[-assert\_diffSize]}
\hyperref[sect:option_nfn]{[-nfn]} 
\hyperref[sect:option_nfno]{[-nfno]}  
\hyperref[sect:option_x]{[-x]}
\hyperref[sect:option_q]{[-q]}
\hyperref[sect:option_option_tmppath]{[tmpPath=]}
\verb|[--help]|
\verb|[--helpl]|
\verb|[--version]|\\


\subsection*{Parameters}
\begin{table}[htbp]
%\begin{center}
{\small
\begin{tabular}{ll}
\verb|c=|    & Define the expression using combinations of operators and functions.  \\
             & Refer to \hyperref[sect:mcal]{mcal} for more details. \\
\verb|o=|    & Records matching the condition will be printed to this output file. \\
\verb|u=|    & Records that do not match the condition will be printed to this output file.\\
\verb|-r|    & Reverse selection\\
             & Select records excluded from the selection condition defined in \verb|c=|\\
\end{tabular} 
}
\end{table} 



\subsection*{Examples}
\subsubsection*{Example 1: Basic example}

Select records where "Amount" is greater than 40. Write the unmatched records to a different output file file \verb|unmatch1.csv|.


\begin{Verbatim}[baselinestretch=0.7,frame=single]
$ more dat1.csv
Customer,Quantity,Amount
A,1,10
A,2,20
B,1,30
B,3,40
B,1,50
$ msel c='${Amount}>40' u=unmatch1.csv i=dat1.csv o=match1.csv
#END# kgsel c=${Amount}>40 i=dat1.csv o=match1.csv u=unmatch1.csv
$ more match1.csv
Customer,Quantity,Amount
B,1,50
$ more unmatch1.csv
Customer,Quantity,Amount
A,1,10
A,2,20
B,1,30
B,3,40
\end{Verbatim}
\subsubsection*{Example 2: Selecting records with null value(s)}

No records will be selected when the condition defined \verb|c=| returned a null value. Records that do not match the condition will be written to a separate file defined in \verb|u=|. 
 
In the following example, the first three rows of data from column \verb|b| are \verb|-1|, null, and \verb|1|. When selecting records where \verb|b| is greater than 0, the query record with a null value will be treated as an exception saved in the unmatched records file. 


\begin{Verbatim}[baselinestretch=0.7,frame=single]
$ more dat2.csv
a,b
A,-1
B,
C,1
$ msel c='${b}>0' i=dat2.csv o=match2.csv u=unmatch2.csv
#END# kgsel c=${b}>0 i=dat2.csv o=match2.csv u=unmatch2.csv
$ more match2.csv
a,b
C,1
$ more unmatch2.csv
a,b
A,-1
B,
\end{Verbatim}
\subsubsection*{Example 3: Specify -r option}

Null value is always evaluated as a unknown value regardless of the condition. Thus,  records with null value is not selected.

In the following example, the reverse selection parameter \verb|-r| is used with the same condition in the previous example. Even though the selection criteria is inverted, the query record with a null value will be treated as an exception saved in the unmatched records file as in the previous example.


\begin{Verbatim}[baselinestretch=0.7,frame=single]
$ msel -r c='${b}>0' i=dat2.csv o=match3.csv u=unmatch3.csv
#END# kgsel -r c=${b}>0 i=dat2.csv o=match3.csv u=unmatch3.csv
$ more match3.csv
a,b
A,-1
$ more unmatch3.csv
a,b
B,
C,1
\end{Verbatim}


\subsection*{Related Commands}
\hyperref[sect:mselnum]{mselnum} : Select records with simple numeric range. 

\hyperref[sect:mselstr]{mselstr} : Select records matching query string

\hyperref[sect:mcal]{mcal} : Return the calculated results instead of selecting records. 

%\end{document}
