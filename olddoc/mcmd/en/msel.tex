
%\begin{document}

\section{msel - Select Records with Conditions\label{sect:msel}}
\index{msel@msel}
Define the computation criteria at \verb|c=| parameter, the record is selected if condition returns true. 
All operators and functions available in mcal command can be used in the conditional function. For more details, please refer to \hyperref[sect:mcal]{mcal}.

\subsection*{Format}
\verb|msel c=  [u=] [-r]|
\hyperref[sect:option_i]{[i=]}
\hyperref[sect:option_o]{[o=]}
\hyperref[sect:option_assert_diffSize]{[-assert\_diffSize]}
\hyperref[sect:option_nfn]{[-nfn]} 
\hyperref[sect:option_nfno]{[-nfno]}  
\hyperref[sect:option_x]{[-x]}
\hyperref[sect:option_q]{[-q]}
\hyperref[sect:option_option_tmppath]{[tmpPath=]}
\verb|[--help]|
\verb|[--helpl]|
\verb|[--version]|\\


\subsection*{Parameters}
\begin{table}[htbp]
%\begin{center}
{\small
\begin{tabular}{ll}
\verb|c=|    & Define the expression using combinations of operators and functions.  \\
             & Refer to \hyperref[sect:mcal]{mcal} for more details. \\
\verb|o=|    & Records matching the condition will be printed to this output file. \\
\verb|u=|    & Records that do not match the condition will be printed to this output file.\\
\verb|-r|    & Reverse selection\\
             & Select records excluded from the selection condition defined in \verb|c=|\\
\end{tabular} 
}
\end{table} 



\subsection*{Examples}
\subsubsection*{例1: 基本例}

「金額」項目の値が40より大きい行を選択する。
それ以外のデータは\verb|unmatch1.csv|に出力する。


\begin{Verbatim}[baselinestretch=0.7,frame=single]
$ more dat1.csv
顧客,数量,金額
A,1,10
A,2,20
B,1,30
B,3,40
B,1,50
$ msel c='${金額}>40' u=unmatch1.csv i=dat1.csv o=match1.csv
#END# kgsel c=${金額}>40 i=dat1.csv o=match1.csv u=unmatch1.csv
$ more match1.csv
顧客,数量,金額
B,1,50
$ more unmatch1.csv
顧客,数量,金額
A,1,10
A,2,20
B,1,30
B,3,40
\end{Verbatim}
\subsubsection*{例2: NULL値の選択規制}

\verb|msel|コマンドでは\verb|c=|で与えられた式を評価した結果がNULL値の場合その行は選択されない。
また、アンマッチ出力ファイルが\verb|u=|によって指定されていれば、そのファイルに出力される。
以下の例では項目\verb|b|に\verb|-1|、NULL値、\verb|1|を持つ3行のデータについて、0より大きい行を選択しているが、
NULL値を含む行はアンマッチ出力ファイルへと出力される。


\begin{Verbatim}[baselinestretch=0.7,frame=single]
$ more dat2.csv
a,b
A,-1
B,
C,1
$ msel c='${b}>0' i=dat2.csv o=match2.csv u=unmatch2.csv
#END# kgsel c=${b}>0 i=dat2.csv o=match2.csv u=unmatch2.csv
$ more match2.csv
a,b
C,1
$ more unmatch2.csv
a,b
A,-1
B,
\end{Verbatim}
\subsubsection*{例3: -rオプション指定}

真偽は逆転するがNULL値の評価に変わりはない。
すなわちNULL値の行は選択されない。
以下の例では、上の例と同様のデータおよび選択条件で\verb|-r|をつけている。
真偽の選択条件は逆転しているが、NULL値を含む行は上記の例と同様にアンマッチファイルへと出力されていることがわかる。


\begin{Verbatim}[baselinestretch=0.7,frame=single]
$ msel -r c='${b}>0' i=dat2.csv o=match3.csv u=unmatch3.csv
#END# kgsel -r c=${b}>0 i=dat2.csv o=match3.csv u=unmatch3.csv
$ more match3.csv
a,b
A,-1
$ more unmatch3.csv
a,b
B,
C,1
\end{Verbatim}


\subsection*{Related Commands}
\hyperref[sect:mselnum]{mselnum} : Select records with simple numeric range. 

\hyperref[sect:mselstr]{mselstr} : Select records matching query string

\hyperref[sect:mcal]{mcal} : Return the calculated results instead of selecting records. 

%\end{document}
