%\documentclass[a4paper]{book}
%\usepackage{mcmd}
%\begin{document}

\subsection{k= Key field name\label{sect:option_k}}

Specify the key field name.
A key field  uniquely identifies individual rows or an entity in the data, it is used as unit of aggregation, or used as common key for joining fields between two files. 

For example, in  \verb|msum| command,  aggregate computation is carried out for records with the same key (aggregate key break processing).  Whereas in \verb|mjoin| command, the size of key items in the two data files are compared (join key break processing) and joined. 

When \verb|k=| command is specified, the field(s) specified are first sorted in character string ascending order, afterwards, corresponding processing is carried out. 

and is considered as the default field for sorting character strings in ascending order (except for \hyperref[sect:mhashsum]{mhashsum}). 
Key break process refers to the processing method for every same key field with the same value assuming that the items are sorted beforehand (However, \hyperref[sect:mhashsum]{mhashsum} command is an exception). 

For details on key break process, please refer to \hyperref[sect:keybreak]{Key break processing}. 
Since frequent sorting may decrease the processing performance, understanding the need for key break processing would help reduce the instances for sorting, desirable for optimizing script performance. 

\subsection*{Examples}
\subsubsection*{Example 1: Basic Example}

Compute sum on \verb|val| column by \verb|id|.


\begin{Verbatim}[baselinestretch=0.7,frame=single]
$ more dat1.csv
id,val
A,1
B,1
B,2
A,2
B,3
$ msum i=dat1.csv k=id f=val o=rsl1.csv
#END# kgsum f=val i=dat1.csv k=id o=rsl1.csv
$ more rsl1.csv
id%0,val
A,3
B,6
\end{Verbatim}
\subsubsection*{Example 2: Join Process}

Use the join key “id” from \verb|dat1.csv|, and join the field “name” from \verb|ref1.csv|.


\begin{Verbatim}[baselinestretch=0.7,frame=single]
$ more dat1.csv
id,val
A,1
B,1
B,2
A,2
B,3
$ more ref1.csv
id,name
A,nysol
B,mcmd
$ mjoin k=id i=dat1.csv m=ref1.csv f=name o=rsl4.csv
#END# kgjoin f=name i=dat1.csv k=id m=ref1.csv o=rsl4.csv
$ more rsl4.csv
id%0,val,name
A,1,nysol
A,2,nysol
B,1,mcmd
B,2,mcmd
B,3,mcmd
\end{Verbatim}


\subsubsection*{Related commands}
\hyperref[sect:msum]{msum},
\hyperref[sect:mslide]{mslide},
\hyperref[sect:mjoin]{mjoin},
\hyperref[sect:mrjoin]{mrjoin},
\hyperref[sect:mcommon]{mcommon}, etc.

%\end{document}

