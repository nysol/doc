
%\documentclass[a4paper]{jsbook}
%\usepackage{mcmd_jp}
%\begin{document}

\section{mminput Display form input screen\label{sect:mminput}}
\index{mminput@mminput}
\underline{Note: This command is a beta. Its specifications may be changed.}

This command displays a data input screen using the text file specified by the \verb|i=| parameter as the screen form. The character strings on the screen form is displayed as is, and the area between square brackets (\verb|[]|) is shown as a freehand input field. There can be two or more input fields. The data entered by the user is output to the CSV file specified by the \verb|o=| parameter. The output data consists of one row. When there are two or more input fields, a multiple-field CSV file is output.
When the command ends with nothing entered in the input frame, null is output. To output a fieldname, use the \verb|f=| parameter. With \verb|f=| omitted, no fieldname header is output. 
The operation is indefinite if the specified coordinate is outside the scope of the terminal.

\subsection*{Format}
\verb/mminput i= [f=] /
\hyperref[sect:option_o]{o=}
\hyperref[sect:option_nfn]{[-nfn]} 
\hyperref[sect:option_nfno]{[-nfno]}  
\hyperref[sect:option_x]{[-x]}
\verb|[--help]|
\verb|[--helpl]|
\verb|[--version]|\\

\subsection*{Parameters}
\begin{table}[htbp]
%\begin{center}
{\small
\begin{tabular}{ll}
\verb|i=| & Specify the text file containing the screen form.\\
\verb|f=| & Specify the output fieldname.\\
\end{tabular} 
}
\end{table} 

\subsection*{Examples}

\subsubsection*{Example 1: Basic example}
A \verb|name| and \verb|address| input screen is displayed. The entered name and address are output to \verb|rsl1.csv| under fieldnames name,address.

\begin{Verbatim}[baselinestretch=0.7,frame=single]
$ more screen.txt

     name   :[               ]
     address:[               ]

$ mminput i=screen.txt f=name,address o=rsl1.csv
$ more rsl1.csv
name,address
Taro,Japan


The following will be displayed:
+--------------------------------------
|
|     name   :[Taro           ]
|     address:[Japan          ]
|
\end{Verbatim}

\subsubsection*{Example 2: Judging end status}
The parameters are the same as in Example 1. This script judges the end status and performs different operations accordingly.

\begin{Verbatim}[baselinestretch=0.7,frame=single]
$ more scp.sh
rm -f rsl3.csv
clear
mminput i=screen.txt f=name,address o=rsl3.csv
if [ $? = 0 ] ; then
  clear ; echo "end by enter key"
else
  clear ; echo "end by escape key"
fi

# Result of typing Taro and Japan and pressing Enter
$ bash scp.sh
end by enter key
$ more rsl3.csv
name,address
Taro,Japan

# Result of typing Taro and Japan and pressing ESC
$ bash scp.sh
end by escape key
$ more rsl3.csv
name,address
Taro,Japan
\end{Verbatim}

\subsection*{Related Commands}
\hyperref[sect:minput] {minput} : Displays the input screen.
\hyperref[sect:mdsp] {mdsp} : Displays a character string at the specified position on the screen.
\hyperref[sect:mseldsp] {mseldsp} : Displays a single-choice input window on the screen.
\hyperref[sect:mmseldsp] {mmseldsp} : Displays a multiple-choice input window on the screen.

%\end{document}
