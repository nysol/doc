
%\begin{document}

\section{muniq - Unique Records\label{sect:muniq}}
\index{muniq@muniq}
Remove duplicate values and create unique records. 

\subsection*{Format}
\verb|muniq [k=] |
\hyperref[sect:option_i]{[i=]}
\hyperref[sect:option_o]{[o=]}
\hyperref[sect:option_assert_diffSize]{[-assert\_diffSize]}
\hyperref[sect:option_assert_nullkey]{[-assert\_nullkey]}
\hyperref[sect:option_nfn]{[-nfn]} 
\hyperref[sect:option_nfno]{[-nfno]}  
\hyperref[sect:option_x]{[-x]}
\hyperref[sect:option_q]{[-q]}
\hyperref[sect:option_option_tmppath]{[tmpPath=]}
\verb|[--help]|
\verb|[--helpl]|
\verb|[--version]|\\

\subsection*{Parameter}
\begin{table}[htbp]
%\begin{center}
{\small
\begin{tabular}{ll}
\verb|k=|    &  Specify the field name (s) as the unique identifier of the records. \\
\end{tabular} 
}
\end{table} 

\subsection*{Examples}
\subsubsection*{Example 1: Basic Example}

Remove duplicate records in the \verb|date| field.


\begin{Verbatim}[baselinestretch=0.7,frame=single]
$ more dat1.csv
date,customer
20081201,A
20081202,A
20081202,B
20081202,B
20081203,C
$ muniq k=date i=dat1.csv o=rsl1.csv
#END# kguniq i=dat1.csv k=date o=rsl1.csv
$ more rsl1.csv
date%0,customer
20081201,A
20081202,B
20081203,C
\end{Verbatim}
\subsubsection*{Example 2: Delete duplicate rows in multiple columns}

Remove duplicate records based on unique values in \verb|date| and \verb|customer| field.


\begin{Verbatim}[baselinestretch=0.7,frame=single]
$ muniq k=date,customer i=dat1.csv o=rsl2.csv
#END# kguniq i=dat1.csv k=date,customer o=rsl2.csv
$ more rsl2.csv
date%0,customer%1
20081201,A
20081202,A
20081202,B
20081203,C
\end{Verbatim}

\subsection*{Related Command}
\hyperref[sect:mbest]{mbest} : Use \verb|mbest| command to select the line number for records with the same key. 

%\end{document}
