
%\begin{document}

\section{Specify Fields\label{sect:fieldname}}
MCMD reads the field names in the first row of CSV data; the field can also be specified with a field number without the field names. There are four options for handling the row of field names, which are: -nfn,-nfno,-nfni and -x. Its usage will be illustrated with examples as follows. In addition, note that the field number starts at 0 from the left such as 0, 1, 2.

\subsubsection*{Example 1: Specify -nfn}

When \verb|-nfn| (no field name) is specified, the first row in the data will not be considered as field names. Thus, each field is specified as a number (note that the number starts from 0). 


\begin{Verbatim}[baselinestretch=0.7,frame=single]
$ more dat2.csv
a,2
b,5
b,4
$ msum -nfn k=0 f=1 i=dat2.csv o=rsl1.csv
#END# kgsum -nfn f=1 i=dat2.csv k=0 o=rsl1.csv
$ more rsl1.csv
a,2
b,9
\end{Verbatim}
\subsubsection*{Example 2: Specify -nfno}

When \verb|-nfno| (no field name for output) is specified, the first row of input data is initialized as field names, but the field names is removed from the output data.


\begin{Verbatim}[baselinestretch=0.7,frame=single]
$ more dat1.csv
key,val
a,2
b,5
b,4
$ msum k=key f=val -nfno i=dat1.csv o=rsl2.csv
#END# kgsum -nfno f=val i=dat1.csv k=key o=rsl2.csv
$ more rsl2.csv
a,2
b,9
\end{Verbatim}
\subsubsection*{Example 3: Specify -nfni}

The option \verb|-nfni| (no field names or input) is only available in the mcut command. This option does the opposite of - nfno; the first row of input data is not treated as a fieldname row, but the fieldnames will be shown in the output data. Thus, you need to specify an output fieldname after a colon (:) following an input field number.


\begin{Verbatim}[baselinestretch=0.7,frame=single]
$ mcut f=0:key,1:val -nfni i=dat2.csv o=rsl3.csv
#END# kgcut -nfni f=0:key,1:val i=dat2.csv o=rsl3.csv
$ more rsl3.csv
key,val
a,2
b,5
b,4
\end{Verbatim}
\subsubsection*{Example 4: Specify -x}

For CSV data with a field names, use the \verb|-x| option to specify the field number.


\begin{Verbatim}[baselinestretch=0.7,frame=single]
$ msum -x k=0 f=1 i=dat1.csv o=rsl4.csv
#END# kgsum -x f=1 i=dat1.csv k=0 o=rsl4.csv
$ more rsl4.csv
key%0,val
a,2
b,9
\end{Verbatim}


\subsection{Valid fieldnames}
Fieldnames can contain the characters stated as follows:

\begin{itemize}
\item Alphabetic characters (a-z, A-Z)
\item Numerals (0-9)
\item Multibyte characters (such as UTF-8)
\item Symbols
\end{itemize}

However, it is recommended to avoid using the following symbols. Using the symbols will not return an error; however, the symbols may be used as special characters in MCMD and the special character function may not be available if it is used as a field name.

\begin{itemize}
\item \verb|,| Comma
\item \verb|:| Colon
\item \verb|%| Percent
\item \verb|*| Asterisk
\item \verb|?| Question mark
\item \verb|&| And
\item \verb|\| Backslash
\item \verb|]| Square brackets, right
\item \verb|[| Square brackets, left
\end{itemize}

\subsection{Valid item number}
When specifying field number, the field numbers can be listed with a comma delimiter. Alternatively, it is also possible to specify the number at the end of the field name (add \verb|"L"|) or specify the range (\verb|-|).

For example, the argument \verb|0L| specifies the last field, and \verb|2L| specifies the 2nd field from the end (note that field number starts from 0). Furthermore, when \verb|0-5| is specified, six fields starting from 0 to 5 are selected, which is equivalent to \verb|0,1,2,3,4,5|.

\subsubsection*{Example 1: Specify range}

By specifying "0-4", fields 0,1,2,3,4 are specified.  


\begin{Verbatim}[baselinestretch=0.7,frame=single]
$ more dat1.csv
brand,quantity01,quantity02,quantity03,quantity04,quantity05,quantity06,quantity07,quantity08,quantity09,quantity10
A,10,50,90,130,170,210,250,290,330,370
B,20,60,100,140,180,220,260,300,340,380
C,30,70,110,150,190,230,270,310,350,390
D,40,80,120,160,200,240,280,320,360,400
$ mcut -x f=0-4 i=dat1.csv o=rsl1.csv
#END# kgcut -x f=0-4 i=dat1.csv o=rsl1.csv
$ more rsl1.csv
brand,quantity01,quantity02,quantity03,quantity04
A,10,50,90,130
B,20,60,100,140
C,30,70,110,150
D,40,80,120,160
\end{Verbatim}
\subsubsection*{Example 2: Specify range in reverse order}

By specifying “4-0”,  fields 0,1,2,3,4 are specified.


\begin{Verbatim}[baselinestretch=0.7,frame=single]
$ mcut -x f=4-0 i=dat1.csv o=rsl2.csv
#END# kgcut -x f=4-0 i=dat1.csv o=rsl2.csv
$ more rsl2.csv
quantity04,quantity03,quantity02,quantity01,brand
130,90,50,10,A
140,100,60,20,B
150,110,70,30,C
160,120,80,40,D
\end{Verbatim}
\subsubsection*{Example 3: Specify Multiple ranges}

By specifying “1-0,2-4”, fields “1,0,2,3,4” are specified.EOF
scp=<<'EOF'
mcut -x f=1-0,2-4 i=dat1.csv o=rsl3.csv
more rsl3.csv


\begin{Verbatim}[baselinestretch=0.7,frame=single]
$ mcut -x f=4-0 i=dat1.csv o=rsl2.csv
#END# kgcut -x f=4-0 i=dat1.csv o=rsl2.csv
$ more rsl2.csv
quantity04,quantity03,quantity02,quantity01,brand
130,90,50,10,A
140,100,60,20,B
150,110,70,30,C
160,120,80,40,D
\end{Verbatim}
\subsubsection*{Example 4: Specified field from the end}

By specifying "2L", the second field from the end is specified (quantity 08).


\begin{Verbatim}[baselinestretch=0.7,frame=single]
$ mcut -x f=2L i=dat1.csv o=rsl4.csv
#END# kgcut -x f=2L i=dat1.csv o=rsl4.csv
$ more rsl4.csv
quantity08
290
300
310
320
\end{Verbatim}
\subsubsection*{Example 5: Specify the range of fields from the end}

By specifying "5-3L", the 5th to the 3rd item from end is specified, i.e. "5,6,7".


\begin{Verbatim}[baselinestretch=0.7,frame=single]
$ mcut -x f=5-3L i=dat1.csv o=rsl5.csv
#END# kgcut -x f=5-3L i=dat1.csv o=rsl5.csv
$ more rsl5.csv
quantity05,quantity06,quantity07
170,210,250
180,220,260
190,230,270
200,240,280
\end{Verbatim}


\subsection{Input and output fields}
The \verb|f=| parameter is used to specify the field(s) in many commands. The format of \verb|f=| is defined as “ input field:output field”. If an output fieldname is not specified, the input fieldname will be used as the output fieldname. In addition, you can specify the \verb|-x| option to combine a fieldname with a number, like \verb|f=0:Quantity|.

\subsubsection*{Example 1: Basic Example}

By specifying "quantity:unit sales", the field name is converted from “quantity” to “unit sales” in the output.


\begin{Verbatim}[baselinestretch=0.7,frame=single]
$ more dat1.csv
brand,quantity
A,10
B,20
C,30
D,40
$ mcut f=brand,quantity:salesquantity i=dat1.csv o=rsl1.csv
#END# kgcut f=brand,quantity:salesquantity i=dat1.csv o=rsl1.csv
$ more rsl1.csv
brand,salesquantity
A,10
B,20
C,30
D,40
\end{Verbatim}
\subsubsection*{Example 2: Add field name}

The maccum command accumulates the values in the "quantity" field, and add the field name "cumulative quantity" in the output results. If the parameter is specified as "f=quantity", the field name of the cumulative result will remain as "quantity", thus results in error because the same field name “quantity” exists in the output. 


\begin{Verbatim}[baselinestretch=0.7,frame=single]
$ maccum f=quantity:accumulationquantity i=dat1.csv o=rsl2.csv
#ERROR# parameter s= is mandatory without -q,-nfn (kgaccum)
$ more rsl2.csv
$ maccum f=quantity i=dat1.csv o=rsl2.csv
#ERROR# parameter s= is mandatory without -q,-nfn (kgaccum)
\end{Verbatim}
\subsubsection*{Example 3: Mixing field name and field number}

The field name and field number can be specified at the same time.


\begin{Verbatim}[baselinestretch=0.7,frame=single]
$ mcut f=0,1:salesquantity -x i=dat1.csv o=rsl3.csv
#END# kgcut -x f=0,1:salesquantity i=dat1.csv o=rsl3.csv
$ more rsl3.csv
brand,salesquantity
A,10
B,20
C,30
D,40
\end{Verbatim}


\subsection{Wildcard}
The wildcard characters  \verb|"*"| and  \verb|"?"|  can be used to specify multiple field names.
The asterisk sign \verb|"*"| matches 0 or more characters, and the question mark \verb|"?"| matches a single character.
Note that the order of evaluation of wildcard characters follows the order of the fields in the input data.
For example, if the order of the fields in input data is \verb|A5,A3,A4,A2,A1|, the parameter \verb|f=A*| is evaluated as \verb|f=A5,A3,A4,A2,A1|.

\subsubsection*{Example 1: Basic Example}

The expression "quantity*" matches field names starting with quantity ("quantity10", "quantity11", "quantity12" and "quantity123").  


\begin{Verbatim}[baselinestretch=0.7,frame=single]
$ more dat1.csv
brand,quantity10,quantity11,quantity12,quantity123
A,10,15,9,1
B,20,16,8,2
C,30,17,7,3
D,40,18,6,4
$ mcut f= quantity* i=dat1.csv o=rsl1.csv
#ERROR# invalid argument: quantity* (kgcut)
$ more rsl1.csv
rsl1.csv: No such file or directory
\end{Verbatim}
\subsubsection*{Example 2: Wildcard character “?”}

Select field names which begin with "quantity" followed by 1, and match any single character after 1. In this case, the wildcard does not match with field name “quantity123”. 


\begin{Verbatim}[baselinestretch=0.7,frame=single]
$ mcut f= quantity 1? i=dat1.csv o=rsl2.csv
#ERROR# invalid argument: quantity (kgcut)
$ more rsl2.csv
rsl2.csv: No such file or directory
\end{Verbatim}


\subsection{Replace the name of an output field}
The special character \verb|"&"| specified in the output field name can be replaced with the current field name.
For example, the parameter \verb|f=abc:xx&xx| returns \verb|xxabcxx| as the output field name.
The \verb|"&"| character can be specified at any position as many times as required in the output field name.
However, the ampersand is a special character in shell which is interpreted as "background execution". Thus, it is necessary to escape and enclose the field name in double quotes when including \verb|"&"| in field name.

\subsubsection*{Example 1: Basic Example}

In this example, \verb|"&"| is replaced with “\verb|brand|” in the input field name, which is equivalent to the expression "\verb|f=brand:brand code|".


\begin{Verbatim}[baselinestretch=0.7,frame=single]
$ more dat1.csv
brand,quantity10,quantity11,quantity12,quantity123
A,10,15,9,1
B,20,16,8,2
C,30,17,7,3
D,40,18,6,4
$ mcut f="brand:& code" i=dat1.csv o=rsl1.csv
#END# kgcut f=brand:& code i=dat1.csv o=rsl1.csv
$ more rsl1.csv
brand code
A
B
C
D
\end{Verbatim}
\subsubsection*{Example 2: Combine with wildcard}

Attach “\verb|&|” after \verb|sales&| to replace the character with input field name (e.g. "quantity10") in the output field name.  For all input fields name beginning with “\verb|quantity|”, attach “\verb|sales|” as the prefix in the output field name. 


\begin{Verbatim}[baselinestretch=0.7,frame=single]
$ mcut f="brand,quantity*:sales&" i=dat1.csv o=rsl2.csv
#END# kgcut f=brand,quantity*:sales& i=dat1.csv o=rsl2.csv
$ more rsl2.csv
brand,salesquantity10,salesquantity11,salesquantity12,salesquantity123
A,10,15,9,1
B,20,16,8,2
C,30,17,7,3
D,40,18,6,4
\end{Verbatim}


\subsection{Notes on summation commands}
A summation command sums records in specified fields for each key field. When one row is output per key field, records to be output are indefinite for non-specified fields. Take the msum command, for example, where there are four fields: customer, date, product, and amount. To sum the amounts for each customer, you would specify \verb|msum k=customer f=amount|. Regarding the records in the date and product fields, the output is indefinite.
%\end{document}
