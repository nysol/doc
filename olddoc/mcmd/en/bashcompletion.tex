
%\begin{document}
%\newcommand{\ctab2}{\fcolorbox{red}{yellow}{TABを2回}}

\section{bash completion\label{sect:whatis}}
The bash-completion tool assists command input on the command line.
To use the input completion functions shown below,install the 
\href{https://github.com/scop/bash-completion}{bash-completion}
configuration file for MCMD.
It will greatly enhance the ease of using MCMD on the command line.
\begin{itemize}
\item Listing parameters available for each command and completing them
\item Completing input/output filenames 
\item Completing fieldnames in the input CSV data 
\item Completing values of choice parameters
\end{itemize}

\subsection{Installation\label{sect:bash_comp_install}}
To check whether bash-completion has been installed, check whether the \verb|complete| command can be used as described below.
If it has not been installed, see the 
\href{https://github.com/scop/bash-completion}{development page}
and install it.
You can search the Internet to find simple installation procedures for different platforms.

\begin{Verbatim}[baselinestretch=0.7,frame=single]
$ complete --help
-bash: complete: --: invalid option
complete: usage: complete [-abcdefgjksuv] [-pr] [-o option] [-A action]
[-G globpat] [-W wordlist] [-P prefix] [-S suffix] [-X filterpat] [-F function]
[-C command] [name ...]
\end{Verbatim}

For the MCMD configuration file, find the \verb|bash_completion_mcmd.sh| file in the root of the \verb|mcmd/unitilies| directory of the MCMD source tree, copy it to the root of the \verb|home| directory, and rename it.

%\begin{figure}[htbp]
\begin{Verbatim}[baselinestretch=0.7,frame=single]
$ cp mcmd/unitilies/bash_completion_mcmd.sh ~/.bash_completion_mcmd.sh
\end{Verbatim}
%\caption{bash completionのMCMD用設定ファイルのインストール\label{fig:bash_comp_install}}
%\end{figure}
Then, as described below, add an instruction to run the script to \verb|~/.bash_profile|. 
That way, the configuration is automatically loaded each time you log in.

\begin{Verbatim}[baselinestretch=0.7,frame=single]
. ~/.bash_completion_mcmd.sh
\end{Verbatim}

\subsection{Completing parameters\label{sect:bash_comp_param}}
After typing a command name, press the \verb|TAB| key twice to list the parameters and options that can be specified for that command. As an example, the following shows the list for the \verb|msum| command.

\begin{Verbatim}[baselinestretch=0.7,frame=single]
$ msum [Press TAB twice]
-assert_diffSize  -assert_nullkey   -n      -nfno     -q      f=      k=      precision=
-assert_nullin    -assert_nullout   -nfn    -params   -x      i=      o=      tmpPath=
\end{Verbatim}

With the above list displayed, typing \verb|p| and pressing \verb|TAB| completes the keyword character string "\verb|precision=|” because it is the only parameter name that starts with a \verb|p|.
Following the completion, a space is inserted after the keyword. To enter another value, you need to move back one character.

\begin{Verbatim}[baselinestretch=0.7,frame=single]
$ msum p[Press TAB]
# The string will be completed as shown below. 
$ msum precision= 
\end{Verbatim}

Typing \verb|-a| and pressing TAB completes the keyword character string only up to “\verb|-assert_|” because it is common to the multiple options starting with -a. Then, pressing TAB again shows the four parameters that start with \verb|-assert_|.

\begin{Verbatim}[baselinestretch=0.7,frame=single]
$ msum -a[Press TAB]
-assert_diffSize  -assert_nullin    -assert_nullkey   -assert_nullout   
\end{Verbatim}

\subsection{Completing input/output filenames\label{sect:bash_comp_file}}
In MCMD, several parameters are used to specify a filename or a directory name, such as \verb|i=|, \verb|m=|, and \verb|o=|. You can use filename completion by typing any of such parameters and pressing TAB. Typing \verb|=| only and pressing TAB lists the files in the root of the current directory. Typing a certain character string and pressing TAB shows the one filename or multiple candidate filenames starting with that character string.
Only one filename can be completed per command, except for the \verb|mcat| command. When using the mcat command, you can enter multiple filenames by completion by pressing TAB after a comma.

\begin{Verbatim}[baselinestretch=0.7,frame=single]
$ msum i=[Press TAB twice]
test1.csv   test2.csv ...

$ mcat i=test1.csv,[Press TAB twice]
test1.csv   test2.csv ...
\end{Verbatim}

\subsection{Completing fieldnames\label{sect:bash_comp_field}}
In MCMD, many parameters are used to specify a fieldname. If an input file has already been entered, the fieldnames in the file can be entered by completion.

\begin{Verbatim}[baselinestretch=0.7,frame=single]
$ cat test1.csv 
codeA,codeB,date,value1,value2
A,aaa,20170101,3,120
A,bbb,20170102,6,100

$ msum i=test1.csv k=[Press TAB twice]
codeA   codeB   date    value1  value2  
~ $ msum i=test1.csv k=c[Press TAB]
# The string will be completed as shown below.
~ $ msum i=test1.csv k=code[Press TAB]
codeA  codeB
~ $ msum i=test1.csv k=codeA f=value[Press TAB twice]
value1  value2
~ $ msum i=test1.csv k=codeA f=value*
codeA%0,codeB,date,value1,value2
A,bbb,20170102,9,220
#END# kgsum f=value* i=test1.csv k=codeA; IN=2 OUT=1;
\end{Verbatim}

For some commands like \verb|mjoin|, two input files are specified. If that is the case, which file to use has been predetermined for each keyword used to specify a filename. Thus, fieldnames in the file corresponding to the keyword are used for completion.

\begin{Verbatim}[baselinestretch=0.7,frame=single]
$ cat test1.csv 
codeA,codeB,date,value1,value2
A,aaa,20170101,3,120
A,bbb,20170102,6,100

$ cat test2.csv
code,name
A,aaa
B,bbb

$ mjoin i=test1.csv m=test2.csv k=[Press TAB twice]
codeA   codeB   date    value1  value2
$ mjoin i=test1.csv m=test2.csv k=codeA K=[Press TAB twice]
codeA  name
$ mjoin i=test1.csv m=test2.csv k=codeA K=code f=[Press TAB twice]
codeA  name
$ mjoin i=test1.csv m=test2.csv k=codeA K=code f=name
codeA%0,codeB,date,value1,value2,name
A,aaa,20170101,3,120,aaa
A,bbb,20170102,6,100,aaa
#END# kgjoin K=code f=name i=test1.csv k=codeA m=test2.csv; IN=2 OUT=2;
\end{Verbatim}

\subsection{Completing choices\label{sect:bash_comp_items}}
Some parameters allow you to choose one of several predetermined items, like the \verb|c=| parameter in the \verb|mstats| command. You can enter such parameters by completion.

\begin{Verbatim}[baselinestretch=0.7,frame=single]
$ msim c=[Press TAB twice]
chi       confMax convMax cosine euclid  jaccard kendall oddsRatio phi      supportr yuleQ
cityblock confMin convMin covar  hamming kappa   lift    pearson   spearman ucovar   yuleY

~ $ mstats c=[Press TAB twice]
USD     cv      kurt    mean    min     qrange  qtile3  sd      sum     ukurt   uskew   var
count   devsq   max     median  mode    qtile1  range   skew    ucount  usd     uvar

\end{Verbatim}

%\end{document}
