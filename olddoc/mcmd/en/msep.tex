
%\begin{document}

\section{msep - Partition Records\label{sect:msep}}
\index{msep@msep}

Define the file name and location the save the separate records at the \verb|d=| parameter. Since it is possible to embed the \verb|${fieldname}| in the file name specified, as a result, the data can be split into separate files for corresponding fields. For example, the argument \verb|d=./out/${date}.csv| specifies the \verb|out| directory in the current directory, and the file is created from corresponding values in the \verb|date| field.

%Value of embedded field name is recognized as a key internally, since the output file is opened upon each keybreak, it is necessary to sort the input data by the embedded field in advance.



\subsection*{Format}
\verb|msep d= [-p] [f=] |  
\hyperref[sect:option_i]{[i=]}
\hyperref[sect:option_assert_nullin]{[-assert\_nullin]}
\hyperref[sect:option_nfn]{[-nfn]} 
\hyperref[sect:option_nfno]{[-nfno]}  
\hyperref[sect:option_x]{[-x]}
\hyperref[sect:option_q]{[-q]}
\hyperref[sect:option_option_tmppath]{[tmpPath=]}
\verb|[--help]|
\verb|[--helpl]|
\verb|[--version]|\\

\subsection*{Parameters}
\begin{table}[htbp]
%\begin{center}
{\small
\begin{tabular}{ll}
\verb|d=|    & Specify the field name used for splitting to different data files.\\
             & String specify here will be added as file name for each record. \\
             & Embed field name as \verb|${field name}|.\\
\verb|-p|    & Create the new directory name specify at the \verb|d=| parameter which does not currently exist.\\
\end{tabular} 
}
\end{table} 


\subsection*{Examples }
\subsubsection*{例1: 基本例}

\verb|dat|という名前のディレクトリを作成し、
そのディレクトリに日付項目値\verb|date|ごとに異なるファイルに出力する。


\begin{Verbatim}[baselinestretch=0.7,frame=single]
$ more dat1.csv
item,date,quantity,price
A,20081201,1,10
B,20081201,4,40
A,20081202,2,20
A,20081203,3,30
B,20081203,5,50
$ msep d='./dat/${date}.csv' -p i=dat1.csv
#END# kgsep -p d=./dat/${date}.csv i=dat1.csv
$ ls ./dat
20081201.csv
20081202.csv
20081203.csv
$ more ./dat/20081201.csv
item,date%0,quantity,price
A,20081201,1,10
B,20081201,4,40
$ more ./dat/20081202.csv
item,date%0,quantity,price
A,20081202,2,20
$ more ./dat/20081203.csv
item,date%0,quantity,price
A,20081203,3,30
B,20081203,5,50
\end{Verbatim}


\subsection*{Related Commands}
\hyperref[sect:msep2]{msep2} : While the functionality is similar with \verb|msep|, serial number is used to name the output file, and prints a list of corresponding key and file name to a separate file. 

\hyperref[sect:mcat] {mcat} : This command restore and merge all the partitioned files by \verb|sep| to the original file.

%\end{document}
