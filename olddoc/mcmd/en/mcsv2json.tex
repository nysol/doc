
%\documentclass[a4paper]{jsbook}
%\usepackage{mcmd_jp}
%\begin{document}

\section{mcsv2json Convert CSV to JSON format\label{sect:mcsv2json}}
\index{mcsv2json@mcsv2json}
\underline{Note: This command is a beta. Its specifications may be changed.}

This command converts CSV data into JSON format, which is the object notation syntax for JavaScript. 
This command has the following characteristics. 
For details, see the examples.

\begin{itemize}
 \item Only character strings are supported on JSON.
 \item Outputs can be arrays and objects (key-value pairs).
 \item With a key field specified, outputs can be in nesting structures of arrays.
\end{itemize}

\subsection*{Format}
\verb/mcsv2json [k=] [s=] f=|h=|p= -flat /
\hyperref[sect:option_i]{i=}
\hyperref[sect:option_o]{[o=]}
\hyperref[sect:option_assert_nullin]{[-assert\_nullin]}
\hyperref[sect:option_nfn]{[-nfn]}
\hyperref[sect:option_nfno]{[-nfno]}
\hyperref[sect:option_x]{[-x]}
\hyperref[sect:option_q]{[-q]}
\hyperref[sect:option_option_tmppath]{[tmpPath=]}
\verb|[--help]|
\verb|[--helpl]|
\verb|[--version]|\\

\subsection*{Parameters}
\begin{table}[htbp]
%\begin{center}
{\small
\begin{tabular}{ll}
\verb|k=|    & Specify a list of fieldnames for which arrays are nested on JSON\\
             & Specifying three fields makes three-fold JSON arrays.\\
\verb|s=|    & Specify the order of values. Specify \verb|%n| for numerical order or \verb|%r| for reverse order.\\
\verb|f=|    & Specify the field whose value is to be output as a JSON array.\\
\verb|h=|    & Specify the fieldname used as the key for outputting a JSON object (hash structure).\\
\verb|p=|    & Specify pairs of key and value fieldnames of JSON objects.\\
             & Use colons as delimiters between the two fieldnames as follows: \verb|p=key fieldname 1: value fieldname 1,key fieldname 2:value fieldname 2,...|\\
\end{tabular} 
}
\end{table} 


\subsection*{Examples}
\subsubsection*{Example 1: Example 1: Outputting arrays}



\begin{Verbatim}[baselinestretch=0.7,frame=single]
$ more dat1.csv
key1,key2,v1,v2
A,X,1,a
A,Y,2,b
A,Y,3,c
B,X,4,d
B,Y,5,e
$ mcsv2json f=v1,v2 i=dat1.csv
[["1","a"],["2","b"],["3","c"],["4","d"],["5","e"]]
#END# kgcsv2json f=v1,v2 i=dat1.csv
\end{Verbatim}
\subsubsection*{Example 2: Example 2: Outputting objects (key-value pairs)}



\begin{Verbatim}[baselinestretch=0.7,frame=single]
$ mcsv2json h=v1,v2 i=dat1.csv
[{"v1":"1","v2":"a"},{"v1":"2","v2":"b"},{"v1":"3","v2":"c"},{"v1":"4","v2":"d"},{"v1":"5","v2":"e"}]
#END# kgcsv2json h=v1,v2 i=dat1.csv
\end{Verbatim}
\subsubsection*{Example 3: Example 3: Outputting objects according to field specification}



\begin{Verbatim}[baselinestretch=0.7,frame=single]
$ mcsv2json p=v2:v1 i=dat1.csv
[{"a":"1"},{"b":"2"},{"c":"3"},{"d":"4"},{"e":"5"}]
#END# kgcsv2json i=dat1.csv p=v2:v1
\end{Verbatim}
\subsubsection*{Example 4: Example 4: Specifying a key field}

The three rows whose \verb|key1| field is \verb|A| are output as a single array. Then, the two rows whose \verb|key1| field is \verb|B| are output as a single array.


\begin{Verbatim}[baselinestretch=0.7,frame=single]
$ mcsv2json k=key1 f=v1 i=dat1.csv
[[["1"],["2"],["3"]],
[["4"],["5"]]]
#END# kgcsv2json f=v1 i=dat1.csv k=key1
\end{Verbatim}
\subsubsection*{Example 5: Example 5: Nesting key fields}

One row where \verb|key1=A| and \verb|key2=X| is output as a single array, two rows where \verb|key1=A| and \verb|key2=Y| are output as a single array, and the two arrays (that is, rows where \verb|key1=A|) are bundled as a single array.


\begin{Verbatim}[baselinestretch=0.7,frame=single]
$ mcsv2json k=key1,key2 f=v1 i=dat1.csv
[[[["1"]],
[["2"],["3"]]],
[[["4"]],
[["5"]]]]
#END# kgcsv2json f=v1 i=dat1.csv k=key1,key2
\end{Verbatim}
\subsubsection*{Example 6: Example 6: Outputting rows flatly without bundling them as an array}



\begin{Verbatim}[baselinestretch=0.7,frame=single]
$ mcsv2json f=v1,v2 -flat i=dat1.csv
["1","a","2","b","3","c","4","d","5","e"]
#END# kgcsv2json -flat f=v1,v2 i=dat1.csv
\end{Verbatim}

\subsection*{Related Commands}

\subsection*{Reference}
http://www.rfc-editor.org/rfc/rfc4627.txt

%\end{document}
