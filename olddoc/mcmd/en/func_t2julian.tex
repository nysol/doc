
%\begin{document}

\section{t2julian(date and time)\label{sect:t2julian}}
\index{t2julian@t2julian}
Use this function to obtain a Julian day from the date and time.

\begin{verbatim}
A Julian day is the number of days from noon of January 1, 4713 B.C. (universal time).
The effective range of the Gregorian calendar, however, is 1400-1-1 to 9999-12-31,
so the corresponding range of the Julius days is 2232300 to 5373484. NULL is output if the result is outside the range.
During the conversion, the time is discarded from the specified date and time. Only the date is converted.
\end{verbatim}

\subsection*{Usage}
\begin{verbatim}
c=t2julian(0t20080822101010)
\end{verbatim}

\subsection*{Example}

\begin{table}[hbt]
\begin{center}
 \caption{Input data}
  \begin{tabular}{|c|c|c|} \hline
Date&Time&Number\\ \hline\hline
20020824&20020824145408&10660\\ \hline
20020622&20020622173449&22740\\ \hline
20020824&20020824145408&14800\\ \hline
20021009&20021009095743&54510\\ \hline
20020121&20020121173449&18750\\ \hline
  \end{tabular}
  \end{center}
\end{table}
Examples based on the above data are shown below.

\subsubsection*{Execution example 1)}
The value in the “Time” field is input and converted into a Julian day. A “Julian day” field is newly created, and the result is output.

\begin{verbatim}
------------------------------------------------
mcal c='t2julian($t{Time})' a="Julian day" i=date.csv o=ot2julian.csv
------------------------------------------------
\end{verbatim}

\begin{table}[hbt]
\begin{center}
 \caption{Output file(ot2julian.csv)}
  \begin{tabular}{|c|c|c|c|} \hline
Date&Time&Number&Julian day\\ \hline\hline
20020824&20020824145408&10660&2452511.621\\ \hline
20020622&20020622173449&22740&2452448.733\\ \hline
20020824&20020824145408&14800&2452511.621\\ \hline
20021009&20021009095743&54510&2452557.415\\ \hline
20020121&20020121173449&18750&2452296.733\\ \hline
  \end{tabular}
  \end{center}
\end{table}

\href{run:hizuke.pdf}{Return to mcal - date related }\\
%\end{document}

