
%\documentclass[a4paper]{jsbook}
%\usepackage{mcmd_jp}
%\begin{document}

\section{mtab2csv Convert TSV data into CSV data\label{sect:mtab2csv}}
\index{mtab2csv@mtab2csv}
This command converts tab-separated values into comma-separated values. The data in the source text file does not have to be tab-separated; use the \verb|d=| parameter to specify a delimiter other than the tab. If the numbers of fields differ before and after the conversion, the data is output until the immediately preceding row and the command abends.

\subsection*{Format}
\verb|mtab2csv [d=] [-r] |
\hyperref[sect:option_i]{[i=]}
\hyperref[sect:option_o]{[o=]}
\hyperref[sect:option_nfn]{[-nfn]} 
\hyperref[sect:option_nfno]{[-nfno]}  
\hyperref[sect:option_x]{[-x]}
\hyperref[sect:option_q]{[-q]}
\hyperref[sect:option_option_tmppath]{[tmpPath=]}
\verb|[--help]|
\verb|[--helpl]|
\verb|[--version]|\\

\subsection*{Parameters}
\begin{table}[htbp]
%\begin{center}
{\small
\begin{tabular}{ll}
\verb|d=|    & Specify a delimiter. (Only a single-byte character can be specified.)\\
\verb|-r|    & Use this option to remove carriage return codes (\verb|CR:\r|).\\
             & When handling CSV data, MCMD recognizes only LF(\verb|\n|) as the line feed code. If the CSV data\\
             & contains Windows text return codes CR+LF(\verb|\r\n|) or Mac text return codes CR(\verb|\r|), conversion can\\
             & result in an error because MCMD handles them as mere characters. This option is for avoiding this\\
             & issue.\\
\end{tabular} 
}
\end{table} 

\subsection*{Examples}
\subsubsection*{例1: 基本例}

 タブ区切りデータをcsvへ変換


\begin{Verbatim}[baselinestretch=0.7,frame=single]
$ more dat1.tab
id	data	data2
A	1102	a
A	2203	aaa
B	1155	bbbb
B	3104	c
B	1206	de
$ mtab2csv i=dat1.tab o=rsl1.csv
#END# kgtab2csv i=dat1.tab o=rsl1.csv
$ more rsl1.csv
id,data,data2
A,1102,a
A,2203,aaa
B,1155,bbbb
B,3104,c
B,1206,de
\end{Verbatim}
\subsubsection*{例2: d=指定}

\verb|d=|を使用してtab以外の区切り文字を使う


\begin{Verbatim}[baselinestretch=0.7,frame=single]
$ more dat2.bar
id-data-data2
A-1102-a
A-2203-aaa
B-1155-bbbb
B-3104-c
B-1206-de
$ mtab2csv d=- i=dat2.bar o=rsl2.csv
#END# kgtab2csv d=- i=dat2.bar o=rsl2.csv
$ more rsl2.csv
id,data,data2
A,1102,a
A,2203,aaa
B,1155,bbbb
B,3104,c
B,1206,de
\end{Verbatim}

\subsection*{Related Commands}
\hyperref[sect:mxml2csv]{mxml2csv}: Converts XML data into CSV data.
\hyperref[sect:msplit]{msplit}: Partitions fields by delimiters.
%\end{document}
