
%\begin{document}

\section{mfsort - Sort Field\label{sect:mfsort}}
\index{mfsort@mfsort}
Sort according to the values of the specified fields at \verb|f=| within each record (in default ascending order by character string). Note that this does not change the sequence of field names. 

\subsection*{Format}
\verb|mfsort f= [-r] [-n]| 
\hyperref[sect:option_i]{[i=]}
\hyperref[sect:option_o]{[o=]}
\hyperref[sect:option_assert_diffSize]{[-assert\_diffSize]}
\hyperref[sect:option_assert_nullin]{[-assert\_nullin]}
\hyperref[sect:option_nfn]{[-nfn]} 
\hyperref[sect:option_nfno]{[-nfno]}  
\hyperref[sect:option_x]{[-x]}
\hyperref[sect:option_q]{[-q]}
\hyperref[sect:option_option_tmppath]{[tmpPath=]}
\verb|[--help]|
\verb|[--helpl]|
\verb|[--version]|\\

\subsection*{Parameters}
\begin{table}[htbp]
%\begin{center}
{\small
\begin{tabular}{ll}
\verb|f=| & Specify multiple fields where data items are sorted. The result remains the same when one field is defined. \\
\verb|-n| & Arrange in numerical order. \\
\verb|-r| & Arrange in reverse order. \\
\end{tabular} 
}
\end{table} 

\subsection*{Examples}
\subsubsection*{Example 1: Basic Example}

Arrange the values in \verb|v1,v2,v3| in ascending order for each record, and output the data items in sequential order corresponding to fields \verb|v1,v2,v3|.


\begin{Verbatim}[baselinestretch=0.7,frame=single]
$ more dat1.csv
id,v1,v2,v3
1,b,a,c
2,a,b,a
3,b,,e
$ mfsort f=v* i=dat1.csv o=rsl1.csv
#END# kgfsort f=v* i=dat1.csv o=rsl1.csv
$ more rsl1.csv
id,v1,v2,v3
1,a,b,c
2,a,a,b
3,,b,e
\end{Verbatim}
\subsubsection*{Example 2: Descending Order}

Add \verb|-r| to arrange in descending order.


\begin{Verbatim}[baselinestretch=0.7,frame=single]
$ mfsort f=v* -r i=dat1.csv o=rsl2.csv
#END# kgfsort -r f=v* i=dat1.csv o=rsl2.csv
$ more rsl2.csv
id,v1,v2,v3
1,c,b,a
2,b,a,a
3,e,b,
\end{Verbatim}


\subsection*{Related Command}

%\end{document}
