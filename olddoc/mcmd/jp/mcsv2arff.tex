
%\documentclass[a4paper]{jsbook}
%\usepackage{mcmd_jp}
%\begin{document}

\section{mcsv2arff CSVをARFF形式に変換\label{sect:mcsv2arff}}
\index{mcsv2arff@mcsv2arff}
csv形式のデータからarff形式(WEKA用のデータフォーマット)のデータへ変換する。
arffでは、属性の型を指定する必要があり、
\verb|d=|でカテゴリ型項目を、\verb|n=|で数値型項目を、\verb|s=|で文字列型項目を、
そして\verb|D=|で日付型項目をそれぞれ指定する。
日付型項目名に\verb|%t|を付ければ時刻を含んだ値と見なし、付けなければ日付のみの値と見なす。

\subsection*{書式}
\verb/mcsv2arff n=|d=|D=|s= [T=]/
\hyperref[sect:option_i]{i=}
\hyperref[sect:option_o]{[o=]}
\hyperref[sect:option_assert_nullin]{[-assert\_nullin]}
\hyperref[sect:option_nfn]{[-nfn]}
\hyperref[sect:option_nfno]{[-nfno]}
\hyperref[sect:option_x]{[-x]}
\hyperref[sect:option_option_tmppath]{[tmpPath=]}
\hyperref[sect:option_precision]{[precision=]}
\verb|[-params]|
\verb|[--help]|
\verb|[--helpl]|
\verb|[--version]|\\

\subsection*{パラメータ}
\begin{table}[htbp]
%\begin{center}
{\small
\begin{tabular}{ll}
\verb|i=|    & 入力ファイル名を指定する。\\
\verb|o=|    & 出力ファイル名を指定する。\\
\verb|n=|    & 数値項目名(複数項目指定可)を指定する。\\
\verb|d=|    & カテゴリ項目名(複数項目指定可)を指定する。\\
\verb|D=|    & 日付(時刻)項目名(複数項目指定可)リストを指定する。 [\%{t}]\\
             & \verb|%t|を指定しなかった場合:yyyyMMdd\\
             & \verb|%t|を指定した場合   :yyyyMMddHHmmss\\
\verb|s=|    & 文字列項目名(複数項目指定可)を指定する。\\
\verb|T=|    & タイトルにする文字列を指定する。\\
\end{tabular} 
}
\end{table} 


\subsection*{利用例}
\subsubsection*{Example 1: Convert csv format data to arff format}

Convert data to arff format and define "customer" field as string type, "product" field as category type, "date" field as date type (exclude time), “quantity” and “amount” fields as numeric attributes.


\begin{Verbatim}[baselinestretch=0.7,frame=single]
$ more dat1.csv
customer,product,date,quantity,amount
No.1,A,20081201,1,10
No.2,A,20081202,2,20
No.3,A,20081203,3,30
No.4,B,20081201,4,40
No.5,B,20081203,5,50
$ mcsv2arff s=customer d=product D=date n=quantity,amount T=Purchase_Data i=dat1.csv  o=rsl1.csv
#END# kgcsv2arff D=date T=Purchase_Data d=product i=dat1.csv n=quantity,amount o=rsl1.csv s=customer
$ more rsl1.csv
@RELATION	Purchase_Data

@ATTRIBUTE	customer	string
@ATTRIBUTE	date	date yyyyMMdd
@ATTRIBUTE	quantity	numeric
@ATTRIBUTE	amount	numeric
@ATTRIBUTE	product	{A,B}

@DATA
No.1,20081201,1,10,A
No.2,20081202,2,20,A
No.3,20081203,3,30,A
No.4,20081201,4,40,B
No.5,20081203,5,50,B
\end{Verbatim}
\subsubsection*{Example 2: Convert csv format data to arff format (include time in the date attribute)}

Specify the date with the time information by adding \verb|%t| such that \verb|D=date%t|.


\begin{Verbatim}[baselinestretch=0.7,frame=single]
$ more dat2.csv
customer,product,date,quantity,amount
No.1,A,20081201102030,1,10
No.2,A,20081202123010,2,20
No.3,A,20081203153010,3,30
No.4,B,20081201174010,4,40
No.5,B,20081203133010,5,50
$ mcsv2arff s=customer d=product D=date%t n=quantity,amount T=Purchase_Data i=dat2.csv  o=rsl2.csv
#END# kgcsv2arff D=date%t T=Purchase_Data d=product i=dat2.csv n=quantity,amount o=rsl2.csv s=customer
$ more rsl2.csv
@RELATION	Purchase_Data

@ATTRIBUTE	customer	string
@ATTRIBUTE	date	date yyyyMMddHHmmss
@ATTRIBUTE	quantity	numeric
@ATTRIBUTE	amount	numeric
@ATTRIBUTE	product	{A,B}

@DATA
No.1,20081201102030,1,10,A
No.2,20081202123010,2,20,A
No.3,20081203153010,3,30,A
No.4,20081201174010,4,40,B
No.5,20081203133010,5,50,B
\end{Verbatim}

\subsection*{関連コマンド}

\hyperref[sect:marff2csv]{marff2csv} : 逆変換


\subsection*{参考資料}
\href{http://weka.wikispaces.com/ARFF}{http://weka.wikispaces.com/ARFF}

%\end{document}
