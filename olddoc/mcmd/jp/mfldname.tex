
%\documentclass[a4paper]{jsbook}
%\usepackage{mcmd_jp}
%\begin{document}

\section{mfldname 項目名の変更\label{sect:mfldname}}
\index{mfldname@mfldname}
\verb|f=|で指定した項目名を変更する。又、\verb|n=|で項目名を新規設定する。

\subsection*{書式}
\verb/mfldname f=|n= [-nfni]/
\hyperref[sect:option_i]{[i=]}
\hyperref[sect:option_o]{[o=]}
\hyperref[sect:option_assert_diffSize]{[-assert\_diffSize]}
\hyperref[sect:option_nfn]{[-nfn]}
\hyperref[sect:option_nfno]{[-nfno]}
\hyperref[sect:option_x]{[-x]}
\hyperref[sect:option_q]{[-q]}
\hyperref[sect:option_option_tmppath]{[tmpPath=]}
\hyperref[sect:option_precision]{[precision=]}
\verb|[-params]|
\verb|[--help]|
\verb|[--helpl]|
\verb|[--version]|\\

\subsection*{パラメータ}
\begin{table}[htbp]
%\begin{center}
{\small
\begin{tabular}{ll}
\verb|i=|    & 入力ファイル名を指定する。\\
\verb|o=|    & 出力ファイル名を指定する。\\
\verb|f=|    & ここで指定された項目名のみを変更する。(現項目名:新項目名)\\
             & 指定していない項目名は変更せずに元の項目名が出力される。\\
\verb|n=|    & ここで指定された項目名が新たに採用される。\\
             & データ項目数と同じ数の項目名を指定する必要がある。\\
\verb|-nfni| & 入力データの1行目を項目名行とみなさない。このオプションが指定された場合は\verb|f=|は利用できない。\\
\end{tabular} 
}
\end{table} 

\subsection*{利用例}
\subsubsection*{例1: 基本例}

項目名の「顧客」を「cust」に、「10月」を「Oct.」に変更する。


\begin{Verbatim}[baselinestretch=0.7,frame=single]
$ more dat1.csv
顧客,itemID,10月
a,xx,11
b,yy,122
c,zz,
$ mfldname f=顧客:cust,10月:Oct. i=dat1.csv o=rsl1.csv
#END# kgfldname f=顧客:cust,10月:Oct. i=dat1.csv o=rsl1.csv
$ more rsl1.csv
cust,itemID,Oct.
a,xx,11
b,yy,122
c,zz,
\end{Verbatim}
\subsubsection*{例2: 項目名変更}

項目名を\verb|x,y,z|に変更する。


\begin{Verbatim}[baselinestretch=0.7,frame=single]
$ mfldname n=x,y,z i=dat1.csv o=rsl2.csv
#END# kgfldname i=dat1.csv n=x,y,z o=rsl2.csv
$ more rsl2.csv
x,y,z
a,xx,11
b,yy,122
c,zz,
\end{Verbatim}
\subsubsection*{例3: 項目名行がないデータ}



\begin{Verbatim}[baselinestretch=0.7,frame=single]
$ more dat2.csv
a,xx,11
b,yy,122
c,zz,
$ mfldname -nfni n=x,y,z i=dat2.csv o=rsl3.csv
#END# kgfldname -nfni i=dat2.csv n=x,y,z o=rsl3.csv
$ more rsl3.csv
x,y,z
a,xx,11
b,yy,122
c,zz,
\end{Verbatim}

\subsection*{関連コマンド}
\hyperref[sect:mcut]{mcut} : \verb|mfldname|と同じことができるが、一部の項目名を変更するには少し面倒。また\verb|mfldname|の方が少しだけ高速。

%\end{document}
