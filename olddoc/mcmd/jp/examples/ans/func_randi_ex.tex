
\subsubsection*{例1: 基本例}

100から999(3桁整数900種類)の整数乱数を生成する。
乱数の種を指定しているので、何度実行しても同じ乱数系列が生成される。

\begin{Verbatim}[baselinestretch=0.7,frame=single]
$ cat dat1.csv
id
1
2
3
4

$ mcal c='randi(100,999,1)' a=rsl i=dat1.csv o=rsl1.csv
#END# kgcal a=rsl c=randi(100,999,1) i=dat1.csv o=rsl1.csv

$ cat rsl1.csv
id,rsl
1,475
2,997
3,748
4,939
\end{Verbatim}

\subsubsection*{例2: 0,1の整数乱数}

0と1の2種類の整数乱数を生成する。
乱数の種を指定していないので、実行の度に異なる乱数系列が生成される。

\begin{Verbatim}[baselinestretch=0.7,frame=single]
$ mcal c='randi(0,1)' a=rsl i=dat1.csv o=rsl2.csv
#END# kgcal a=rsl c=randi(0,1) i=dat1.csv o=rsl2.csv

$ cat rsl2.csv
id,rsl
1,0
2,1
3,0
4,1
\end{Verbatim}
