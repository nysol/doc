
\subsubsection*{例1: 基本例}


\begin{Verbatim}[baselinestretch=0.7,frame=single]
$ cat dat1.csv
id,v1,v2,v3
1,1,2,3
2,-5,2,1
3,1,,3
4,,,

$ mcal c='product(${v1},${v2},${v3})' a=rsl i=dat1.csv o=rsl1.csv
#END# kgcal a=rsl c=product(${v1},${v2},${v3}) i=dat1.csv o=rsl1.csv

$ cat rsl1.csv
id,v1,v2,v3,rsl
1,1,2,3,6
2,-5,2,1,-10
3,1,,3,3
4,,,,
\end{Verbatim}

\subsubsection*{例2: ワイルドカードを利用した例}

\verb|v|から始まる項目(\verb|v1,v2,v3|)をワイルドカード「\verb|v*|」によって指定している。

\begin{Verbatim}[baselinestretch=0.7,frame=single]
$ mcal c='product(${v*})' a=rsl i=dat1.csv o=rsl2.csv
#END# kgcal a=rsl c=product(${v*}) i=dat1.csv o=rsl2.csv

$ cat rsl2.csv
id,v1,v2,v3,rsl
1,1,2,3,6
2,-5,2,1,-10
3,1,,3,3
4,,,,
\end{Verbatim}
