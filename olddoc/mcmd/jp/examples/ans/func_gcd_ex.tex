
\subsubsection*{例1: 基本例}


\begin{Verbatim}[baselinestretch=0.7,frame=single]
$ cat dat1.csv
id,val1,val2
1,12,36
2,6,5
3,,
4,12.1,36.2

$ mcal c='gcd(${val1},${val2})' a=rsl i=dat1.csv o=rsl1.csv
#END# kgcal a=rsl c=gcd(${val1},${val2}) i=dat1.csv o=rsl1.csv

$ cat rsl1.csv
id,val1,val2,rsl
1,12,36,12
2,6,5,1
3,,,
4,12.1,36.2,12
\end{Verbatim}

\subsubsection*{例2: 定数を与える例}

val1項目と36の最大公約数を求める。

\begin{Verbatim}[baselinestretch=0.7,frame=single]
$ mcal c='gcd(${val1},36)' a=rsl i=dat1.csv o=rsl2.csv
#END# kgcal a=rsl c=gcd(${val1},36) i=dat1.csv o=rsl2.csv

$ cat rsl2.csv
id,val1,val2,rsl
1,12,36,12
2,6,5,6
3,,,
4,12.1,36.2,12
\end{Verbatim}
