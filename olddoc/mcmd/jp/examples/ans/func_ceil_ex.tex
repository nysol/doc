
\subsubsection*{例1: 基本例}

小数点以下一桁目を切り上げる。

\begin{Verbatim}[baselinestretch=0.7,frame=single]
$ cat dat1.csv
id,val
1,3.28
2,3.82
3,
4,-0.6

$ mcal c='floor(${val})' a=rsl i=dat1.csv o=rsl1.csv
#END# kgcal a=rsl c=floor(${val}) i=dat1.csv o=rsl1.csv

$ cat rsl1.csv
id,val,rsl
1,3.28,3
2,3.82,3
3,,
4,-0.6,-1
\end{Verbatim}

\subsubsection*{例2: 基本例}

小数点以下二桁目を切り上げる。

\begin{Verbatim}[baselinestretch=0.7,frame=single]
$ mcal c='floor(${val},0.1)' a=rsl i=dat1.csv o=rsl2.csv
#END# kgcal a=rsl c=floor(${val},0.1) i=dat1.csv o=rsl2.csv

$ cat rsl2.csv
id,val,rsl
1,3.28,3.2
2,3.82,3.8
3,,
4,-0.6,-0.6
\end{Verbatim}

\subsubsection*{例3: 基数0.5の例}

0.5を基数として切り上げる。

\begin{Verbatim}[baselinestretch=0.7,frame=single]
$ mcal c='floor(${val},0.5)' a=rsl i=dat1.csv o=rsl3.csv
#END# kgcal a=rsl c=floor(${val},0.5) i=dat1.csv o=rsl3.csv

$ cat rsl3.csv
id,val,rsl
1,3.28,3
2,3.82,3.5
3,,
4,-0.6,-1
\end{Verbatim}

\subsubsection*{例4: 基数10の例}

一桁目を切り上げる。

\begin{Verbatim}[baselinestretch=0.7,frame=single]
$ cat dat2.csv
id,val
1,1341.28
2,188
3,1.235E+3
4,-1.235E+3

$ mcal c='round(${val},10)' a=rsl i=dat2.csv o=rsl4.csv
#END# kgcal a=rsl c=round(${val},10) i=dat2.csv o=rsl4.csv

$ cat rsl4.csv
id,val,rsl
1,1341.28,1340
2,188,190
3,1.235E+3,1240
4,-1.235E+3,-1230
\end{Verbatim}
