\subsubsection*{例1: 基本例}

「数量:売上数量」の指定により、項目名が「数量」から「売上数量」に変換されて出力される。


\begin{Verbatim}[baselinestretch=0.7,frame=single]
$ more dat1.csv
ブランド,数量
A,10
B,20
C,30
D,40
$ mcut f=ブランド,数量:売上数量 i=dat1.csv o=rsl1.csv
#END# kgcut f=ブランド,数量:売上数量 i=dat1.csv o=rsl1.csv
$ more rsl1.csv
ブランド,売上数量
A,10
B,20
C,30
D,40
\end{Verbatim}
\subsubsection*{例2: 追加項目名}

以下のmaccumコマンドでは、「ブランド」項目で並べ替えた後、
「数量」項目の値を累積し、「累積数量」項目として追加出力する。
もし、「f=数量」とだけすれば、累積結果も「数量」という名の項目となり、
オリジナルの「数量」項目とダブってしまいエラーとなる。


\begin{Verbatim}[baselinestretch=0.7,frame=single]
$ maccum s=ブランド f=数量:累積数量 i=dat1.csv o=rsl2.csv
#END# kgaccum f=数量:累積数量 i=dat1.csv o=rsl2.csv s=ブランド
$ more rsl2.csv
ブランド%0,数量,累積数量
A,10,10
B,20,30
C,30,60
D,40,100
$ maccum s=ブランド f=数量 i=dat1.csv o=rsl2.csv
#ERROR# same field name is specified: 数量 (kgaccum)
\end{Verbatim}
\subsubsection*{例3: 番号指定との混在}

番号指定と出力項目名指定を混在させることも可能である。


\begin{Verbatim}[baselinestretch=0.7,frame=single]
$ mcut f=0,1:売上数量 -x i=dat1.csv o=rsl3.csv
#END# kgcut -x f=0,1:売上数量 i=dat1.csv o=rsl3.csv
$ more rsl3.csv
ブランド,売上数量
A,10
B,20
C,30
D,40
\end{Verbatim}
