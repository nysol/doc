\subsubsection*{例1: 基本例}

\verb|dat1.csv|ファイルを\verb|rsl1.csv|と\verb|rsl2.csv|という2つのファイルにコピーする。
また、標準出力に出力されるので、画面上に内容が出力される。


\begin{Verbatim}[baselinestretch=0.7,frame=single]
$ more dat1.csv
customer,quantity,price
A,1,10
A,2,20
B,1,15
$ mtee i=dat1.csv o=rsl1.csv,rsl2.csv
customer,quantity,price
A,1,10
A,2,20
B,1,15
#END# kgtee i=dat1.csv o=rsl1.csv,rsl2.csv
$ more rsl1.csv
customer,quantity,price
A,1,10
A,2,20
B,1,15
$ more rsl2.csv
customer,quantity,price
A,1,10
A,2,20
B,1,15
\end{Verbatim}
\subsubsection*{例2: 標準出力なし}

\verb|-nostdout|を指定すると、\verb|rsl1.csv|と\verb|rsl2.csv|という2つのファイルにコピーのみ行い、
標準出力には出力しない。


\begin{Verbatim}[baselinestretch=0.7,frame=single]
$ mtee i=dat1.csv o=rsl1.csv,rsl2,csv -nostdout
#END# kgtee -nostdout i=dat1.csv o=rsl1.csv,rsl2,csv
\end{Verbatim}
