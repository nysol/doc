\subsubsection*{例1: 基本例}

入力ファイルにある\verb|item|項目と、
参照ファイルにある\verb|item|項目を比較し同じ値の場合、\verb|cost|項目を結合する。
入力ファイル、参照ファイル共に\verb|item=A|が2行あり、結果、出力ファイルには2$\times$2=4行の\verb|item=A|が出力されている。


\begin{Verbatim}[baselinestretch=0.7,frame=single]
$ more dat1.csv
item,date,price
A,20081201,100
A,20081213,98
B,20081002,400
B,20081209,450
C,20081201,100
$ more ref1.csv
item,cost
A,50
A,70
B,300
E,200
$ mnjoin k=item f=cost m=ref1.csv i=dat1.csv o=rsl1.csv
#END# kgnjoin f=cost i=dat1.csv k=item m=ref1.csv o=rsl1.csv
$ more rsl1.csv
item%0,date,price,cost
A,20081201,100,50
A,20081201,100,70
A,20081213,98,50
A,20081213,98,70
B,20081002,400,300
B,20081209,450,300
\end{Verbatim}
\subsubsection*{例2: 未結合データ出力}

\verb|-n|を指定することで、参照ファイルにマッチしない入力ファイルの行(\verb|item="C"|の行)も出力し、
\verb|-N|を指定することで、入力ファイルにマッチしない参照ファイルの行(\verb|item="E"|の行)も出力する。


\begin{Verbatim}[baselinestretch=0.7,frame=single]
$ more ref2.csv
item,cost
A,50
B,300
E,200
$ mnjoin k=item f=cost m=ref2.csv -n -N i=dat1.csv o=rsl2.csv
#END# kgnjoin -N -n f=cost i=dat1.csv k=item m=ref2.csv o=rsl2.csv
$ more rsl2.csv
item%0,date,price,cost
A,20081201,100,50
A,20081213,98,50
B,20081002,400,300
B,20081209,450,300
C,20081201,100,
E,,,200
\end{Verbatim}
