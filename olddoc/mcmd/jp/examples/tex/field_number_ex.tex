\subsubsection*{例1: 範囲指定}

「0-4」の指定により、「0,1,2,3,4」を指定したことになる。


\begin{Verbatim}[baselinestretch=0.7,frame=single]
$ more dat1.csv
ブランド,数量01,数量02,数量03,数量04,数量05,数量06,数量07,数量08,数量09,数量10
A,10,50,90,130,170,210,250,290,330,370
B,20,60,100,140,180,220,260,300,340,380
C,30,70,110,150,190,230,270,310,350,390
D,40,80,120,160,200,240,280,320,360,400
$ mcut -x f=0-4 i=dat1.csv o=rsl1.csv
#END# kgcut -x f=0-4 i=dat1.csv o=rsl1.csv
$ more rsl1.csv
ブランド,数量01,数量02,数量03,数量04
A,10,50,90,130
B,20,60,100,140
C,30,70,110,150
D,40,80,120,160
\end{Verbatim}
\subsubsection*{例2: 範囲指定逆順}

「4-0」の指定により、「4,3,2,1,0」を指定したことになる。


\begin{Verbatim}[baselinestretch=0.7,frame=single]
$ mcut -x f=4-0 i=dat1.csv o=rsl2.csv
#END# kgcut -x f=4-0 i=dat1.csv o=rsl2.csv
$ more rsl2.csv
数量04,数量03,数量02,数量01,ブランド
130,90,50,10,A
140,100,60,20,B
150,110,70,30,C
160,120,80,40,D
\end{Verbatim}
\subsubsection*{例3: 複数範囲指定}

「1-0,2-4」の指定により、「1,0,2,3,4」を指定したことになる。


\begin{Verbatim}[baselinestretch=0.7,frame=single]
$ mcut -x f=1-0,2-4 i=dat1.csv o=rsl3.csv
#END# kgcut -x f=1-0,2-4 i=dat1.csv o=rsl3.csv
$ more rsl3.csv
数量01,ブランド,数量02,数量03,数量04
10,A,50,90,130
20,B,60,100,140
30,C,70,110,150
40,D,80,120,160
\end{Verbatim}
\subsubsection*{例4: 末尾項目からの指定}

「2L」の指定により、項目の後ろから数えた2番項目(数量08)を指定したことになる。


\begin{Verbatim}[baselinestretch=0.7,frame=single]
$ mcut -x f=2L i=dat1.csv o=rsl4.csv
#END# kgcut -x f=2L i=dat1.csv o=rsl4.csv
$ more rsl4.csv
数量08
290
300
310
320
\end{Verbatim}
\subsubsection*{例5: 末尾項目からの指定と範囲指定}

「5-3L」の指定により、5番項目~後ろから3番目の項目、すなわち「5,6,7」を指定したことになる。


\begin{Verbatim}[baselinestretch=0.7,frame=single]
$ mcut -x f=5-3L i=dat1.csv o=rsl5.csv
#END# kgcut -x f=5-3L i=dat1.csv o=rsl5.csv
$ more rsl5.csv
数量05,数量06,数量07
170,210,250
180,220,260
190,230,270
200,240,280
\end{Verbatim}
