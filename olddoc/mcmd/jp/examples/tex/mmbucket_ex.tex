\subsubsection*{例1: 基本例}

\verb|x、y|項目の件数ができるだけ多次元均等になるように2分割する。
その際、各バケットの数値範囲を\verb|rng.csv|という名前のファイルに出力する。


\begin{Verbatim}[baselinestretch=0.7,frame=single]
$ more dat1.csv
id,x,y
A,2,7
B,6,7
C,5,6
D,7,5
E,6,4
F,1,3
G,3,3
H,4,2
I,7,2
J,2,1
$ mmbucket f=x:xb,y:yb n=2,2 O=rng.csv i=dat1.csv o=rsl1.csv
calculating on dimension ... #0 #1 done. VAR=30 updated!
calculating on dimension ... #0 #1 done. VAR=28 updated!
calculating on dimension ... #0 #1 done. VAR=28
#END# kgmbucket O=rng.csv f=x:xb,y:yb i=dat1.csv n=2,2 o=rsl1.csv
$ more rsl1.csv
id,x,y,xb,yb
A,2,7,1,2
B,6,7,2,2
C,5,6,2,2
D,7,5,2,2
E,6,4,2,1
F,1,3,1,1
G,3,3,1,1
H,4,2,2,1
I,7,2,2,1
J,2,1,1,1
$ more rng.csv
fieldName,bucketNo,rangeFrom,rangeTo
x,1,,3.5
x,2,3.5,
y,1,,4.5
y,2,4.5,
\end{Verbatim}
\subsubsection*{例2: 出力形式}

\verb|id|項目を単位に件数ができるだけ多次元均等になるように\verb|x,y|項目を2分割する。
出力形式はバケット番号とバケット範囲の両方を表示する。


\begin{Verbatim}[baselinestretch=0.7,frame=single]
$ more dat2.csv
id,x,y
A,2,7
A,6,7
A,5,6
B,7,5
B,6,4
B,1,3
C,3,3
C,4,2
C,7,2
C,2,1
$ mmbucket k=id f=x:xb,y:yb n=2,2 F=2 i=dat2.csv o=rsl2.csv
calculating on dimension ... #0 #1 done. VAR=3 updated!
calculating on dimension ... #0 #1 done. VAR=3
calculating on dimension ... #0 #1 done. VAR=3 updated!
calculating on dimension ... #0 #1 done. VAR=3
calculating on dimension ... #0 #1 done. VAR=6 updated!
calculating on dimension ... #0 #1 done. VAR=6
#END# kgmbucket F=2 f=x:xb,y:yb i=dat2.csv k=id n=2,2 o=rsl2.csv
$ more rsl2.csv
id%0,x,y,xb,yb
A,2,7,1:_3.5,2:6.5_
A,6,7,2:3.5_,2:6.5_
A,5,6,2:3.5_,1:_6.5
B,7,5,2:3.5_,2:4.5_
B,6,4,2:3.5_,1:_4.5
B,1,3,1:_3.5,1:_4.5
C,3,3,1:_3.5,2:1.5_
C,4,2,2:3.5_,2:1.5_
C,7,2,2:3.5_,2:1.5_
C,2,1,1:_3.5,1:_1.5
\end{Verbatim}
