\subsubsection*{例1: 基本例}

\verb|no|項目が整数(\%n)として連続するようにレコードをパディングする。
\verb|1|とverb|4|の間に\verb|2,3|を、\verb|4|と\verb|2|の間に\verb|3|が挿入されている。


\begin{Verbatim}[baselinestretch=0.7,frame=single]
$ more dat1.csv
no
3
6
8
$ mpadding f=no%n i=dat1.csv o=rsl1.csv
#END# kgpadding f=no%n i=dat1.csv o=rsl1.csv
$ more rsl1.csv
no%0n
3
4
5
6
7
8
\end{Verbatim}
\subsubsection*{例2: 開始値、終了値の指定}

行間のパディングだけでなく、先頭行/終端行の前後もパディングする。
前後の範囲は\verb|S=,E=|で指定する。


\begin{Verbatim}[baselinestretch=0.7,frame=single]
$ mpadding f=no%n S=1 E=10 i=dat1.csv o=rsl2.csv
#END# kgpadding E=10 S=1 f=no%n i=dat1.csv o=rsl2.csv
$ more rsl2.csv
no%0n
1
2
3
4
5
6
7
8
9
10
\end{Verbatim}
\subsubsection*{例3: 日付パディング}

\verb|date|項目が日付(\%d)として連続するようにレコードをパディングする。
\verb|k=,f=|で指定した以外の項目は、直前の行の項目値でパディングする。


\begin{Verbatim}[baselinestretch=0.7,frame=single]
$ more dat2.csv
date,dummy
20130929,a
20131002,b
20131004,c
$ mpadding f=date%d i=dat2.csv o=rsl3.csv
#END# kgpadding f=date%d i=dat2.csv o=rsl3.csv
$ more rsl3.csv
date%0,dummy
20130929,a
20130930,a
20131001,a
20131002,b
20131003,b
20131004,c
\end{Verbatim}
\subsubsection*{例4: パディング用文字列指定}

\verb|v=|にてパディング文字列を指定することもできる。


\begin{Verbatim}[baselinestretch=0.7,frame=single]
$ mpadding f=date%d v=padding i=dat2.csv o=rsl4.csv
#END# kgpadding f=date%d i=dat2.csv o=rsl4.csv v=padding
$ more rsl4.csv
date%0,dummy
20130929,a
20130930,padding
20131001,padding
20131002,b
20131003,padding
20131004,c
\end{Verbatim}
\subsubsection*{例5: パディングにNULL値を指定}

\verb|-n|を指定してNULL値でパディングすることも可能。


\begin{Verbatim}[baselinestretch=0.7,frame=single]
$ mpadding f=date%d -n i=dat2.csv o=rsl5.csv
#END# kgpadding -n f=date%d i=dat2.csv o=rsl5.csv
$ more rsl5.csv
date%0,dummy
20130929,a
20130930,
20131001,
20131002,b
20131003,
20131004,c
\end{Verbatim}
