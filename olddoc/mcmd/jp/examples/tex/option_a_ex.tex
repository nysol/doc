\subsubsection*{例1: 基本例}

精算日という新しい項目を追加する。


\begin{Verbatim}[baselinestretch=0.7,frame=single]
$ more dat1.csv
id
A
B
C
$ msetstr v=20070101 a=精算日 i=dat1.csv o=rsl1.csv
#END# kgsetstr a=精算日 i=dat1.csv o=rsl1.csv v=20070101
$ more rsl1.csv
id,精算日
A,20070101
B,20070101
C,20070101
\end{Verbatim}
\subsubsection*{例2: 複数項目を追加}

id項目のデータ\verb|A,B,C|の2つの組合わせを2つの項目(id1,id2)として出力する。


\begin{Verbatim}[baselinestretch=0.7,frame=single]
$ mcombi f=id n=2 a=id1,id2 i=dat1.csv o=rsl2.csv
#END# kgcombi a=id1,id2 f=id i=dat1.csv n=2 o=rsl2.csv
$ more rsl2.csv
id,id1,id2
C,A,B
C,A,C
C,B,C
\end{Verbatim}
