\subsubsection*{例1: 基本例}

項目\verb|val1|と\verb|val2|を切り出す。


\begin{Verbatim}[baselinestretch=0.7,frame=single]
$ more dat1.csv
id,val1,val2
A,1,2
B,2,3
C,3,4
$ mcut f=val1,val2 i=dat1.csv o=rsl1.csv
#END# kgcut f=val1,val2 i=dat1.csv o=rsl1.csv
$ more rsl1.csv
val1,val2
1,2
2,3
3,4
\end{Verbatim}
\subsubsection*{例2: 出力項目名の指定}

\verb|val1,val2|を集計し、\verb|sum1,sum2|という項目名で出力する。


\begin{Verbatim}[baselinestretch=0.7,frame=single]
$ msum f=val1:sum1,val2:sum2 i=dat1.csv o=rsl2.csv
#END# kgsum f=val1:sum1,val2:sum2 i=dat1.csv o=rsl2.csv
$ more rsl2.csv
id,sum1,sum2
C,6,9
\end{Verbatim}
