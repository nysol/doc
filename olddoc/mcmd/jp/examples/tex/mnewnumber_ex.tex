\subsubsection*{例1: 基本例}

1から始まる間隔が1の連番をふり、\verb|No.|という項目名で新規に5行のデータを作成する。


\begin{Verbatim}[baselinestretch=0.7,frame=single]
$ mnewnumber a=No. I=1 S=1 l=5 o=rsl1.csv
#END# kgNewnumber I=1 S=1 a=No. l=5 o=rsl1.csv
$ more rsl1.csv
No.
1
2
3
4
5
\end{Verbatim}
\subsubsection*{例2: 開始番号と間隔の変更}

10から始まる間隔が5の連番をふり、\verb|No.|という項目名で新規に5行のデータを作成する。


\begin{Verbatim}[baselinestretch=0.7,frame=single]
$ mnewnumber a=No. I=5 S=10 l=5 o=rsl2.csv
#END# kgNewnumber I=5 S=10 a=No. l=5 o=rsl2.csv
$ more rsl2.csv
No.
10
15
20
25
30
\end{Verbatim}
\subsubsection*{例3: アルファベット連番}

Aから始まる間隔が1のアルファベット連番をふり、\verb|No.|という項目名で新規に5行のデータを作成する。


\begin{Verbatim}[baselinestretch=0.7,frame=single]
$ mnewnumber a=No. I=1 S=A l=5 o=rsl3.csv
#END# kgNewnumber I=1 S=A a=No. l=5 o=rsl3.csv
$ more rsl3.csv
No.
A
B
C
D
E
\end{Verbatim}
\subsubsection*{例4: ヘッダ行なしで新規作成}

Bから始まる間隔が3のアルファベット連番をふり、ヘッダなしで新規に11行のデータを作成する。


\begin{Verbatim}[baselinestretch=0.7,frame=single]
$ mnewnumber  -nfn  I=3 l=11 S=B o=rsl4.csv
#END# kgNewnumber -nfn I=3 S=B l=11 o=rsl4.csv
$ more rsl4.csv
B
E
H
K
N
Q
T
W
Z
AC
AF
\end{Verbatim}
