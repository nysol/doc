\subsubsection*{例1: 基本例}

「顧客」項目を単位にして、「数量」と「金額」項目の平均を計算する。


\begin{Verbatim}[baselinestretch=0.7,frame=single]
$ more dat1.csv
顧客,数量,金額
A,1,
B,,15
A,2,20
B,3,10
B,1,20
$ mhashavg k=顧客 f=数量,金額 i=dat1.csv o=rsl1.csv
#END# kghashavg f=数量,金額 i=dat1.csv k=顧客 o=rsl1.csv
$ more rsl1.csv
顧客,数量,金額
A,1.5,20
B,2,15
\end{Verbatim}
\subsubsection*{例2: NULL値の出力}

\verb|-n|オプションを指定することで、NULL値が含まれている場合は、結果もNULL値として出力する。


\begin{Verbatim}[baselinestretch=0.7,frame=single]
$ mhashavg k=顧客 f=数量,金額 -n i=dat1.csv o=rsl2.csv
#END# kghashavg -n f=数量,金額 i=dat1.csv k=顧客 o=rsl2.csv
$ more rsl2.csv
顧客,数量,金額
A,1.5,
B,,15
\end{Verbatim}
