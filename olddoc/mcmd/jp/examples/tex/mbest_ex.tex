\subsubsection*{例1: 基本例}

この例では、「数量」と「金額」項目値の大きい順(降順)に並べ替えている。
\verb|from=|,\verb|to=|,\verb|size=|を指定しなければ先頭行(0行目)のみ選択する


\begin{Verbatim}[baselinestretch=0.7,frame=single]
$ more dat1.csv
顧客,数量,金額
A,20,5200 
B,18,4000   
C,15,3500 
D,10,2000 
E,3,800 
$ mbest s=数量%nr,金額%nr i=dat1.csv o=rsl1.csv
#END# kgbest i=dat1.csv o=rsl1.csv s=数量%nr,金額%nr
$ more rsl1.csv
顧客,数量%0nr,金額%1nr
A,20,5200 
\end{Verbatim}
\subsubsection*{例2: 基本例2}

「顧客」で並べ替えた後、先頭行(0行目)から3行選択する


\begin{Verbatim}[baselinestretch=0.7,frame=single]
$ mbest s=顧客 from=0 size=3 i=dat1.csv o=rsl2.csv
#END# kgbest from=0 i=dat1.csv o=rsl2.csv s=顧客 size=3
$ more rsl2.csv
顧客%0,数量,金額
A,20,5200 
B,18,4000   
C,15,3500 
\end{Verbatim}
\subsubsection*{例3: 基本例3}

並べ替えを行わず(もとのレコード順序を維持したまま)、0行目から1行目まで選択する


\begin{Verbatim}[baselinestretch=0.7,frame=single]
$ mbest -q from=0 to=1 i=dat1.csv o=rsl3.csv
#END# kgbest -q from=0 i=dat1.csv o=rsl3.csv to=1
$ more rsl3.csv
顧客,数量,金額
A,20,5200 
B,18,4000   
\end{Verbatim}
\subsubsection*{例4: 条件反転}

顧客の初回来店日以外の行を選択する。
初回来店日は\verb|fvd.csv|というファイルに出力する。


\begin{Verbatim}[baselinestretch=0.7,frame=single]
$ more dat2.csv
顧客,日付,金額
A,20081201,10
A,20081207,20
A,20081213,30
B,20081002,40
B,20081209,50
$ mbest s=顧客,日付 k=顧客 -r u=fvd.csv i=dat2.csv o=rsl4.csv
#END# kgbest -r i=dat2.csv k=顧客 o=rsl4.csv s=顧客,日付 u=fvd.csv
$ more rsl4.csv
顧客,日付,金額
A,20081207,20
A,20081213,30
B,20081209,50
$ more fvd.csv
顧客,日付,金額
A,20081201,10
B,20081002,40
\end{Verbatim}
