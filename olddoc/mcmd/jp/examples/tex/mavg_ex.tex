\subsubsection*{例1: 基本例}

「顧客」項目を単位に「数量」と「金額」項目の平均値を計算し、「数量平均」と「金額平均」という項目名で出力する。


\begin{Verbatim}[baselinestretch=0.7,frame=single]
$ more dat1.csv
顧客,数量,金額
A,1,5
A,2,20
B,1,15
B,,10
B,5,20
$ mavg k=顧客 f=数量:数量平均,金額:金額平均 i=dat1.csv o=rsl1.csv
#END# kgavg f=数量:数量平均,金額:金額平均 i=dat1.csv k=顧客 o=rsl1.csv
$ more rsl1.csv
顧客%0,数量平均,金額平均
A,1.5,12.5
B,3,15
\end{Verbatim}
\subsubsection*{例2: NULL値がある場合の出力}

「顧客」項目を単位に「数量」と「金額」項目の平均値を計算し、「数量平均」と「金額平均」という項目名で出力する。
\verb|-n|オプションを指定することで、NULL値が含まれている場合は、結果もNULL値として出力する。


\begin{Verbatim}[baselinestretch=0.7,frame=single]
$ mavg k=顧客 f=数量:数量平均,金額:金額平均 -n i=dat1.csv o=rsl2.csv
#END# kgavg -n f=数量:数量平均,金額:金額平均 i=dat1.csv k=顧客 o=rsl2.csv
$ more rsl2.csv
顧客%0,数量平均,金額平均
A,1.5,12.5
B,,15
\end{Verbatim}
\subsubsection*{例3: 顧客項目を単位としない例}

「数量」と「金額」項目の平均値を計算し、「数量平均」と「金額平均」という項目名で出力する。


\begin{Verbatim}[baselinestretch=0.7,frame=single]
$ mavg f=数量:数量平均,金額:金額平均 i=dat1.csv o=rsl3.csv
#END# kgavg f=数量:数量平均,金額:金額平均 i=dat1.csv o=rsl3.csv
$ more rsl3.csv
顧客,数量平均,金額平均
B,2.25,14
\end{Verbatim}
