\subsubsection*{例1: 基本例}

\verb|customer|を単位に\verb|item|をスペース区切りで結合し、
\verb|transaction|という項目名で出力する。


\begin{Verbatim}[baselinestretch=0.7,frame=single]
$ more dat1.csv
customer,item
A,a
A,b
B,c
B,d
B,e
$ mtra k=customer f=item:transaction i=dat1.csv o=rsl1.csv
#END# kgtra f=item:transaction i=dat1.csv k=customer o=rsl1.csv
$ more rsl1.csv
customer%0,transaction
A,a b
B,c d e
\end{Verbatim}
\subsubsection*{例2: アイテムの区切り文字を-(ハイフン)で実行}



\begin{Verbatim}[baselinestretch=0.7,frame=single]
$ mtra k=customer f=item:transaction delim=- i=dat1.csv o=rsl2.csv
#END# kgtra delim=- f=item:transaction i=dat1.csv k=customer o=rsl2.csv
$ more rsl2.csv
customer%0,transaction
A,a-b
B,c-d-e
\end{Verbatim}
\subsubsection*{例3: アイテムを降順に並べ替えてから変換}



\begin{Verbatim}[baselinestretch=0.7,frame=single]
$ mtra k=customer s=item%r f=item:transaction i=dat1.csv o=rsl3.csv
#END# kgtra f=item:transaction i=dat1.csv k=customer o=rsl3.csv s=item%r
$ more rsl3.csv
customer%0,transaction
A,b a
B,e d c
\end{Verbatim}
