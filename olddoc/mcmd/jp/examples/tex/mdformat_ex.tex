\subsubsection*{例1: 基本例}

\verb|fld|項目から日付・時刻を抽出し変換する。
\verb|fld|項目には「date:年月日 time:時分秒.マイクロ秒」の形式で日付・時刻が格納されているので、
\verb|c=|パラメータには「\verb|date:%Y%m%d time:%H%M%s|」と指定している。


\begin{Verbatim}[baselinestretch=0.7,frame=single]
$ more dat1.csv
fld
date:20120304 time:121212
date:20101204 time:011309.1234
$ mdformat f=fld c='date:%Y%m%d time:%H%M%s' i=dat1.csv o=rsl1.csv
#END# kgdformat c=date:%Y%m%d time:%H%M%s f=fld i=dat1.csv o=rsl1.csv
$ more rsl1.csv
fld
20120304121212
20101204011309.1234
\end{Verbatim}
\subsubsection*{例2: 項目の追加}

\verb|fld1|項目、\verb|fld2|項目には「年/月/日」形式で日付が格納されているので、
\verb|c=|パラメータには「\verb|%Y/%m/%d|」と指定している。
\verb|-A|オプションを使用し、変換結果を新たな\verb|f1|、\verb|f2|項目に抽出する。


\begin{Verbatim}[baselinestretch=0.7,frame=single]
$ more dat2.csv
fld,fld2
2010/11/20,2010/11/21
2010/1/1,2010/1/2
2011/01/01,2010/01/02
2010/1/01,2010/1/02
$ mdformat f=fld:f1,fld2:f2 c=%Y/%m/%d i=dat2.csv -A o=rsl2.csv
#END# kgdformat -A c=%Y/%m/%d f=fld:f1,fld2:f2 i=dat2.csv o=rsl2.csv
$ more rsl2.csv
fld,fld2,f1,f2
2010/11/20,2010/11/21,20101120,20101121
2010/1/1,2010/1/2,20100101,20100102
2011/01/01,2010/01/02,20110101,20100102
2010/1/01,2010/1/02,20100101,20100102
\end{Verbatim}
\subsubsection*{例3: 抽出がうまくいかない例}

\verb|fld|項目には「年 月 日 時:分:秒」形式で日付が格納されているので、
\verb|c=|パラメータには「\verb|%Y %m %d %H:%M:%S|」と指定している。
しかし形式が異なる行は抽出に失敗している。


\begin{Verbatim}[baselinestretch=0.7,frame=single]
$ more dat3.csv
fld
2010 11 20 12:34:56
2011 01 01 23:34:56
2010  1 01 123455
$ mdformat f=fld:f1 c='%Y %m %d %H:%M:%S' i=dat3.csv -A o=rsl3.csv
#END# kgdformat -A c=%Y %m %d %H:%M:%S f=fld:f1 i=dat3.csv o=rsl3.csv
$ more rsl3.csv
fld,f1
2010 11 20 12:34:56,20101120123456
2011 01 01 23:34:56,20110101233456
2010  1 01 123455,
\end{Verbatim}
