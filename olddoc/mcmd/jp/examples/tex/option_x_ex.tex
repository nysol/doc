\subsubsection*{例1: 基本例}

0番目項目を集計キーとして1番目と2番目の項目を合計する。


\begin{Verbatim}[baselinestretch=0.7,frame=single]
$ more dat1.csv
顧客,数量,金額
A,1,10
A,2,20
B,1,15
B,3,10
B,1,20
$ msum -x k=0 f=1,2 i=dat1.csv o=rsl1.csv
#END# kgsum -x f=1,2 i=dat1.csv k=0 o=rsl1.csv
$ more rsl1.csv
顧客%0,数量,金額
A,3,30
B,5,45
\end{Verbatim}
\subsubsection*{例2: 出力項目名も利用可能}

1番目と2番目の項目は、\verb|a,b|という名前で出力する。


\begin{Verbatim}[baselinestretch=0.7,frame=single]
$ msum -x k=0 f=1:a,2:b i=dat1.csv o=rsl2.csv
#END# kgsum -x f=1:a,2:b i=dat1.csv k=0 o=rsl2.csv
$ more rsl2.csv
顧客%0,a,b
A,3,30
B,5,45
\end{Verbatim}
\subsubsection*{例3: -nfnではうまくいかない}

\verb|-nfn|は、最初の行をデータ行としてみなすので、「数量」「金額」というデータを合計しようとしてしまい、うまくいかない。
\verb|-x|は、あくまでも最初の行は項目名行とみなす点が\verb|-nfn|と異なる。


\begin{Verbatim}[baselinestretch=0.7,frame=single]
$ msum -nfn k=0 f=1,2 i=dat1.csv o=rsl3.csv
#END# kgsum -nfn f=1,2 i=dat1.csv k=0 o=rsl3.csv
$ more rsl3.csv
顧客,0,0
A,3,30
B,5,45
\end{Verbatim}
