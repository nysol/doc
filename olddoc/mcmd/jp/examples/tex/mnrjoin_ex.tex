\subsubsection*{例1: 基本例}

日付項目の値が\verb|20080203|で、「金額」項目の値が\verb|5|以上\verb|15|未満の入力データ行には\verb|avg=150|を、
\verb|40|以上\verb|50|未満の行には\verb|avg=200|を結合する。


\begin{Verbatim}[baselinestretch=0.7,frame=single]
$ more dat1.csv
date,price
20080123,10
20080123,20
20080203,10
20080203,35
200804l0,50
$ more ref1.csv
date,priceF,priceT,avg
20080203,5,15,150
20080203,40,50,200
$ mnrjoin k=date f=avg m=ref1.csv R=priceF,priceT r=price%n i=dat1.csv o=rsl1.csv
#END# kgnrjoin R=priceF,priceT f=avg i=dat1.csv k=date m=ref1.csv o=rsl1.csv r=price%n
$ more rsl1.csv
date%0,price,avg
20080203,10,150
\end{Verbatim}
\subsubsection*{例2: 未結合データ出力}

\verb|-n|を指定することで、参照ファイルにマッチしない入力ファイルの行(\verb|avg=|がNULL値の行)も出力し、
\verb|-N|を指定することで、入力ファイルにマッチしない参照ファイルの行(\verb|price=|がNULL値の行)も出力する。
いわゆる外部結合である。


\begin{Verbatim}[baselinestretch=0.7,frame=single]
$ mnrjoin k=date f=avg m=ref1.csv R=priceF,priceT r=price%n -n -N i=dat1.csv o=rsl2.csv
#END# kgnrjoin -N -n R=priceF,priceT f=avg i=dat1.csv k=date m=ref1.csv o=rsl2.csv r=price%n
$ more rsl2.csv
date%0,price,avg
20080123,10,
20080123,20,
20080203,10,150
20080203,35,
20080203,,200
200804l0,50,
\end{Verbatim}
