\subsubsection*{例1: 基本例}

「数量」と「金額」項目がNULL値の行を削除する。
NULL値の行は\verb|oth.csv|に出力する。


\begin{Verbatim}[baselinestretch=0.7,frame=single]
$ more dat1.csv
顧客,数量,金額
A,1,10
A,,20
B,1,15
B,3,
C,1,20
$ mdelnull f=数量,金額 u=oth.csv i=dat1.csv o=rsl1.csv
#END# kgdelnull f=数量,金額 i=dat1.csv o=rsl1.csv u=oth.csv
$ more rsl1.csv
顧客,数量,金額
A,1,10
B,1,15
C,1,20
$ more oth.csv
顧客,数量,金額
A,,20
B,3,
\end{Verbatim}
\subsubsection*{例2: NULL値の行を選択}

\verb|-r|を指定することで、削除ではなく選択することになる。


\begin{Verbatim}[baselinestretch=0.7,frame=single]
$ mdelnull f=数量,金額 -r i=dat1.csv o=rsl2.csv
#END# kgdelnull -r f=数量,金額 i=dat1.csv o=rsl2.csv
$ more rsl2.csv
顧客,数量,金額
A,,20
B,3,
\end{Verbatim}
\subsubsection*{例3: キー項目でのNULL値の行の削除}

\verb|k=|を指定することで、集計キー単位で削除することになる。
以下では「顧客」項目を単位にして、「数量」と「金額」項目にNULL値が一つでも含まれていれば削除する。


\begin{Verbatim}[baselinestretch=0.7,frame=single]
$ mdelnull k=顧客 f=数量,金額 i=dat1.csv o=rsl3.csv
#END# kgdelnull f=数量,金額 i=dat1.csv k=顧客 o=rsl3.csv
$ more rsl3.csv
顧客%0,数量,金額
C,1,20
\end{Verbatim}
\subsubsection*{例4: 項目間AND条件の例}

「数量」と「金額」項目の両方がNULL値の行を削除する。


\begin{Verbatim}[baselinestretch=0.7,frame=single]
$ more dat2.csv
顧客,数量,金額
A,1,10
A,,
B,1,15
B,3,
C,1,20
$ mdelnull f=数量,金額 -F i=dat2.csv o=rsl4.csv
#END# kgdelnull -F f=数量,金額 i=dat2.csv o=rsl4.csv
$ more rsl4.csv
顧客,数量,金額
A,1,10
B,1,15
B,3,
C,1,20
\end{Verbatim}
\subsubsection*{例5: レコード間AND条件の例}

「顧客」項目を単位にして、「数量」項目が全てNULL値の行を削除する。


\begin{Verbatim}[baselinestretch=0.7,frame=single]
$ mdelnull k=顧客 f=数量 -R i=dat1.csv o=rsl5.csv
#END# kgdelnull -R f=数量 i=dat1.csv k=顧客 o=rsl5.csv
$ more rsl5.csv
顧客%0,数量,金額
A,1,10
A,,20
B,1,15
B,3,
C,1,20
\end{Verbatim}
