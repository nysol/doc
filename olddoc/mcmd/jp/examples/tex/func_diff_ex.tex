\subsubsection*{例1: 月単位での期間}

\verb|date|項目から2013年9月1日までの経過期間を日数で計算する。


\begin{Verbatim}[baselinestretch=0.7,frame=single]
$ more dat1.csv
id,date
1,19641010
2,20000101
3,20130831
4,20130901
5,20130902
$ mcal c='diffday(0d20130901,$d{date})' a=rsl i=dat1.csv o=rsl1.csv
#END# kgcal a=rsl c=diffday(0d20130901,$d{date}) i=dat1.csv o=rsl1.csv
$ more rsl1.csv
id,date,rsl
1,19641010,17858
2,20000101,4992
3,20130831,1
4,20130901,0
5,20130902,-1
\end{Verbatim}
\subsubsection*{例2: 分単位での期間}

\verb|time|項目から2012年1月1日 00時00分00秒までの経過期間を分単位で計算する。


\begin{Verbatim}[baselinestretch=0.7,frame=single]
$ more dat2.csv
id,time
1,20120101000000
2,20120101011112
3,
4,20111231235000
5,20111231235000.123456
$ mcal c='diffmonth(0t20120101000000,$t{time})' a=rsl i=dat2.csv o=rsl2.csv
#END# kgcal a=rsl c=diffmonth(0t20120101000000,$t{time}) i=dat2.csv o=rsl2.csv
$ more rsl2.csv
id,time,rsl
1,20120101000000,0
2,20120101011112,-1
3,,
4,20111231235000,0
5,20111231235000.123456,0
\end{Verbatim}
\subsubsection*{例3: マイクロ秒単位での期間}

\verb|time|項目から2012年1月1日 00時00分00秒までの経過期間を秒単位およびマイクロ秒単位で計算する。


\begin{Verbatim}[baselinestretch=0.7,frame=single]
$ mcal c='diffsecond(0t20120101000000,$t{time})' a=rsl i=dat2.csv o=rsl3.csv
#END# kgcal a=rsl c=diffsecond(0t20120101000000,$t{time}) i=dat2.csv o=rsl3.csv
$ more rsl3.csv
id,time,rsl
1,20120101000000,0
2,20120101011112,-4272
3,,
4,20111231235000,600
5,20111231235000.123456,599
$ mcal c='diffusecond(0t20120101000000,$t{time})' a=rsl i=dat2.csv o=rsl4.csv
#END# kgcal a=rsl c=diffusecond(0t20120101000000,$t{time}) i=dat2.csv o=rsl4.csv
$ more rsl4.csv
id,time,rsl
1,20120101000000,0
2,20120101011112,-4272
3,,
4,20111231235000,600
5,20111231235000.123456,599.876544
\end{Verbatim}
