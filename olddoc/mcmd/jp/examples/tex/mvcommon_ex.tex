\subsubsection*{例1: 複数項目に対して結合する例}



\begin{Verbatim}[baselinestretch=0.7,frame=single]
$ more dat1.csv
items1,items2
b a c,b b
c c,a d
e a a,a a
$ more ref1.csv
item
a
c
e
$ mvcommon vf=items1,items2 K=item m=ref1.csv i=dat1.csv o=rsl1.csv
#END# kgvcommon K=item i=dat1.csv m=ref1.csv o=rsl1.csv vf=items1,items2
$ more rsl1.csv
items1,items2
a c,
c c,a
e a a,a a
\end{Verbatim}
\subsubsection*{例2: 項目名を変更する例}

\verb|item2|に新項目名\verb|new2|を指定しているので、
項目名が変更され出力される。


\begin{Verbatim}[baselinestretch=0.7,frame=single]
$ mvcommon vf=items1,items2:new2 K=item m=ref1.csv i=dat1.csv o=rsl2.csv
#END# kgvcommon K=item i=dat1.csv m=ref1.csv o=rsl2.csv vf=items1,items2:new2
$ more rsl2.csv
items1,new2
a c,
c c,a
e a a,a a
\end{Verbatim}
\subsubsection*{例3: 項目を追加する例}

\verb|item1|に新項目名\verb|new1|を、
\verb|item2|に新項目名\verb|new2|を指定し、
\verb|-A|オプションを付けているので
新項目\verb|new1|と\verb|new2|が追加され出力される。


\begin{Verbatim}[baselinestretch=0.7,frame=single]
$ mvcommon vf=items1:new1,items2:new2 -A K=item m=ref1.csv i=dat1.csv o=rsl3.csv
#END# kgvcommon -A K=item i=dat1.csv m=ref1.csv o=rsl3.csv vf=items1:new1,items2:new2
$ more rsl3.csv
items1,items2,new1,new2
b a c,b b,a c,
c c,a d,c c,a
e a a,a a,e a a,a a
\end{Verbatim}
