\subsubsection*{例1: 基本例}

0.0から1.0の範囲の実数乱数を生成する。


\begin{Verbatim}[baselinestretch=0.7,frame=single]
$ more dat1.csv
顧客
A
B
C
D
E
$ mrand a=rand i=dat1.csv o=rsl1.csv
#END# kgrand a=rand i=dat1.csv o=rsl1.csv
$ more rsl1.csv
顧客,rand
A,0.5477480278
B,0.3656605978
C,0.04199989187
D,0.9302980842
E,0.3048081712
\end{Verbatim}
\subsubsection*{例2: 基本例2}

-intで整数乱数


\begin{Verbatim}[baselinestretch=0.7,frame=single]
$ mrand a=rand -int i=dat1.csv o=rsl2.csv
#END# kgrand -int a=rand i=dat1.csv o=rsl2.csv
$ more rsl2.csv
顧客,rand
A,1083858011
B,549225333
C,2044140724
D,114682354
E,65869181
\end{Verbatim}
\subsubsection*{例3: 最小値、最大値を決めた乱数の生成}

最小値が10、最大値が100の整数の乱数を生成し、\verb|rand|という項目名で出力する。


\begin{Verbatim}[baselinestretch=0.7,frame=single]
$ mrand a=rand -int min=10 max=100 S=1 i=dat1.csv o=rsl3.csv
#END# kgrand -int S=1 a=rand i=dat1.csv max=100 min=10 o=rsl3.csv
$ more rsl3.csv
顧客,rand
A,47
B,100
C,75
D,94
E,10
\end{Verbatim}
\subsubsection*{例4: キー単位の乱数生成}

以下の例は、顧客\verb|A,B,C,D|の4人について同じ顧客には同じ乱数値を振る。


\begin{Verbatim}[baselinestretch=0.7,frame=single]
$ more dat2.csv
顧客
A
A
A
B
B
C
D
D
D
$ mrand k=顧客 -int min=0 max=1 a=rand i=dat2.csv o=rsl4.csv
#END# kgrand -int a=rand i=dat2.csv k=顧客 max=1 min=0 o=rsl4.csv
$ more rsl4.csv
顧客%0,rand
A,1
A,1
A,1
B,0
B,0
C,0
D,1
D,1
D,1
\end{Verbatim}
