\subsubsection*{例1: 基本例}

\verb|k1|項目で並べ替えた後、\verb|k1|キー項目の先頭(\verb|top|項目)と終端(\verb|bottom|項目)に印(\verb|1|)をつける。


\begin{Verbatim}[baselinestretch=0.7,frame=single]
$ more dat1.csv
id,k1,k2,val
1,A,a,1
2,A,b,2
3,A,b,3
4,B,a,4
5,B,a,5
$ mkeybreak k=k1 i=dat1.csv o=rsl1.csv
#END# kgkeybreak i=dat1.csv k=k1 o=rsl1.csv
$ more rsl1.csv
id,k1%0,k2,val,top,bot
1,A,a,1,1,
2,A,b,2,,
3,A,b,3,,1
4,B,a,4,1,
5,B,a,5,,1
\end{Verbatim}
\subsubsection*{例2: 2項目キー}

\verb|k1|・\verb|k2|項目で並べ替えた後、\verb|k1|キー項目の先頭(\verb|top|項目)と終端(\verb|bottom|項目)に印(\verb|1|)をつける。


\begin{Verbatim}[baselinestretch=0.7,frame=single]
$ mkeybreak s=k1,k2 k=k1 i=dat1.csv o=rsl2.csv
#END# kgkeybreak i=dat1.csv k=k1 o=rsl2.csv s=k1,k2
$ more rsl2.csv
id,k1,k2,val,top,bot
1,A,a,1,1,
2,A,b,2,,
3,A,b,3,,1
4,B,a,4,1,
5,B,a,5,,1
\end{Verbatim}
