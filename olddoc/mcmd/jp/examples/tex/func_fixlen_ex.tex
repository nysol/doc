\subsubsection*{例1: 基本例}

str項目を5文字の固定長文字列に変換する。
5文字に満たない文字列は右詰(\verb|"R"|)で\verb|"#"|を埋める。


\begin{Verbatim}[baselinestretch=0.7,frame=single]
$ more dat1.csv
id,str
1,abc
2,123
3,
4,1234567
$ mcal c='fixlen($s{str},5,"R","#")' a=rsl i=dat1.csv o=rsl1.csv
#END# kgcal a=rsl c=fixlen($s{str},5,"R","#") i=dat1.csv o=rsl1.csv
$ more rsl1.csv
id,str,rsl
1,abc,##abc
2,123,##123
3,,#####
4,1234567,34567
\end{Verbatim}
\subsubsection*{例2: 左詰め例}

左詰(\verb|"L"|)で\verb|"#"|を埋める。


\begin{Verbatim}[baselinestretch=0.7,frame=single]
$ mcal c='fixlen($s{str},5,"L","#")' a=rsl i=dat1.csv o=rsl2.csv
#END# kgcal a=rsl c=fixlen($s{str},5,"L","#") i=dat1.csv o=rsl2.csv
$ more rsl2.csv
id,str,rsl
1,abc,abc##
2,123,123##
3,,#####
4,1234567,12345
\end{Verbatim}
\subsubsection*{例3: マルチバイト文字を含む場合}



\begin{Verbatim}[baselinestretch=0.7,frame=single]
$ more dat2.csv
id,str
1,こんにちは
2,大阪
$ mcal c='fixlenw($s{str},4,"R","埋")' a=rsl i=dat2.csv o=rsl3.csv
#END# kgcal a=rsl c=fixlenw($s{str},4,"R","埋") i=dat2.csv o=rsl3.csv
$ more rsl3.csv
id,str,rsl
1,こんにちは,んにちは
2,大阪,埋埋大阪
\end{Verbatim}
