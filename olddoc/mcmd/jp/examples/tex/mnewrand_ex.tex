\subsubsection*{例1: 基本例}

実数乱数を10行生成する。乱数の種は1に固定しているので、いつ実行しても乱数系列は同じになる。


\begin{Verbatim}[baselinestretch=0.7,frame=single]
$ mnewrand a=rand S=1 o=rsl1.csv
#END# kgnewrand S=1 a=rand o=rsl1.csv
$ more rsl1.csv
rand
0.4170219984
0.9971848081
0.7203244893
0.9325573612
0.0001143810805
0.1281244478
0.3023325677
0.9990405154
0.1467558926
0.2360889763
\end{Verbatim}
\subsubsection*{例2: 整数乱数}

最小値が0、最大値が1000、乱数の種が1の整数乱数を5行作成する。


\begin{Verbatim}[baselinestretch=0.7,frame=single]
$ mnewrand a=rand -int max=1000 min=0 l=5 S=1 o=rsl2.csv
#END# kgnewrand -int S=1 a=rand l=5 max=1000 min=0 o=rsl2.csv
$ more rsl2.csv
rand
417
998
721
933
0
\end{Verbatim}
\subsubsection*{例3: ヘッダ行なしで出力}

\verb|-nfn|でヘッダーなしのデータが生成される。


\begin{Verbatim}[baselinestretch=0.7,frame=single]
$ mnewrand -nfn l=5 S=1 o=rsl3.csv
#END# kgnewrand -nfn S=1 l=5 o=rsl3.csv
$ more rsl3.csv
0.4170219984
0.9971848081
0.7203244893
0.9325573612
0.0001143810805
\end{Verbatim}
