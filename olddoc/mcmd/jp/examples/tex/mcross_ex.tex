\subsubsection*{例1: 基本例}

\verb|item|項目を単位に\verb|date|項目を横に展開し、
\verb|quantity|項目を出力する。


\begin{Verbatim}[baselinestretch=0.7,frame=single]
$ more dat1.csv
item,date,quantity,price
A,20081201,1,10
A,20081202,2,20
A,20081203,3,30
B,20081201,4,40
B,20081203,5,50
$ mcross k=item f=quantity s=date i=dat1.csv o=rsl1.csv
#END# kgcross f=quantity i=dat1.csv k=item o=rsl1.csv s=date
$ more rsl1.csv
item%0,fld,20081201,20081202,20081203
A,quantity,1,2,3
B,quantity,4,,5
\end{Verbatim}
\subsubsection*{例2: 元の入力データに戻す例}

例1の出力結果を元に戻すには、同じく\verb|mcross|を以下のよう用いればよい。


\begin{Verbatim}[baselinestretch=0.7,frame=single]
$ more rsl1.csv
item%0,fld,20081201,20081202,20081203
A,quantity,1,2,3
B,quantity,4,,5
$ mcross k=item f=2008* s=fld a=date i=rsl1.csv o=rsl2.csv
#END# kgcross a=date f=2008* i=rsl1.csv k=item o=rsl2.csv s=fld
$ more rsl2.csv
item%0,date,quantity
A,20081201,1
A,20081202,2
A,20081203,3
B,20081201,4
B,20081202,
B,20081203,5
\end{Verbatim}
\subsubsection*{例3: 複数の値を出力}

\verb|quantity,price|の2項目を出力する。


\begin{Verbatim}[baselinestretch=0.7,frame=single]
$ mcross k=item f=quantity,price s=date i=dat1.csv o=rsl3.csv
#END# kgcross f=quantity,price i=dat1.csv k=item o=rsl3.csv s=date
$ more rsl3.csv
item%0,fld,20081201,20081202,20081203
A,quantity,1,2,3
A,price,10,20,30
B,quantity,4,,5
B,price,40,,50
\end{Verbatim}
\subsubsection*{例4: 並びを逆順する例}

横に展開する項目名の並びを逆順にする。


\begin{Verbatim}[baselinestretch=0.7,frame=single]
$ mcross k=item f=quantity,price s=date%r i=dat1.csv o=rsl4.csv
#END# kgcross f=quantity,price i=dat1.csv k=item o=rsl4.csv s=date%r
$ more rsl4.csv
item%0,fld,20081203,20081202,20081201
A,quantity,3,2,1
A,price,30,20,10
B,quantity,5,,4
B,price,50,,40
\end{Verbatim}
