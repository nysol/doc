\subsubsection*{例1: 基本例}

入力ファイルにある「顧客」項目と、参照ファイルにある「顧客」項目が同じ値を持つ入力ファイルの行を選択する。
それ以外のデータは\verb|oth.csv|に出力する。


\begin{Verbatim}[baselinestretch=0.7,frame=single]
$ more dat1.csv
顧客,数量
A,1
B,2
C,1
D,3
E,1
$ more ref1.csv
顧客,性別
A,女性
B,男性
E,女性
$ mcommon k=顧客 m=ref1.csv u=oth.csv i=dat1.csv o=rsl1.csv
#END# kgcommon i=dat1.csv k=顧客 m=ref1.csv o=rsl1.csv u=oth.csv
$ more rsl1.csv
顧客%0,数量
A,1
B,2
E,1
$ more oth.csv
顧客%0,数量
C,1
D,3
\end{Verbatim}
\subsubsection*{例2: 同じ値を持たない入力ファイルの行選択}

\verb|-r|オプションを付けることで、条件が逆転し、参照ファイルにない「顧客」を選択することになる。


\begin{Verbatim}[baselinestretch=0.7,frame=single]
$ mcommon k=顧客 m=ref1.csv -r i=dat1.csv o=rsl2.csv
#END# kgcommon -r i=dat1.csv k=顧客 m=ref1.csv o=rsl2.csv
$ more rsl2.csv
顧客%0,数量
C,1
D,3
\end{Verbatim}
\subsubsection*{例3: 結合キー項目名が異なる場合}

結合キーの項目名が異なる場合は、\verb|K=|で指定する。


\begin{Verbatim}[baselinestretch=0.7,frame=single]
$ more ref2.csv
顧客ID,性別
A,女性
B,男性
E,女性
$ mcommon k=顧客 K=顧客ID i=dat1.csv m=ref2.csv o=rsl3.csv
#END# kgcommon K=顧客ID i=dat1.csv k=顧客 m=ref2.csv o=rsl3.csv
$ more rsl3.csv
顧客%0,数量
A,1
B,2
E,1
\end{Verbatim}
\subsubsection*{例4: キー項目に重複行がある場合の例}

参照ファイルと入力ファイルのキー項目に重複行があっても選択可能。


\begin{Verbatim}[baselinestretch=0.7,frame=single]
$ more dat3.csv
顧客,数量
A,1
A,2
A,3
B,1
D,1
D,2
$ more ref3.csv
顧客
A
A
D
$ mcommon k=顧客 m=ref3.csv -r i=dat3.csv o=rsl4.csv
#END# kgcommon -r i=dat3.csv k=顧客 m=ref3.csv o=rsl4.csv
$ more rsl4.csv
顧客%0,数量
B,1
\end{Verbatim}
