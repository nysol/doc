\subsubsection*{例1: 基本例}



\begin{Verbatim}[baselinestretch=0.7,frame=single]
$ more dat1.csv
id,str
1,abc
2,3.1415
3,
4,hello world!
$ mcal c='length($s{str})' a=rsl i=dat1.csv o=rsl1.csv
#END# kgcal a=rsl c=length($s{str}) i=dat1.csv o=rsl1.csv
$ more rsl1.csv
id,str,rsl
1,abc,3
2,3.1415,6
3,,0
4,hello world!,12
\end{Verbatim}
\subsubsection*{例2: マルチバイト文字を含む例}

以下はutf-8でエンコーディングされた日本語を用いた例である。
utf-8の日本語は1文字3バイトでエンコーディングされているので、
length関数では日本語としての文字数ではなく、そのバイト数を返す。


\begin{Verbatim}[baselinestretch=0.7,frame=single]
$ more dat2.csv
id,str
1,こんにちは
2,大阪
$ mcal c='length($s{str})' a=rsl i=dat2.csv o=rsl2.csv
#END# kgcal a=rsl c=length($s{str}) i=dat2.csv o=rsl2.csv
$ more rsl2.csv
id,str,rsl
1,こんにちは,15
2,大阪,6
\end{Verbatim}
\subsubsection*{例3: ワイド文字として扱う例}

lengthwを使うと、内部で文字列をワイド文字に変換するので、マルチバイト文字1文字を正しく認識して計算する。


\begin{Verbatim}[baselinestretch=0.7,frame=single]
$ mcal c='lengthw($s{str})' a=rsl i=dat2.csv o=rsl3.csv
#END# kgcal a=rsl c=lengthw($s{str}) i=dat2.csv o=rsl3.csv
$ more rsl3.csv
id,str,rsl
1,こんにちは,5
2,大阪,2
\end{Verbatim}
