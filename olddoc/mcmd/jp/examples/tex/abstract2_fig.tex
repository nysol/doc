\begin{figure}[htbp]
\begin{Verbatim}[baselinestretch=0.7,frame=single]
$ mdata -man1 >man1.csv
#END# kgData -man1
$ more man1.csv
顧客,日付,商品
A,20130916,a
A,20130916,c
A,20130917,a
A,20130917,e
B,20130916,d
B,20130917,a
B,20130917,d
B,20130917,f
$ mcut f=顧客,商品 i=man1.csv o=xxa
#END# kgcut f=顧客,商品 i=man1.csv o=xxa
$ more xxa
顧客,商品
A,a
A,c
A,a
A,e
B,d
B,a
B,d
B,f
# 顧客商品別に行数をカウントする。
$ mcount k=顧客,商品 a=件数 i=xxa o=xxb
#END# kgcount a=件数 i=xxa k=顧客,商品 o=xxb
$ more xxb
顧客%0,商品%1,件数
A,a,2
A,c,1
A,e,1
B,a,1
B,d,2
B,f,1
# 商品を項目にしたクロス集計を実行。購入されていない商品の個数は0にしている。
$ mcross k=顧客 s=商品 f=件数 v=0 i=xxb o=xxc
#END# kgcross f=件数 i=xxb k=顧客 o=xxc s=商品 v=0
$ more xxc
顧客%0,fld,a,c,d,e,f
A,件数,2,1,0,1,0
B,件数,1,0,2,0,1
# 余分な項目"fld"を除いている。
$ mcut f=fld -r i=xxc o=output.csv
#END# kgcut -r f=fld i=xxc o=output.csv
$ more output.csv
顧客%0,a,c,d,e,f
A,2,1,0,1,0
B,1,0,2,0,1
\end{Verbatim}
\caption{顧客別商品購入数量マトリックス\label{fig:abstract2_1}}
\end{figure}
