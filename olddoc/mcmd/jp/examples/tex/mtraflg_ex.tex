\subsubsection*{例1: 基本例}

\verb|egg|と\verb|milk|項目の値がNULL値以外なら、それら項目名を要素としたベクトルを作成する。


\begin{Verbatim}[baselinestretch=0.7,frame=single]
$ more dat1.csv
customer,egg,milk
A,1,1
B,,1
C,1,
D,1,1
$ mtraflg f=egg,milk a=transaction i=dat1.csv o=rsl1.csv
#END# kgtraflg a=transaction f=egg,milk i=dat1.csv o=rsl1.csv
$ more rsl1.csv
customer,transaction
A,egg milk
B,milk
C,egg
D,egg milk
\end{Verbatim}
\subsubsection*{例2: 基本例2}

出力された結果を\verb|-r|をつけて再実行し元に戻す。


\begin{Verbatim}[baselinestretch=0.7,frame=single]
$ mtraflg -r f=egg,milk a=transaction i=rsl1.csv o=rsl2.csv
#END# kgtraflg -r a=transaction f=egg,milk i=rsl1.csv o=rsl2.csv
$ more rsl2.csv
customer,egg,milk
A,1,1
B,,1
C,1,
D,1,1
\end{Verbatim}
\subsubsection*{例3: 区切り文字の指定}

区切り文字を-(ハイフン)で連結し、\verb|transaction|という項目名で出力する。


\begin{Verbatim}[baselinestretch=0.7,frame=single]
$ mtraflg f=egg,milk a=transaction delim=- i=dat1.csv o=rsl3.csv
#END# kgtraflg a=transaction delim=- f=egg,milk i=dat1.csv o=rsl3.csv
$ more rsl3.csv
customer,transaction
A,egg-milk
B,milk
C,egg
D,egg-milk
\end{Verbatim}
