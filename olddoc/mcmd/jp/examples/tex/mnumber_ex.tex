\subsubsection*{例1: 数字の連番}

Customer項目名順(昇順)に連番を振り「No」という項目名で出力する。


\begin{Verbatim}[baselinestretch=0.7,frame=single]
$ more dat1.csv
Customer,Val,Sum
A,29,300
B,35,250
C,15,200
D,23,150
E,10,100
$ mnumber s=Customer a=No i=dat1.csv o=rsl1.csv
#END# kgnumber a=No i=dat1.csv o=rsl1.csv s=Customer
$ more rsl1.csv
Customer%0,Val,Sum,No
A,29,300,0
B,35,250,1
C,15,200,2
D,23,150,3
E,10,100,4
\end{Verbatim}
\subsubsection*{例2: Date項目順の連番}

Date項目順(昇順)に連番をふる。その際、同じDateには同じNoを振り「No」という項目名で出力する。


\begin{Verbatim}[baselinestretch=0.7,frame=single]
$ more dat2.csv
Date
20090101
20090101
20090102
20090103
20090103
$ mnumber k=Date a=No -B i=dat2.csv o=rsl2.csv
#END# kgnumber -B a=No i=dat2.csv k=Date o=rsl2.csv
$ more rsl2.csv
Date%0,No
20090101,0
20090101,0
20090102,1
20090103,2
20090103,2
\end{Verbatim}
\subsubsection*{例3: Sum項目順の連番(同Rankは同じアルファベットをふる)}

Sum項目の多い順(降順)にアルファベットのAから順に連文字を振り「Rank」という項目名で出力する。
また、同Rankの場合は同じアルファベット文字を振ることにする。


\begin{Verbatim}[baselinestretch=0.7,frame=single]
$ more dat3.csv
Customer,Val,Sum
A,3,300
B,1,250
C,2,250
D,1,150
E,1,100
$ mnumber a=Rank e=same s=Sum%nr S=A  i=dat3.csv o=rsl3.csv
#END# kgnumber S=A a=Rank e=same i=dat3.csv o=rsl3.csv s=Sum%nr
$ more rsl3.csv
Customer,Val,Sum%0nr,Rank
A,3,300,A
B,1,250,B
C,2,250,B
D,1,150,C
E,1,100,D
\end{Verbatim}
\subsubsection*{例4: Sum項目順の連番(同Rankは並び順でNoをふる)}

Sum項目の多い順(降順)に連番を振り「Rank」という項目名で出力する。
その際、同Rankの場合は並び順でNoを振ることにする。


\begin{Verbatim}[baselinestretch=0.7,frame=single]
$ mnumber a=Rank e=seq s=Sum%nr i=dat3.csv o=rsl4.csv
#END# kgnumber a=Rank e=seq i=dat3.csv o=rsl4.csv s=Sum%nr
$ more rsl4.csv
Customer,Val,Sum%0nr,Rank
A,3,300,0
B,1,250,1
C,2,250,2
D,1,150,3
E,1,100,4
\end{Verbatim}
\subsubsection*{例5: Sum項目順の連番(同Rankは同じNoをふる)}

Sum項目の多い順(降順)に連番を振り「Rank」という項目名で出力する。
その際、同Rankの場合は同じNoを振ることにする。


\begin{Verbatim}[baselinestretch=0.7,frame=single]
$ mnumber a=Rank e=same s=Sum%nr i=dat3.csv o=rsl5.csv
#END# kgnumber a=Rank e=same i=dat3.csv o=rsl5.csv s=Sum%nr
$ more rsl5.csv
Customer,Val,Sum%0nr,Rank
A,3,300,0
B,1,250,1
C,2,250,1
D,1,150,2
E,1,100,3
\end{Verbatim}
\subsubsection*{例6: Sum項目順の連番(同Rankの場合は同じRankNoを振り、次のNoはスキップ)}

Sum項目の多い順(降順)に連番を振り「Rank」という項目名で出力する。
その際、同Rankの場合は同じRankNoを振り、次のNoはスキップするようにNoを振ることにする。


\begin{Verbatim}[baselinestretch=0.7,frame=single]
$ mnumber a=Rank e=skip s=Sum%nr i=dat3.csv o=rsl6.csv
#END# kgnumber a=Rank e=skip i=dat3.csv o=rsl6.csv s=Sum%nr
$ more rsl6.csv
Customer,Val,Sum%0nr,Rank
A,3,300,0
B,1,250,1
C,2,250,1
D,1,150,3
E,1,100,4
\end{Verbatim}
\subsubsection*{例7: 10から始まる連番}

Sum項目の小さい順(昇順)に10から始まる連番を振り「Score」という項目名で出力する。
その際、同Rankの場合は同じRankNoを振り、次のNoはスキップするようにNoを振ることにする。


\begin{Verbatim}[baselinestretch=0.7,frame=single]
$ more dat4.csv
Customer,Val,Sum
A,1,100
B,1,150
C,1,250
D,2,250
E,3,300
$ mnumber a=Score e=same s=Sum%n S=10 i=dat4.csv o=rsl7.csv
#END# kgnumber S=10 a=Score e=same i=dat4.csv o=rsl7.csv s=Sum%n
$ more rsl7.csv
Customer,Val,Sum%0n,Score
A,1,100,10
B,1,150,11
C,1,250,12
D,2,250,12
E,3,300,13
\end{Verbatim}
\subsubsection*{例8: 10から始まる5つ飛びの連番}

Sum項目の小さい順番(昇順)に10から始まる5つ飛びの連番を振り「Score」という項目名で出力する。
また、同Rankの場合は同じNoを振ることにする。


\begin{Verbatim}[baselinestretch=0.7,frame=single]
$ mnumber a=Score e=same s=Sum%n S=10 I=5 i=dat4.csv o=rsl8.csv
#END# kgnumber I=5 S=10 a=Score e=same i=dat4.csv o=rsl8.csv s=Sum%n
$ more rsl8.csv
Customer,Val,Sum%0n,Score
A,1,100,10
B,1,150,15
C,1,250,20
D,2,250,20
E,3,300,25
\end{Verbatim}
