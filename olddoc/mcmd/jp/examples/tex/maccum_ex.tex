\subsubsection*{例1: 基本例}

「数量」と「金額」項目の累積値を計算し、「数量累計」と「金額累計」という項目名で出力する。


\begin{Verbatim}[baselinestretch=0.7,frame=single]
$ more dat1.csv
顧客,数量,金額
A,1,10
A,2,20
B,1,15
B,3,10
B,1,20
$ maccum s=顧客 f=数量:数量累計,金額:金額累計 i=dat1.csv o=rsl1.csv
#END# kgaccum f=数量:数量累計,金額:金額累計 i=dat1.csv o=rsl1.csv s=顧客
$ more rsl1.csv
顧客%0,数量,金額,数量累計,金額累計
A,1,10,1,10
A,2,20,3,30
B,1,15,4,45
B,3,10,7,55
B,1,20,8,75
\end{Verbatim}
\subsubsection*{例2: キー項目を指定する例}

「顧客」項目を単位に「数量」と「金額」項目の累積値を計算し、「数量累計」と「金額累計」という項目名で出力する。


\begin{Verbatim}[baselinestretch=0.7,frame=single]
$ more dat1.csv
顧客,数量,金額
A,1,10
A,2,20
B,1,15
B,3,10
B,1,20
$ maccum k=顧客 s=顧客 f=数量:数量累計,金額:金額累計 i=dat1.csv o=rsl2.csv
#END# kgaccum f=数量:数量累計,金額:金額累計 i=dat1.csv k=顧客 o=rsl2.csv s=顧客
$ more rsl2.csv
顧客,数量,金額,数量累計,金額累計
A,1,10,1,10
A,2,20,3,30
B,1,15,1,15
B,3,10,4,25
B,1,20,5,45
\end{Verbatim}
\subsubsection*{例3: NULL値を含む累計}

「数量」と「金額」項目の累積値を計算し、「数量累計」と「金額累計」という項目名で出力する。
NULLは無視される。結果もNULLが出力される。


\begin{Verbatim}[baselinestretch=0.7,frame=single]
$ more dat2.csv
顧客,数量,金額
A,1,10
A,,20
B,1,15
B,3,
B,1,20
$ maccum s=顧客 f=数量:数量累計,金額:金額累計 i=dat2.csv o=rsl3.csv
#END# kgaccum f=数量:数量累計,金額:金額累計 i=dat2.csv o=rsl3.csv s=顧客
$ more rsl3.csv
顧客%0,数量,金額,数量累計,金額累計
A,1,10,1,10
A,,20,,30
B,1,15,2,45
B,3,,5,
B,1,20,6,65
\end{Verbatim}
