\subsubsection*{例1: 基本例}

ここでは、\verb|"&"|が入力項目名である「\verb|ブランド|」に置換され、「\verb|f=ブランド:ブランドコード|」と指定したことと同等となる。


\begin{Verbatim}[baselinestretch=0.7,frame=single]
$ more dat1.csv
ブランド,数量10,数量11,数量12,数量123
A,10,15,9,1
B,20,16,8,2
C,30,17,7,3
D,40,18,6,4
$ mcut f="ブランド:&コード" i=dat1.csv o=rsl1.csv
#END# kgcut f=ブランド:&コード i=dat1.csv o=rsl1.csv
$ more rsl1.csv
ブランドコード
A
B
C
D
\end{Verbatim}
\subsubsection*{例2: ワイルドカードとの併用}

出力項目名指定における\verb|売上&|の\verb|&|が入力項目名(例えば「数量10」)に置き換わる。
結果として、「\verb|数量|」で始まる項目全てに対して「\verb|売上|」を先頭に加えて出力することになる。


\begin{Verbatim}[baselinestretch=0.7,frame=single]
$ mcut f="ブランド,数量*:売上&" i=dat1.csv o=rsl2.csv
#END# kgcut f=ブランド,数量*:売上& i=dat1.csv o=rsl2.csv
$ more rsl2.csv
ブランド,売上数量10,売上数量11,売上数量12,売上数量123
A,10,15,9,1
B,20,16,8,2
C,30,17,7,3
D,40,18,6,4
\end{Verbatim}
