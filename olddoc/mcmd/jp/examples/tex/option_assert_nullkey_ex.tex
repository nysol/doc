\subsubsection*{(基本例) }
例えば、msum(項目値の合計)コマンドを例にあげる。k=パラメータで指定した項目の値が同じ行に対して、f=パラメータで指定した集計項目の項目値を合計することを想定した場合、データによっては、k=パラメータで指定したキー項目の値にNULL値が含まれている場合がある。-assert\_nullkeyパラメータを指定すると、キー項目の値にNULL値が含まれているかどうかのチェックを行い、NULL値が含まれていた場合に、\verb|「#WARNING# ; exist NULL in key filed」|というメッセージを表示するのでキー項目にNULL値が含まれていたかどうかを確認することができる。

\begin{Verbatim}[baselinestretch=0.7,frame=single]
$ more dat1.csv
顧客,数量,金額
A,1,10
,1,10
B,1,15
A,2,20
B,3,10
B,1,20

$ msum k=顧客 f=数量:数量合計,金額:金額合計 -assert_nullkey i=dat1.csv o=rsl1.csv
#WARNING# ; exist NULL in key filed
#END# kgsum -assert_nullkey f=数量:数量合計,金額:金額合計 i=dat1.csv k=顧客 o=rsl1.csv

$ more rsl1.csv
顧客%0,数量合計,金額合計
,1,10
A,3,30
B,5,45
\end{Verbatim}
