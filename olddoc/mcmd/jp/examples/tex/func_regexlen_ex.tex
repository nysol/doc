\subsubsection*{例1: 基本例}

正規表現\verb|c.*a|に最も長くマッチする部分文字列の長さを得る。
マッチした部分文字列については\verb|regexstr|と同じ入力データを使っているので、比較すると分かりやすい。


\begin{Verbatim}[baselinestretch=0.7,frame=single]
$ more dat1.csv
id,str
1,xcbbbayy
2,xxcbaay
3,
4,bacabbca
$ mcal c='regexlen($s{str},"c.*a")' a=rsl i=dat1.csv o=rsl1.csv
#END# kgcal a=rsl c=regexlen($s{str},"c.*a") i=dat1.csv o=rsl1.csv
$ more rsl1.csv
id,str,rsl
1,xcbbbayy,5
2,xxcbaay,4
3,,0
4,bacabbca,6
\end{Verbatim}
\subsubsection*{例2: マルチバイト文字}

正規表現\verb|"あ.*い"|に最も長くマッチする部分文字列の長さを得る。
ただし、以下ではマルチバイト文字対応でないregexlen関数を使っているので、
文字数ではなくバイト数を返している。


\begin{Verbatim}[baselinestretch=0.7,frame=single]
$ more dat2.csv
id,str
1,漢漢あbbbいyy
2,漢あbいいy
3,
4,bあいあbbいあ
$ mcal c='regexlen($s{str},"あ.*い")' a=rsl i=dat2.csv o=rsl2.csv
#END# kgcal a=rsl c=regexlen($s{str},"あ.*い") i=dat2.csv o=rsl2.csv
$ more rsl2.csv
id,str,rsl
1,漢漢あbbbいyy,9
2,漢あbいいy,10
3,,0
4,bあいあbbいあ,14
\end{Verbatim}
\subsubsection*{例3: マルチバイト文字2}

正規表現\verb|"あ.*い"|に最も長くマッチする部分文字列の長さを得る。
regexlenw関数を使うと、正しくマルチバイト文字を扱ってくれる。


\begin{Verbatim}[baselinestretch=0.7,frame=single]
$ mcal c='regexlenw($s{str},"あ.*い")' a=rsl i=dat2.csv o=rsl3.csv
#END# kgcal a=rsl c=regexlenw($s{str},"あ.*い") i=dat2.csv o=rsl3.csv
$ more rsl3.csv
id,str,rsl
1,漢漢あbbbいyy,5
2,漢あbいいy,4
3,,0
4,bあいあbbいあ,6
\end{Verbatim}
