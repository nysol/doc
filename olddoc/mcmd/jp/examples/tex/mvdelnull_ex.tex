\subsubsection*{例1: nullを削除する基本例}



\begin{Verbatim}[baselinestretch=0.7,frame=single]
$ more dat1.csv
items
b a  c
 c c
e a   b 
$ mvdelnull vf=items i=dat1.csv o=rsl1.csv
#END# kgvdelnull i=dat1.csv o=rsl1.csv vf=items
$ more rsl1.csv
items
b a c
c c
e a b
\end{Verbatim}
\subsubsection*{例2: 分かりやすく区切り文字を.(ドット)にした例}



\begin{Verbatim}[baselinestretch=0.7,frame=single]
$ more dat2.csv
items
b.a..c
.c.c
e.a...b.
$ mvdelnull vf=items delim=. i=dat2.csv o=rsl2.csv
#END# kgvdelnull delim=. i=dat2.csv o=rsl2.csv vf=items
$ more rsl2.csv
items
b.a.c
c.c
e.a.b
\end{Verbatim}
\subsubsection*{例3: 項目名を変更して出力}



\begin{Verbatim}[baselinestretch=0.7,frame=single]
$ mvdelnull vf=items:new i=dat1.csv o=rsl3.csv
#END# kgvdelnull i=dat1.csv o=rsl3.csv vf=items:new
$ more rsl3.csv
new
b a c
c c
e a b
\end{Verbatim}
\subsubsection*{例4: -Aを指定することで追加項目として出力}



\begin{Verbatim}[baselinestretch=0.7,frame=single]
$ mvdelnull vf=items:new -A i=dat1.csv o=rsl4.csv
#END# kgvdelnull -A i=dat1.csv o=rsl4.csv vf=items:new
$ more rsl4.csv
items,new
b a  c,b a c
 c c,c c
e a   b ,e a b
\end{Verbatim}
