\subsubsection*{例1: 基本例}

コロンを区切り文字として、ベクトル項目\verb|vec|の要素を並べ替える。


\begin{Verbatim}[baselinestretch=0.7,frame=single]
$ more dat1.csv
vec
b:a:c
x:p
$ mvsort vf=vec delim=: i=dat1.csv o=rsl1.csv
#END# kgvsort delim=: i=dat1.csv o=rsl1.csv vf=vec
$ more rsl1.csv
vec
a:b:c
p:x
\end{Verbatim}
\subsubsection*{例2: delimを指定しないと}

delimを指定していないので\verb|b:a:c|や\verb|x:p|は一つの要素として解釈される。


\begin{Verbatim}[baselinestretch=0.7,frame=single]
$ mvsort vf=vec i=dat1.csv o=rsl2.csv
#END# kgvsort i=dat1.csv o=rsl2.csv vf=vec
$ more rsl2.csv
vec
b:a:c
x:p
\end{Verbatim}
\subsubsection*{例3: カンマを区切り文字にする}

区切り文字をカンマにした場合は、ベクトル全体がダブルクオーテーションで囲われることで
CSVの区切り文字との区別がつけられる。


\begin{Verbatim}[baselinestretch=0.7,frame=single]
$ more dat2.csv
id,vec1,vec2
1,a,b
2,p,q
$ mvcat vf=vec1,vec2 a=vec3 delim=, i=dat2.csv o=rsl3.csv
#END# kgvcat a=vec3 delim=, i=dat2.csv o=rsl3.csv vf=vec1,vec2
$ more rsl3.csv
id,vec3
1,"a,b"
2,"p,q"
\end{Verbatim}
