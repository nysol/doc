\subsubsection*{例1: 基本例}

「顧客」と「金額」項目を選択する。ただし、「金額」項目は「売上」と名前を変更して出力している。


\begin{Verbatim}[baselinestretch=0.7,frame=single]
$ more dat1.csv
顧客,数量,金額
A,1,10
A,2,20
B,1,15
B,3,10
B,1,20
$ mcut f=顧客,金額:売上 i=dat1.csv o=rsl1.csv
#END# kgcut f=顧客,金額:売上 i=dat1.csv o=rsl1.csv
$ more rsl1.csv
顧客,売上
A,10
A,20
B,15
B,10
B,20
\end{Verbatim}
\subsubsection*{例2: 項目削除}

\verb|-r|を指定することで、項目を削除できる。


\begin{Verbatim}[baselinestretch=0.7,frame=single]
$ mcut f=顧客,金額 -r i=dat1.csv o=rsl2.csv
#END# kgcut -r f=顧客,金額 i=dat1.csv o=rsl2.csv
$ more rsl2.csv
数量
1
2
1
3
1
\end{Verbatim}
\subsubsection*{例3: 項目名なしデータ}

ヘッダなし入力ファイルから、0,2番目の項目を選択し、
「顧客」と「金額」という名前で出力する。


\begin{Verbatim}[baselinestretch=0.7,frame=single]
$ more dat2.csv
A,1,10
A,2,20
B,1,15
B,3,10
B,1,20
$ mcut f=0:顧客,2:金額 -nfni i=dat2.csv o=rsl3.csv
#END# kgcut -nfni f=0:顧客,2:金額 i=dat2.csv o=rsl3.csv
$ more rsl3.csv
顧客,金額
A,10
A,20
B,15
B,10
B,20
\end{Verbatim}
