\subsubsection*{例1: 基本例}

\verb|dat1.csv|を入力データとして\verb|mcut|は実行される。


\begin{Verbatim}[baselinestretch=0.7,frame=single]
$ more dat1.csv
顧客,数量,金額
A,1,10
A,2,20
$ mcut f=顧客,金額 i=dat1.csv o=rsl1.csv
#END# kgcut f=顧客,金額 i=dat1.csv o=rsl1.csv
$ more rsl1.csv
顧客,金額
A,10
A,20
\end{Verbatim}
\subsubsection*{例2: 出力項目名の指定}

標準入力をリダイレクト(\verb|"<"|記号)して読み込む。


\begin{Verbatim}[baselinestretch=0.7,frame=single]
$ mcut f=顧客,金額 o=rsl2.csv <dat1.csv
#END# kgcut f=顧客,金額 o=rsl2.csv
$ more rsl2.csv
顧客,金額
A,10
A,20
\end{Verbatim}
