\subsubsection*{例1: 基本例}

\verb|zipCode|項目の値が00から始まる4桁文字列を\verb|####|に置換する。


\begin{Verbatim}[baselinestretch=0.7,frame=single]
$ more dat1.csv
customer,zipCode
A,6230041
B,6240053
C,6330032
D,6230087
E,6530095
$ msed f=zipCode c=00.. v=#### i=dat1.csv o=rsl1.csv
#END# kgsed c=00.. f=zipCode i=dat1.csv o=rsl1.csv v=####
$ more rsl1.csv
customer,zipCode
A,623####
B,624####
C,633####
D,623####
E,653####
\end{Verbatim}
\subsubsection*{例2: 項目名指定}

\verb|zipCode|の値が00から始まる4桁の数字を\verb|####|に置換し、\verb|zipCode4|という項目名で出力する。


\begin{Verbatim}[baselinestretch=0.7,frame=single]
$ msed f=zipCode:zipCode4 c='00\d\d' v=#### i=dat1.csv o=rsl2.csv
#END# kgsed c=00\d\d f=zipCode:zipCode4 i=dat1.csv o=rsl2.csv v=####
$ more rsl2.csv
customer,zipCode4
A,623####
B,624####
C,633####
D,623####
E,653####
\end{Verbatim}
\subsubsection*{例3: グローバル置換}

\verb|zipCode|の値が\verb|0|を全て\verb|-|にグローバル置換する。


\begin{Verbatim}[baselinestretch=0.7,frame=single]
$ msed f=zipCode c=0 v=- -g i=dat1.csv o=rsl3.csv
#END# kgsed -g c=0 f=zipCode i=dat1.csv o=rsl3.csv v=-
$ more rsl3.csv
customer,zipCode
A,623--41
B,624--53
C,633--32
D,623--87
E,653--95
\end{Verbatim}
\subsubsection*{例4: 部分置換}

\verb|item|の先頭の\verb|fruit|を削除する。
先頭一致(\verb|^|)を指定しているので、最後の行の\verb|grapefruit|は削除されていないことに注意する。


\begin{Verbatim}[baselinestretch=0.7,frame=single]
$ more dat2.csv
item,price
fruit:apple,100
fruit:peach,250
fruit:pineapple,300
fruit:orange,450
fruit:grapefruit,500
$ msed f=item c='^fruit' v= -g i=dat2.csv o=rsl4.csv
#END# kgsed -g c=^fruit f=item i=dat2.csv o=rsl4.csv v=
$ more rsl4.csv
item,price
:apple,100
:peach,250
:pineapple,300
:orange,450
:grapefruit,500
\end{Verbatim}
\subsubsection*{例5: マッチ結果を用いた置換}

\verb|v=|の中で\verb|$&|を用いれば、マッチした文字列(連続した\verb|b|)に置き換わる。


\begin{Verbatim}[baselinestretch=0.7,frame=single]
$ more dat3.csv
str1
abc
abbc
ac
$ msed f=str1 c='b+' v='#$&#' i=dat3.csv o=rsl5.csv
#END# kgsed c=b+ f=str1 i=dat3.csv o=rsl5.csv v=#$&#
$ more rsl5.csv
str1
a#b#c
a#bb#c
ac
\end{Verbatim}
\subsubsection*{例6: グローバルマッチとの組み合せ}

グローバルマッチにすると、個々のマッチ毎に\verb|v=|の内容が評価される。


\begin{Verbatim}[baselinestretch=0.7,frame=single]
$ msed f=str1 c=b v='#$&#' -g i=dat3.csv o=rsl6.csv
#END# kgsed -g c=b f=str1 i=dat3.csv o=rsl6.csv v=#$&#
$ more rsl6.csv
str1
a#b#c
a#b##b#c
ac
\end{Verbatim}
\subsubsection*{例7: プレフィックス置換}

\verb|$`|にて、マッチした箇所の前の文字列(プレフィックス)に置換される。


\begin{Verbatim}[baselinestretch=0.7,frame=single]
$ msed f=str1 c=b v='#$`#' i=dat3.csv o=rsl7.csv
#END# kgsed c=b f=str1 i=dat3.csv o=rsl7.csv v=#$`#
$ more rsl7.csv
str1
a#a#c
a#a#bc
ac
\end{Verbatim}
\subsubsection*{例8: サフィックス置換}

\verb|$'|にて、マッチした箇所の後の文字列(サフィックス)に置換される。


\begin{Verbatim}[baselinestretch=0.7,frame=single]
$ msed f=str1 c=b v="#$'#" i=dat3.csv o=rsl8.csv
#END# kgsed c=b f=str1 i=dat3.csv o=rsl8.csv v=#$'#
$ more rsl8.csv
str1
a#c#c
a#bc#bc
ac
\end{Verbatim}
