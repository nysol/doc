\subsubsection*{例1: 基本例}

3つの項目str1,str2,str3を\verb|"#"|の区切り文字を挿入して併合する。


\begin{Verbatim}[baselinestretch=0.7,frame=single]
$ more dat1.csv
id,str1,str2,str3
1,abc,def,ghi
2,A,,CDE
3,,,
4,,,XY
$ mcal c='cat("#",$s{str1},$s{str2},$s{str3})' a=rsl i=dat1.csv o=rsl1.csv
#END# kgcal a=rsl c=cat("#",$s{str1},$s{str2},$s{str3}) i=dat1.csv o=rsl1.csv
$ more rsl1.csv
id,str1,str2,str3,rsl
1,abc,def,ghi,abc#def#ghi
2,A,,CDE,A##CDE
3,,,,##
4,,,XY,##XY
\end{Verbatim}
\subsubsection*{例2: tokenが空文字の場合}



\begin{Verbatim}[baselinestretch=0.7,frame=single]
$ mcal c='cat("",$s{str1},$s{str2},$s{str3})' a=rsl i=dat1.csv o=rsl2.csv
#END# kgcal a=rsl c=cat("",$s{str1},$s{str2},$s{str3}) i=dat1.csv o=rsl2.csv
$ more rsl2.csv
id,str1,str2,str3,rsl
1,abc,def,ghi,abcdefghi
2,A,,CDE,ACDE
3,,,,
4,,,XY,XY
\end{Verbatim}
\subsubsection*{例3: ワイルドカードを利用した例}

\verb|str|から始まる項目(\verb|str1,str2,str3|)をワイルドカード「\verb|str*|」によって指定している。


\begin{Verbatim}[baselinestretch=0.7,frame=single]
$ mcal c='cat("",$s{str*})' a=rsl i=dat1.csv o=rsl3.csv
#END# kgcal a=rsl c=cat("",$s{str*}) i=dat1.csv o=rsl3.csv
$ more rsl3.csv
id,str1,str2,str3,rsl
1,abc,def,ghi,abcdefghi
2,A,,CDE,ACDE
3,,,,
4,,,XY,XY
\end{Verbatim}
