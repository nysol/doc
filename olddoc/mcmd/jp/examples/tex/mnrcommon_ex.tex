\subsubsection*{例1: 基本例}

日付項目の値が\verb|20080203|で、「金額」項目の値が\verb|5|以上\verb|15|未満の行、および\verb|40|以上\verb|50|未満の行を選択する。


\begin{Verbatim}[baselinestretch=0.7,frame=single]
$ more dat1.csv
日付,金額
20080123,10
20080203,10
20080203,20
20080203,45
200804l0,50
$ more ref1.csv
日付,金額F,金額T
20080203,5,15
20080203,40,50
$ mnrcommon k=日付 m=ref1.csv R=金額F,金額T r=金額%n i=dat1.csv o=rsl1.csv
#END# kgnrcommon R=金額F,金額T i=dat1.csv k=日付 m=ref1.csv o=rsl1.csv r=金額%n
$ more rsl1.csv
日付%0,金額
20080203,10
20080203,45
\end{Verbatim}
\subsubsection*{例2: 条件反転}

\verb|-r|を付けると選択条件は反転する。


\begin{Verbatim}[baselinestretch=0.7,frame=single]
$ mnrcommon k=日付 m=ref1.csv R=金額F,金額T r=金額%n -r i=dat1.csv o=rsl2.csv
#END# kgnrcommon -r R=金額F,金額T i=dat1.csv k=日付 m=ref1.csv o=rsl2.csv r=金額%n
$ more rsl2.csv
日付%0,金額
20080123,10
20080203,20
200804l0,50
\end{Verbatim}
