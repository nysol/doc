\subsubsection*{例1: 基本例}

\verb|item|項目別にデータを分割する。
出力ファイル名は0から始まる連番であり、どの番号がどのキーに対応しているかが\verb|table.csv|に出力される。


\begin{Verbatim}[baselinestretch=0.7,frame=single]
$ more dat1.csv
item,no
A,1
A,1
A,2
B,1
B,2
$ msep2 k=item O=./output a=fileName o=table.csv i=dat1.csv
#END# kgsep2 O=./output a=fileName i=dat1.csv k=item o=table.csv
$ ls ./output
0
1
$ more table.csv
item%0,fileName
A,./output/0
B,./output/1
$ more output/0
item%0,no
A,1
A,1
A,2
$ more output/1
item%0,no
B,1
B,2
\end{Verbatim}
\subsubsection*{例2: 複数キー項目}

複数のキー項目\verb|item,no|を設定しても同様に各ファイル名は連番で作成される。
\verb|table.csv|に複数のキー項目と番号の対応表が出力されている。


\begin{Verbatim}[baselinestretch=0.7,frame=single]
$ more dat1.csv
item,no
A,1
A,1
A,2
B,1
B,2
$ msep2 k=item,no O=./output2 a=fileName o=table.csv i=dat1.csv
#END# kgsep2 O=./output2 a=fileName i=dat1.csv k=item,no o=table.csv
$ ls ./output2
0
1
2
3
$ more table.csv
item%0,no%1,fileName
A,1,./output2/0
A,2,./output2/1
B,1,./output2/2
B,2,./output2/3
$ more output/0
item%0,no
A,1
A,1
A,2
\end{Verbatim}
