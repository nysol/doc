\subsubsection*{例1: 基本例}

「顧客」を単位にして「数量」と「金額」項目を基準化(z得点)し、
「数量基準値」と「金額基準値」という項目名で出力する。


\begin{Verbatim}[baselinestretch=0.7,frame=single]
$ more dat1.csv
顧客,数量,金額
A,1,10
A,2,20
B,1,15
B,3,10
B,1,20
$ mnormalize c=z k=顧客 f=数量:数量基準値,金額:金額基準値 i=dat1.csv o=rsl1.csv
#END# kgnormalize c=z f=数量:数量基準値,金額:金額基準値 i=dat1.csv k=顧客 o=rsl1.csv
$ more rsl1.csv
顧客%0,数量,金額,数量基準値,金額基準値
A,1,10,-0.7071067812,-0.7071067812
A,2,20,0.7071067812,0.7071067812
B,1,15,-0.5773502692,0
B,3,10,1.154700538,-1
B,1,20,-0.5773502692,1
\end{Verbatim}
\subsubsection*{例2: 偏差値}



\begin{Verbatim}[baselinestretch=0.7,frame=single]
$ mnormalize c=Z k=顧客 f=数量:数量基準値,金額:金額基準値 i=dat1.csv o=rsl2.csv
#END# kgnormalize c=Z f=数量:数量基準値,金額:金額基準値 i=dat1.csv k=顧客 o=rsl2.csv
$ more rsl2.csv
顧客%0,数量,金額,数量基準値,金額基準値
A,1,10,42.92893219,42.92893219
A,2,20,57.07106781,57.07106781
B,1,15,44.22649731,50
B,3,10,61.54700538,40
B,1,20,44.22649731,60
\end{Verbatim}
\subsubsection*{例3: 0から1への線形変換}



\begin{Verbatim}[baselinestretch=0.7,frame=single]
$ mnormalize c=range k=顧客 f=数量:数量基準値,金額:金額基準値 i=dat1.csv o=rsl3.csv
#END# kgnormalize c=range f=数量:数量基準値,金額:金額基準値 i=dat1.csv k=顧客 o=rsl3.csv
$ more rsl3.csv
顧客%0,数量,金額,数量基準値,金額基準値
A,1,10,0,0
A,2,20,1,1
B,1,15,0,0.5
B,3,10,1,0
B,1,20,0,1
\end{Verbatim}
