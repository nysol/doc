\subsubsection*{(基本例) }
例えば、mjoin(参照ファイルの項目結合)コマンドを利用する際に、入力ファイルのキー項目(k=パラメータで指定する項目)と参照ファイルのキー項目(K=パラメータで指定する項目)が完全に一致しているかどうかを確認したい場合を想定する。mjoinコマンドでNULL値を出力する-n、-Nパラメータを指定しない場合は、入力ファイルと参照ファイルで共通のキー項目のみが結合され、一致しないキー項目の値は除外される為、入力データと出力データの件数が異なる。その際、-assert\_diffSizeパラメータを指定しておくと、入力ファイルと出力ファイルの件数の比較を行い、入力ファイルと出力ファイルの件数が異なる場合に\verb|「#WARNING# ; the number of lines is different」|というメッセージを表示するので入力ファイルと参照ファイルのキー項目が完全に一致していないことを確認することができる。

\begin{Verbatim}[baselinestretch=0.7,frame=single]
$ more dat1.csv
item,date,price
A,20081201,100
A,20081213,98
B,20081002,400
B,20081209,450
C,20081201,100

$ more ref1.csv
item,cost
A,50
B,300
E,200

$ mjoin k=item f=cost m=ref1.csv -assert_diffSize i=dat1.csv o=rsl1.csv
#WARNING# ; the number of lines is different
#END# kgjoin -assert_diffSize f=cost i=dat1.csv k=item m=ref1.csv o=rsl1.csv; IN=5 OUT=4

$ more rsl1.csv
item%0,date,price,cost
A,20081201,100,50
A,20081213,98,50
B,20081002,400,300
B,20081209,450,300
\end{Verbatim}
