\subsubsection*{例1: -nfn指定}

\verb|-nfn|(no field names)を指定すると,先頭行を項目名行と見なさない。
そして項目は必ず番号で指定する(番号は0から始まることに注意する)。


\begin{Verbatim}[baselinestretch=0.7,frame=single]
$ more dat2.csv
a,2
b,5
b,4
$ msum -nfn k=0 f=1 i=dat2.csv o=rsl1.csv
#END# kgsum -nfn f=1 i=dat2.csv k=0 o=rsl1.csv
$ more rsl1.csv
a,2
b,9
\end{Verbatim}
\subsubsection*{例2: -nfno指定}

\verb|-nfno|(no field names for output)を指定すると,入力データの先頭行は項目名行として扱うが,出力データには項目名を出力しない.


\begin{Verbatim}[baselinestretch=0.7,frame=single]
$ more dat1.csv
key,val
a,2
b,5
b,4
$ msum k=key f=val -nfno i=dat1.csv o=rsl2.csv
#END# kgsum -nfno f=val i=dat1.csv k=key o=rsl2.csv
$ more rsl2.csv
a,2
b,9
\end{Verbatim}
\subsubsection*{例3: -nfni指定}

\verb|-nfni|(no field name for input)の指定はmcutでのみ可能であるオプションである.
このオプションは-nfnoと逆の働きをする.
すなわち,入力データの先頭行は項目名行として扱わないが,
出力データには項目名を出力する.
よって、入力項目番号に続けて出力項目名を":"に続けて指定する必要がある.


\begin{Verbatim}[baselinestretch=0.7,frame=single]
$ mcut f=0:key,1:val -nfni i=dat2.csv o=rsl3.csv
#END# kgcut -nfni f=0:key,1:val i=dat2.csv o=rsl3.csv
$ more rsl3.csv
key,val
a,2
b,5
b,4
\end{Verbatim}
\subsubsection*{例4: -x指定}

項目名行があるCSVデータに対して、項目番号で指定したい場合は\verb|-x|オプションを利用する。


\begin{Verbatim}[baselinestretch=0.7,frame=single]
$ msum -x k=0 f=1 i=dat1.csv o=rsl4.csv
#END# kgsum -x f=1 i=dat1.csv k=0 o=rsl4.csv
$ more rsl4.csv
key%0,val
a,2
b,9
\end{Verbatim}
