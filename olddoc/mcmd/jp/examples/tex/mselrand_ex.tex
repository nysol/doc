\subsubsection*{例1: 基本例}

一人の顧客につきランダムに1行を選択する。


\begin{Verbatim}[baselinestretch=0.7,frame=single]
$ more dat1.csv
顧客,日付,金額
A,20081201,10
A,20081207,20
A,20081213,30
B,20081002,40
B,20081209,50
$ mselrand k=顧客 c=1 S=1 i=dat1.csv o=rsl1.csv
#END# kgselrand S=1 c=1 i=dat1.csv k=顧客 o=rsl1.csv
$ more rsl1.csv
顧客%0,日付,金額
A,20081201,10
B,20081002,40
\end{Verbatim}
\subsubsection*{例2: ランダムに一定割合の行を選択}

一人の顧客につきランダムに50\%の行を選択する。
また、それ以外の不一致データは\verb|oth2.csvと|いうファイルに出力する。


\begin{Verbatim}[baselinestretch=0.7,frame=single]
$ mselrand k=顧客 p=50 S=1 u=oth2.csv i=dat1.csv o=rsl2.csv
#END# kgselrand S=1 i=dat1.csv k=顧客 o=rsl2.csv p=50 u=oth2.csv
$ more rsl2.csv
顧客%0,日付,金額
A,20081201,10
B,20081002,40
$ more oth2.csv
顧客%0,日付,金額
A,20081207,20
A,20081213,30
B,20081209,50
\end{Verbatim}
\subsubsection*{例3: キー単位の選択}

以下の例は、顧客\verb|A,B,C,D|の4人からランダムに2人を選ぶ。
顧客\verb|D|が選ばれると、顧客\verb|D|の行は全て出力される。


\begin{Verbatim}[baselinestretch=0.7,frame=single]
$ more dat2.csv
顧客,日付,金額
A,20081201,10
A,20081207,20
A,20081213,30
B,20081002,40
B,20081209,50
C,20081210,60
D,20081201,70
D,20081205,80
D,20081209,90
$ mselrand k=顧客 c=2 S=1 -B i=dat2.csv o=rsl3.csv
#END# kgselrand -B S=1 c=2 i=dat2.csv k=顧客 o=rsl3.csv
$ more rsl3.csv
顧客%0,日付,金額
C,20081210,60
D,20081201,70
D,20081205,80
D,20081209,90
\end{Verbatim}
