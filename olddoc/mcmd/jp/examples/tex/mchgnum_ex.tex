\subsubsection*{例1: 基本例}

\verb|quantity|項目の値が最小以上10未満を\verb|low|、
10以上20未満を\verb|middle|、20以上最大未満を\verb|high|という文字列に置換する。


\begin{Verbatim}[baselinestretch=0.7,frame=single]
$ more dat1.csv
customer,quantity
A,5
B,10
C,15
D,2
E,50
$ mchgnum f=quantity R=MIN,10,20,MAX v=low,middle,high i=dat1.csv o=rsl1.csv
#END# kgchgnum R=MIN,10,20,MAX f=quantity i=dat1.csv o=rsl1.csv v=low,middle,high
$ more rsl1.csv
customer,quantity
A,low
B,middle
C,middle
D,low
E,high
\end{Verbatim}
\subsubsection*{例2: パラメータ範囲にイコールをつける例}

\verb|quantity|項目の値が最小より多く10以下を\verb|low|、
10より多く20以下を\verb|middle|、20より多く最大以下を\verb|high|という文字列に置換する。


\begin{Verbatim}[baselinestretch=0.7,frame=single]
$ mchgnum f=quantity R=MIN,10,20,MAX v=low,middle,high -r i=dat1.csv o=rsl2.csv
#END# kgchgnum -r R=MIN,10,20,MAX f=quantity i=dat1.csv o=rsl2.csv v=low,middle,high
$ more rsl2.csv
customer,quantity
A,low
B,low
C,middle
D,low
E,high
\end{Verbatim}
\subsubsection*{例3: 数値範囲リストに合致しない値を置換}

\verb|quantity|項目の値が10以上20未満を\verb|low|、
20以上30未満を\verb|middle|、30以上最大未満を\verb|high|、
数量が10より小さい値は\verb|out of range|という文字列に置換する。


\begin{Verbatim}[baselinestretch=0.7,frame=single]
$ mchgnum f=quantity R=10,20,30,MAX v=low,middle,high O="out of range" i=dat1.csv o=rsl3.csv
#END# kgchgnum O=out of range R=10,20,30,MAX f=quantity i=dat1.csv o=rsl3.csv v=low,middle,high
$ more rsl3.csv
customer,quantity
A,out of range
B,low
C,low
D,out of range
E,high
\end{Verbatim}
\subsubsection*{例4: 新たな項目の追加}

\verb|quantity|項目の値が最小以上10未満を\verb|low|、
10以上20未満を\verb|middle|、20以上最大未満を\verb|high|という文字列に置換し
\verb|evaluate|という項目名で出力する。


\begin{Verbatim}[baselinestretch=0.7,frame=single]
$ mchgnum f=quantity:evaluate R=MIN,10,20,MAX v=low,middle,high -A i=dat1.csv o=rsl4.csv
#END# kgchgnum -A R=MIN,10,20,MAX f=quantity:evaluate i=dat1.csv o=rsl4.csv v=low,middle,high
$ more rsl4.csv
customer,quantity,evaluate
A,5,low
B,10,middle
C,15,middle
D,2,low
E,50,high
\end{Verbatim}
\subsubsection*{例5: 範囲外を項目の値として出力}

\verb|quantity|項目の値が10以上20未満を\verb|low|、20以上30未満を\verb|middle|、
30以上最大未満を\verb|high|、数量が10より小さい値は置換しないでそのまま出力する。


\begin{Verbatim}[baselinestretch=0.7,frame=single]
$ mchgnum f=quantity R=10,20,30,MAX v=low,middle,high -F i=dat1.csv o=rsl5.csv
#END# kgchgnum -F R=10,20,30,MAX f=quantity i=dat1.csv o=rsl5.csv v=low,middle,high
$ more rsl5.csv
customer,quantity
A,5
B,low
C,low
D,2
E,high
\end{Verbatim}
