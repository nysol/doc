\subsubsection*{例1: 基本例}

概要にて解説した例。
/a/bをキー要素として、5つのCSV項目を出力する。


\begin{Verbatim}[baselinestretch=0.7,frame=single]
$ more dat1.xml
<a att="aa">
  <b att="bb1">
    <c p="pp1" q="qq1"/>
    <d>text1</d>
  </b>
  <b att="bb2">
    <c q="qq2"></c>
    <d>text2</d>
  </b>
  <b>
    <c p="pp3"/>
    <d>text3</d>
  </b>
</a>
$ mxml2csv k=/a/b f=@att:b_att,c@p:c_p,c@p%f:c_p_f,d:d,/a:a i=dat1.xml  o=rsl1.csv
#END# kgxml2csv f=@att:b_att,c@p:c_p,c@p%f:c_p_f,d:d,/a:a i=dat1.xml k=/a/b o=rsl1.csv
$ more rsl1.csv
b_att,c_p,c_p_f,d,a
bb1,pp1,1,text1,text1
bb2,,,text2,text1text2
,pp3,1,text3,text1text2text3
\end{Verbatim}
\subsubsection*{例2: 絶対パス}

基本例と同じ要素を絶対パスで指定する例。
/a/bをキー要素として、5つのCSV項目を出力する。


\begin{Verbatim}[baselinestretch=0.7,frame=single]
$ mxml2csv k=/a/b f=/a/b@att:b_att,/a/b/c@p:c_p,/a/b/c@p%f:c_p_f,/a/b/d:d,/a:a i=dat1.xml  o=rsl2.csv
#END# kgxml2csv f=/a/b@att:b_att,/a/b/c@p:c_p,/a/b/c@p%f:c_p_f,/a/b/d:d,/a:a i=dat1.xml k=/a/b o=rsl2.csv
$ more rsl2.csv
b_att,c_p,c_p_f,d,a
bb1,pp1,1,text1,text1
bb2,,,text2,text1text2
,pp3,1,text3,text1text2text3
\end{Verbatim}
\subsubsection*{例3: キー要素の変更}

絶対パスの例でキー要素をaに変更した例。
aの終了タグは一つしかないので、一行だけ出力されている。
f=で指定した/a/b@attは、2回出現しているが、最後に出現した値が出力されている。


\begin{Verbatim}[baselinestretch=0.7,frame=single]
$ mxml2csv k=/a f=/a/b@att:b_att,/a/b/c@p:c_p,/a/b/c@p%f:c_p_f,/a/b/d:d,/a:a i=dat1.xml o=rsl3.csv
#END# kgxml2csv f=/a/b@att:b_att,/a/b/c@p:c_p,/a/b/c@p%f:c_p_f,/a/b/d:d,/a:a i=dat1.xml k=/a o=rsl3.csv
$ more rsl3.csv
b_att,c_p,c_p_f,d,a
bb2,pp3,1,text3,text1text2text3
\end{Verbatim}
