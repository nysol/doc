\subsubsection*{例1: 基本例}

\verb|item|項目を単位に\verb|date|項目を横に展開し、
\verb|quantity|項目を出力する。


\begin{Verbatim}[baselinestretch=0.7,frame=single]
$ more dat1.csv
item,date,quantity
A,20081201,1
A,20081202,2
A,20081203,3
B,20081201,4
B,20081203,5
$ m2cross k=item f=quantity s=date i=dat1.csv o=rsl1.csv
#END# kg2cross f=quantity i=dat1.csv k=item o=rsl1.csv s=date
$ more rsl1.csv
item%0,20081201,20081202,20081203
A,1,2,3
B,4,,5
\end{Verbatim}
\subsubsection*{例2: 元の入力データに戻す例}

例1の出力結果を元に戻すには、同じく\verb|m2cross|を以下のよう用いればよい。


\begin{Verbatim}[baselinestretch=0.7,frame=single]
$ more rsl1.csv
item%0,20081201,20081202,20081203
A,1,2,3
B,4,,5
$ m2cross f=2008* a=date,quantity i=rsl1.csv o=rsl2.csv
#END# kg2cross a=date,quantity f=2008* i=rsl1.csv o=rsl2.csv
$ more rsl2.csv
item%0,date,quantity
A,20081201,1
A,20081202,2
A,20081203,3
B,20081201,4
B,20081202,
B,20081203,5
\end{Verbatim}
\subsubsection*{例3: 並びを逆順する例}

横に展開する項目名の並びを逆順にする。


\begin{Verbatim}[baselinestretch=0.7,frame=single]
$ m2cross k=item f=quantity s=date%r i=dat1.csv o=rsl3.csv
#END# kg2cross f=quantity i=dat1.csv k=item o=rsl3.csv s=date%r
$ more rsl3.csv
item%0,20081203,20081202,20081201
A,3,2,1
B,5,,4
\end{Verbatim}
\subsubsection*{例4: データがない場合に項目を追加する例}

横に展開する際に、データがない場合に項目を追加する"


\begin{Verbatim}[baselinestretch=0.7,frame=single]
$ more dat2.csv
item,week,quantity
A,Monday,1
A,Tuesday,2
A,Wednesday,3
B,Thursday,4
B,Friday,5
$ m2cross k=item f=quantity s=week i=dat2.csv fixfld=Sunday,Monday,Tuesday,Wednesday,Thursday,Friday,Saturday o=rsl4.csv
#END# kg2cross f=quantity fixfld=Sunday,Monday,Tuesday,Wednesday,Thursday,Friday,Saturday i=dat2.csv k=item o=rsl4.csv s=week
$ more rsl4.csv
item%0,Friday,Monday,Saturday,Sunday,Thursday,Tuesday,Wednesday
A,,1,,,,2,3
B,5,,,,4,,
\end{Verbatim}
