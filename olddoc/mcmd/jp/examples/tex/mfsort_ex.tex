\subsubsection*{例1: 例1: 基本例}

各行において \verb|v1,v2,v3| の値を昇順にならべ、その順番で \verb|v1,v2,v3| 項目として出力する。


\begin{Verbatim}[baselinestretch=0.7,frame=single]
$ more dat1.csv
id,v1,v2,v3
1,b,a,c
2,a,b,a
3,b,,e
$ mfsort f=v* i=dat1.csv o=rsl1.csv
#END# kgfsort f=v* i=dat1.csv o=rsl1.csv
$ more rsl1.csv
id,v1,v2,v3
1,a,b,c
2,a,a,b
3,,b,e
\end{Verbatim}
\subsubsection*{例2: 例2: 降順}

降順にしたければ\verb|-r|を付ける。


\begin{Verbatim}[baselinestretch=0.7,frame=single]
$ mfsort f=v* -r i=dat1.csv o=rsl2.csv
#END# kgfsort -r f=v* i=dat1.csv o=rsl2.csv
$ more rsl2.csv
id,v1,v2,v3
1,c,b,a
2,b,a,a
3,e,b,
\end{Verbatim}
