\subsubsection*{例1: 基本例}

\verb|birthday|項目がNULL値の場合は\verb|"no value"|に置換する。


\begin{Verbatim}[baselinestretch=0.7,frame=single]
$ more dat1.csv
customer,birthday
A,19690103
B,
C,19500501
D,
E,
$ mnullto f=birthday v="no value" i=dat1.csv o=rsl1.csv
#END# kgnullto f=birthday i=dat1.csv o=rsl1.csv v=no value
$ more rsl1.csv
customer,birthday
A,19690103
B,no value
C,19500501
D,no value
E,no value
\end{Verbatim}
\subsubsection*{例2: NULL値以外の置換}

\verb|birthday|項目がNULL値の場合は、\verb|"no value"|
値がある場合は\verb|"value"|置換し\verb|entry|という項目名に変更して出力する。


\begin{Verbatim}[baselinestretch=0.7,frame=single]
$ mnullto f=birthday:entry v="no value" O="value" i=dat1.csv o=rsl2.csv
#END# kgnullto O=value f=birthday:entry i=dat1.csv o=rsl2.csv v=no value
$ more rsl2.csv
customer,entry
A,value
B,no value
C,value
D,no value
E,no value
\end{Verbatim}
\subsubsection*{例3: 新しい項目を出力}

\verb|birthday|項目がNULL値の場合は\verb|"no value"|、値がある場合は\verb|"value"|に置換し\verb|entry|という項目名で出力する。


\begin{Verbatim}[baselinestretch=0.7,frame=single]
$ mnullto f=birthday:entry v="no value" O="value" -A i=dat1.csv o=rsl3.csv
#END# kgnullto -A O=value f=birthday:entry i=dat1.csv o=rsl3.csv v=no value
$ more rsl3.csv
customer,birthday,entry
A,19690103,value
B,,no value
C,19500501,value
D,,no value
E,,no value
\end{Verbatim}
\subsubsection*{例4: 前行の値に置換}



\begin{Verbatim}[baselinestretch=0.7,frame=single]
$ more dat3.csv
id,seq,val
A,1,1
A,3,2
A,2,
B,2,3
B,1,
$ mnullto f=val -p i=dat3.csv o=rsl4.csv
#END# kgnullto -p f=val i=dat3.csv o=rsl4.csv
$ more rsl4.csv
id,seq,val
A,1,1
A,3,2
A,2,2
B,2,3
B,1,3
\end{Verbatim}
\subsubsection*{例5: キー項目を指定した場合の例}



\begin{Verbatim}[baselinestretch=0.7,frame=single]
$ mnullto k=id s=seq f=val -p i=dat3.csv o=rsl5.csv
#END# kgnullto -p f=val i=dat3.csv k=id o=rsl5.csv s=seq
$ more rsl5.csv
id%0,seq%1,val
A,1,1
A,2,1
A,3,2
B,1,
B,2,3
\end{Verbatim}
