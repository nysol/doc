\subsubsection*{例1: 基本例}

日付計算で必要となる基準日を(2007年01月01日と定義した場合)すべての行に「\verb|20070101|」という文字列を追加し「基準日」という項目名で出力する。


\begin{Verbatim}[baselinestretch=0.7,frame=single]
$ more dat1.csv
顧客,日付
A,20081202
A,20081204
B,20081203
$ msetstr v=20070101 a=基準日 i=dat1.csv o=rsl1.csv
#END# kgsetstr a=基準日 i=dat1.csv o=rsl1.csv v=20070101
$ more rsl1.csv
顧客,日付,基準日
A,20081202,20070101
A,20081204,20070101
B,20081203,20070101
\end{Verbatim}
\subsubsection*{例2: 複数項目を追加}



\begin{Verbatim}[baselinestretch=0.7,frame=single]
$ msetstr v=20070101,20070201 a=基準日1,基準日2 i=dat1.csv o=rsl2.csv
#END# kgsetstr a=基準日1,基準日2 i=dat1.csv o=rsl2.csv v=20070101,20070201
$ more rsl2.csv
顧客,日付,基準日1,基準日2
A,20081202,20070101,20070201
A,20081204,20070101,20070201
B,20081203,20070101,20070201
\end{Verbatim}
\subsubsection*{例3: null値項目追加}



\begin{Verbatim}[baselinestretch=0.7,frame=single]
$ msetstr v= a=追加項目 i=dat1.csv o=rsl3.csv
#END# kgsetstr a=追加項目 i=dat1.csv o=rsl3.csv v=
$ more rsl3.csv
顧客,日付,追加項目
A,20081202,
A,20081204,
B,20081203,
\end{Verbatim}
