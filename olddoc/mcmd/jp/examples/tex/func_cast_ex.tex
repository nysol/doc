\subsubsection*{例1: 乱数の固定長化}

1から9999の整数乱数を発生させ、4桁固定長で出力する。
fixlen関数は整数型のデータ(ここではrandiの結果)には対応していないので、
n2s関数で文字列型に変換する必要がある。


\begin{Verbatim}[baselinestretch=0.7,frame=single]
$ more dat1.csv
id
1
2
3
4
$ mcal c='fixlen(n2s(randi(1,9999,11)),4,"R","0")' a=rsl i=dat1.csv o=rsl1.csv
#END# kgcal a=rsl c=fixlen(n2s(randi(1,9999,11)),4,"R","0") i=dat1.csv o=rsl1.csv
$ more rsl1.csv
id,rsl
1,1803
2,0684
3,0195
4,6647
\end{Verbatim}
\subsubsection*{例2: 真偽パターン}

項目v1,v2,v3が10異常かどうかを判定し、01のパターンを出力する。


\begin{Verbatim}[baselinestretch=0.7,frame=single]
$ more dat2.csv
id,v1,v2,v3
1,10,5,7
2,5,12,11
3,3,6,2
4,14,16,11
$ mcal c='cat("",b2s(${v1}>=10),b2s(${v2}>=10),b2s(${v3}>=10))' a=rsl i=dat2.csv o=rsl2.csv
#END# kgcal a=rsl c=cat("",b2s(${v1}>=10),b2s(${v2}>=10),b2s(${v3}>=10)) i=dat2.csv o=rsl2.csv
$ more rsl2.csv
id,v1,v2,v3,rsl
1,10,5,7,100
2,5,12,11,011
3,3,6,2,000
4,14,16,11,111
\end{Verbatim}
