\subsubsection*{例1: 基本例}

 半角スペースで分割


\begin{Verbatim}[baselinestretch=0.7,frame=single]
$ more dat1.csv
id,data
A,1 10 2
A,2 20 3
B,1 15 5
B,3 10 4
B,1 20 6
$ msplit f=data a=d1,d2,d3 i=dat1.csv o=rsl1.csv
#END# kgsplit a=d1,d2,d3 f=data i=dat1.csv o=rsl1.csv
$ more rsl1.csv
id,data,d1,d2,d3
A,1 10 2,1,10,2
A,2 20 3,2,20,3
B,1 15 5,1,15,5
B,3 10 4,3,10,4
B,1 20 6,1,20,6
\end{Verbatim}
\subsubsection*{例2: -r利用}

\verb|-r|を指定することで、\verb|f=|で項目を削除できる。


\begin{Verbatim}[baselinestretch=0.7,frame=single]
$ msplit f=data a=d1,d2,d3 -r i=dat1.csv o=rsl2.csv
#END# kgsplit -r a=d1,d2,d3 f=data i=dat1.csv o=rsl2.csv
$ more rsl2.csv
id,d1,d2,d3
A,1,10,2
A,2,20,3
B,1,15,5
B,3,10,4
B,1,20,6
\end{Verbatim}
\subsubsection*{例3: 分割数不一致}

\verb|a=|で指定した項目数よりも分割できる項目数が少ない場合は、NULLが追加され、
多い場合、先頭から指定した分割数まで出力する


\begin{Verbatim}[baselinestretch=0.7,frame=single]
$ more dat2.csv
id,data
A,1 10 2
A,2 20 3
B,1 15 5
B,3 4
B,1
$ msplit f=data a=d1,d2 i=dat2.csv o=rsl3.csv
#END# kgsplit a=d1,d2 f=data i=dat2.csv o=rsl3.csv
$ more rsl3.csv
id,data,d1,d2
A,1 10 2,1,10
A,2 20 3,2,20
B,1 15 5,1,15
B,3 4,3,4
B,1,1,
\end{Verbatim}
\subsubsection*{例4: delim指定}

\verb|delim=|を使用して半角スペース以外の文字で分割する


\begin{Verbatim}[baselinestretch=0.7,frame=single]
$ more dat3.csv
id,data
A,1_10_3
A,2_20_5
B,1_15_6
B,3_10_7
B,1_20_8
$ msplit f=data a=d1,d2,d3 delim=_ i=dat3.csv o=rsl4.csv
#END# kgsplit a=d1,d2,d3 delim=_ f=data i=dat3.csv o=rsl4.csv
$ more rsl4.csv
id,data,d1,d2,d3
A,1_10_3,1,10,3
A,2_20_5,2,20,5
B,1_15_6,1,15,6
B,3_10_7,3,10,7
B,1_20_8,1,20,8
\end{Verbatim}
