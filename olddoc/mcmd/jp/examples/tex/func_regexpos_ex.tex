\subsubsection*{例1: 基本例}

正規表現\verb|c.*a|に最も長くマッチする部分文字列の位置を得る。
先頭文字の位置は0であることに注意する。
マッチした部分文字列については\verb|regexstr|と同じ入力データを使っているので、比較すると分かりやすい。


\begin{Verbatim}[baselinestretch=0.7,frame=single]
$ more dat1.csv
id,str
1,xcbbbayy
2,xxcbaay
3,
4,bacabbca
$ mcal c='regexpos($s{str},"c.*a")' a=rsl i=dat1.csv o=rsl1.csv
#END# kgcal a=rsl c=regexpos($s{str},"c.*a") i=dat1.csv o=rsl1.csv
$ more rsl1.csv
id,str,rsl
1,xcbbbayy,1
2,xxcbaay,2
3,,
4,bacabbca,2
\end{Verbatim}
\subsubsection*{例2: マルチバイト文字}

正規表現\verb|"い.*あ"|に最も長くマッチする部分文字列の長さを得る。
ただし、以下ではマルチバイト文字対応でないregexpos関数を使っているので、
文字数ではなくバイト数でカウントした場合の位置を返している。


\begin{Verbatim}[baselinestretch=0.7,frame=single]
$ more dat2.csv
id,str
1,漢漢あbbbいyy
2,漢あbいいy
3,
4,bあいあbbいあ
$ mcal c='regexpos($s{str},"あ.*い")' a=rsl i=dat2.csv o=rsl2.csv
#END# kgcal a=rsl c=regexpos($s{str},"あ.*い") i=dat2.csv o=rsl2.csv
$ more rsl2.csv
id,str,rsl
1,漢漢あbbbいyy,6
2,漢あbいいy,3
3,,
4,bあいあbbいあ,1
\end{Verbatim}
\subsubsection*{例3: マルチバイト文字2}

正規表現\verb|"い.*あ"|に最も長くマッチする部分文字列の長さを得る。
regexposw関数を使うと、正しくマルチバイト文字を扱ってくれる。


\begin{Verbatim}[baselinestretch=0.7,frame=single]
$ mcal c='regexposw($s{str},"あ.*い")' a=rsl i=dat2.csv o=rsl3.csv
#END# kgcal a=rsl c=regexposw($s{str},"あ.*い") i=dat2.csv o=rsl3.csv
$ more rsl3.csv
id,str,rsl
1,漢漢あbbbいyy,2
2,漢あbいいy,1
3,,
4,bあいあbbいあ,1
\end{Verbatim}
