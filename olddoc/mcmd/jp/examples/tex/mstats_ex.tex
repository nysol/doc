\subsubsection*{例1: 基本例}

「顧客」項目を単位に「数量」と「金額」項目の
各統計量合計値を計算する。


\begin{Verbatim}[baselinestretch=0.7,frame=single]
$ more dat1.csv
顧客,数量,金額
A,1,10
B,5,20
B,2,10
C,1,15
C,3,10
C,1,21
$ mstats k=顧客 f=数量,金額 c=sum i=dat1.csv o=rsl1.csv
#END# kgstats c=sum f=数量,金額 i=dat1.csv k=顧客 o=rsl1.csv
$ more rsl1.csv
顧客%0,数量,金額
A,1,10
B,7,30
C,5,46
\end{Verbatim}
\subsubsection*{例2: 基本例2}

各統計量最大値を計算する。


\begin{Verbatim}[baselinestretch=0.7,frame=single]
$ mstats k=顧客 f=数量,金額 c=max i=dat1.csv o=rsl2.csv
#END# kgstats c=max f=数量,金額 i=dat1.csv k=顧客 o=rsl2.csv
$ more rsl2.csv
顧客%0,数量,金額
A,1,10
B,5,20
C,3,21
\end{Verbatim}
