
%\documentclass[a4paper]{jsbook}
%\usepackage{mcmd_jp}
%\begin{document}

\section{msep レコードの分割\label{sect:msep}}
\index{msep@msep}
\verb|d=|パラメータで指定したファイル名のデータに各レコードを出力する。
指定するファイル名に項目名を埋め込むことができるので、結果としてレコード分割することになる。
埋め込むファイル名は\verb|${項目名}|によって指定する。
例えば、\verb|d=./out/${date}.csv|と指定すれば、カレントディレクトリの下の\verb|out|ディレクトリの下に、
\verb|date|項目の値別にファイルが作成されることになる。

内部的には、埋め込んだ項目の値をキーとして認識し、並べ替えが行われた後レコードが分割される。

\subsection*{書式}
\verb|msep d= [-p] [f=] |  
\hyperref[sect:option_i]{[i=]}
\hyperref[sect:option_assert_nullin]{[-assert\_nullin]}
\hyperref[sect:option_nfn]{[-nfn]} 
\hyperref[sect:option_nfno]{[-nfno]}  
\hyperref[sect:option_x]{[-x]}
\hyperref[sect:option_q]{[-q]}
\hyperref[sect:option_option_tmppath]{[tmpPath=]}
\hyperref[sect:option_precision]{[precision=]}
\verb|[-params]|
\verb|[--help]|
\verb|[--helpl]|
\verb|[--version]|\\

\subsection*{パラメータ}
\begin{table}[htbp]
%\begin{center}
{\small
\begin{tabular}{ll}
\verb|i=|    & 入力ファイル名を指定する。\\
\verb|d=|    & 異なるデータファイルに分割する項目名を指定する。\\
             & ここで指定した文字列をファイル名として各レコードが追記されていく。\\
             & 項目名は\verb|${項目名}|によって埋め込む。\\
\verb|f=|		 & 出力する項目名を指定する。 \\
\verb|-p|    & \verb|d=| パラメータで指定したディレクトリ名が存在しなければ作成する。\\
\end{tabular} 
}
\end{table} 


\subsection*{利用例}
\subsubsection*{例1: 基本例}

\verb|dat|という名前のディレクトリを作成し、
そのディレクトリに日付項目値\verb|date|ごとに異なるファイルに出力する。


\begin{Verbatim}[baselinestretch=0.7,frame=single]
$ more dat1.csv
item,date,quantity,price
A,20081201,1,10
B,20081201,4,40
A,20081202,2,20
A,20081203,3,30
B,20081203,5,50
$ msep d='./dat/${date}.csv' -p i=dat1.csv
#END# kgsep -p d=./dat/${date}.csv i=dat1.csv
$ ls ./dat
20081201.csv
20081202.csv
20081203.csv
$ more ./dat/20081201.csv
item,date%0,quantity,price
A,20081201,1,10
B,20081201,4,40
$ more ./dat/20081202.csv
item,date%0,quantity,price
A,20081202,2,20
$ more ./dat/20081203.csv
item,date%0,quantity,price
A,20081203,3,30
B,20081203,5,50
\end{Verbatim}


\subsection*{関連コマンド}
\hyperref[sect:msep2]{msep2} : \verb|msep|と同じような機能だが、ファイル名は連番で出力し、キー項目との対応表を別途ファイルに出力する。

\hyperref[sect:mcat] {mcat} : \verb|msep|で分割したファイルをこのコマンドで併合すると元に戻る。

%\end{document}
