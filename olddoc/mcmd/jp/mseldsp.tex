
%\documentclass[a4paper]{jsbook}
%\usepackage{mcmd_jp}
%\begin{document}

\section{mseldsp 選択画面入力\label{sect:mseldsp}}
\index{mseldsp@mseldsp}
\underline{注)本コマンドは開発バージョンであり、仕様が変更される可能性があります。}

座標\verb|x=,y=|で指定したターミナル上の位置に\verb|i=|、
もしくは\verb|seldata=|で指定した文字列リストの選択画面を表示し、
ユーザが選んだ文字列を\verb|o=|で指定したファイルに出力する。
利用者は一つの選択肢しか選択できない。
複数の選択肢を選択せたければ\hyperref[sect:mmseldsp]{mmseldsp}コマンドを利用する。
出力されるCSVファイルは、一項目一行のデータとなる。
入力枠に何も入力せずに終了した場合は、null値が出力される(すなわち改行だけが出力される)。
\verb|f=|を指定すれば、項目名を出力できる。
\verb|f=|を省略すれば、項目名ヘッダーは出力されない。
選択肢の数が多くて画面をはみ出る場合は、\verb|height=|で
スクロール窓の行数を指定すればよい。

選択画面でエンターキーを押すと、終了ステータス0を返して終了し、
エスケープキーを押すと、終了ステータス1を返して終了する。
いずれのキーで終了しても、選択画面で選ばれた内容はファイルに出力される。

座標は左上が\verb|x=1,y=1|である(エスケープシーケンスの仕様)。
\verb|x=|もしくは\verb|y=|で指定した値が1より小さい場合は、1を指定したものとして動作する。
またターミナルの範囲を超えた座標が指定された場合の動作は不定である。


\subsection*{書式}
\verb/mseldsp x= y= [f=] [height=] i=|seldata=/
\hyperref[sect:option_o]{o=}
\hyperref[sect:option_nfn]{[-nfn]} 
\hyperref[sect:option_nfno]{[-nfno]}  
\hyperref[sect:option_x]{[-x]}
\hyperref[sect:option_option_tmppath]{[tmpPath=]}
\hyperref[sect:option_precision]{[precision=]}
\verb|[-params]|
\verb|[--help]|
\verb|[--helpl]|
\verb|[--version]|\\

\subsection*{パラメータ}
\begin{table}[htbp]
%\begin{center}
{\small
\begin{tabular}{ll}
\verb|o=|   & 出力ファイル名を指定する。\\
\verb|x=|   & x軸(左から右への横方向)表示開始位置(1以上の値)を指定する。\\
\verb|y=|   & y軸(上から下への縦方向)表示開始位置(1以上の値)を指定する。\\
\verb|height=| & 選択肢を表示する行数。 \\
\verb|i=|   & 選択肢の文字列を項目として持つCSVファイル名 \\
\verb|f=|   & 選択肢の文字列を項目として持つCSVファイル名 \\
\verb|seldata=| & カンマで区切られた選択肢の文字列リスト \\
\end{tabular} 
}
\end{table} 

\subsection*{利用例}

\subsubsection*{例1: 基本例}

ターミナルのx=10,y=3の位置に\verb|sel.txt|の内容を表示し、
利用者が選んだ文字列を\verb|rsl1.txt|に出力する。

\begin{Verbatim}[baselinestretch=0.7,frame=single,commandchars=\\\{\}]
$ more sel.txt
1:そう思う
2:どちらでもない
3:そう思わない
$ mseldsp x=10 y=3 i=sel.txt o=rsl1.txt
# 利用者が一行目を選んだとする。
$ mose rsl1.txt
1:そう思う


以下、画面イメージ
+--------------------------------------
|
|
|          \textColor{red}{black}{1:そう思う   }
|          \textColor{white}{black}{2:どちらでもない}
|          \textColor{white}{black}{3:そう思わない}
|
\end{Verbatim}

\subsubsection*{例2: 引数で与える例}

例1と同様で、選択肢の文字列を\verb|seldata=|で与える。

\begin{Verbatim}[baselinestretch=0.7,frame=single,commandchars=\\\{\}]
$ mseldsp x=10 y=3 seldata=1:そう思う,2:どちらでもない,3:そう思わない o=rsl2.txt
# 利用者が二行目を選んだとする。
$ mose rsl2.txt
2:どちらでもない


以下、画面イメージ
+--------------------------------------
|
|
|          \textColor{white}{black}{1:そう思う   }
|          \textColor{red}{black}{2:どちらでもない}
|          \textColor{white}{black}{3:そう思わない}
|
\end{Verbatim}


\subsubsection*{例3: 終了ステータスを判定する例}

例1と同じパラメータで実行し、終了ステータスを判定して異なる動作をするスクリプトの例。

\begin{Verbatim}[baselinestretch=0.7,frame=single]
$ more scp.sh
rm -f rsl3.csv
clear
mseldsp x=10 y=3 seldata=aaaa,bbbb,cccc o=rsl3.csv
if [ $? = 0 ] ; then
  clear ; echo "end by enter key"
else
  clear ; echo "end by escape key"
fi

# aaaaを選択後enterキーを入力した場合の結果
$ bash scp.sh
end by enter key
$ more rsl3.csv
aaaa

# bbbbを選択後escapeキーを入力した場合の結果
$ bash scp.sh
end by escape key
$ more rsl3.csv
bbbb
\end{Verbatim}

\subsection*{関連コマンド}
\hyperref[sect:minput] {minput} :入力画面を表示する。

\hyperref[sect:mminput] {mminput} : 複数入力枠による入力画面を表示する。

\hyperref[sect:mdsp] {mdsp} : 画面の指定位置に文字列を表示する。

\hyperref[sect:mmseldsp] {mmseldsp} : 画面に複数選択入力窓を表示する。


%\end{document}
