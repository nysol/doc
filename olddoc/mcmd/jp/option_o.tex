
%\begin{document}

\subsection{o= 出力ファイル名\label{sect:option_o}}
出力ファイル名を指定する。
単一のファイルのみ指定可能である。
ただし、例外として、\verb|mtee|コマンドは複数の出力ファイルを指定でき、
また出力データを必要としないコマンド、例えば、\verb|msep|などもある。

このパラメータが省略された時には標準出力にデータを書き込む。
この機能があるために、パイプラインによる接続が可能となる。
例えば、以下の例では、\verb|msortf|で\verb|o=|を指定していないが、
これは\verb|msortf|の結果が標準出力を通じてパイプラインに出力されるためである。

\begin{Verbatim}[baselinestretch=0.7,frame=single]
$ msortf f=a i=dat.csv | msum k=a f=b o=rsl.csv
\end{Verbatim}

また、上記の似たような処理ではあるが、以下に示した例はうまく動作しない。
上記との違いは\verb|msortf|に\verb|o=|パラメータが指定されている点である。
\verb|msortf|の結果は\verb|tmp.csv|に出力されるが、標準出力に出力するデータがなく、
パイプ接続された\verb|msum|はいつまでも入力データが来るのを待つこととなり、一見動いているように見えて、いつまでも終了しない。

\begin{Verbatim}[baselinestretch=0.7,frame=single]
$ msortf f=a i=dat.csv o=tmp.csv | msum k=a f=b o=rsl.csv
\end{Verbatim}

少し複雑な例であるが、上記の例は以下のように\verb|mtee|コマンドを利用することでうまく動作するようになる。

\begin{Verbatim}[baselinestretch=0.7,frame=single]
$ msortf f=a i=dat.csv | mtee o=tmp.csv | msum k=a f=b o=rsl.csv
\end{Verbatim}

\verb|mtee|コマンドは、標準入力を\verb|o=|で指定されたファイルおよび標準出力に書き出す。
結果として、\verb|msortf|の結果は\verb|tmp.csv|に書きこまれ、\verb|msum|も\verb|mtee|よりパイプラインを通じてデータ供給を受け、
最終結果を\verb|rsl.csv|に書き出す。


\subsection*{利用例}
\subsubsection*{Example 1: Basic Example}

The result of \verb|mcut| is saved to \verb|rsl1.csv| as specified in \verb|o=| parameter.


\begin{Verbatim}[baselinestretch=0.7,frame=single]
$ more dat1.csv
customer,quantity,amount
A,1,10
A,2,20
$ mcut f=customer,amount i=dat1.csv o=rsl1.csv
#END# kgcut f=customer,amount i=dat1.csv o=rsl1.csv
$ more rsl1.csv
customer,amount
A,10
A,20
\end{Verbatim}
\subsubsection*{Example 2: Redirect}

Write to standard input using redirection (\verb|">"|).


\begin{Verbatim}[baselinestretch=0.7,frame=single]
$ mcut f=customer,amount i=dat1.csv >rsl2.csv
#END# kgcut f=customer,amount i=dat1.csv
$ more rsl2.csv
customer,amount
A,10
A,20
\end{Verbatim}


\subsubsection*{対応コマンド}
\verb|sep|など一部のコマンドを除いて全てのコマンドで利用できる。

%\end{document}

