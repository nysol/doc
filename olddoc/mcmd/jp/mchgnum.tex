
%\documentclass[a4paper]{jsbook}
%\usepackage{mcmd_jp}
%\begin{document}

\section{mchgnum 数値範囲による置換\label{sect:mchgnum}}
\index{mchgnum@mchgnum}
\verb|f=|パラメータで指定した項目について、\verb|R=|パラメータで指定する
数値範囲条件と\verb|v=|パラメータで指定する置換文字列により、項目の値を置換する。

\subsection*{書式}
\verb/mchgnum f= R= [O=|-F] [v=] [-A] [-r]/
\hyperref[sect:option_i]{[i=]}
\hyperref[sect:option_o]{[o=]}
\hyperref[sect:option_assert_diffSize]{[-assert\_diffSize]}
\hyperref[sect:option_assert_nullin]{[-assert\_nullin]}
\hyperref[sect:option_assert_nullout]{[-assert\_nullout]}
\hyperref[sect:option_nfn]{[-nfn]} 
\hyperref[sect:option_nfno]{[-nfno]}
\hyperref[sect:option_x]{[-x]}
\hyperref[sect:option_option_tmppath]{[tmpPath=]}
\hyperref[sect:option_precision]{[precision=]}
\verb|[-params]|
\verb|[--help]|
\verb|[--helpl]|
\verb|[--version]|\\

\subsection*{パラメータ}
\begin{table}[htbp]
%\begin{center}
{\small
\begin{tabular}{ll}
\verb|i=|    & 入力ファイル名を指定する。\\
\verb|o=|    & 出力ファイル名を指定する。\\
\verb|f=|    & ここで指定した項目(複数項目指定可)の数値を\verb|R=|と\verb|v=|パラメータで指定した\\
             & 数値範囲リストおよび置換文字列リストに従って置換する。\\
\verb|R=|    & 置換対象となる数値範囲を指定(複数項目指定可)する(\verb|1.1,2.5| : 1.1以上2.5未満)。\\
             & 最小値、最大値として\verb|MIN,MAX|を使うことができる(\verb|MIN,2.5| : 2.5未満)。\\
\verb|O=|    & 範囲外文字列\\
             & \verb|R=|パラメータで指定した数値範囲リストのいずれとも合致しない値を\\
             & 置換するときの文字列(指定がなければNULL値となる)を指定する。\\
\verb|-F|    & 範囲外を項目の値として出力\\
             & \verb|R=|パラメータで指定した数値範囲リストのいずれとも\\
             & 合致しない値は、その項目の値のまま出力する。\\
\verb|v=|    & \verb|R=|パラメータで指定した数値範囲に対応する置換文字列を指定する。\\
             & \verb|R=|で指定した値の個数より1つ少い個数でなければならない。\\
\verb|-A|    & このオプションにより、指定した項目を置き換えるのではなく、新たに項目が追加される。\\
\verb|-r|    & \verb|R=|パラメータの範囲を'〜より大きく〜以下'として扱う。\\
             & 例えば、\verb|1.1_2.5|は「1.1より大きく2.5以下」として扱う。\\
\end{tabular} 
}
\end{table} 

\subsection*{利用例}
\subsubsection*{Example 1: Basic Example}

Encodes the numeric values in \verb|quantity| column to character strings where values of less than but not equal to 10 are treated as \verb|low|, 10 or more but less than 20 are treated as \verb|middle|, values of 20 or more is treated as \verb|high|.


\begin{Verbatim}[baselinestretch=0.7,frame=single]
$ more dat1.csv
customer,quantity
A,5
B,10
C,15
D,2
E,50
$ mchgnum f=quantity R=MIN,10,20,MAX v=low,middle,high i=dat1.csv o=rsl1.csv
#END# kgchgnum R=MIN,10,20,MAX f=quantity i=dat1.csv o=rsl1.csv v=low,middle,high
$ more rsl1.csv
customer,quantity
A,low
B,middle
C,middle
D,low
E,high
\end{Verbatim}
\subsubsection*{Example 2: Equal to paramter range}

Replace the numeric values in \verb|quantity| column to character strings where 10 or below is treated as \verb|low|, more than 10 but less than or equal to 20 is treated as \verb|middle|, more than 20 is treated as \verb|high|.


\begin{Verbatim}[baselinestretch=0.7,frame=single]
$ mchgnum f=quantity R=MIN,10,20,MAX v=low,middle,high -r i=dat1.csv o=rsl2.csv
#END# kgchgnum -r R=MIN,10,20,MAX f=quantity i=dat1.csv o=rsl2.csv v=low,middle,high
$ more rsl2.csv
customer,quantity
A,low
B,low
C,middle
D,low
E,high
\end{Verbatim}
\subsubsection*{Example 3: Replace values out of the list of numeric range}

Replace numeric values in \verb|quantity| column to character strings  where 10 or above and less than 20 is coded as \verb|low|, 20 or above and less than 30 is coded as \verb|middle|, 30 or more is coded as \verb|high|, values that are less than 10 is coded as \verb|OutOfRange|.


\begin{Verbatim}[baselinestretch=0.7,frame=single]
$ mchgnum f=quantity R=10,20,30,MAX v=low,middle,high O="OutOfRange" i=dat1.csv o=rsl3.csv
#END# kgchgnum O=OutOfRange R=10,20,30,MAX f=quantity i=dat1.csv o=rsl3.csv v=low,middle,high
$ more rsl3.csv
customer,quantity
A,OutOfRange
B,low
C,low
D,OutOfRange
E,high
\end{Verbatim}
\subsubsection*{Example 4: Add a new column}

Replace the numeric values in \verb|quantity| column to character strings where values less than 10  is treated as \verb|low|, 10 or above but less than 20 is treated as \verb|middle|, 20 or above is treated as \verb|high|. Store the output of replacement strings in a new column as \verb|evaluate|.


\begin{Verbatim}[baselinestretch=0.7,frame=single]
$ mchgnum f=quantity:evaluate R=MIN,10,20,MAX v=low,middle,high -A i=dat1.csv o=rsl4.csv
#END# kgchgnum -A R=MIN,10,20,MAX f=quantity:evaluate i=dat1.csv o=rsl4.csv v=low,middle,high
$ more rsl4.csv
customer,quantity,evaluate
A,5,low
B,10,middle
C,15,middle
D,2,low
E,50,high
\end{Verbatim}
\subsubsection*{Example 5: Display original values in column if out of defined range}

Replace the numeric values in \verb|quantity| column to character strings where values of 10 or above but less than 20 is coded as \verb|low|, 20 or above but less than 30 is coded as \verb|middle|, 30 or above is coded as \verb|high|. Retain original values in the output if the value is less than 10.


\begin{Verbatim}[baselinestretch=0.7,frame=single]
$ mchgnum f=quantity R=10,20,30,MAX v=low,middle,high -F i=dat1.csv o=rsl5.csv
#END# kgchgnum -F R=10,20,30,MAX f=quantity i=dat1.csv o=rsl5.csv v=low,middle,high
$ more rsl5.csv
customer,quantity
A,5
B,low
C,low
D,2
E,high
\end{Verbatim}

\subsection*{関連コマンド}

\hyperref[sect:mchgstr]{mchgstr} : 文字列の置換であればこちら。

\hyperref[sect:msed]{msed} : 正規表現を使った置換が可能。

%\end{document}
