
%\begin{document}

\section{dist 距離\label{sect:dist}}
\index{dist@dist}

書式: dist(タイプ,$num_1,num_2,\cdots,n_k,num_{k+1},num_{k+2},\cdots,num_{2k}$)

2つのk次元ベクトル($num_1,num_2,\cdots,n_k),(num_{k+1},num_{k+2},\cdots,num_{2k}$)の距離を計算する。
距離としては以下のものが利用できる。
詳細な定義はmsimを参照のこと。

\begin{itemize}
\item \verb|euclid|: ユークリッド距離
\item \verb|cityblock|: 都市ブロック距離
\item \verb|hamming|: ハミング距離
\end{itemize}

ハミング距離については、値は文字型として指定しなければならない(以下の例を参考のこと)。

\subsection*{利用例}
\subsubsection*{Example 1: Euclidean distance }



\begin{Verbatim}[baselinestretch=0.7,frame=single]
$ more dat1.csv
id,x1,y1,x2,y2
1,0,0,1,1
2,0,1,2,0
3,,,,
$ mcal c='dist("euclid",${x1},${y1},${x2},${y2})' a=rsl i=dat1.csv o=rsl1.csv
#END# kgcal a=rsl c=dist("euclid",${x1},${y1},${x2},${y2}) i=dat1.csv o=rsl1.csv
$ more rsl1.csv
id,x1,y1,x2,y2,rsl
1,0,0,1,1,1.414213562
2,0,1,2,0,2.236067977
3,,,,,
\end{Verbatim}
\subsubsection*{Example 2: City Block distance}



\begin{Verbatim}[baselinestretch=0.7,frame=single]
$ mcal c='dist("cityblock",${x1},${y1},${x2},${y2})' a=rsl i=dat1.csv o=rsl2.csv
#END# kgcal a=rsl c=dist("cityblock",${x1},${y1},${x2},${y2}) i=dat1.csv o=rsl2.csv
$ more rsl2.csv
id,x1,y1,x2,y2,rsl
1,0,0,1,1,2
2,0,1,2,0,3
3,,,,,
\end{Verbatim}
\subsubsection*{Example 3: Hamming distance}

Hamming distance must be specified in character string format for the calculation of Hamming distance.


\begin{Verbatim}[baselinestretch=0.7,frame=single]
$ more dat2.csv
id,x1,y1,x2,y2
1,a,b,a,c
2,0,1,0,1
3,,,,
$ mcal c='dist("hamming",$s{x1},$s{y1},$s{x2},$s{y2})' a=rsl i=dat2.csv o=rsl3.csv
#END# kgcal a=rsl c=dist("hamming",$s{x1},$s{y1},$s{x2},$s{y2}) i=dat2.csv o=rsl3.csv
$ more rsl3.csv
id,x1,y1,x2,y2,rsl
1,a,b,a,c,1
2,0,1,0,1,2
3,,,,,
\end{Verbatim}


%\end{document}

