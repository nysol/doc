
%\documentclass[a4paper]{jsbook}
%\usepackage{mcmd_jp}
%\begin{document}

\section{mtra 縦型データをベクトル項目に変換\label{sect:mtra}}
\index{mtra@mtra}
\verb|f=|パラメータで指定した項目値をアイテムとし、それらのアイテムを連結し新しいベクトル項目(トランザクション項目とも呼ぶ)として出力する。
アイテムの区切り文字は\verb|delim=|パラメータで指定する。

\subsection*{書式}
\verb|mtra f= [s=] [k=] [delim=] [-r]  | 
\hyperref[sect:option_i]{[i=]}
\hyperref[sect:option_o]{[o=]}
\hyperref[sect:option_assert_diffSize]{[-assert\_diffSize]}
\hyperref[sect:option_assert_nullkey]{[-assert\_nullkey]}
\hyperref[sect:option_assert_nullin]{[-assert\_nullin]}
\hyperref[sect:option_assert_nullout]{[-assert\_nullout]}
\hyperref[sect:option_nfn]{[-nfn]} 
\hyperref[sect:option_nfno]{[-nfno]}  
\hyperref[sect:option_x]{[-x]}
\hyperref[sect:option_q]{[-q]}
\hyperref[sect:option_option_tmppath]{[tmpPath=]}
\hyperref[sect:option_precision]{[precision=]}
\verb|[-params]|
\verb|[--help]|
\verb|[--helpl]|
\verb|[--version]|\\

\subsection*{パラメータ}
\begin{table}[htbp]
%\begin{center}
{\small
\begin{tabular}{ll}
\verb|i=|        & 入力ファイル名を指定する。\\
\verb|o=|        & 出力ファイル名を指定する。\\
\verb|f=|        & ここで指定した項目(複数項目指定可)の値がアイテムとして連結されトランザクション項目となる。\\
                 & NULL値は無視される。\\
\verb|s=|        & ここで指定した項目(複数項目指定可)で並べ替えられた後、変換が行われる。\\
%\verb|k=|        & 文字列パターンの単位となる項目名(複数項目指定可)リスト。【\hyperref[sect:option_k]{集計キーブレイク処理}】\\
\verb|k=|        & 文字列パターンの単位となる項目名(複数項目指定可)リスト。\\
                 & \verb|-r|オプションが指定された時は指定できない。\\
\verb|delim=|    & ここで指定した文字を区切り文字とする(省略時はスペース)。\\
\verb|-r|        & 条件反転\\
                 & トランザクション項目を縦型データに変換する。\\
\end{tabular} 
}
\end{table} 

\subsection*{利用例}
\subsubsection*{例1: 基本例}

\verb|customer|を単位に\verb|item|をスペース区切りで結合し、
\verb|transaction|という項目名で出力する。


\begin{Verbatim}[baselinestretch=0.7,frame=single]
$ more dat1.csv
customer,item
A,a
A,b
B,c
B,d
B,e
$ mtra k=customer f=item:transaction i=dat1.csv o=rsl1.csv
#END# kgtra f=item:transaction i=dat1.csv k=customer o=rsl1.csv
$ more rsl1.csv
customer%0,transaction
A,a b
B,c d e
\end{Verbatim}
\subsubsection*{例2: アイテムの区切り文字を-(ハイフン)で実行}



\begin{Verbatim}[baselinestretch=0.7,frame=single]
$ mtra k=customer f=item:transaction delim=- i=dat1.csv o=rsl2.csv
#END# kgtra delim=- f=item:transaction i=dat1.csv k=customer o=rsl2.csv
$ more rsl2.csv
customer%0,transaction
A,a-b
B,c-d-e
\end{Verbatim}
\subsubsection*{例3: アイテムを降順に並べ替えてから変換}



\begin{Verbatim}[baselinestretch=0.7,frame=single]
$ mtra k=customer s=item%r f=item:transaction i=dat1.csv o=rsl3.csv
#END# kgtra f=item:transaction i=dat1.csv k=customer o=rsl3.csv s=item%r
$ more rsl3.csv
customer%0,transaction
A,b a
B,e d c
\end{Verbatim}

\subsection*{関連コマンド}
\hyperref[sect:mvsort]{mvsort} : トランザクションデータは、ベクトル型データを処理する一連の処理コマンド(\verb|mv|から始まる)によって加工できる。

\hyperref[sect:mcross]{mcross} : トランザクションデータとしてではなく、個々のアイテムを独立した項目として出力し、その出現件数を出力する。

\hyperref[sect:mtrafld]{mtrafld} : 「項目名=値」の形式でトランザクションデータを作成する。

\hyperref[sect:mtraflg]{mtraflg} : 項目名をアイテムとしてトランザクションデータを作成する。

%\end{document}
