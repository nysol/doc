
%\documentclass[a4paper]{jsbook}
%\usepackage{mcmd_jp}
%\begin{document}

\section{mcombi 組合せ計算\label{sect:mcombi}}
\index{mcombi@mcombi}
\verb|f=|パラメータで指定した項目について、
\verb|n=|パラメータで指定した数の組み合せを求め、
\verb|a=|パラメータで指定した項目名で出力する。
\verb|-p|を指定することで順列として出力することも可能である。

\subsection*{書式}
\verb/mcombi a= f= n= [s=] [k=] [-p] [-dup]/
\hyperref[sect:option_i]{[i=]}
\hyperref[sect:option_o]{[o=]}
\hyperref[sect:option_assert_diffSize]{[-assert\_diffSize]}
\hyperref[sect:option_assert_nullin]{[-assert\_nullin]}
\hyperref[sect:option_nfn]{[-nfn]} 
\hyperref[sect:option_nfno]{[-nfno]}  
\hyperref[sect:option_x]{[-x]}
\hyperref[sect:option_q]{[-q]}
\hyperref[sect:option_option_tmppath]{[tmpPath=]}
\hyperref[sect:option_precision]{[precision=]}
\verb|[-params]|
\verb|[--help]|
\verb|[--helpl]|
\verb|[--version]|\\

\subsection*{パラメータ}
\begin{table}[htbp]
%\begin{center}
{\small
\begin{tabular}{ll}
\verb|i=|    & 入力ファイル名を指定する。\\
\verb|o=|    & 出力ファイル名を指定する。\\
\verb|a=|    & 新たに追加される項目の名前を指定する。\\
\verb|f=|    & 組合せを求める項目名リスト(複数項目指定可)を指定する。\\
             & ここで指定した項目の値の全組合せを出力する。\\
\verb|n=|    & 組合せの数を指定する。\\
             & 組み合わせ数を大きくすると、出力レコードが爆発的に大きくなることに注意する。\\
\verb|s=|    & ここで指定した項目(複数項目指定可)で並べ替えられた後、\verb|f=|で指定した項目の組み合わせを求める。\\
\verb|k=|    & キー項目名リスト(複数項目指定可)\\
             & 組合せを求める単位となる項目名リスト。\\
\verb|-p|    & 組合せでなく順列を求める。\\
\verb|-dup|  & 同一の値の組み合せも出力する\\
\end{tabular} 
}
\end{table} 

\subsection*{利用例}
\subsubsection*{Example 1: Basic Example}

Enumerate all combinations of two items in the \verb|item| field for each \verb|customer|, and save the output in \verb|item1,item2|.  Fields not specified at \verb|k=,f=| (\verb|item| field in this case) remains after the key field column.


\begin{Verbatim}[baselinestretch=0.7,frame=single]
$ more dat1.csv
customer,item
A,a1
A,a2
A,a3
B,a4
B,a5
$ mcombi k=customer f=item n=2 a=item1,item2 i=dat1.csv o=rsl1.csv
#END# kgcombi a=item1,item2 f=item i=dat1.csv k=customer n=2 o=rsl1.csv
$ more rsl1.csv
customer%0,item,item1,item2
A,a3,a1,a2
A,a3,a1,a3
A,a3,a2,a3
B,a5,a4,a5
\end{Verbatim}
\subsubsection*{Example 2: Basic Example 2}

When you specify the \verb|-dup| option, the output includes combination of the same field.


\begin{Verbatim}[baselinestretch=0.7,frame=single]
$ mcombi k=customer f=item n=2 a=item1,item2 i=dat1.csv o=rsl2.csv -dup
#END# kgcombi -dup a=item1,item2 f=item i=dat1.csv k=customer n=2 o=rsl2.csv
$ more rsl2.csv
customer%0,item,item1,item2
A,a3,a1,a1
A,a3,a1,a2
A,a3,a1,a3
A,a3,a2,a2
A,a3,a2,a3
A,a3,a3,a3
B,a5,a4,a4
B,a5,a4,a5
B,a5,a5,a5
\end{Verbatim}
\subsubsection*{Example 3: Compute permutation}

Enumerate permutation of two items in the \verb|item| field for each \verb|customer|, and save the output in column \verb|item1,item2|.


\begin{Verbatim}[baselinestretch=0.7,frame=single]
$ mcombi k=customer f=item n=2 a=item1,item2 -p i=dat1.csv o=rsl3.csv
#END# kgcombi -p a=item1,item2 f=item i=dat1.csv k=customer n=2 o=rsl3.csv
$ more rsl3.csv
customer%0,item,item1,item2
A,a3,a1,a2
A,a3,a2,a1
A,a3,a1,a3
A,a3,a3,a1
A,a3,a2,a3
A,a3,a3,a2
B,a5,a4,a5
B,a5,a5,a4
\end{Verbatim}

\subsection*{関連コマンド}

%\end{document}
