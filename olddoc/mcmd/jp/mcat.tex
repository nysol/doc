
%\documentclass[a4paper]{jsbook}
%\usepackage{mcmd_jp}
%\begin{document}

\section{mcat 併合\label{sect:mcat}}
\index{mcat@mcat}
\verb|i=|パラメータで指定した全ファイルのレコードを、指定した順に併合する。
ワイルドカードでファイル名を指定した場合は、ファイル名のアルファベット順に併合される。

\subsection*{書式}
\verb/mcat [f=] [-skip_fnf] [-skip_zero] [flist=] [kv=] [-nostop|-skip|-force]/
\hyperref[sect:option_i]{[i=]}
\hyperref[sect:option_o]{[o=]}
\hyperref[sect:option_add_fname]{[-add\_fname]}
\hyperref[sect:option_-stdin]{[-stdin]}
\hyperref[sect:option_assert_diffSize]{[-assert\_diffSize]}
\hyperref[sect:option_assert_nullin]{[-assert\_nullin]}
\hyperref[sect:option_nfn]{[-nfn]} 
\hyperref[sect:option_nfno]{[-nfno]}  
\hyperref[sect:option_x]{[-x]}
\hyperref[sect:option_option_tmppath]{[tmpPath=]}
\hyperref[sect:option_precision]{[precision=]}
\verb|[-params]|
\verb|[--help]|
\verb|[--helpl]|
\verb|[--version]|\\

\subsection*{パラメータ}
\begin{table}[htbp]
%\begin{center}
{\small
\begin{tabular}{ll}
\verb|i=|        & 入力ファイル名リストを指定する。\\
                 & 複数のファイルをカンマで区切って指定する。ワイルドカードを用いることができる。\\
\verb|o=|  			 & 出力ファイル名を指定する。\\
\verb|f=|        & 併合する項目名を指定する。\\
                 & 指定を省略すれば\verb|i=|で指定した1つ目のファイルの項目名が使われる。\\
\verb|-skip_fnf| & \verb|i=|で指定したファイルが存在しなくてもエラー終了しない。\\
                 & ただし、全ファイルがなければエラーとなる。\\
\verb|-nostop|   & \verb|-nostop| ,\verb|-skip|,\verb|-force|は、指定の項目名がなかったときの動作を制御するフラグである。\\
                 & \verb|-nostop|は、指定の項目名がなければnullを出力する。\\
                 & \verb|-nfn|が同時に指定された場合,項目数が異なればエラー終了する。\\
\verb|-skip|     & 指定の項目名がなければそのファイルは併合しない。\\
                 & \verb|-nfn|が同時に指定された場合、項目数が異なればそのファイルは併合しない。\\
\verb|-skip_zero|	& -nfnを指定していない場合でも0バイトファイルでエラーにならないようにする。 \\
\verb|flist=|		 & 併合するファイルリストをCSVデータとして指定する。flist=fileName:fldNameで指定する。\\
\verb|kv=|			 & パス名に含めた”key-value”の文字列を抜き出し項目名とその値としてデータに付加する。\\
\verb|-force|    & 指定の項目名がなければ,項目番号で強制併合する。\\
                 & 指定の項目番号がなければnullを出力する。\\
\verb|-stdin|    & 標準入力も併合する。\\
\verb|-add_fname|& 併合元のファイル名を最終項目として追加する。\\
                 & 標準入力は\verb|/dev/stdin|という名称になる。\\
                 & 項目名は\verb|"fileName"|固定なので、入力データに同一の項目名があるとエラーとなる。\\
\end{tabular}
}
\end{table}

\subsection*{備考}
\begin{itemize}
\item 複数ファイルの指定にワイルドカード("*"と"?")を利用することができる。ファイル名だけでなくディレクトリ名に対しても指定することができる。
\item ホームディレクトリ記号(\verb|~/|)も利用可能。
\item 併合される順序は\verb|i=|で指定したファイルの出現順。ワイルドカードを指定した場合は、アルファベット順。標準入力は最初に併合される。
\end{itemize}

\subsection*{利用例}
\subsubsection*{Example 1: Concatenate files with the same header}



\begin{Verbatim}[baselinestretch=0.7,frame=single]
$ more dat1.csv
customer,date,amount
A,20081201,10
B,20081002,40
$ more dat2.csv
customer,date,amount
A,20081207,20
A,20081213,30
B,20081209,50
$ mcat i=dat1.csv,dat2.csv o=rsl1.csv
#END# kgcat i=dat1.csv,dat2.csv o=rsl1.csv
$ more rsl1.csv
customer,date,amount
A,20081201,10
B,20081002,40
A,20081207,20
A,20081213,30
B,20081209,50
\end{Verbatim}
\subsubsection*{Example 2: Concatenate files with different header}

The first file \verb|dat1.csv| defined at \verb|i=| contains columns "customer,date,amount". However, since "amount" is not present in \verb|dat3.csv|, it will return an error. Nevertheless, the contents in the first file \verb|dat1.csv| is merged and saved in the output.


\begin{Verbatim}[baselinestretch=0.7,frame=single]
$ more dat3.csv
customer,date,quantity
A,20081201,3
B,20081002,1
$ mcat i=dat1.csv,dat3.csv o=rsl2.csv
#ERROR# field name [amount] not found on file [dat3.csv] (kgcat)
$ more rsl2.csv
customer,date,amount
A,20081201,10
B,20081002,40
\end{Verbatim}
\subsubsection*{Example 3: Concatenate files with different header2}

When previous example is attached with \verb|-nostop| option, the command will continue processing and return NULL value for the data item not found. Other options such as \verb|skip,force| handle conditions when the field name is not found. For details, refer to the description of parameters.


\begin{Verbatim}[baselinestretch=0.7,frame=single]
$ more dat3.csv
customer,date,quantity
A,20081201,3
B,20081002,1
$ mcat -nostop i=dat1.csv,dat3.csv o=rsl3.csv
#END# kgcat -nostop i=dat1.csv,dat3.csv o=rsl3.csv
$ more rsl3.csv
customer,date,amount
A,20081201,10
B,20081002,40
A,20081201,
B,20081002,
\end{Verbatim}
\subsubsection*{Example 4: Concatenate specific field names from input files}

Merge field names specified at \verb|f=|.


\begin{Verbatim}[baselinestretch=0.7,frame=single]
$ mcat f=customer,date i=dat2.csv,dat3.csv o=rsl4.csv
#END# kgcat f=customer,date i=dat2.csv,dat3.csv o=rsl4.csv
$ more rsl4.csv
customer,date
A,20081207
A,20081213
B,20081209
A,20081201
B,20081002
\end{Verbatim}
\subsubsection*{Example 5: Merge from standard input}

Read file \verb|dat2.csv| from standard input by specifying \verb|-stdin| option.



\begin{Verbatim}[baselinestretch=0.7,frame=single]
$ mcat -stdin i=dat1.csv o=rsl5.csv <dat2.csv
#END# kgcat -stdin i=dat1.csv o=rsl5.csv
$ more rsl5.csv
customer,date,amount
A,20081207,20
A,20081213,30
B,20081209,50
A,20081201,10
B,20081002,40
\end{Verbatim}
\subsubsection*{Example 6: Add file name as new column}

When \verb|-add_fname| is specified, the original file name \verb|fileName| is added as a new column.
File name of standard input is \verb|/dev/stdin|.


\begin{Verbatim}[baselinestretch=0.7,frame=single]
$ mcat -add_fname -stdin i=dat1.csv o=rsl6.csv <dat2.csv
#END# kgcat -add_fname -stdin i=dat1.csv o=rsl6.csv
$ more rsl6.csv
customer,date,amount,fileName
A,20081207,20,/dev/stdin
A,20081213,30,/dev/stdin
B,20081209,50,/dev/stdin
A,20081201,10,dat1.csv
B,20081002,40,dat1.csv
\end{Verbatim}
\subsubsection*{Example 7: Specify wild card}

Specifying wild card \verb|dat*.csv| to concatenate the three CSV files \verb|dat1.csv,dat2.csv,dat3.csv| in the current directory.


\begin{Verbatim}[baselinestretch=0.7,frame=single]
$ more dat1.csv
customer,date,amount
A,20081201,10
B,20081002,40
$ more dat2.csv
customer,date,amount
A,20081207,20
A,20081213,30
B,20081209,50
$ more dat3.csv
customer,date,quantity
A,20081201,3
B,20081002,1
$ mcat -force i=dat*.csv o=rsl7.csv
#END# kgcat -force i=dat*.csv o=rsl7.csv
$ more rsl7.csv
customer,date,amount
A,20081201,10
B,20081002,40
A,20081207,20
A,20081213,30
B,20081209,50
A,20081201,3
B,20081002,1
\end{Verbatim}
\subsubsection*{Example 8: Concatenate the same file multiple times}

Same file can be specified more than one time.


\begin{Verbatim}[baselinestretch=0.7,frame=single]
$ mcat i=dat1.csv,dat1.csv,dat1.csv o=rsl8.csv
#END# kgcat i=dat1.csv,dat1.csv,dat1.csv o=rsl8.csv
$ more rsl8.csv
customer,date,amount
A,20081201,10
B,20081002,40
A,20081201,10
B,20081002,40
A,20081201,10
B,20081002,40
\end{Verbatim}


\subsection*{関連コマンド}
\hyperref[sect:msep]{msep} : ちょうど逆の動きをする。

%\end{document}

