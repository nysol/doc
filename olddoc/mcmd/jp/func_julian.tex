
%\begin{document}

\section{julian ユリウス暦変換\label{sect:julian}}
\index{julian@julian}

書式1: julian($date$)

書式2: julian($time$)

書式3: julian2d($num$)

書式4: julian2t($num$)

書式1,2では、日付$date$もしくは時刻$time$をユリウス通日に変換する。
逆に書式3,4では、ユリウス通日を日付型もしくは時刻型に変換する。
ここで、日付型が与えられたときは、その日の最初の時刻である\verb|00:00:00|として計算される。

\subsection*{利用例}
\subsubsection*{例1: 基本例}

日付型の\verb|date|項目を\verb|julian|関数でユリウス通日に変換し、\verb|julian2d|関数でまたもとに戻す。


\begin{Verbatim}[baselinestretch=0.7,frame=single]
$ more dat1.csv
id,date
1,20000101
2,20121021
3,
4,19700101
$ mcal c='julian($d{date})' a=julian i=dat1.csv o=rsl1.csv
#END# kgcal a=julian c=julian($d{date}) i=dat1.csv o=rsl1.csv
$ more rsl1.csv
id,date,julian
1,20000101,2451545
2,20121021,2456222
3,,
4,19700101,2440588
$ mcal c='julian2d(${julian})' a=date2 i=rsl1.csv o=rsl2.csv
#END# kgcal a=date2 c=julian2d(${julian}) i=rsl1.csv o=rsl2.csv
$ more rsl2.csv
id,date,julian,date2
1,20000101,2451545,20000101
2,20121021,2456222,20121021
3,,,
4,19700101,2440588,19700101
\end{Verbatim}
\subsubsection*{例2: 時刻型も同様}



\begin{Verbatim}[baselinestretch=0.7,frame=single]
$ more dat2.csv
id,time
1,20000101000000
2,20121021111213
3,
4,19700101000100
$ mcal c='julian($t{time})' a=julian i=dat2.csv o=rsl3.csv
#END# kgcal a=julian c=julian($t{time}) i=dat2.csv o=rsl3.csv
$ more rsl3.csv
id,time,julian
1,20000101000000,2451544.5
2,20121021111213,2456221.967
3,,
4,19700101000100,2440587.501
$ mcal c='julian2t(${julian})' a=time2 i=rsl3.csv o=rsl4.csv
#END# kgcal a=time2 c=julian2t(${julian}) i=rsl3.csv o=rsl4.csv
$ more rsl4.csv
id,time,julian,time2
1,20000101000000,2451544.5,20000101000000
2,20121021111213,2456221.967,20121021111228.800015
3,,,
4,19700101000100,2440587.501,19700101000126.400014
\end{Verbatim}


%\end{document}

