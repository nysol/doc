
%\begin{document}

\section{mnewstr 固定文字列データの新規生成\label{sect:mnewstr}}
\index{mnewstr@mnewstr}
\verb|v=|パラメータで指定した文字列データを新規作成し、\verb|a=|パラメータで指定した項目名で出力する。
一度に複数の項目を生成することも可能。

\subsection*{書式}
\verb|mnewstr a= [v=] [l=]|
\hyperref[sect:option_o]{[o=]}
\hyperref[sect:option_nfn]{[-nfn]} 
\hyperref[sect:option_nfno]{[-nfno]}  
\hyperref[sect:option_x]{[-x]}
\hyperref[sect:option_option_tmppath]{[tmpPath=]}
\hyperref[sect:option_precision]{[precision=]}
\verb|[-params]|
\verb|[--help]|
\verb|[--helpl]|
\verb|[--version]|\\

\subsection*{パラメータ}
\begin{table}[htbp]
%\begin{center}
{\small
\begin{tabular}{ll}
\verb|o=|    & 出力ファイル名を指定する。\\
\verb|a=|    & 新規に作成するデータの項目名を指定する。\\
             & 複数の項目を生成する場合は、項目名をカンマで区切る。\\
             & \verb|-nfn,-nfno|オプション指定時は指定の必要はない。\\
\verb|v=|    & 新しく作成する文字列を指定する。\\
             & 複数の項目を生成する場合は、値をカンマで区切る。\verb|a=|で指定した個数と同数でなければならない。\\
\verb|l=|    & 新規作成する乱数データの行数を指定する。【デフォルト値:10】\\
\end{tabular} 
}
\end{table} 


\subsection*{利用例}
\subsubsection*{Example 1: Basic Example}

Generate a new dataset with characters strings \verb|custNo| and \verb|A0001| printed in 5 rows, and name the fields as \verb|attribute| and \verb|code| respectively.


\begin{Verbatim}[baselinestretch=0.7,frame=single]
$ mnewstr a=attribute,code v=custNo,A0001 l=5 o=rsl1.csv
#END# kgnewstr a=attribute,code l=5 o=rsl1.csv v=custNo,A0001
$ more rsl1.csv
attribute,code
custNo,A0001
custNo,A0001
custNo,A0001
custNo,A0001
custNo,A0001
\end{Verbatim}

\subsection*{関連コマンド}
\hyperref[sect:mnewnumber] {mnewnumber} : 連番を生成する。

\hyperref[sect:mnewrand] {mnewrand} : 乱数を生成する。

%\end{document}
