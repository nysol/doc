
%\documentclass[a4paper]{jsbook}
%\usepackage{mcmd_jp}
%\begin{document}

\section{mpadding (行補完) コマンド\label{sect:mpadding}}
\index{mpadding@mpadding}
\verb|k=|パラメータで指定した項目をキーとして、\verb|f=|パラメータで指定した項目の値が連続するようにレコードを作成する。\verb|v=|パラメータを指定した場合は、\verb|k=|,\verb|f=|で指定した以外の項目値を指定した文字列でパディングし、\verb|-n|オプション指定時は、nullでパディングする。(\verb|v=|,\verb|-n|共に指定がなければ直前の項目値でパディングする)

\subsection*{書式}
\verb|mpadding [k=] f= [v=] [S=] [E=] [-n] | 
\hyperref[sect:option_i]{[i=]}
\hyperref[sect:option_o]{[o=]}
\hyperref[sect:option_assert_diffSize]{[-assert\_diffSize]}
\hyperref[sect:option_assert_nullkey]{[-assert\_nullkey]}
\hyperref[sect:option_assert_nullin]{[-assert\_nullin]}
\hyperref[sect:option_assert_nullout]{[-assert\_nullout]}
\hyperref[sect:option_nfn]{[-nfn]} 
\hyperref[sect:option_nfno]{[-nfno]}  
\hyperref[sect:option_x]{[-x]}
\hyperref[sect:option_x]{[-q]}
\hyperref[sect:option_option_tmppath]{[tmpPath=]}
\hyperref[sect:option_precision]{[precision=]}
\verb|[-params]|
\verb|[--help]|
\verb|[--helpl]|
\verb|[--version]|\\

\subsection*{パラメータ}
\begin{table}[htbp]
%\begin{center}
{\small
\begin{tabular}{ll}
%\verb|k=|    & ここで指定された項目をキーとする。【\hyperref[sect:option_k]{集計キーブレイク処理}】\\
\verb|i=|    & 入力ファイル名を指定する。\\
\verb|o=|    & 出力ファイル名を指定する。\\
\verb|k=|    & ここで指定された項目をキーとする。\\
\verb|f=|    & 連続パディング対象項目名 \\
             & ここで指定された項目の値が連続するようにレコードをパディングする。\\
			 & 数字としてパディングするときは、no\%nのように\%nを指定する。\\
			 & 日付としてパディングするときは\%d、時刻としてパディングするときは\%tを指定する。\\
			 & 降順でパディングしたいときはno\%d\%rのようにrを追加する。\\
%             & k=で指定した項目と合わせて事前に並べ替えておく必要がある。\\
%             & 並べ方は、type=の指定によって異なる。整数としてパディングするのであればmsortf f=name\%nのように、\\
%             & 数値昇順で並べ替え、日付としてパディングするのであればmsortf f=nameのように\\
%             & 文字列昇順に並べ替えておかなければならない。\\
%             & 降順にパディングするのであれば、f=name\%rのように項目名の後ろに\%rを付け、\\
%             & 事前にmsortfにより降順に並べ替えておかなければならない。\\
%             & また、k=をあわせて指定する場合は、事前にk=で指定した項目と合わせて事前に並べ替えておく必要がある。\\
%\verb|type=| & f=で指定した項目の型を指定する。\\
%             & int:数値/date:日付/time:時間の指定が可能。(それ以外の指定は不可)\\
\verb|v=|    & パディング用文字列指定\\
             & k=,f=で指定した以外の項目値を指定した文字列で出力する。\\
\verb|S=|    & 開始値\\
             & f=で指定した項目の値の開始値を指定する。\\
\verb|E=|    & 終了値\\
             & f=で指定した項目の値の終了値を指定する。\\
\verb|-n|    & パディングにnull値指定\\
             & k=,f=で指定した以外の項目値をnullで出力する。\\
\end{tabular} 
}
\end{table} 

\subsection*{利用例}
\subsubsection*{Example 1: Basic Example}

Create padding with integer values (\verb|type=int|) between records in \verb|no| column.
Insert \verb|4,5| between \verb|3| and \verb|6|, and \verb|7| between \verb|6| and \verb|8|.


\begin{Verbatim}[baselinestretch=0.7,frame=single]
$ more dat1.csv
no
3
6
8
$ mpadding f=no%n i=dat1.csv o=rsl1.csv
#END# kgpadding f=no%n i=dat1.csv o=rsl1.csv
$ more rsl1.csv
no%0n
3
4
5
6
7
8
\end{Verbatim}
\subsubsection*{Example 2: Specify the starting and ending value}

Insert padding between records as well as before and after the first and last records from the input data.
Specify the starting and ending range at \verb|S=,E=|.


\begin{Verbatim}[baselinestretch=0.7,frame=single]
$ mpadding f=no%n S=1 E=10 i=dat1.csv o=rsl2.csv
#END# kgpadding E=10 S=1 f=no%n i=dat1.csv o=rsl2.csv
$ more rsl2.csv
no%0n
1
2
3
4
5
6
7
8
9
10
\end{Verbatim}
\subsubsection*{Example 3: Padding with date}

Create padding to fill in values between dates (\verb|type=date|) in the \verb|date| column.
Create padding values in columns other than those specified at \verb|k=,f=|.


\begin{Verbatim}[baselinestretch=0.7,frame=single]
$ more dat2.csv
date,dummy
20130929,a
20131002,b
20131004,c
$ mpadding f=date%d i=dat2.csv o=rsl3.csv
#END# kgpadding f=date%d i=dat2.csv o=rsl3.csv
$ more rsl3.csv
date%0,dummy
20130929,a
20130930,a
20131001,a
20131002,b
20131003,b
20131004,c
\end{Verbatim}
\subsubsection*{Example 4: Specify character string for padding}

Specify the character string padding value at \verb|v=|.


\begin{Verbatim}[baselinestretch=0.7,frame=single]
$ mpadding f=date%d v=padding i=dat2.csv o=rsl4.csv
#END# kgpadding f=date%d i=dat2.csv o=rsl4.csv v=padding
$ more rsl4.csv
date%0,dummy
20130929,a
20130930,padding
20131001,padding
20131002,b
20131003,padding
20131004,c
\end{Verbatim}
\subsubsection*{Example 5: Specify NULL value as padding character}

NULL value can be used as padding when the \verb|-n| option is specified.


\begin{Verbatim}[baselinestretch=0.7,frame=single]
$ mpadding f=date%d -n i=dat2.csv o=rsl5.csv
#END# kgpadding -n f=date%d i=dat2.csv o=rsl5.csv
$ more rsl5.csv
date%0,dummy
20130929,a
20130930,
20131001,
20131002,b
20131003,
20131004,c
\end{Verbatim}


\subsection*{関連コマンド}

%\end{document}
