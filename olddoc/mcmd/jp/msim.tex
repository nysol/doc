
%\documentclass[a4paper]{jsbook}
%\usepackage{mcmd_jp}
%\begin{document}

\section{msim 二変数間の類似度の計算\label{sect:msim}}
\index{msim@msim}
\verb|f=|パラメータで指定した項目の二変数間の類似度(距離)を
\verb|c=|パラメータで指定した類似度(距離)関数で計算し類似度行列として出力する。

\subsection*{書式}
\verb/msim c= f= [a=] [k=] [-n] [-d]/
\hyperref[sect:option_i]{[i=]}
\hyperref[sect:option_o]{[o=]}
\hyperref[sect:option_bufcount]{[bufcount=]} 
\hyperref[sect:option_assert_diffSize]{[-assert\_diffSize]}
\hyperref[sect:option_assert_nullkey]{[-assert\_nullkey]}
\hyperref[sect:option_assert_nullin]{[-assert\_nullin]}
\hyperref[sect:option_assert_nullout]{[-assert\_nullout]}
\hyperref[sect:option_nfn]{[-nfn]} 
\hyperref[sect:option_nfno]{[-nfno]}  
\hyperref[sect:option_x]{[-x]}
\hyperref[sect:option_x]{[-q]}
\hyperref[sect:option_option_tmppath]{[tmpPath=]}
\hyperref[sect:option_precision]{[precision=]}
\verb|[-params]|
\verb|[--help]|
\verb|[--helpl]|
\verb|[--version]|\\

\subsection*{パラメータ}
\begin{table}[htbp]
%\begin{center}
{\small
\begin{tabular}{ll}
%\verb|k=|    & ここで指定された項目(複数項目指定可)を単位として求める。【\hyperref[sect:option_k]{集計キーブレイク処理}】 \\
\verb|i=|    & 入力ファイル名を指定する。\\
\verb|o=|    & 出力ファイル名を指定する。\\ 
\verb|k=|    & ここで指定された項目(複数項目指定可)を単位として求める。 \\
\verb|f=|    & ここで指定された項目全ての二項目間の類似度を求める。\\
\verb|c=|    & 類似度(距離)名リスト(複数項目指定可)\\
             & 次項に示した類似度(距離)名を指定する。\\
             & 項目名は以下のように,(:)コロンに続けて指定して変更可能。\\
             & コロンに続く名称を省略した場合は類似度(距離)関数名がそのまま項目名として利用される。\\
             & 例) \verb|msim f=x,y,z c=pearson:ピアソン積率相関係数,euclid:ユークリッド距離,cosine:コサイン|\\
             & 類似度\verb/=covar|ucovar|pearson|spearman|kendall|euclid|cosine|/\\
             &       ~~\verb/cityblock|hamming|chi|phi|jaccard|supportr|lift|confMax|/\\
             &       ~~\verb/confMin|yuleQ|yuleY|kappa|oddsRatio|convMax|convMin/\\
\verb|a=|    & 2変数の名称を示す項目名を指定する。カンマで区切って2つ指定する。\\
             & 省略すると\verb|fld1,fld2|が使われる。\\
\verb|-d|    & 対角行列、上三角行列を出力する。\\
             & \verb|-d|オプションが指定されないと類似度行列の下三角行列のみ出力されるが、\\
             & \verb|-d|オプションを指定することにより対角行列及び上三角行列も出力される。\\
\verb|-n|    & NULL値が1つでも含まれていると結果もNULL値とする。\\
\verb|bufcount=| & バッファのサイズ数を指定する。 \\
\end{tabular} 
}
\end{table} 

\subsection*{類似度(距離)の定義}

\subsubsection*{実数ベクトル}
サイズが同じ2つの実数ベクトル${\bf x}=(x_1,x_2,\cdots,x_n),{\bf y}=(x_1,x_2,\cdots,x_n)$に関する類似度(もしくは距離)
の定義を表\ref{tbl:simReal}に示す。

\begin{table}[htbp]
\begin{center}
\caption{実数ベクトルの類似度一覧\label{tbl:simReal}}
{\small 
\renewcommand{\arraystretch}{2.0}
\begin{tabular}{lllll}
\hline
パラメータ値 & 内容 & 距離/類似度 & 定義式 & 範囲\\
\hline
covar	& 共分散	& 類似度 & 
$
\frac{1}{n}\sum_{i=1}^n~(x_i-\bar{x})(y_i-\bar{y})
$
 & $-\infty$ 〜 $\infty$\\

ucovar	& 不偏共分散	& 類似度 & 
$
\frac{1}{n-1}\sum_{i=1}^n~(x_i-\bar{x})(y_i-\bar{y})
$
 & $-\infty$ 〜 $\infty$\\

pearson	& ピアソンの積率相関係数	& 類似度 & 
$
\frac{\frac{1}{n}\sum_{i=1}^n~(x_i-\bar{x})(y_i-\bar{y})}{\sqrt{\frac{1}{n}\sum_{i=1}^n~(x_i-\bar{x})^2}\sqrt{\frac{1}{n}\sum_{i=1}^n~(y_i-\bar{y})^2}}~
$
 & $-1.0$ 〜 $1.0$ \\

spearman & スピアマンの順位相関係数	& 類似度 &
$\bf{x},\bf{y}$を順位に変換しての積率相関係数 
 & $-1.0$ 〜 $1.0$ \\

kendall  & ケンドールの順位相関係数	& 類似度 &
$
\frac{c-d}{\frac{1}{2}n(n-1)} ^{注1,2)}
$
 & $-1.0$ 〜 $1.0$ \\

euclid   & ユークリッド距離(数値)   & 距離 & 
$
\sqrt{\sum_{i=1}^n~(x_i-y_i)^2}~
$
 & $0$ 〜 $\infty$ \\

cosine   & コサイン   & 類似度 & 
$
\frac{\bf{x}\cdot~\bf{y}}{\mid\bf{x}\mid\mid\bf{y}\mid}=\frac{\sum_{i=1}^n~x_i~y_i}{\sqrt{\sum_{i=1}^n~x_i^2}\sqrt{\sum_{i=1}^n~y_i^2}}
$
 & $-1.0$ 〜 $1.0$ \\

cityblock   & 都市ブロック距離   & 距離 & 
$
\sum_{i=1}^n~\mid~x_i-y_i\mid
$
 & $-\infty$ 〜 $\infty$\\

hamming   & ハミング距離   & 距離 & 
$
\mid\{i \mid x_i\ne y_i, i=1,2,\cdots,n\}\mid
$
 & $0$ 〜 $n$\\

\hline
\end{tabular}
}
{\footnotesize
\\
注1)
$c=|\{(i,j)|(x_i>x_j ~{\rm and}~ y_i>y_j) ~{\rm or}~ (x_i<x_j ~{\rm and}~ y_i<y_j), i>j, i=1,2,\cdots,n, j=1,2,\cdots,n\}|$\\
注2)
$d=|\{(i,j)|(x_i>x_j ~{\rm and}~ y_i<y_j) ~{\rm or}~ (x_i<x_j ~{\rm and}~ y_i>y_j), i>j, i=1,2,\cdots,n, j=1,2,\cdots,n\}|$
}

\end{center}
\end{table}

\subsubsection*{0-1ベクトル}
値として0もしくは1をとる2つの0-1ベクトル${\bf a}=(a_1,a_2,\cdots,a_n),{\bf b}=(b_1,b_2,\cdots,b_n)$に関する類似度
の定義を表\ref{tbl:sim01}に示す。
表の中で使われている記号$f_{jk}$は、
$a_i,b_i$がとる値(0,1)の組み合せ別の件数で、
表\ref{tbl:matrix}に示されている。

\begin{table}[htbp]
\begin{center}
\caption{2変数の値の組みわせによる$2\times 2$分割表\label{tbl:matrix}}
{\small 
\begin{tabular}{l|ll|l}
\hline
 & $b_i=1$ & $b_i=0$ & 計\\
\hline
$a_i=1$ & $f_{11}$ & $f_{10}$ & $f_{1.}$\\
$a_i=0$ & $f_{01}$ & $f_{00}$ & $f_{0.}$\\
\hline
計 & $f_{.1}$ & $f_{.0}$ & $f_{..}$\\
\hline
\end{tabular}
}
\end{center}
\end{table}

また$P(\cdot)$の意味は以下に示すとおりである。
\begin{table}[htbp]
\begin{center}
{\small 
\begin{tabular}{l}
\hline
$P(a)=f_{1.}/f_{..}$\\
$P(b)=f_{.1}/f_{..}$\\
$P({\bar a})=f_{0.}/f_{..}$\\
$P(a,b)=f_{11}/f_{..}$\\
$P(a|b)=f_{11}/f_{.1}$\\
\hline
\end{tabular}
}
\end{center}
\end{table}


\begin{table}[htbp]
\begin{center}
\caption{0-1ベクトルの類似度一覧\label{tbl:sim01}}
{\small 
\renewcommand{\arraystretch}{2.0}
\begin{tabular}{lllll}
\hline
パラメータ値 & 内容 & 距離/類似度 & 定義式 & 範囲\\
\hline

chi   & カイ2乗値   & 類似度 & 
$
\sum_{i=0}^1~\sum_{j=0}^1~\frac{f_{ij}-e_{ij}}{e_{ij}}~ ^{注1)}
$
 & $0$ 〜 $\infty$\\


phi   & ファイ係数   & 類似度 & 
$
\frac{f_{11}f_{00}-f_{10}f_{01}}{\sqrt{f_{1.}f_{0.}f_{.1}f_{.0}}}
$
 & $-1.0$ 〜 $1.0$ \\

jaccard   & ジャックカード係数   & 類似度 & 
$
\frac{P(a,b)}{P(a)+P(b)-P(a,b)}
$
 & $0.0$ 〜 $1.0$ \\

support   & 支持度   & 類似度 & 
$
P(a,b)
$
 & $0.0$ 〜 $1.0$ \\

lift   & リフト値   & 類似度 & 
$
\frac{P(a,b)}{P(a)P(b)}
$
 & $0$ 〜 $\infty$\\

confMax  & 最大確信度   & 類似度 & 
$
\max(P(a|b),P(b|a))
$
 & $0.0$ 〜 $1.0$ \\

confMin  & 最小確信度   & 類似度 & 
$
\min((P(a|b),P(b|a))
$
 & $0.0$ 〜 $1.0$ \\

yuleQ  & yuleの連関係数(Q)  & 類似度 & 
$
\frac{\alpha-1}{\alpha+1} ^{注2)}
$
 & $-1.0$ 〜 $1.0$ \\

yuleY  & yuleの連関係数(Y)  & 類似度 & 
$
\frac{\sqrt{\alpha}-1}{\sqrt{\alpha}+1} ^{注2)}
$
 & $-1.0$ 〜 $1.0$ \\

kappa  & kappa & 類似度 & 
$
\frac{P(a,b)+P(\bar{a},\bar{b})-P(a)P(b)-P(\bar{a})P(\bar{b})}{1-P(a)P(b)-P(\bar{a})P(\bar{b})}
$
 & $-1.0$ 〜 $1.0$ \\

oddsRatio  & oddsRatio & 類似度 & 
$
\frac{P(a,b)P(\bar{a},\bar{b})}{P(a,\bar{b})P(\bar{a},b)}
$
 & $0$ 〜 $\infty$\\

convMax  & 最大conviction & 類似度 & 
$
\max(\frac{P(a)P(\bar{b})}{P(a,\bar{b})},\frac{P(\bar{a})P(b)}{P(\bar{a},b)})
$
 & $0.5$ 〜 $\infty$\\

convMin  & 最小conviction & 類似度 & 
$
\min(\frac{P(a)P(\bar{b})}{P(a,\bar{b})},\frac{P(\bar{a})P(b)}{P(\bar{a},b)})
$
 & $0.5$ 〜 $\infty$\\


\hline
\end{tabular}
}

{\footnotesize
注1) $e_{ij}=\frac{f_{i.}f_{.j}}{f_{..}}$
注2) $\alpha=\frac{f_{11}f_{00}}{f_{10}f_{01}}$
}
\end{center}
\end{table}


\subsection*{利用例}
\subsubsection*{Example 1: Basic Example}

Calculate the cosine and Pearson's product-moment correlation coefficient for the combination of two items among \verb|x, y, z| fields.


\begin{Verbatim}[baselinestretch=0.7,frame=single]
$ more dat1.csv
x,y,z
14,0.17,-14
11,0.2,-1
32,0.15,-2
13,0.33,-2
$ msim c=pearson,cosine f=x,y,z i=dat1.csv o=rsl1.csv
#END# kgsim c=pearson,cosine f=x,y,z i=dat1.csv o=rsl1.csv
$ more rsl1.csv
fld1,fld2,pearson,cosine
x,y,-0.5088704666,0.7860308044
x,z,0.1963041929,-0.5338153343
y,z,0.3311001423,-0.5524409416
\end{Verbatim}
\subsubsection*{Example 2: Output diagonal matrix, the upper triangular matrix}

Calculate the cosine and Pearson's product-moment correlation coefficient for the combination of two items between \verb|x, y, z|  fields (with d option).


\begin{Verbatim}[baselinestretch=0.7,frame=single]
$ msim c=pearson,cosine f=x,y,z -d i=dat1.csv o=rsl2.csv
#END# kgsim -d c=pearson,cosine f=x,y,z i=dat1.csv o=rsl2.csv
$ more rsl2.csv
fld1,fld2,pearson,cosine
x,x,1,1
x,y,-0.5088704666,0.7860308044
x,z,0.1963041929,-0.5338153343
y,x,-0.5088704666,0.7860308044
y,y,1,1
y,z,0.3311001423,-0.5524409416
z,x,0.1963041929,-0.5338153343
z,y,0.3311001423,-0.5524409416
z,z,1,1
\end{Verbatim}
\subsubsection*{Example 3: Calculation based on key unit}

Calculate using \verb|key| field as unit.


\begin{Verbatim}[baselinestretch=0.7,frame=single]
$ more dat2.csv
key,x,y,z
A,14,0.17,-14
A,11,0.2,-1
A,32,0.15,-2
B,13,0.33,-2
B,10,0.8,-5
B,15,0.45,-9
$ msim k=key c=pearson,cosine f=x,y,z i=dat2.csv o=rsl3.csv
#END# kgsim c=pearson,cosine f=x,y,z i=dat2.csv k=key o=rsl3.csv
$ more rsl3.csv
key%0,fld1,fld2,pearson,cosine
A,x,y,-0.8746392857,0.8472573627
A,x,z,0.3164384831,-0.521983618
A,y,z,0.1830936883,-0.6719258683
B,x,y,-0.7919009884,0.8782575583
B,x,z,-0.471446429,-0.9051543403
B,y,z,-0.1651896746,-0.8514129252
\end{Verbatim}
\subsubsection*{Example 4: Specify the type of degree of similarity}

Using the data with 01 values, compute the phi coefficient and Hamming distance.


\begin{Verbatim}[baselinestretch=0.7,frame=single]
$ more dat3.csv
x,y,z
1,1,0
1,0,1
1,0,1
0,1,1
$ msim c=hamming,phi f=x,y,z i=dat3.csv o=rsl4.csv
#END# kgsim c=hamming,phi f=x,y,z i=dat3.csv o=rsl4.csv
$ more rsl4.csv
fld1,fld2,hamming,phi
x,y,0.75,-0.5773502692
x,z,0.5,-0.3333333333
y,z,0.75,-0.5773502692
\end{Verbatim}
\subsubsection*{Example 5: Rename the column containing degree of similarity}

Using the data with 01 values, compute the phi coefficient and Hamming distance and change the output field name.


\begin{Verbatim}[baselinestretch=0.7,frame=single]
$ msim c=hamming:HammingDist,phi:PhiCoeff a=variable1,variable2 f=x,y,z i=dat3.csv o=rsl5.csv
#END# kgsim a=variable1,variable2 c=hamming:HammingDist,phi:PhiCoeff f=x,y,z i=dat3.csv o=rsl5.csv
$ more rsl5.csv
variable1,variable2,HammingDist,PhiCoeff
x,y,0.75,-0.5773502692
x,z,0.5,-0.3333333333
y,z,0.75,-0.5773502692
\end{Verbatim}

\subsection*{関連コマンド}
\hyperref[sect:mstats]{mstats} : 1変量の統計量を計算するときはこちら。

\hyperref[sect:mmvsim]{mmvsim} : 移動窓を設定した類似度計算。

%\end{document}
