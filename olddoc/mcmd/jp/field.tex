
%\begin{document}

\section{項目の指定\label{sect:fieldname}}
MCMDでは、CSVデータの先頭行を項目名として扱うことも可能であるし、
また項目名行がなくても項目番号で項目を指定することもできる。
項目名行の扱いに関係するパラメータは-nfn,-nfno,-nfni,-xの4つある。
以下では、例を示しながら、その利用方法について説明する。
なお、項目番号は左から0,1,2のように0から始まることに注意する。

\subsubsection*{Example 1: Specify -nfn}

When \verb|-nfn| (no field name) is specified, the first row in the data will not be considered as field names. Thus, each field is specified as a number (note that the number starts from 0). 


\begin{Verbatim}[baselinestretch=0.7,frame=single]
$ more dat2.csv
a,2
b,5
b,4
$ msum -nfn k=0 f=1 i=dat2.csv o=rsl1.csv
#END# kgsum -nfn f=1 i=dat2.csv k=0 o=rsl1.csv
$ more rsl1.csv
a,2
b,9
\end{Verbatim}
\subsubsection*{Example 2: Specify -nfno}

When \verb|-nfno| (no field name for output) is specified, the first row of input data is initialized as field names, but the field names is removed from the output data.


\begin{Verbatim}[baselinestretch=0.7,frame=single]
$ more dat1.csv
key,val
a,2
b,5
b,4
$ msum k=key f=val -nfno i=dat1.csv o=rsl2.csv
#END# kgsum -nfno f=val i=dat1.csv k=key o=rsl2.csv
$ more rsl2.csv
a,2
b,9
\end{Verbatim}
\subsubsection*{Example 3: Specify -nfni}

The option \verb|-nfni| (no field names or input) is only available in the mcut command. This option does the opposite of - nfno; the first row of input data is not treated as a fieldname row, but the fieldnames will be shown in the output data. Thus, you need to specify an output fieldname after a colon (:) following an input field number.


\begin{Verbatim}[baselinestretch=0.7,frame=single]
$ mcut f=0:key,1:val -nfni i=dat2.csv o=rsl3.csv
#END# kgcut -nfni f=0:key,1:val i=dat2.csv o=rsl3.csv
$ more rsl3.csv
key,val
a,2
b,5
b,4
\end{Verbatim}
\subsubsection*{Example 4: Specify -x}

For CSV data with a field names, use the \verb|-x| option to specify the field number.


\begin{Verbatim}[baselinestretch=0.7,frame=single]
$ msum -x k=0 f=1 i=dat1.csv o=rsl4.csv
#END# kgsum -x f=1 i=dat1.csv k=0 o=rsl4.csv
$ more rsl4.csv
key%0,val
a,2
b,9
\end{Verbatim}


\subsection{有効な項目名}

項目名として利用可能な文字は以下の通りである。
\begin{itemize}
\item アルファベット(a-z,A-Z)
\item 数字(0-9)
\item マルチバイト文字(UTF-8など)
\item 記号
\end{itemize}

ただし、記号については、以下の9つの利用は避けることを推奨する。
利用してエラーとなるわけではないが、いずれもMCMDの中で項目指定時の特殊文字として利用しており
その特殊用途を利用できなくなる可能性があるからである。

\begin{itemize}
\item \verb|,| カンマ
\item \verb|:| コロン
\item \verb|%| パーセント
\item \verb|*| アスタリスク
\item \verb|?| クエスチョンマーク
\item \verb|&| アンド
\item \verb|\| バックスラッシュ
\item \verb|]| 四角括弧右
\item \verb|[| 四角括弧左
\end{itemize}

\subsection{有効な項目番号}

項目番号の指定では、単純に項目番号をカンマで区切って列挙する以外にも、
後ろの項目からの番号指定(\verb|"L"|を付ける)や範囲(\verb|-|)を指定することが可能である。
例えば\verb|0L|とすれば、最後の項目を指定したことになり、
\verb|2L|とすれば、最後から数えて2番目の項目(0番から始まることに注意)を指定したことになる。
また\verb|0-5|と指定すれば、0番項目から5番項目まで6つの項目を指定したことになる。
すなわち\verb|0,1,2,3,4,5|と指定したことと同等である。

\subsubsection*{Example 1: Specify range}

By specifying "0-4", fields 0,1,2,3,4 are specified.  


\begin{Verbatim}[baselinestretch=0.7,frame=single]
$ more dat1.csv
brand,quantity01,quantity02,quantity03,quantity04,quantity05,quantity06,quantity07,quantity08,quantity09,quantity10
A,10,50,90,130,170,210,250,290,330,370
B,20,60,100,140,180,220,260,300,340,380
C,30,70,110,150,190,230,270,310,350,390
D,40,80,120,160,200,240,280,320,360,400
$ mcut -x f=0-4 i=dat1.csv o=rsl1.csv
#END# kgcut -x f=0-4 i=dat1.csv o=rsl1.csv
$ more rsl1.csv
brand,quantity01,quantity02,quantity03,quantity04
A,10,50,90,130
B,20,60,100,140
C,30,70,110,150
D,40,80,120,160
\end{Verbatim}
\subsubsection*{Example 2: Specify range in reverse order}

By specifying “4-0”,  fields 0,1,2,3,4 are specified.


\begin{Verbatim}[baselinestretch=0.7,frame=single]
$ mcut -x f=4-0 i=dat1.csv o=rsl2.csv
#END# kgcut -x f=4-0 i=dat1.csv o=rsl2.csv
$ more rsl2.csv
quantity04,quantity03,quantity02,quantity01,brand
130,90,50,10,A
140,100,60,20,B
150,110,70,30,C
160,120,80,40,D
\end{Verbatim}
\subsubsection*{Example 3: Specify Multiple ranges}

By specifying “1-0,2-4”, fields “1,0,2,3,4” are specified.EOF
scp=<<'EOF'
mcut -x f=1-0,2-4 i=dat1.csv o=rsl3.csv
more rsl3.csv


\begin{Verbatim}[baselinestretch=0.7,frame=single]
$ mcut -x f=4-0 i=dat1.csv o=rsl2.csv
#END# kgcut -x f=4-0 i=dat1.csv o=rsl2.csv
$ more rsl2.csv
quantity04,quantity03,quantity02,quantity01,brand
130,90,50,10,A
140,100,60,20,B
150,110,70,30,C
160,120,80,40,D
\end{Verbatim}
\subsubsection*{Example 4: Specified field from the end}

By specifying "2L", the second field from the end is specified (quantity 08).


\begin{Verbatim}[baselinestretch=0.7,frame=single]
$ mcut -x f=2L i=dat1.csv o=rsl4.csv
#END# kgcut -x f=2L i=dat1.csv o=rsl4.csv
$ more rsl4.csv
quantity08
290
300
310
320
\end{Verbatim}
\subsubsection*{Example 5: Specify the range of fields from the end}

By specifying "5-3L", the 5th to the 3rd item from end is specified, i.e. "5,6,7".


\begin{Verbatim}[baselinestretch=0.7,frame=single]
$ mcut -x f=5-3L i=dat1.csv o=rsl5.csv
#END# kgcut -x f=5-3L i=dat1.csv o=rsl5.csv
$ more rsl5.csv
quantity05,quantity06,quantity07
170,210,250
180,220,260
190,230,270
200,240,280
\end{Verbatim}


\subsection{入力項目と出力項目}

多くのコマンドで項目の指定には\verb|f=|が利用される。
\verb|f=|の書式は、「入力項目:出力項目」で、
出力項目名の指定を省略すれば、入力項目名が出力項目名として利用される。
また、\verb|-x|を指定することで\verb|f=0:数量|のように、番号指定と混在させることも可能である。

\subsubsection*{Example 1: Basic Example}

By specifying "quantity:unit sales", the field name is converted from “quantity” to “unit sales” in the output.


\begin{Verbatim}[baselinestretch=0.7,frame=single]
$ more dat1.csv
brand,quantity
A,10
B,20
C,30
D,40
$ mcut f=brand,quantity:salesquantity i=dat1.csv o=rsl1.csv
#END# kgcut f=brand,quantity:salesquantity i=dat1.csv o=rsl1.csv
$ more rsl1.csv
brand,salesquantity
A,10
B,20
C,30
D,40
\end{Verbatim}
\subsubsection*{Example 2: Add field name}

The maccum command accumulates the values in the "quantity" field, and add the field name "cumulative quantity" in the output results. If the parameter is specified as "f=quantity", the field name of the cumulative result will remain as "quantity", thus results in error because the same field name “quantity” exists in the output. 


\begin{Verbatim}[baselinestretch=0.7,frame=single]
$ maccum f=quantity:accumulationquantity i=dat1.csv o=rsl2.csv
#ERROR# parameter s= is mandatory without -q,-nfn (kgaccum)
$ more rsl2.csv
$ maccum f=quantity i=dat1.csv o=rsl2.csv
#ERROR# parameter s= is mandatory without -q,-nfn (kgaccum)
\end{Verbatim}
\subsubsection*{Example 3: Mixing field name and field number}

The field name and field number can be specified at the same time.


\begin{Verbatim}[baselinestretch=0.7,frame=single]
$ mcut f=0,1:salesquantity -x i=dat1.csv o=rsl3.csv
#END# kgcut -x f=0,1:salesquantity i=dat1.csv o=rsl3.csv
$ more rsl3.csv
brand,salesquantity
A,10
B,20
C,30
D,40
\end{Verbatim}


\subsection{ワイルドカード}
複数項目を指定する際にはには、項目名に\verb|"*"|と\verb|"?"|のワイルドカードを利用することができる。
\verb|"*"|は任意の長さの任意の文字列にマッチし、\verb|"?"|は任意の1文字にマッチする。
また、ワイルドカードの評価順は入力データ上の項目の並び順となることに注意する。
例えば、入力データの項目の並びが、\verb|A5,A3,A4,A2,A1|であれば、\verb|f=A*|は
\verb|f=A5,A3,A4,A2,A1|と評価される。

\subsubsection*{Example 1: Basic Example}

The expression "quantity*" matches field names starting with quantity ("quantity10", "quantity11", "quantity12" and "quantity123").  


\begin{Verbatim}[baselinestretch=0.7,frame=single]
$ more dat1.csv
brand,quantity10,quantity11,quantity12,quantity123
A,10,15,9,1
B,20,16,8,2
C,30,17,7,3
D,40,18,6,4
$ mcut f= quantity* i=dat1.csv o=rsl1.csv
#ERROR# invalid argument: quantity* (kgcut)
$ more rsl1.csv
rsl1.csv: No such file or directory
\end{Verbatim}
\subsubsection*{Example 2: Wildcard character “?”}

Select field names which begin with "quantity" followed by 1, and match any single character after 1. In this case, the wildcard does not match with field name “quantity123”. 


\begin{Verbatim}[baselinestretch=0.7,frame=single]
$ mcut f= quantity 1? i=dat1.csv o=rsl2.csv
#ERROR# invalid argument: quantity (kgcut)
$ more rsl2.csv
rsl2.csv: No such file or directory
\end{Verbatim}


\subsection{出力項目名の置換}

出力項目名で指定された\verb|"&"|は特殊な意味を持ち、入力項目名に置換される。
例えば、\verb|f=abc:xx&xx|では、出力項目名は\verb|xxabcxx|に置換される。
\verb|"&"|は、出力項目名の任意の位置に指定することができ、またその指定数に制限はない。
ただし、\verb|"&"|記号は、シェルにおいて「バックグラウンド実行」と解釈されてしまうので、
ダブルクォーツで囲うなどしてエスケープする必要がある。

\subsubsection*{Example 1: Basic Example}

In this example, \verb|"&"| is replaced with “\verb|brand|” in the input field name, which is equivalent to the expression "\verb|f=brand:brand code|".


\begin{Verbatim}[baselinestretch=0.7,frame=single]
$ more dat1.csv
brand,quantity10,quantity11,quantity12,quantity123
A,10,15,9,1
B,20,16,8,2
C,30,17,7,3
D,40,18,6,4
$ mcut f="brand:& code" i=dat1.csv o=rsl1.csv
#END# kgcut f=brand:& code i=dat1.csv o=rsl1.csv
$ more rsl1.csv
brand code
A
B
C
D
\end{Verbatim}
\subsubsection*{Example 2: Combine with wildcard}

Attach “\verb|&|” after \verb|sales&| to replace the character with input field name (e.g. "quantity10") in the output field name.  For all input fields name beginning with “\verb|quantity|”, attach “\verb|sales|” as the prefix in the output field name. 


\begin{Verbatim}[baselinestretch=0.7,frame=single]
$ mcut f="brand,quantity*:sales&" i=dat1.csv o=rsl2.csv
#END# kgcut f=brand,quantity*:sales& i=dat1.csv o=rsl2.csv
$ more rsl2.csv
brand,salesquantity10,salesquantity11,salesquantity12,salesquantity123
A,10,15,9,1
B,20,16,8,2
C,30,17,7,3
D,40,18,6,4
\end{Verbatim}


\subsection{集計処理コマンドについての注意点}
集計処理コマンドでは、キー項目ごとに指定された項目が集計処理され、キー項目ごとに1行出力される際、指定した項目以外の項目はどのレコードが出力さるかについては不定である。
例えば、msumコマンドで、顧客、日付、商品、金額という項目のデータを想定し、
msumコマンドで顧客別に金額を集計するのに、\verb|msum k=顧客 f=金額|と指定した場合、
それ以外の日付、商品項目についてはどのレコードが出力されるか不定であることに注意する。

%\end{document}

