
%\documentclass[a4paper]{jsbook}
%\usepackage{mcmd_jp}
%\begin{document}

\section{maccum 累積計算\label{sect:maccum}}
\index{maccum@maccum}
\verb|f=|パラメータで指定した項目の累積を計算し、新しい項目として追加する。
\verb|k=|を指定することで、キー単位毎に累積計算が可能となる。

\subsection*{書式}
\verb|maccum f= s= [k=] |
\hyperref[sect:option_i]{[i=]}
\hyperref[sect:option_o]{[o=]}
\hyperref[sect:option_assert_diffSize]{[-assert\_diffSize]}
\hyperref[sect:option_assert_nullkey]{[-assert\_nullkey]}
\hyperref[sect:option_assert_nullin]{[-assert\_nullin]}
\hyperref[sect:option_assert_nullout]{[-assert\_nullout]}
\hyperref[sect:option_nfn]{[-nfn]} 
\hyperref[sect:option_nfno]{[-nfno]}  
\hyperref[sect:option_x]{[-x]}
\hyperref[sect:option_q]{[-q]}
\hyperref[sect:option_option_tmppath]{[tmpPath=]}
\hyperref[sect:option_precision]{[precision=]}
\verb|[-params]|
\verb|[--help]|
\verb|[--helpl]|
\verb|[--version]|\\

\subsection*{パラメータ}
\begin{table}[htbp]
%\begin{center}
{\small
\begin{tabular}{ll}
\verb|i=|    & 入力ファイル名を指定する。\\
\verb|o=|    & 出力ファイル名を指定する。\\
\verb|f=|    & ここで指定した項目(複数項目指定可)の値が累積される。\\
             & 項目の値がNULL値である場合は無視される。\\
             & :(コロン)で新項目名を指定する必要がある。例)\verb|f=数量:数量累計| \\
\verb|s=|    & ここで指定した項目(複数項目指定可)で並べ替えられた後、累積が計算される。\\
             & \verb|-q|オプションを指定しないとき、\verb|s=|パラメータは必須。\\
\verb|k=|    & 累積の単位となる項目名リスト(複数項目指定可)を指定する。\\
\end{tabular} 
}
\end{table} 

\subsection*{利用例}
\subsubsection*{例1: 基本例}

「数量」と「金額」項目の累積値を計算し、「数量累計」と「金額累計」という項目名で出力する。


\begin{Verbatim}[baselinestretch=0.7,frame=single]
$ more dat1.csv
顧客,数量,金額
A,1,10
A,2,20
B,1,15
B,3,10
B,1,20
$ maccum s=顧客 f=数量:数量累計,金額:金額累計 i=dat1.csv o=rsl1.csv
#END# kgaccum f=数量:数量累計,金額:金額累計 i=dat1.csv o=rsl1.csv s=顧客
$ more rsl1.csv
顧客%0,数量,金額,数量累計,金額累計
A,1,10,1,10
A,2,20,3,30
B,1,15,4,45
B,3,10,7,55
B,1,20,8,75
\end{Verbatim}
\subsubsection*{例2: キー項目を指定する例}

「顧客」項目を単位に「数量」と「金額」項目の累積値を計算し、「数量累計」と「金額累計」という項目名で出力する。


\begin{Verbatim}[baselinestretch=0.7,frame=single]
$ more dat1.csv
顧客,数量,金額
A,1,10
A,2,20
B,1,15
B,3,10
B,1,20
$ maccum k=顧客 s=顧客 f=数量:数量累計,金額:金額累計 i=dat1.csv o=rsl2.csv
#END# kgaccum f=数量:数量累計,金額:金額累計 i=dat1.csv k=顧客 o=rsl2.csv s=顧客
$ more rsl2.csv
顧客,数量,金額,数量累計,金額累計
A,1,10,1,10
A,2,20,3,30
B,1,15,1,15
B,3,10,4,25
B,1,20,5,45
\end{Verbatim}
\subsubsection*{例3: NULL値を含む累計}

「数量」と「金額」項目の累積値を計算し、「数量累計」と「金額累計」という項目名で出力する。
NULLは無視される。結果もNULLが出力される。


\begin{Verbatim}[baselinestretch=0.7,frame=single]
$ more dat2.csv
顧客,数量,金額
A,1,10
A,,20
B,1,15
B,3,
B,1,20
$ maccum s=顧客 f=数量:数量累計,金額:金額累計 i=dat2.csv o=rsl3.csv
#END# kgaccum f=数量:数量累計,金額:金額累計 i=dat2.csv o=rsl3.csv s=顧客
$ more rsl3.csv
顧客%0,数量,金額,数量累計,金額累計
A,1,10,1,10
A,,20,,30
B,1,15,2,45
B,3,,5,
B,1,20,6,65
\end{Verbatim}


\subsection*{関連コマンド}
\hyperref[sect:mshare]{mshare} : 構成比を計算する。\verb|maccum|と組み合わせて累積相対度数が計算できる。

\hyperref[sect:mcal]{mcal} : 前行の計算結果\verb|#{}|を利用することで累計計算ができる。

%\end{document}
