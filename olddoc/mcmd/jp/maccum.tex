
%\documentclass[a4paper]{jsbook}
%\usepackage{mcmd_jp}
%\begin{document}

\section{maccum 累積計算\label{sect:maccum}}
\index{maccum@maccum}
\verb|f=|パラメータで指定した項目の累積を計算し、新しい項目として追加する。
\verb|k=|を指定することで、キー単位毎に累積計算が可能となる。

\subsection*{書式}
\verb|maccum f= s= [k=] |
\hyperref[sect:option_i]{[i=]}
\hyperref[sect:option_o]{[o=]}
\hyperref[sect:option_assert_diffSize]{[-assert\_diffSize]}
\hyperref[sect:option_assert_nullkey]{[-assert\_nullkey]}
\hyperref[sect:option_assert_nullin]{[-assert\_nullin]}
\hyperref[sect:option_assert_nullout]{[-assert\_nullout]}
\hyperref[sect:option_nfn]{[-nfn]} 
\hyperref[sect:option_nfno]{[-nfno]}  
\hyperref[sect:option_x]{[-x]}
\hyperref[sect:option_q]{[-q]}
\hyperref[sect:option_option_tmppath]{[tmpPath=]}
\hyperref[sect:option_precision]{[precision=]}
\verb|[-params]|
\verb|[--help]|
\verb|[--helpl]|
\verb|[--version]|\\

\subsection*{パラメータ}
\begin{table}[htbp]
%\begin{center}
{\small
\begin{tabular}{ll}
\verb|i=|    & 入力ファイル名を指定する。\\
\verb|o=|    & 出力ファイル名を指定する。\\
\verb|f=|    & ここで指定した項目(複数項目指定可)の値が累積される。\\
             & 項目の値がNULL値である場合は無視される。\\
             & :(コロン)で新項目名を指定する必要がある。例)\verb|f=数量:数量累計| \\
\verb|s=|    & ここで指定した項目(複数項目指定可)で並べ替えられた後、累積が計算される。\\
             & \verb|-q|オプションを指定しないとき、\verb|s=|パラメータは必須。\\
\verb|k=|    & 累積の単位となる項目名リスト(複数項目指定可)を指定する。\\
\end{tabular} 
}
\end{table} 

\subsection*{利用例}
\subsubsection*{Example 1: Basic Example}

Calculate the cumulative values ​​of "Quantity" and "Amount" fields for each "Customer", save output as new data attributes in new columns named "AccumQuantity" and "AccumlAmount".


\begin{Verbatim}[baselinestretch=0.7,frame=single]
$ more dat1.csv
Customer,Quantity,Amount
A,1,10
A,2,20
B,1,15
B,3,10
B,1,20
$ maccum s=Customer f=Quantity:AccumQuantity,Amount:AccumAmount i=dat1.csv o=rsl1.csv
#END# kgaccum f=Quantity:AccumQuantity,Amount:AccumAmount i=dat1.csv o=rsl1.csv s=Customer
$ more rsl1.csv
Customer%0,Quantity,Amount,AccumQuantity,AccumAmount
A,1,10,1,10
A,2,20,3,30
B,1,15,4,45
B,3,10,7,55
B,1,20,8,75
\end{Verbatim}
\subsubsection*{Example 2: Specify Calculation by Key}

Calculates the cumulative value of "Quantity" and "Amount" fields for each "Customer", and save the output in new columns named "AccumQuantity" and  "AccumAmount".


\begin{Verbatim}[baselinestretch=0.7,frame=single]
$ more dat1.csv
Customer,Quantity,Amount
A,1,10
A,2,20
B,1,15
B,3,10
B,1,20
$ maccum k=Customer s=Customer f=Quantity:AccumQuantity,Amount:AccumAmount i=dat1.csv o=rsl2.csv
#END# kgaccum f=Quantity:AccumQuantity,Amount:AccumAmount i=dat1.csv k=Customer o=rsl2.csv s=Customer
$ more rsl2.csv
Customer,Quantity,Amount,AccumQuantity,AccumAmount
A,1,10,1,10
A,2,20,3,30
B,1,15,1,15
B,3,10,4,25
B,1,20,5,45
\end{Verbatim}
\subsubsection*{Example 3: Cumulative computation with NULL values}

Calculate the cumulative values ​​of "Quantity" and "Amount" item, and save the output as new columns named "AccumQuantity" and "AccumAmount". NULL values are ignored. Records with NULL values will be retained in the output.


\begin{Verbatim}[baselinestretch=0.7,frame=single]
$ more dat2.csv
Customer,Quantity,Amount
A,1,10
A,,20
B,1,15
B,3,
B,1,20
$ maccum s=Customer f=Quantity:AccumQuantity,Amount:AccumAmount i=dat2.csv o=rsl3.csv
#END# kgaccum f=Quantity:AccumQuantity,Amount:AccumAmount i=dat2.csv o=rsl3.csv s=Customer
$ more rsl3.csv
Customer%0,Quantity,Amount,AccumQuantity,AccumAmount
A,1,10,1,10
A,,20,,30
B,1,15,2,45
B,3,,5,
B,1,20,6,65
\end{Verbatim}


\subsection*{関連コマンド}
\hyperref[sect:mshare]{mshare} : 構成比を計算する。\verb|maccum|と組み合わせて累積相対度数が計算できる。

\hyperref[sect:mcal]{mcal} : 前行の計算結果\verb|#{}|を利用することで累計計算ができる。

%\end{document}
