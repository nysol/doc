

%\documentclass[a4paper]{jsbook}
%\usepackage{mcmd_jp}
%\begin{document}

\section{mchkcsv CSVデータのチェック\label{sect:mchkcsv}}
\index{mchkcsv@mchkcsv}
\hyperref[sect:csv]{MCMDが前提とするCSVの仕様}を満たしていないデータを自動修復する(項目数の統一など)。
また、\verb|-diag|オプションを指定することで、CSVデータのチェックのみ実行する。

\subsection*{書式}
\verb/mchkcsv [a=] [-diagl|-diag] [-r]/
\hyperref[sect:option_o]{[i=]}
\hyperref[sect:option_o]{[o=]}
\hyperref[sect:option_assert_nullout]{[-assert\_nullout]}
\hyperref[sect:option_nfn]{[-nfn]} 
\hyperref[sect:option_nfno]{[-nfno]}  
\hyperref[sect:option_x]{[-x]}
\hyperref[sect:option_option_tmppath]{[tmpPath=]}
\hyperref[sect:option_precision]{[precision=]}
\verb|[-params]|
\verb|[--help]|
\verb|[--helpl]|
\verb|[--version]|\\

\subsection*{パラメータ}
\begin{table}[htbp]
%\begin{center}
{\small
\begin{tabular}{ll}
\verb|i=|    & 入力ファイル名を指定する。\\
             & このパラメータで指定したCSVデータに不備な箇所がないかチェックを行い、\\
             & 不備がある場合その箇所を自動的に補完する。\\
             & このパラメータが省略された時には標準入力が用いられる。\\
\verb|o=|    & 出力ファイル名を指定する。\\
\verb|a=|    & 入力データの項目名を無視し、ここで指定した項目名を出力する。\\
             & ここで指定した項目数が入力データの項目数より少ない場合は、\\
             & 入力データの左から順番に指定の個数分の項目が出力される。\\
             & 逆に、ここで指定した項目数が入力データの項目数より多い場合、不足分はNULL値が出力される。\\
\verb|-diagl| &このオプションを指定した場合、データの修正/補完は行わず、\\
             & データ項目の統計情報を出力するとともに、CSVお妥当性チェックの結果を標準出力に出力する。\\
             & チェックする内容は以下の通り。\\
             & a): 同一項目名がないか。\\
             & b): 項目名に不正な文字がないか。\\
             & c): 改行がLFか。\\
             & d): 最終行に改行が存在するか。\\
             & e): データファイル内に\verb|\0|がないか。\\
             & f): 一行の最大長を超過していないか。\\
             & g): 全行同じ項目数か。\\
             & h): 制御文字(0x01~0x1F,0x7F)が入り込んでいないか。\\
             & i): TABが入っていないか。\\
             & j): DQUOTEで囲われていない中でDQUOTEがない。\\
             & k): DQUOTEで囲われている中で単一のDQUOTEがないか。\\
             & l): 先頭にBOM(Bit Order Mark)が含まれていないか。\\
\verb|-diag| & \verb|-diagl|の英語版\\
\verb|-r|    & 制御文字を無視\\
             & ここではASCII文字コードの0x00〜0x1f,0x7f(0x09,0x0a,0x0dは除く)を制御文字と定義している。\\
             & このオプションを指定しなければ制御コードは自動的に\&\#xという文字列に変換される。\\
\end{tabular} 
}
\end{table} 

\subsection*{利用例}
%\if0 #no help# following sentences will not apear on the help document. \fi
\subsubsection*{Example 1: Repair data}

This data contains different number of columns in all records. For instance, only 3 records have 4 columns. Use M-Command to repair and standardize 3rd and 5th rows to 4 columns.


\begin{Verbatim}[baselinestretch=0.7,frame=single]
$ more dat1.csv
product,date,quantity,amount
A,20081201,1,10
A,20081202,2,
A,*,3
B,20081201,4,40
B,20081203,50
$ mchkcsv i=dat1.csv o=rsl1.csv
#END# kgchkcsv i=dat1.csv o=rsl1.csv
$ more rsl1.csv
product,date,quantity,amount
A,20081201,1,10
A,20081202,2,
A,*,3,
B,20081201,4,40
B,20081203,50,
\end{Verbatim}
\subsubsection*{Example 2: Change field name after repairing the data}

This data contains different number of columns in all records. For instance, only 3 records have 4 columns. Use M Command to repair and standardize 3rd and 5th rows to 4 columns. At the same time, label the field names from the input data as ¥verb|“product,date,quantity,amount”| starting from the left.


\begin{Verbatim}[baselinestretch=0.7,frame=single]
$ more dat2.csv
fld1,fld2,fld3,fld4
A,20081201,1,10
A,20081202,2,
A,*,3
B,20081201,4,40
B,20081203,50
$ mchkcsv a=product,date,quantity,amount i=dat2.csv o=rsl2.csv
#END# kgchkcsv a=product,date,quantity,amount i=dat2.csv o=rsl2.csv
$ more rsl2.csv
product,date,quantity,amount
A,20081201,1,10
A,20081202,2,
A,*,3,
B,20081201,4,40
B,20081203,50,
\end{Verbatim}
\subsubsection*{Example 3: Check data integrity and output diagnostic results}

It merely checks for incomplete data structure in the CSV data, and save the diagnosis result in CSV file.


\begin{Verbatim}[baselinestretch=0.7,frame=single]
$ mchkcsv -diag i=dat1.csv o=rsl3.csv
#END# kgchkcsv -diag i=dat1.csv o=rsl3.csv
$ more rsl3.csv
#===================================================
# diagnosis for the CSV file 
# file name : dat1.csv
#---------------------------------------------------
# meaning of the first charactor on each line
# # : commnet or resutl with no error
#   : that is, mcmd can handle the CSV file if all lines begin with '#'
# ? : error that mcmd cannot handle
#   : refer the 'explanation' section for the meaning of the alphabet next of '?'
#===================================================
############################ CSV header for field names(first line) ###
# the number of fields : 4
# fieldNo.  name
# 1   product
# 2   date
# 3   quantity
# 4   amount
#
############## about EOL(End Of Line) (including a CSV header) ##
# the number of lines with LF : 6 (LineNo: 0 1 2 ... )
#
################# about data lines (no including a CSV header) ###
# the number of lines : 5
# data volume in byte : 66
# the average number of lines : 13.2
# the maximum number of lines : 16 (LineNo:2)
# the maximum number of lines : 6 (LineNo:4)
# note: EOL charactor is counted in the numbers
#
################################# aoubt the number of fields ###
?g lines with different number of fields are detected
?g  # of fields:3 (LineNo:4)
?g  # of fields:3 (LineNo:6)
#
####################################### about field ###
#  fieldNo[1] name[product]
#    # of lines in NULL : 0 
#    # of lines not enclosed with double quotation : 5 (LineNo: 1 2 3 ... )
#    # of lines enclosed with double quotation : 0 
#
#  fieldNo[2] name[date]
#    # of lines in NULL : 0 
#    # of lines not enclosed with double quotation : 5 (LineNo: 1 2 3 ... )
#    # of lines enclosed with double quotation : 0 
#
#  fieldNo[3] name[quantity]
#    # of lines in NULL : 0 
#    # of lines not enclosed with double quotation : 5 (LineNo: 1 2 3 ... )
#    # of lines enclosed with double quotation : 0 
#
#  fieldNo[4] name[amount]
#    # of lines in NULL : 1 (LineNo: 2 )
#    # of lines not enclosed with double quotation : 3 (LineNo: 1 2 4 )
#    # of lines enclosed with double quotation : 0 
#
################################### explanation ###
# ?a : It cannot assign the field if there are duplicat name of fields.
#  [how to] Specify a complete set of field names like 'mchkcsv a=x,y,z'
# ?b : MCMD has a valid set of charactors for a field name.
#  [how to]Specify a complete set of field names like 'mchkcsv a=x,y,z'
# ?c : MCMD can handle only lines with LF as a EOL.
#      This is MCMD original restriction, conforming to RFC4180.
#  [how to] Run mchkcsv command that convert CR,CRLF to LF.
#
# ?d : The last line does not have a EOL charactor.
#      It does not conform to RFC4180.
#  [how to] Run mchkcsv command add LF at the end.
#
# ?e : Data file has '\0' charactor.
#      The data may not be a text.
#      It does not conform to RFC4180.
#  [how to] Run mchkcsv that convert the charactor to "&#x00
#           Run mchkcsv command with '-r' option that delete '\0' characotr.
#
# ?f : The number of charactors per line exceed the limit MCMD can handle.
#      MCMD can handle a line less than or equal to 1024000 bytes of charactors.
#  [how to] Change a enviroment variable KG_MaxRecLen.
#        ex) export KG_MaxRecLen=2048000
#      You cannot specify the number greater than 10240000 bytes.
#      This is MCMD original restriction, conforming to RFC4180.
#
# ?g : MCMD can handle only a CSV data that all lines have same number of fields.
#      This is MCMD original restriction, conforming to RFC4180.
#  [how to]
#       1) Use mchkcsv command that aligns all lines with same number of the field name header.
#            Exceeded field value will be cut off, and it will add a NULL value for missing field.
#       2) Use mchkcsv command with a= parameter.
#            It uses the field names on a= parameter just as a header line (the header line will be skipped).
#
# ?h : Field value include a control charactors (0x01~0x1F,0x7F).
#      The data may not be a text.
#      It does not conform to RFC4180.
#  [how to] Run mchkcsv that convert the charactor to text like "&#x01
#           Run mchkcsv command with '-r' option that delete the control characotrs.
#
# ?i : TAB cannot be used.
#      It does not conform to RFC4180.
#  [how to] Run mchkcsv that convert the TAB to "&#x09
#           Run mchkcsv command with '-r' option that delete the TAB.
#
# ?j : Double quotation charactor is found in a value not enclosed by double quotation.
#        ex) NG: xxx,oo"oo,xxx  -> OK: xxx,"oo""oo",xxx
#      It does not conform to RFC4180.
#  [how to] Run mchkcsv that makes convertion in the above example.
#
# ?k : Double quotation charactor is found in a value enclosed by double quotation.
#      ex) NG: xxx,"oo"oo",xxx  -> OK: xxx,"oo""oo",xxx
#      It does not conform to RFC4180.
#  [how to] Run mchkcsv that makes convertion in the above example.
# ?l : It has BOM (Bite Order Mark) at the beginning of data.
#  [how to] Run mchkcsv command that remove the BOM.
#-------------------------------------------------------------
\end{Verbatim}


\subsection*{関連コマンド}

%\end{document}
