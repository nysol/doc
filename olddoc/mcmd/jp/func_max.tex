
%\begin{document}

\section{max 最大値\label{sect:max}}
\index{max@max}

書式1: max($num_1,num_2,\cdots$)

書式2: max($num_1,num_2,\cdots,str$)

$num_i$で与えられた数値の最大値を計算する。
書式1では、NULL値は無視されるが、全てがNULL値であれば結果もNULLとなる。

書式2において最後の引数として"-n"を与えると、
NULL値に対する扱いが変わり、
項目値に一つでもNULL値がある場合は、結果もNULL値となる。

\subsection*{利用例}
\subsubsection*{例1: 基本例}



\begin{Verbatim}[baselinestretch=0.7,frame=single]
$ more dat1.csv
id,v1,v2,v3
1,1,2,3
2,-5,2,1
3,1,,3
4,,,
$ mcal c='max(${v1},${v2},${v3})' a=rsl i=dat1.csv o=rsl1.csv
#END# kgcal a=rsl c=max(${v1},${v2},${v3}) i=dat1.csv o=rsl1.csv
$ more rsl1.csv
id,v1,v2,v3,rsl
1,1,2,3,3
2,-5,2,1,2
3,1,,3,3
4,,,,
\end{Verbatim}
\subsubsection*{例2: ワイルドカードを利用した例}

\verb|v|から始まる項目(\verb|v1,v2,v3|)をワイルドカード「\verb|v*|」によって指定している。


\begin{Verbatim}[baselinestretch=0.7,frame=single]
$ mcal c='max(${v*})' a=rsl i=dat1.csv o=rsl2.csv
#END# kgcal a=rsl c=max(${v*}) i=dat1.csv o=rsl2.csv
$ more rsl2.csv
id,v1,v2,v3,rsl
1,1,2,3,3
2,-5,2,1,2
3,1,,3,3
4,,,,
\end{Verbatim}
\subsubsection*{例3: -nを利用した例}

\verb|v2|にNULL値を含む\verb|id=3|の行の結果もNULLとなる。


\begin{Verbatim}[baselinestretch=0.7,frame=single]
$ mcal c='max(${v1},${v2},${v3},"-n")' a=rsl i=dat1.csv o=rsl3.csv
#END# kgcal a=rsl c=max(${v1},${v2},${v3},"-n") i=dat1.csv o=rsl3.csv
$ more rsl3.csv
id,v1,v2,v3,rsl
1,1,2,3,3
2,-5,2,1,2
3,1,,3,
4,,,,
\end{Verbatim}


%\end{document}

