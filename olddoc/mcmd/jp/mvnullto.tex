
%\documentclass[a4paper]{jsbook}
%\usepackage{mcmd_jp}
%\begin{document}

\section{mvnullto ベクトル要素のNULL置換\label{sect:mvnullto}}
\index{mvreplace@mvreplace}
ベクトル要素でNULLの要素を任意の値に置換する。
ベクトル要素がNULLであれば、要素の区切り文字が連続する。
以下に示したベクトルは全てNULLを含む。
ただし、わかりやすさのためにベクトルの末尾に\verb|`\n'|を記している。
上から順番に、3番目、1番目、4番目の要素がNULLである。
\begin{Verbatim}[baselinestretch=0.7,frame=single]
a b  c\n
 a b\n
a b c \n
\end{Verbatim}

\subsection*{書式}
\verb/mvnullto vf= [v=|-p] [O=] [-A] /
\hyperref[sect:option_i]{i=}
\hyperref[sect:option_o]{[o=]}
\hyperref[sect:option_delim]{[delim=]} 
\hyperref[sect:option_assert_diffSize]{[-assert\_diffSize]}
\hyperref[sect:option_assert_nullin]{[-assert\_nullin]}
\hyperref[sect:option_assert_nullout]{[-assert\_nullout]}
\hyperref[sect:option_nfn]{[-nfn]} 
\hyperref[sect:option_nfno]{[-nfno]}  
\hyperref[sect:option_x]{[-x]}
\hyperref[sect:option_option_tmppath]{[tmpPath=]}
\hyperref[sect:option_precision]{[precision=]}
\verb|[-params]|
\verb|[--help]|
\verb|[--helpl]|
\verb|[--version]|\\

\subsection*{パラメータ}
\begin{table}[htbp]
%\begin{center}
{\small
\begin{tabular}{ll}
\verb|i=|    & 入力ファイル名を指定する。\\
\verb|o=|    & 出力ファイル名を指定する。\\
\verb|vf=| & NULL置換の対象となる項目名(\verb|i=|ファイル上)を指定する。\\
           & 複数項目指定可能。\\
		   & 結果の項目名を変更したいときは、:(コロン)に続けて新項目名を指定する。\\
\verb|-A|  & \verb|vf=|で:(コロン)に続けて指定した項目名で、新たな項目が追加される。\\
           & なお\verb|-A|オプションを指定した場合、\verb|vf=|パラメータで指定するすべての\\
           & 項目に新項目名を指定しなければならない。\\
\verb|v=|  & 置換文字列を指定する。\\
\verb|-p|  & 直前の要素で置換する。v=と同時に指定はできない。\\
\verb|O=|  & NULL値以外の要素を全て、ここで指定した文字列に置換する。\\
           & 指定しなければNULL値以外は置換しない。\\
\verb|delim=| & ベクトル型データの区切り文字を指定する。\\
\end{tabular}
}
\end{table} 

\subsection*{利用例}
\subsubsection*{例1: nullを文字列`null'に置換する例}



\begin{Verbatim}[baselinestretch=0.7,frame=single]
$ more dat1.csv
items
b a  c
 c c
e a   b 
$ mvnullto vf=items v=null i=dat1.csv o=rsl1.csv
#END# kgvnullto i=dat1.csv o=rsl1.csv v=null vf=items
$ more rsl1.csv
items
b a null c
null c c
e a null null b null
\end{Verbatim}
\subsubsection*{例2: 分かりやすく区切り文字を.(ドット)にした例}



\begin{Verbatim}[baselinestretch=0.7,frame=single]
$ more dat2.csv
items
b.a..c
.c.c
e.a...b.
$ mvnullto vf=items v=null delim=. i=dat2.csv o=rsl2.csv
#END# kgvnullto delim=. i=dat2.csv o=rsl2.csv v=null vf=items
$ more rsl2.csv
items
b.a.null.c
null.c.c
e.a.null.null.b.null
\end{Verbatim}
\subsubsection*{例3: nullを直前の値に置換する例}



\begin{Verbatim}[baselinestretch=0.7,frame=single]
$ mvnullto vf=items -p i=dat1.csv o=rsl3.csv
#END# kgvnullto -p i=dat1.csv o=rsl3.csv vf=items
$ more rsl3.csv
items
b a a c
 c c
e a a a b b
\end{Verbatim}
\subsubsection*{例4: O=を指定することで、null以外は全て指定の値に置換される}



\begin{Verbatim}[baselinestretch=0.7,frame=single]
$ mvnullto vf=items v=null O=X i=dat1.csv o=rsl4.csv
#END# kgvnullto O=X i=dat1.csv o=rsl4.csv v=null vf=items
$ more rsl4.csv
items
X X null X
null X X
X X null null X null
\end{Verbatim}


\subsection*{関連コマンド}
\hyperref[sect:mvdelnull]{mvdelnull} : NULL要素を削除する。

%\end{document}
