

%\documentclass[a4paper]{jsbook}
%\usepackage{mcmd_jp}
%\begin{document}

\subsection{-assert\_nullkey キー項目にNULL値が含まれているかどうかのチェックを行う \label{sect:option_assert_nullkey}}

このパラメータを指定すると、キー項目(k=またはK=パラメータで指定する項目)にNULL値が含まれているかどうかのチェックを行い、NULL値が含まれていた場合に、\verb|「#WARNING# ; exist NULL in key filed」|というメッセージを表示する。

\subsection*{利用例}
\subsubsection*{(Basic example) }
Assume, for instance, the msum command (sum the field values) is used. When the values in the fields specified by the f= parameter is summed for the rows that have the same value in the fields specified by the k= parameter, the value of the key field specified by the k= parameter may contain a NULL value. Specify the -assert\_nullkey option to check whether a key field contains a NULL value. When a NULL value is contained, the \verb|“#WARNING# ; exist NULL in key filed”| message is shown.

\begin{Verbatim}[baselinestretch=0.7,frame=single]
$ more dat1.csv
customer,quantity,金額
A,1,10
,1,10
B,1,15
A,2,20
B,3,10
B,1,20

$ msum k=customer f=quantity:quantityT,Amount:AmountT -assert_nullkey i=dat1.csv o=rsl1.csv
#WARNING# ; exist NULL in key filed
#END# kgsum -assert_nullkey f=quantity:quantityT,Amount:AmountT i=dat1.csv k=customer o=rsl1.csv

$ more rsl1.csv
customer%0,quantityT,AmountT
,1,10
A,3,30
B,5,45
\end{Verbatim}


\subsubsection*{対応コマンド}
\hyperref[sect:maccum]{maccum},
\hyperref[sect:mavg]{mavg},
\hyperref[sect:mbest]{mbest},
\hyperref[sect:mbucket]{mbucket},
\hyperref[sect:mcal]{mcal},
\hyperref[sect:mcommon]{mcommon},
\hyperref[sect:mcount]{mcount},
\hyperref[sect:mcross]{mcross},
\hyperref[sect:mdelnull]{mdelnull},
\hyperref[sect:mhashavg]{mhashavg},
\hyperref[sect:mhashsum]{mhashsum},
\hyperref[sect:mjoin]{mjoin},
\hyperref[sect:mkeybreak]{mkeybreak},
\hyperref[sect:mmbucket]{mmbucket},
\hyperref[sect:mmvavg]{mmvavg},
\hyperref[sect:mmvsim]{mmvsim},
\hyperref[sect:mstats]{mstats},
\hyperref[sect:mnjoin]{mnjoin},
\hyperref[sect:mnormalize]{mnormalize},
\hyperref[sect:mnrcommon]{mnrcommon},
\hyperref[sect:mnrjoin]{mnrjoin},
\hyperref[sect:mnumber]{mnumber},
\hyperref[sect:mpadding]{mpadding},
\hyperref[sect:mrand]{mrand},
\hyperref[sect:mrjoin]{mrjoin},
\hyperref[sect:mselnum]{mselnum},
\hyperref[sect:mselrand]{mselrand},
\hyperref[sect:mselstr]{mselstr},
\hyperref[sect:msep2]{msep2},
\hyperref[sect:mshare]{mshare},
\hyperref[sect:msim]{msim},
\hyperref[sect:mslide]{mslide},
\hyperref[sect:mstats]{mstats},
\hyperref[sect:msum]{msum},
\hyperref[sect:msummary]{msummary},
\hyperref[sect:mtra]{mtra},
\hyperref[sect:muniq]{muniq},
\hyperref[sect:mwindow]{mwindow}\\
上記のコマンドで利用できる。

%\end{document}

