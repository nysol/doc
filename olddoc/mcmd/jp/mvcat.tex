
%\documentclass[a4paper]{jsbook}
%\usepackage{mcmd_jp}
%\begin{document}

\section{mvcat ベクトルの併合\label{sect:mvcat}}
\index{mvcat@mvcat}
複数のベクトルを併合して新しいベクトルを生成する。

\if0 #no help# following sentences will not apear on the help document. \fi
典型的な例を表\ref{tbl:mvcat_input}〜\ref{tbl:mvcat_out2}に示す。

\begin{table}[htbp]
\begin{center}
\begin{tabular}{ccc}

\begin{minipage}{0.3\hsize}
\begin{center}
\caption{入力データ\label{tbl:mvcat_input}}
\verb|in.csv| \\
{\small
\begin{tabular}{cll}
\hline
no&items1,items2 \\
\hline
1&a c,b \\
2&a d,a e \\
3&b f, \\
4&e,e \\
\hline

\end{tabular}
}
\end{center}
\end{minipage}

\begin{minipage}{0.5\hsize}
\begin{center}
\caption{基本例\label{tbl:mvcat_out1}}
\verb|mvcat vf=item1,items2 a=catItems i=in.csv| \\
{\small
\begin{tabular}{lll}
\hline
no&catItems \\
\hline
1&a c b \\
2&a d a e \\
3&b f \\
4&e e \\
\hline
\end{tabular}
}
\end{center}
\end{minipage}

\end{tabular}
\end{center}
\end{table}

\begin{table}[htbp]
\begin{center}
\begin{tabular}{ccc}

\begin{minipage}{0.5\hsize}
\begin{center}
\caption{併合前のベクトルを残す例\label{tbl:mvcat_out2}}
\verb|mvcat vf=item1,items2 -A i=in.csv| \\
{\small
\begin{tabular}{lll}
\hline
no&items1,items2,new \\
\hline
1&a c,b,a c b \\
2&a d,a e,a d a e \\
3&b f,,b f \\
4&e,e,e e \\
\hline
\end{tabular}
}
\end{center}
\end{minipage}

\end{tabular}
\end{center}
\end{table}

\subsection*{書式}
\verb|mvcat vf= a= [-A]|
\hyperref[sect:option_i]{[i=]}
\hyperref[sect:option_o]{[o=]}
\hyperref[sect:option_delim]{[delim=]} 
\hyperref[sect:option_assert_diffSize]{[-assert\_diffSize]}
\hyperref[sect:option_assert_nullin]{[-assert\_nullin]}
\hyperref[sect:option_assert_nullout]{[-assert\_nullout]}
\hyperref[sect:option_nfn]{[-nfn]} 
\hyperref[sect:option_nfno]{[-nfno]}  
\hyperref[sect:option_x]{[-x]}
\hyperref[sect:option_option_tmppath]{[tmpPath=]}
\hyperref[sect:option_precision]{[precision=]}
\verb|[-params]|
\verb|[--help]|
\verb|[--helpl]|
\verb|[--version]|\\

\subsection*{パラメータ}
\begin{table}[htbp]
%\begin{center}
{\small
\begin{tabular}{ll}
\verb|i=|    & 入力ファイル名を指定する。\\
\verb|o=|    & 出力ファイル名を指定する。\\
\verb|vf=| & 併合する複数のベクトル項目名(\verb|i=|ファイル上)を指定する。\\
           & 項目名にワイルドカードを使うことができる。\\
\verb|a=|  & 併合後の項目名を指定する。\\
\verb|-A|  & 新しい項目として追加する。このオプションを指定しなければ、併合元の項目(\verb|vf=|)は削除される。 \\
\verb|delim=| & ベクトル型データの区切り文字を指定する。\\
\end{tabular}
}
\end{table} 

\subsection*{利用例}
\subsubsection*{例1: ワイルドカードを利用した例}



\begin{Verbatim}[baselinestretch=0.7,frame=single]
$ more dat1.csv
items1,items2,items3,items4
b a c,b,x,y
c c,,x,y
e a a,a a a,x,y
$ mvcat vf=items* a=items i=dat1.csv o=rsl1.csv
#END# kgvcat a=items i=dat1.csv o=rsl1.csv vf=items*
$ more rsl1.csv
items
b a c b x y
c c x y
e a a a a a x y
\end{Verbatim}

\subsection*{関連コマンド}

%\end{document}
