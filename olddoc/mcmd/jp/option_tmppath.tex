
%\begin{document}

\subsection{tmpPath= 作業ファイル格納パス名\label{sect:option_tmpPath}}
コマンドが内部で用いる作業ファイルを格納するディレクトリ名を指定する。
例えば、\verb|msortf|は分割ソートにおいて並べ替えられた中間結果が一時ファイルとして保存される。
指定がなければデフォルトとして\verb|/tmp|が用いられる。
一時ファイル名は、必ず\verb|__KGTMP|から始まる。

作業ファイルは正常に終了すれば(エラー終了も含めてMCMDのコントロール下で正常に終了するという意味)削除されるが、
不測の事態、例えば停電やバグ終了の場合、消されず残る場合がある。
データ量によっては、非常に多くの作業ファイルが生成される可能性があり(100万ファイル以上!!)、
その場合は、コマンドの動作が極端に遅くなる可能性があるので、
定期的に作業パスのファイルを確認しておくべきである。
なお、現在のところ、これらの不要ファイルの自動消去(ガベージコレクション)の機能を実装する予定はない。

また、環境変数\verb|KG_Tmp_Path|を設定することで、作業ディレクトリを変更できる。
ただし、環境変数を変更すると、それ以降に実行するコマンド全てに反映されることに注意する。

\subsection*{利用例}
\subsubsection*{Example 1: Basic Example}

Set the \verb|tmp| directory under the current directory for temporary files.


\begin{Verbatim}[baselinestretch=0.7,frame=single]
$ msortf f=val tmpPath=./tmp i=dat1.csv o=rsl1.csv
#ERROR# internal error: cannot create temp file (kgsortf)
\end{Verbatim}
\subsubsection*{Example 2: Specify the environment variable}

The settings of the environment variable will be applied to subsequent commands.


\begin{Verbatim}[baselinestretch=0.7,frame=single]
$ export KG_TmpPath=~/tmp
$ msortf f=val i=dat1.csv o=rsl1.csv
#END# kgsortf f=val i=dat1.csv o=rsl1.csv
\end{Verbatim}


\subsubsection*{対応コマンド}
\hyperref[sect:msortf]{msortf},
\hyperref[sect:mselstr]{mdelnull}などのキー単位での選択を伴うコマンド、
\hyperref[mbucket]{mbucket}
\hyperref[mnjoin]{mnjoin}
\hyperref[mshare]{mshare}など、キー単位の処理において、データを複数パス走査する必要のあるコマンド。

%\end{document}

