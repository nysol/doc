
%\documentclass[a4paper]{jsbook}
%\usepackage{mcmd_jp}
%\begin{document}

\section{mmvsim 移動窓の類似度計算\label{sect:mmvsim}}
\index{mmvsim@mmvsim}

移動窓を設定し、各種類似度(2変量の統計量)を計算する。
\hyperref[sect:msim]{msim}コマンドの移動窓バージョンとして考えればよい。
\verb|msim|との違いは、指定できる類似度は一つだけで、また類似度計算の対象項目は2つのみである。

\subsection*{書式}
\verb|mmvsim [s=] [k=] f= c= a= [t=] [skip=] [-n] |
\hyperref[sect:option_i]{[i=]}
\hyperref[sect:option_o]{[o=]}
\hyperref[sect:option_assert_diffSize]{[-assert\_diffSize]}
\hyperref[sect:option_assert_nullkey]{[-assert\_nullkey]}
\hyperref[sect:option_assert_nullin]{[-assert\_nullin]}
\hyperref[sect:option_assert_nullout]{[-assert\_nullout]}
\hyperref[sect:option_nfn]{[-nfn]} 
\hyperref[sect:option_nfno]{[-nfno]}  
\hyperref[sect:option_x]{[-x]}
\hyperref[sect:option_q]{[-q]}
\hyperref[sect:option_option_tmppath]{[tmpPath=]}
\hyperref[sect:option_precision]{[precision=]}
\verb|[-params]|
\verb|[--help]|
\verb|[--helpl]|
\verb|[--version]|\\

\subsection*{パラメータ}
\begin{table}[htbp]
%\begin{center}
{\small
\begin{tabular}{ll}
\verb|i=|    & 入力ファイル名を指定する。\\
\verb|o=|    & 出力ファイル名を指定する。\\
\verb|s=|    & ここで指定した項目(複数項目指定可)で並べ替えられた後、各種類似度が計算される。\\
             & \verb|-q|オプションを指定しないとき、\verb|s=|パラメータは必須。\\
\verb|k=|    & ここで指定された項目(複数項目指定可)を単位として集計する。 \\
\verb|f=|    & 集計項目名リスト(複数項目指定可)を指定する。\\
\verb|t=|    & 期間数を1以上の整数で指定する。 \\
\verb|c=|    & 類似度(以下のリストから一つだけ)指定する。\\
             & \verb/covar|ucovar|pearson|spearman|kendall|euclid|/\\
             & \verb/cosine|cityblock|hamming|chi|phi|jaccard|support|lift/ \\
             & 詳細な定義は\hyperref[sect:msim]{msim}コマンドを参照のこと。\\
\verb|skip=| & 出力を抑制する最初の行数を指定する。【デフォルト値:\verb|skip=(t=の値-1)|】\\
\verb|a=| & 計算結果の出力として追加される項目の名前を指定する。 \\
\verb|-n| & 期間内にNULL値が1つでも含まれていると結果もNULL値とする。\\
\end{tabular} 
}
\end{table} 

\subsection*{利用例}
\subsubsection*{Example 1: Basic Example}

Calculate the Pearson product-moment correlation coefficient for 3 window intervals for fields \verb|x,y|.


\begin{Verbatim}[baselinestretch=0.7,frame=single]
$ more dat1.csv
t,x,y
1,14,0.17
2,11,0.2
3,32,0.15
4,13,0.33
5,8,0.1
6,19,0.56
$ mmvsim s=t t=3 c=pearson f=x,y a=sim i=dat1.csv o=rsl1.csv
#END# kgmvsim a=sim c=pearson f=x,y i=dat1.csv o=rsl1.csv s=t t=3
$ more rsl1.csv
t%0,x,y,sim
3,32,0.15,-0.8746392857
4,13,0.33,-0.6515529194
5,8,0.1,-0.1164257338
6,19,0.56,0.9986254289
\end{Verbatim}


\subsection*{関連コマンド}
\hyperref[sect:msim] {msim} : 移動窓を設定せずに類似度計算を行う。

\hyperref[sect:mwindow] {mwindow} : 動窓のデータを作成するので、そのデータを使えば\verb|mmvstats|で計算できない統計量も計算可能。

\hyperref[sect:mmvavg] {mmvavg} : 移動平均に限定した計算を行う。

%\end{document}
