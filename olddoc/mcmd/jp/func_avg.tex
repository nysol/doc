
%\begin{document}

\section{avg 平均\label{sect:avg}}
\index{avg@avg}

書式1: avg($num_1,num_2,\cdots$)

書式2: avg($num_1,num_2,\cdots,str$)

$num_i$で与えられた数値の平均を計算する。
書式1では、NULL値は無視されるが、全てがNULL値であれば結果もNULLとなる。

書式2において最後の引数として"-n"を与えると、
NULL値に対する扱いが変わり、
項目値に一つでもNULL値がある場合は、結果もNULL値となる。

\subsection*{利用例}
\subsubsection*{Example 1: Basic Example}



\begin{Verbatim}[baselinestretch=0.7,frame=single]
$ more dat1.csv
id,v1,v2,v3
1,1,2,3
2,-5,2,1
3,1,,3
4,,,
$ mcal c='avg(${v1},${v2},${v3})' a=rsl i=dat1.csv o=rsl1.csv
#END# kgcal a=rsl c=avg(${v1},${v2},${v3}) i=dat1.csv o=rsl1.csv
$ more rsl1.csv
id,v1,v2,v3,rsl
1,1,2,3,2
2,-5,2,1,-0.6666666667
3,1,,3,2
4,,,,
\end{Verbatim}
\subsubsection*{Example 2: Example of using wildcard}

Specify columns names that start with \verb|v| (\verb|v1,v2,v3|) using a wildcard in column name such as \verb|v*|.


\begin{Verbatim}[baselinestretch=0.7,frame=single]
$ mcal c='avg(${v*})' a=rsl i=dat1.csv o=rsl2.csv
#END# kgcal a=rsl c=avg(${v*}) i=dat1.csv o=rsl2.csv
$ more rsl2.csv
id,v1,v2,v3,rsl
1,1,2,3,2
2,-5,2,1,-0.6666666667
3,1,,3,2
4,,,,
\end{Verbatim}


%\end{document}
