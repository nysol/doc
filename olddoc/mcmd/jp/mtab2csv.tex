
%\documentclass[a4paper]{jsbook}
%\usepackage{mcmd_jp}
%\begin{document}

\section{mtab2csv TSVからCSVデータへの変換\label{sect:mtab2csv}}
\index{mtab2csv@mtab2csv}
タブ区切りデータをCSVデータへ変換する。
\verb|d=|で区切り文字を指定することで、タブ以外の区切り文字のテキストファイルも
変換することが可能である。 
変換後の項目数に違いある場合には、その直前行まで出力され、
その後エラー終了する。


\subsection*{書式}
\verb|mtab2csv [d=] [-r] |
\hyperref[sect:option_i]{[i=]}
\hyperref[sect:option_o]{[o=]}
\hyperref[sect:option_assert_diffSize]{[-assert\_diffSize]}
\hyperref[sect:option_nfn]{[-nfn]} 
\hyperref[sect:option_nfno]{[-nfno]}  
\hyperref[sect:option_x]{[-x]}
\hyperref[sect:option_option_tmppath]{[tmpPath=]}
\hyperref[sect:option_precision]{[precision=]}
\verb|[-params]|
\verb|[--help]|
\verb|[--helpl]|
\verb|[--version]|\\

\subsection*{パラメータ}
\begin{table}[htbp]
%\begin{center}
{\small
\begin{tabular}{ll}
\verb|i=|    & 入力ファイル名を指定する。\\
\verb|o=|    & 出力ファイル名を指定する。\\
\verb|d=|    & 区切り文字の指定(1バイト文字のみ指定可)。\\
\verb|-r|    & 改行コード(CR:\verb|\r|)を取り除く。\\
             & MCMDが扱うCSVは改行コードがLF(\verb|\n|)固定であるため、\\
             & Windowsのテキスト改行CR+LF(\verb|\r\n|)やMacのテキスト改行CR(\verb|\r|)があれば、\\
             & 単なる文字列として扱ってしまい、変換後に不具合が生じる。\\
             & この問題を回避するためのオプションである。\\
\end{tabular} 
}
\end{table} 

\subsection*{利用例}
\subsubsection*{Example 1: Example 1: Basic example}

TSV data is converted into CSV data.


\begin{Verbatim}[baselinestretch=0.7,frame=single]
$ more dat1.tab
id	data	data2
A	1102	a
A	2203	aaa
B	1155	bbbb
B	3104	c
B	1206	de
$ mtab2csv i=dat1.tab o=rsl1.csv
#END# kgtab2csv i=dat1.tab o=rsl1.csv
$ more rsl1.csv
id,data,data2
A,1102,a
A,2203,aaa
B,1155,bbbb
B,3104,c
B,1206,de
\end{Verbatim}
\subsubsection*{Example 2: Example 2: Specifying d=}

The \verb|d=| parameter is specified to use a delimiter other than the tab.


\begin{Verbatim}[baselinestretch=0.7,frame=single]
$ more dat2.bar
id-data-data2
A-1102-a
A-2203-aaa
B-1155-bbbb
B-3104-c
B-1206-de
$ mtab2csv d=- i=dat2.bar o=rsl2.csv
#END# kgtab2csv d=- i=dat2.bar o=rsl2.csv
$ more rsl2.csv
id,data,data2
A,1102,a
A,2203,aaa
B,1155,bbbb
B,3104,c
B,1206,de
\end{Verbatim}

\subsection*{関連コマンド}
\hyperref[sect:mxml2csv]{mxml2csv}:XMLデータをCSVデータへ変換する

\hyperref[sect:msplit]{msplit}:項目値の区切り文字によるデータ分割
%\end{document}
