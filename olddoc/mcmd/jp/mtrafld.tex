
%\documentclass[a4paper]{jsbook}
%\usepackage{mcmd_jp}
%\begin{document}

\section{mtrafld クロス表をトランザクション項目に変換\label{sect:mtrafld}}
\index{mtrafld@mtrafld}
\verb|f=|で指定した項目値とその値のペアのアイテムを作成し、
それらのアイテムを連結し新しいベクトル項目(トランザクション項目とも呼ぶ)として出力する。

\subsection*{書式}
\verb|mtrafld a= [f=] [delim=] [delim2=] [-r] [-valOnly] |   
\hyperref[sect:option_i]{[i=]}
\hyperref[sect:option_o]{[o=]}
\hyperref[sect:option_assert_diffSize]{[-assert\_diffSize]}
\hyperref[sect:option_assert_nullin]{[-assert\_nullin]}
\hyperref[sect:option_assert_nullout]{[-assert\_nullout]}
\hyperref[sect:option_nfn]{[-nfn]} 
\hyperref[sect:option_nfno]{[-nfno]}  
\hyperref[sect:option_x]{[-x]}
\hyperref[sect:option_option_tmppath]{[tmpPath=]}
\hyperref[sect:option_precision]{[precision=]}
\verb|[-params]|
\verb|[--help]|
\verb|[--helpl]|
\verb|[--version]|\\

\subsection*{パラメータ}
\begin{table}[htbp]
%\begin{center}
{\small
\begin{tabular}{ll}
\verb|i=|    & 入力ファイル名を指定する。\\
\verb|o=|    & 出力ファイル名を指定する。\\
\verb|a=|      & トランザクション項目名を指定する。\\
\verb|f=|      & 項目名リスト(複数項目指定可)【\verb|-r|指定時必須、それ以外は任意】\\
               & ここで指定された項目名と値とを連結したアイテムを作成し\\
               & トランザクション項目として出力される。\\
               & \verb|-r|オプションの指定がある時は\\
               & トランザクションデータから抜き出す項目名を指定する。\\
               & \verb|-r|オプションが指定されたとき,このパラメータは省略可能である。\\
               & 省略すると、全ての項目名と値ペアを処理対象とする。\\
               & ただし、\verb|f=|パラメータを省略すると標準入力(パイプ入力)は利用できない。\\
\verb|delim=|  & トランザクション項目のアイテムを区切る文字を指定する(省略時はスペース)。\\
\verb|delim2=| & 項目名と値ペアとを区切る文字を指定する(省略時は=)。\\
\verb|-r|      & 条件反転\\
               & トランザクション項目をクロス表に変換する。\\
\verb|-valOnly|& このオプションが指定されると、アイテムとして「項目名=」は出力しない。\\
\end{tabular} 
}
\end{table} 

\subsection*{利用例}
\subsubsection*{例1: 基本例}

\verb|price|と\verb|quantity|項目を1つの文字列として連結し、
\verb|transaction|という項目名で出力する。


\begin{Verbatim}[baselinestretch=0.7,frame=single]
$ more dat1.csv
customer,price,quantity
A,198,1
B,325,2
C,200,3
D,450,2
E,100,1
$ mtrafld a=transaction f=price,quantity i=dat1.csv o=rsl1.csv
#END# kgtrafld a=transaction f=price,quantity i=dat1.csv o=rsl1.csv
$ more rsl1.csv
customer,transaction
A,price=198 quantity=1
B,price=325 quantity=2
C,price=200 quantity=3
D,price=450 quantity=2
E,price=100 quantity=1
\end{Verbatim}
\subsubsection*{例2: 基本例2}

出力された結果を\verb|-r|をつけて再実行し元に戻す。


\begin{Verbatim}[baselinestretch=0.7,frame=single]
$ mtrafld -r a=transaction f=price,quantity i=rsl1.csv o=rsl2.csv
#END# kgtrafld -r a=transaction f=price,quantity i=rsl1.csv o=rsl2.csv
$ more rsl2.csv
customer,price,quantity
A,198,1
B,325,2
C,200,3
D,450,2
E,100,1
\end{Verbatim}
\subsubsection*{例3: 区切り文字の指定}

\verb|price|と数量\verb|quantity|項目を\_(アンダーバー)で区切り1つの文字列として連結し、
項目名とデータは:(コロン)で区切り\verb|transaction|という項目名で出力する。


\begin{Verbatim}[baselinestretch=0.7,frame=single]
$ mtrafld a=transaction f=price,quantity delim=_ delim2=':' i=dat1.csv o=rsl3.csv
#END# kgtrafld a=transaction delim2=: delim=_ f=price,quantity i=dat1.csv o=rsl3.csv
$ more rsl3.csv
customer,transaction
A,price:198_quantity:1
B,price:325_quantity:2
C,price:200_quantity:3
D,price:450_quantity:2
E,price:100_quantity:1
\end{Verbatim}
\subsubsection*{例4: NULL値を含む場合}



\begin{Verbatim}[baselinestretch=0.7,frame=single]
$ more dat2.csv
customer,price,quantity
A,198,1
B,,2
C,200,
D,450,2
E,,
$ mtrafld a=transaction f=price,quantity i=dat2.csv o=rsl4.csv
#END# kgtrafld a=transaction f=price,quantity i=dat2.csv o=rsl4.csv
$ more rsl4.csv
customer,transaction
A,price=198 quantity=1
B,quantity=2
C,price=200
D,price=450 quantity=2
E,
\end{Verbatim}
\subsubsection*{例5: NULL値を含む場合2}

出力された結果を\verb|-r|をつけて再実行し元に戻す。


\begin{Verbatim}[baselinestretch=0.7,frame=single]
$ mtrafld -r a=transaction f=price,quantity i=rsl4.csv o=rsl5.csv
#END# kgtrafld -r a=transaction f=price,quantity i=rsl4.csv o=rsl5.csv
$ more rsl5.csv
customer,price,quantity
A,198,1
B,,2
C,200,
D,450,2
E,,
\end{Verbatim}
\subsubsection*{例6: -valOnlyの指定}



\begin{Verbatim}[baselinestretch=0.7,frame=single]
$ mtrafld -valOnly f=price,quantity a=transaction i=dat2.csv o=rsl6.csv
#END# kgtrafld -valOnly a=transaction f=price,quantity i=dat2.csv o=rsl6.csv
$ more rsl6.csv
customer,transaction
A,198 1
B,2
C,200
D,450 2
E,
\end{Verbatim}

\subsection*{関連コマンド}
\hyperref[sect:mvsort] {mvsort} : トランザクションデータはベクトル型データを処理する一連の処理コマンド(\verb|mv|から始まる)によって加工できる。

\hyperref[sect:mcross] {mcross} : トランザクションデータとしてではなく、個々のアイテムを独立した項目として出力し、その出現件数を出力する。

\hyperref[sect:mtra] {mtra} : 項目の値をアイテムとしてトランザクションデータを作成する。

\hyperref[sect:mtraflg] {mtraflg} : 項目名をアイテムとしてトランザクションデータを作成する。

%\end{document}
