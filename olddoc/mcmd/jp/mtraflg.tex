
%\documentclass[a4paper]{jsbook}
%\usepackage{mcmd_jp}
%\begin{document}

\section{mtraflg クロス表をトランザクション項目に変換\label{sect:mtraflg}}
\index{mtraflg@mtraflg}
\verb|f=|パラメータで指定した項目値がNULL値かどうかをチェックし、
NULL値以外であれば,それらの項目名を1つのアイテムとして連結し
新しいベクトル項目(トランザクション項目とも呼ぶ)として出力する。

\subsection*{書式}
\verb|mtraflg a= f= [delim=] [-r]  | 
\hyperref[sect:option_i]{[i=]}
\hyperref[sect:option_o]{[o=]}
\hyperref[sect:option_assert_diffSize]{[-assert\_diffSize]}
\hyperref[sect:option_assert_nullin]{[-assert\_nullin]}
\hyperref[sect:option_assert_nullout]{[-assert\_nullout]}
\hyperref[sect:option_nfn]{[-nfn]} 
\hyperref[sect:option_nfno]{[-nfno]}  
\hyperref[sect:option_x]{[-x]}
\hyperref[sect:option_option_tmppath]{[tmpPath=]}
\hyperref[sect:option_precision]{[precision=]}
\verb|[-params]|
\verb|[--help]|
\verb|[--helpl]|
\verb|[--version]|\\

\subsection*{パラメータ}
\begin{table}[htbp]
%\begin{center}
{\small
\begin{tabular}{ll}
\verb|i=|    & 入力ファイル名を指定する。\\
\verb|o=|    & 出力ファイル名を指定する。\\
\verb|a=|      & トランザクション項目名を指定する。\\
\verb|f=|      & ここで指定された項目値(複数項目指定可)をチェックし、トランザクションデータを作成する。\\
               & (\verb|-r|オプションの指定がある時はトランザクションデータから項目名として抜き出す値のリスト)\\
\verb|delim=|  & ここで指定した文字をトランザクション項目のアイテム間の区切りとする(省略時はスペース)。\\
               & 文字列の指定はできない。1バイト文字のみ指定可能。\\
\verb|-r|      & 条件反転\\
               & トランザクション型から縦型へデータを変換する。\\
\end{tabular} 
}
\end{table} 

\subsection*{利用例}
\subsubsection*{Example 1: Basic Example}

Create a string of vector from the list of non-null values in column \verb|egg| and \verb|milk|.


\begin{Verbatim}[baselinestretch=0.7,frame=single]
$ more dat1.csv
customer,egg,milk
A,1,1
B,,1
C,1,
D,1,1
$ mtraflg f=egg,milk a=transaction i=dat1.csv o=rsl1.csv
#END# kgtraflg a=transaction f=egg,milk i=dat1.csv o=rsl1.csv
$ more rsl1.csv
customer,transaction
A,egg milk
B,milk
C,egg
D,egg milk
\end{Verbatim}
\subsubsection*{Example 2: Basic Example 2}

Use \verb|-r| option to revert the output results back to the original data.


\begin{Verbatim}[baselinestretch=0.7,frame=single]
$ mtraflg -r f=egg,milk a=transaction i=rsl1.csv o=rsl2.csv
#END# kgtraflg -r a=transaction f=egg,milk i=rsl1.csv o=rsl2.csv
$ more rsl2.csv
customer,egg,milk
A,1,1
B,,1
C,1,
D,1,1
\end{Verbatim}
\subsubsection*{Example 3: Specify the delimiter}

Combine items using the “-” (hyphen) as delimiter. Save output in column named \verb|transaction|.


\begin{Verbatim}[baselinestretch=0.7,frame=single]
$ mtraflg f=egg,milk a=transaction delim=- i=dat1.csv o=rsl3.csv
#END# kgtraflg a=transaction delim=- f=egg,milk i=dat1.csv o=rsl3.csv
$ more rsl3.csv
customer,transaction
A,egg-milk
B,milk
C,egg
D,egg-milk
\end{Verbatim}

\subsection*{関連コマンド}
\hyperref[sect:mvsort] {mvsort} : トランザクションデータはベクトル型データを処理する一連の処理コマンド(\verb|mv|から始まる)によって加工できる。

\hyperref[sect:mcross] {mcross} : トランザクションデータとしてではなく、個々のアイテムを独立した項目として出力し、その出現件数を出力する。

\hyperref[sect:mtra] {mtra} : 項目の値をアイテムとしてトランザクションデータを作成する。

\hyperref[sect:mtrafld] {mtrafld} : 「項目名=値」の形式でトランザクションデータを作成する。

%\end{document}
