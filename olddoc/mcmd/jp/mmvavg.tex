
%\documentclass[a4paper]{jsbook}
%\usepackage{mcmd_jp}
%\begin{document}

\section{mmvavg 移動平均の算出\label{sect:mmvavg}}
\index{mmvavg@mmvavg}

移動平均(moving average)を算出する。
移動平均としては、単純移動平均($SMA$)、線形荷重移動平均($WMA$)、指数平滑移動平均($EMA$)の3種類の移動平均が計算可能である。

\if0 #no help# following sentences will not apear on the help document. \fi
ある時$t$における値を$x_t$で表したとき、$m$期の各種移動平均は式(\ref{eq:sma},\ref{eq:wma},\ref{eq:ema})で定義される。

\begin{eqnarray}
%\begin{footnotesize}
SMA_t=\frac{1}{m} \sum_{i=0}^{m-1} x_{t-i}
\label{eq:sma}
%\end{footnotesize}
\end{eqnarray}

\begin{eqnarray}
%\begin{footnotesize}
WMA_t=\sum_{i=0}^{m-1} \frac{m-i}{S} x_{t-i},\ \ S=\sum_{i=1}^m i
\label{eq:wma}
%\end{footnotesize}
\end{eqnarray}

\begin{eqnarray}
%\begin{footnotesize}
EMA_t=\alpha x_t + (1-\alpha)EMA_{t-1}
\label{eq:ema}
%\end{footnotesize}
\end{eqnarray}

\subsection*{書式}
\verb/mmvavg [s=] [k=] [-n] f= [t=] [-exp|-w] [alpha=] [skip=]/
\hyperref[sect:option_i]{[i=]}
\hyperref[sect:option_o]{[o=]}
\hyperref[sect:option_assert_diffSize]{[-assert\_diffSize]}
\hyperref[sect:option_assert_nullkey]{[-assert\_nullkey]}
\hyperref[sect:option_assert_nullin]{[-assert\_nullin]}
\hyperref[sect:option_assert_nullout]{[-assert\_nullout]}
\hyperref[sect:option_nfn]{[-nfn]} 
\hyperref[sect:option_nfno]{[-nfno]}  
\hyperref[sect:option_x]{[-x]}
\hyperref[sect:option_q]{[-q]}
\hyperref[sect:option_option_tmppath]{[tmpPath=]}
\hyperref[sect:option_precision]{[precision=]}
\verb|[-params]|
\verb|[--help]|
\verb|[--helpl]|
\verb|[--version]|\\

\subsection*{パラメータ}
\begin{table}[htbp]
%\begin{center}
{\small
\begin{tabular}{ll}
\verb|i=|    & 入力ファイル名を指定する。\\
\verb|o=|    & 出力ファイル名を指定する。\\
\verb|s=|    & ここで指定した項目(複数項目指定可)で並べ替えられた後、移動平均が計算される。\\
             & \verb|-q|オプションを指定しないとき、\verb|s=|パラメータは必須。\\
\verb|k=|    & ここで指定された項目(複数項目指定可)を単位として集計する。 \\
\verb|f=|    & 移動平均を求める項目名リスト(複数項目指定可)を指定する。\\
\verb|t=|    & 期間数を1以上の整数で指定する。 \\
             & \verb|-exp|指定時に\verb|alpha=|を指定すれば\verb|t=|は指定できない。\\
\verb|-w|    & 線形加重移動平均を求める。\\
\verb|-exp|  & 指数平滑移動平均を求める。\\
\verb|alpha=|& \verb|-exp|が指定された時の平滑化係数を実数値で与える。\\
             & 省略時は\verb|alpha=2/(t=の値+1)|。\\
\verb|skip=| & 出力を抑制する最初の行数。\\
             & デフォルト値: \verb|skip=(t=の値-1)|, \verb|-exp|が指定された場合は\verb|skip=0| \\
\verb|-n| & 期間内にNULL値が1つでも含まれていると結果もNULL値とする。 \\
\end{tabular} 
}
\end{table} 


\subsection*{利用例}
\subsubsection*{Example 1: Basic Example}

The first row is not printed as there is less than the number of required intervals for computation.


\begin{Verbatim}[baselinestretch=0.7,frame=single]
$ more dat1.csv
id,value
1,5
2,1
3,3
4,4
5,4
6,6
7,1
8,4
9,7
$ mmvavg s=id f=value t=2 i=dat1.csv o=rsl1.csv
#END# kgmvavg f=value i=dat1.csv o=rsl1.csv s=id t=2
$ more rsl1.csv
id%0,value
2,3
3,2
4,3.5
5,4
6,5
7,3.5
8,2.5
9,5.5
\end{Verbatim}
\subsubsection*{Example 2: Basic Example 2}

The first row is not printed as there is less than the number of required intervals for computation.


\begin{Verbatim}[baselinestretch=0.7,frame=single]
$ mmvavg s=id f=value t=2 -w i=dat1.csv o=rsl2.csv
#END# kgmvavg -w f=value i=dat1.csv o=rsl2.csv s=id t=2
$ more rsl2.csv
id%0,value
2,2.333333333
3,2.333333333
4,3.666666667
5,4
6,5.333333333
7,2.666666667
8,3
9,6
\end{Verbatim}
\subsubsection*{Example 3: Basic Example 3}

Exponential smoothing moving average (\verb|-exp|) includes the first row in the output.


\begin{Verbatim}[baselinestretch=0.7,frame=single]
$ mmvavg s=id f=value t=2 -exp i=dat1.csv o=rsl3.csv
#END# kgmvavg -exp f=value i=dat1.csv o=rsl3.csv s=id t=2
$ more rsl3.csv
id%0,value
1,5
2,2.333333333
3,2.777777778
4,3.592592593
5,3.864197531
6,5.288065844
7,2.429355281
8,3.47645176
9,5.82548392
\end{Verbatim}
\subsubsection*{Example 4: An example of assigning key}



\begin{Verbatim}[baselinestretch=0.7,frame=single]
$ more dat2.csv
id,key,value
1,a,5
2,a,1
3,a,3
4,a,4
5,a,4
6,b,6
7,b,1
8,b,4
9,b,7
$ mmvavg s=key,id k=key f=value t=2 i=dat2.csv o=rsl4.csv
#END# kgmvavg f=value i=dat2.csv k=key o=rsl4.csv s=key,id t=2
$ more rsl4.csv
id,key,value
2,a,3
3,a,2
4,a,3.5
5,a,4
7,b,3.5
8,b,2.5
9,b,5.5
\end{Verbatim}
\subsubsection*{Example 5: Display all records including those that are less than the defined intervals }



\begin{Verbatim}[baselinestretch=0.7,frame=single]
$ more dat3.csv
key,value
a,1
a,2
a,3
a,4
a,5
b,6
b,1
b,4
b,7
$ mmvavg -q k=key f=value t=2 skip=0 i=dat3.csv o=rsl5.csv
#END# kgmvavg -q f=value i=dat3.csv k=key o=rsl5.csv skip=0 t=2
$ more rsl5.csv
key,value
a,1
a,1.5
a,2.5
a,3.5
a,4.5
b,6
b,3.5
b,2.5
b,5.5
\end{Verbatim}

\subsection*{関連コマンド}
\hyperref[sect:mmvstats] {mmvstats} : 平均だけでなく、各種統計量を指定可能。

\hyperref[sect:mmvsim] {mmvsim} : 2変量の統計量を計算する。

\hyperref[sect:mwindow] {mwindow} : 動窓のデータを作成するので、そのデータを使えば\verb|mmvstats|で計算できない統計量も計算可能。

%\end{document}
