
%\begin{document}

\section{ceil 切り上げ\label{sect:ceil}}
\index{ceil@ceil}

書式: ceil($num$,基数)

$num$を切り上げにより丸める。
この時、基数の整数倍の値集合のうち、$num$より大きい最小の目盛点にまるめられる。
例えば、ceil(3.42,0.5)の場合、0.5とびに目盛がうたれた数直線上で、
3.42より大きい最小の目盛点、すなわち3.5が基数0.5における切り上げ点となる。
基数を省略すると1.0がデフォルト値として用いられる。
これは小数点以下1桁目を切り上げて整数値にまるめることに等しい。

\subsection*{利用例}
\subsubsection*{例1: 基本例}

小数点以下一桁目を切り上げる。


\begin{Verbatim}[baselinestretch=0.7,frame=single]
$ more dat1.csv
id,val
1,3.28
2,3.82
3,
4,-0.6
$ mcal c='ceil(${val})' a=rsl i=dat1.csv o=rsl1.csv
#END# kgcal a=rsl c=ceil(${val}) i=dat1.csv o=rsl1.csv
$ more rsl1.csv
id,val,rsl
1,3.28,4
2,3.82,4
3,,
4,-0.6,-0
\end{Verbatim}
\subsubsection*{例2: 基本例}

小数点以下二桁目を切り上げる。


\begin{Verbatim}[baselinestretch=0.7,frame=single]
$ mcal c='ceil(${val},0.1)' a=rsl i=dat1.csv o=rsl2.csv
#END# kgcal a=rsl c=ceil(${val},0.1) i=dat1.csv o=rsl2.csv
$ more rsl2.csv
id,val,rsl
1,3.28,3.3
2,3.82,3.9
3,,
4,-0.6,-0.5
\end{Verbatim}
\subsubsection*{例3: 基数0.5の例}

0.5を基数として切り上げる。


\begin{Verbatim}[baselinestretch=0.7,frame=single]
$ mcal c='ceil(${val},0.5)' a=rsl i=dat1.csv o=rsl3.csv
#END# kgcal a=rsl c=ceil(${val},0.5) i=dat1.csv o=rsl3.csv
$ more rsl3.csv
id,val,rsl
1,3.28,3.5
2,3.82,4
3,,
4,-0.6,-0.5
\end{Verbatim}
\subsubsection*{例4: 基数10の例}

一桁目を切り上げる。


\begin{Verbatim}[baselinestretch=0.7,frame=single]
$ more dat2.csv
id,val
1,1341.28
2,188
3,1.235E+3
4,-1.235E+3
$ mcal c='ceil(${val},10)' a=rsl i=dat2.csv o=rsl4.csv
#END# kgcal a=rsl c=ceil(${val},10) i=dat2.csv o=rsl4.csv
$ more rsl4.csv
id,val,rsl
1,1341.28,1350
2,188,190
3,1.235E+3,1240
4,-1.235E+3,-1230
\end{Verbatim}


%\end{document}

