
%\documentclass[a4paper]{jsbook}
%\usepackage{mcmd_jp}
%\begin{document}

\section{mpaste 参照ファイル項目の行番号マッチング結合\label{sect:mpaste}}
\index{mpaste@mpaste}
入力ファイルと参照ファイルを行番号でマッチングさせて結合する。
データ件数が異なる場合は、少い方のデータに合わせる。
\verb|-n|や\verb|-N|を指定することでマッチングできな行もNULL値で結合することが可能である。

\subsection*{書式}
\verb/mpaste [f=] -n -N m=|/   
\hyperref[sect:option_i]{i=}
\hyperref[sect:option_o]{[o=]}
\hyperref[sect:option_assert_diffSize]{[-assert\_diffSize]}
\hyperref[sect:option_assert_nullin]{[-assert\_nullin]}
\hyperref[sect:option_assert_nullout]{[-assert\_nullout]}
\hyperref[sect:option_nfn]{[-nfn]} 
\hyperref[sect:option_nfno]{[-nfno]}  
\hyperref[sect:option_x]{[-x]}
\hyperref[sect:option_option_tmppath]{[tmpPath=]}
\hyperref[sect:option_precision]{[precision=]}
\verb|[-params]|
\verb|[--help]|
\verb|[--helpl]|
\verb|[--version]|\\

\subsection*{パラメータ}
\begin{table}[htbp]
%\begin{center}
{\small
\begin{tabular}{ll}
\verb|i=|    & 入力ファイル名を指定する。\\
\verb|o=|    & 出力ファイル名を指定する。\\
\verb|f=|    & 結合する参照ファイル上の項目名リスト(複数項目指定可)。\\
             & 省略するとキー項目を除いた全ての項目が結合される。\\
\verb|m=|    & 参照ファイル名を指定する。\\
             & このパラメータが省略された時には標準入力が用いられる。(\verb|i=|指定ありの場合)\\
\verb|-n|    & 入力ファイルにあって参照ファイルにない場合にNULL値を出力する。\\
\verb|-N|    & 参照ファイルにあって入力ファイルにない場合にNULL値を出力する。\\
\end{tabular} 
}
\end{table} 

\subsection*{利用例}
\subsubsection*{Example 1: Basic Example}



\begin{Verbatim}[baselinestretch=0.7,frame=single]
$ more dat1.csv
id1
1
2
3
4
$ more ref1.csv
id2
a
b
c
d
$ mpaste m=ref1.csv i=dat1.csv o=rsl1.csv
#END# kgpaste i=dat1.csv m=ref1.csv o=rsl1.csv
$ more rsl1.csv
id1,id2
1,a
2,b
3,c
4,d
\end{Verbatim}
\subsubsection*{Example 2: Example of merging data of different sizes}

If the number of rows in the input file is different from the reference file , merge records according to the smaller file.


\begin{Verbatim}[baselinestretch=0.7,frame=single]
$ more ref2.csv
id2
a
b
$ mpaste m=ref2.csv i=dat1.csv o=rsl2.csv
#END# kgpaste i=dat1.csv m=ref2.csv o=rsl2.csv
$ more rsl2.csv
id1,id2
1,a
2,b
\end{Verbatim}
\subsubsection*{Example 3: Outer join}

If there are less number of rows in the reference file, NULL values will be assigned to records that did not match with the input file when \verb|-n| option is specified.


\begin{Verbatim}[baselinestretch=0.7,frame=single]
$ mpaste m=ref2.csv -n i=dat1.csv o=rsl3.csv
#END# kgpaste -n i=dat1.csv m=ref2.csv o=rsl3.csv
$ more rsl3.csv
id1,id2
1,a
2,b
3,
4,
\end{Verbatim}
\subsubsection*{Example 4: Define fields to join}



\begin{Verbatim}[baselinestretch=0.7,frame=single]
$ more ref3.csv
id2,val
a,R0
b,R1
c,R2
d,R3
$ mpaste f=val m=ref3.csv i=dat1.csv o=rsl4.csv
#END# kgpaste f=val i=dat1.csv m=ref3.csv o=rsl4.csv
$ more rsl4.csv
id1,val
1,R0
2,R1
3,R2
4,R3
\end{Verbatim}


\subsection*{関連コマンド}
\hyperref[sect:mjoin]{mjoin} : 行番号でなく、キー項目で結合する。

%\end{document}
