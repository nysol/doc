
%\documentclass[a4paper]{jsbook}
%\usepackage{mcmd_jp}
%\begin{document}

\section{minput 画面入力\label{sect:minput}}
\index{minput@minput}
\underline{注)本コマンドは開発バージョンであり、仕様が変更される可能性があります。}

座標\verb|x=,y=|で指定したターミナル上の位置に\verb|len=|で指定した長さの入力枠を表示し、
入力した内容を\verb|o=|で指定したファイルに出力する。
入力画面でエンターキーを押すと、終了ステータス0を返して終了し、
エスケープキーを押すと、終了ステータス1を返して終了する。
いずれのキーで終了しても、そこまでに入力した内容はファイルに出力される。

出力されるCSVファイルは、一項目一行のデータとなる。
入力枠に何も入力せずに終了した場合は、null値が出力される(すなわち改行だけが出力される)。
\verb|f=|を指定すれば、項目名を出力できる。
\verb|f=|を省略すれば、項目名ヘッダーは出力されない。

座標は左上が\verb|x=1,y=1|である(エスケープシーケンスの仕様)。
\verb|x=|もしくは\verb|y=|で指定した値が1より小さい場合は、1を指定したものとして動作する。
またターミナルの範囲を超えた座標が指定された場合の動作は不定である。


\subsection*{書式}
\verb/minput x= y= len= [f=] /
\hyperref[sect:option_o]{o=}
\hyperref[sect:option_nfn]{[-nfn]} 
\hyperref[sect:option_nfno]{[-nfno]}  
\hyperref[sect:option_x]{[-x]}
\hyperref[sect:option_option_tmppath]{[tmpPath=]}
\hyperref[sect:option_precision]{[precision=]}
\verb|[-params]|
\verb|[--help]|
\verb|[--helpl]|
\verb|[--version]|\\

\subsection*{パラメータ}
\begin{table}[htbp]
%\begin{center}
{\small
\begin{tabular}{ll}
\verb|o=|   & 出力ファイル名を指定する。\\
\verb|x=|   & x軸(左から右への横方向)表示開始位置(1以上の値)を指定する。\\
\verb|y=|   & y軸(上から下への縦方向)表示開始位置(1以上の値)を指定する。\\
\verb|len=| & 入力欄の半角文字数 \\
\verb|f=|   & 出力項目名\\
\end{tabular} 
}
\end{table} 

\subsection*{利用例}

\subsubsection*{例1: 基本例}

ターミナルのx=10,y=5の位置に幅12文字の入力領域を示し、入力枠に"abcd"と入力してenterキーを押し、入力結果をrsl1.csvに保存する。

\begin{Verbatim}[baselinestretch=0.7,frame=single]
$ minput x=10 y=5 len=12 o=rsl1.csv
$ more rsl1.csv
abcd


以下、画面イメージ
+--------------------------------------
|
|
|
|
|          [abcd        ]
|
|
\end{Verbatim}

\subsubsection*{例2: 項目名を指定する基本例}

例1と同様でf=で項目名を指定する。

\begin{Verbatim}[baselinestretch=0.7,frame=single]
$ minput x=10 y=5 len=12 f=name o=rsl1.csv
$ more rsl1.csv
name
abcd
\end{Verbatim}


\subsubsection*{例3: 終了ステータスを判定する例}

例1と同じパラメータで実行し、終了ステータスを判定して異なる動作をするスクリプトの例。

\begin{Verbatim}[baselinestretch=0.7,frame=single]
$ more scp.sh
rm -f rsl2.csv
clear
minput x=10 y=5 len=12 o=rsl2.csv
if [ $? = 0 ] ; then
  clear ; echo "end by enter key"
else
  clear ; echo "end by escape key"
fi

# abcdと入力後enterキーを入力した場合の結果
$ bash scp.sh
end by enter key
$ more rsl2.csv
abcd

# abcdと入力後escapeキーを入力した場合の結果
$ bash scp.sh
end by escape key
$ more rsl2.csv
abcd
\end{Verbatim}

\subsection*{関連コマンド}
\hyperref[sect:mminput] {mminput} : 複数入力枠による入力画面を表示する。

\hyperref[sect:mdsp] {mdsp} : 画面の指定位置に文字列を表示する。

\hyperref[sect:mseldsp] {mseldsp} : 画面に単一選択入力窓を表示する。

\hyperref[sect:mmseldsp] {mmseldsp} : 画面に複数選択入力窓を表示する。


%\end{document}
