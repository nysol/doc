
%\documentclass[a4paper]{jsbook}
%\usepackage{mcmd_jp}
%\begin{document}

\section{msum 項目値の合計\label{sect:msum}}
\index{msum@msum}
\verb|k=|パラメータで指定した項目の値が同じ行について、
\verb|f=|パラメータで指定した集計項目の項目値を合計する。\\
\verb|(注)|k=とf=パラメータで指定した項目以外については、どの行が出力されるかは不定であることに注意してください。\\

\subsection*{書式}
\verb|msum f= [k=] [-n]| 
\hyperref[sect:option_i]{[i=]}
\hyperref[sect:option_o]{[o=]}
\hyperref[sect:option_assert_diffSize]{[-assert\_diffSize]}
\hyperref[sect:option_assert_nullkey]{[-assert\_nullkey]}
\hyperref[sect:option_assert_nullin]{[-assert\_nullin]}
\hyperref[sect:option_assert_nullout]{[-assert\_nullout]}
\hyperref[sect:option_nfn]{[-nfn]} 
\hyperref[sect:option_nfno]{[-nfno]}  
\hyperref[sect:option_x]{[-x]}
\hyperref[sect:option_q]{[-q]}
\hyperref[sect:option_tmpPath]{[tmpPath=]} 
\hyperref[sect:option_precision]{[precision=]}
\verb|[-params]|
\verb|[--help]|
\verb|[--helpl]|
\verb|[--version]|\\

\subsection*{パラメータ}
\begin{table}[htbp]
%\begin{center}
{\small
\begin{tabular}{ll}
\verb|i=|    & 入力ファイル名を指定する。\\
\verb|o=|    & 出力ファイル名を指定する。\\ 
\verb|k=|    & 集計の単位となる項目名リスト(複数項目指定可)を指定する。\\
%\verb|k=|    & 集計の単位となる項目名リスト(複数項目指定可)を指定する。【\hyperref[sect:option_k]{集計キーブレイク処理}】\\
%             & このパラメータを指定する場合は事前に、してした項目で並べ替えておく必要がある。\\
\verb|f=|    & ここで指定された項目(複数項目指定可)の値が集計される。NULL値は無視される。 \\
\verb|-n|    & \verb|f=|で指定した項目にNULL値が入っていると計算結果もNULLとする。\\
\end{tabular} 
}
\end{table} 

\subsection*{利用例}
\subsubsection*{Example 1: Basic Example}

Calculate the total value of "quantity" and "amount" for each "customer".  Save the output with field names "total quantity" and "total amount".


\begin{Verbatim}[baselinestretch=0.7,frame=single]
$ more dat1.csv
customer,quantity,amount
A,1,10
A,2,20
B,1,15
B,3,10
B,1,20
$ msum k=customer f=quantity:quantitySum,amount:amountSum i=dat1.csv o=rsl1.csv
#END# kgsum f=quantity:quantitySum,amount:amountSum i=dat1.csv k=customer o=rsl1.csv
$ more rsl1.csv
customer%0,quantitySum,amountSum
A,3,30
B,5,45
\end{Verbatim}

\subsection*{関連コマンド}
\hyperref[sect:mhashsum]{mhashsum} : 集計キーを事前に並べ替えなくても計算できる。

\hyperref[sect:mavg]{mavg} : 平均バージョン。

\hyperref[sect:mstats]{mstats} : その他の多様な統計量を求めるのであればこれ。

%\end{document}
