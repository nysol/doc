
%\documentclass[a4paper]{jsbook}
%\usepackage{mcmd_jp}
%\begin{document}

\section{mproduct 参照ファイルの直積結合\label{sect:mproduct}}
\index{mproduct@mproduct}
入力データ1行に対して、\verb|m=|パラメータで指定した参照データの
\verb|f=|パラメータで指定した項目全行を結合する。

\subsection*{書式}
\verb/mproduct [f=] m=|/
\hyperref[sect:option_i]{i=}
\hyperref[sect:option_o]{[o=]}
\hyperref[sect:option_bufcount]{[bufcount=]} 
\hyperref[sect:option_assert_diffSize]{[-assert\_diffSize]}
\hyperref[sect:option_assert_nullin]{[-assert\_nullin]}
\hyperref[sect:option_nfn]{[-nfn]} 
\hyperref[sect:option_nfno]{[-nfno]}  
\hyperref[sect:option_x]{[-x]}
\hyperref[sect:option_option_tmppath]{[tmpPath=]}
\hyperref[sect:option_precision]{[precision=]}
\verb|[-params]|
\verb|[--help]|
\verb|[--helpl]|
\verb|[--version]|\\

\subsection*{パラメータ}
\begin{table}[htbp]
%\begin{center}
{\small
\begin{tabular}{ll}
\verb|i=|    & 入力ファイル名を指定する。\\
\verb|o=|    & 出力ファイル名を指定する。\\
\verb|f=|    & 結合する参照ファイル上の項目名リスト(複数項目指定可)。\\
             & 省略するとキー項目を除いた全ての項目が結合される。\\
\verb|m=|    & 参照ファイル名を指定する。\\
             & このパラメータが省略された時には標準入力が用いられる。(\verb|i=|指定ありの場合)\\
\verb|bufcount=| & バッファのサイズ数を指定する。 \\
\end{tabular} 
}
\end{table} 

\subsection*{利用例}
\subsubsection*{Example 1: Basic Example}

Combine the \verb|date| column from reference file to the \verb|customer| column from the input file.


\begin{Verbatim}[baselinestretch=0.7,frame=single]
$ more dat1.csv
customer
A
B
$ more ref1.csv
date
20090101
20090201
20090301
$ mproduct f=date m=ref1.csv i=dat1.csv o=rsl1.csv
#END# kgproduct f=date i=dat1.csv m=ref1.csv o=rsl1.csv
$ more rsl1.csv
customer,date
A,20090101
A,20090201
A,20090301
B,20090101
B,20090201
B,20090301
\end{Verbatim}

\subsection*{関連コマンド}

\hyperref[sect:mnjoin]{mnjoin} : 結合キーを指定しての\verb|mproduct|のような結合を行う。

%\end{document}
