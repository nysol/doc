
%\documentclass[a4paper]{jsbook}
%\usepackage{mcmd_jp}
%\begin{document}

\section{mavg 項目値の平均\label{sect:mavg}}
\index{mavg@mavg}
\verb|f=|パラメータで指定した項目の平均値を計算する。
\verb|(注)|k=とf=パラメータで指定した項目以外については、どの行が出力されるか>は不定であることに注意してください。\\

\subsection*{書式}
\verb|mavg f= [k=] [-n]|
\hyperref[sect:option_i]{[i=]}
\hyperref[sect:option_o]{[o=]}
\hyperref[sect:option_assert_diffSize]{[-assert\_diffSize]}
\hyperref[sect:option_assert_nullkey]{[-assert\_nullkey]}
\hyperref[sect:option_assert_nullin]{[-assert\_nullin]}
\hyperref[sect:option_assert_nullout]{[-assert\_nullout]}
\hyperref[sect:option_nfn]{[-nfn]} 
\hyperref[sect:option_nfno]{[-nfno]}  
\hyperref[sect:option_x]{[-x]}
\hyperref[sect:option_q]{[-q]}
\hyperref[sect:option_option_tmppath]{[tmpPath=]}
\hyperref[sect:option_precision]{[precision=]}
\verb|[-params]|
\verb|[--help]|
\verb|[--helpl]|
\verb|[--version]|\\

\subsection*{パラメータ}
\begin{table}[htbp]
%\begin{center}
{\small
\begin{tabular}{ll}
\verb|i=|    & 入力ファイル名を指定する。\\
\verb|o=|    & 出力ファイル名を指定する。\\
\verb|f=|    & ここで指定した項目(複数項目指定可)の値が集計される。\\
             & :(コロン)で新項目名を指定可能。例)\verb|f=|数量:数量平均\\
\verb|k=|    & 集計の単位となる項目(複数項目指定可)名リストを指定する。\\
\verb|-n|    & NULL値が1つでも含まれていると結果もNULL値とする。\\
\end{tabular} 
}
\end{table} 


\subsection*{利用例}
\subsubsection*{例1: 基本例}

「顧客」項目を単位に「数量」と「金額」項目の平均値を計算し、「数量平均」と「金額平均」という項目名で出力する。


\begin{Verbatim}[baselinestretch=0.7,frame=single]
$ more dat1.csv
顧客,数量,金額
A,1,5
A,2,20
B,1,15
B,,10
B,5,20
$ mavg k=顧客 f=数量:数量平均,金額:金額平均 i=dat1.csv o=rsl1.csv
#END# kgavg f=数量:数量平均,金額:金額平均 i=dat1.csv k=顧客 o=rsl1.csv
$ more rsl1.csv
顧客%0,数量平均,金額平均
A,1.5,12.5
B,3,15
\end{Verbatim}
\subsubsection*{例2: NULL値がある場合の出力}

「顧客」項目を単位に「数量」と「金額」項目の平均値を計算し、「数量平均」と「金額平均」という項目名で出力する。
\verb|-n|オプションを指定することで、NULL値が含まれている場合は、結果もNULL値として出力する。


\begin{Verbatim}[baselinestretch=0.7,frame=single]
$ mavg k=顧客 f=数量:数量平均,金額:金額平均 -n i=dat1.csv o=rsl2.csv
#END# kgavg -n f=数量:数量平均,金額:金額平均 i=dat1.csv k=顧客 o=rsl2.csv
$ more rsl2.csv
顧客%0,数量平均,金額平均
A,1.5,12.5
B,,15
\end{Verbatim}
\subsubsection*{例3: 顧客項目を単位としない例}

「数量」と「金額」項目の平均値を計算し、「数量平均」と「金額平均」という項目名で出力する。


\begin{Verbatim}[baselinestretch=0.7,frame=single]
$ mavg f=数量:数量平均,金額:金額平均 i=dat1.csv o=rsl3.csv
#END# kgavg f=数量:数量平均,金額:金額平均 i=dat1.csv o=rsl3.csv
$ more rsl3.csv
顧客,数量平均,金額平均
B,2.25,14
\end{Verbatim}

\subsection*{関連コマンド}

\hyperref[sect:mhashavg]{mhashavg} : 集計キーを事前に並べ替えなくても計算できる。

\hyperref[sect:msum]{msum} : 合計バージョン。

\hyperref[sect:mstats]{mstats} : その他の多様な統計量を求めるのであればこれ。

%\end{document}
