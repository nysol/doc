
%\documentclass[a4paper]{jsbook}
%\usepackage{mcmd_jp}
%\begin{document}

\section{mvuniq ベクトル要素の単一化\label{sect:mvuniq}}
\index{mvuniq@mvuniq}

ベクトル型の項目の要素を単一化する。
内部的にはtree構造を利用して単一化をしているので、
出力される要素の順序は文字列昇順に並び変わる。

一方で、\verb|-n|オプションを指定すると、
ベクトルを系列として考え、
要素を先頭から順番に走査し、互いに隣接した要素のみを単一化し出力する。
%よって、\verb|mvsort|コマンドにてソーティングさせた後に、
%\verb|-n|オプションを指定して本コマンドを利用するとすよって、ベクトル内の全要素を対象に単一化したい場合は事前に
%ておく必要がある。

\if0 #no help# following sentences will not apear on the help document. \fi
典型的な例を表\ref{tbl:mvuniq_out1},\ref{tbl:mvuniq_out2}に示す。
表\ref{tbl:mvuniq_out1}では、全ての要素が単一化されているのが分かる。
一方で、\verb|-n|オプションを指定して実行すると、
表\ref{tbl:mvuniq_out2}の3行目に見られるように、
互いに隣接する\verb|b|のみが単一化される。
%の1行目では、\verb|a|が2つ連続しているので
%入力データ(表\ref{tbl:mvuniq_input})の1行目では、\verb|a|が2つ連続しているので
%1つの\verb|a|に単一化される。
%3行目は、\verb|a|が2つ出現するが連続していないので単一化されない。
%一方で\verb|b|は連続しているので単一化される。

\begin{table}[htbp]
\begin{center}
\begin{tabular}{ccc}

\begin{minipage}{0.28\hsize}
\begin{center}
\caption{入力データ\label{tbl:mvuniq_input}}
\verb|in.csv| \\
{\small
\begin{tabular}{cll}
\hline
no&items \\
\hline
1&b a a \\
2&a a b b b \\
3&a b b a \\
4&a b c \\
\hline

\end{tabular}
}
\end{center}
\end{minipage}

\begin{minipage}{0.33\hsize}
\begin{center}
\caption{基本的な例\label{tbl:mvuniq_out1}}
\verb|vf=items i=in.csv| \\
{\small
\begin{tabular}{ll}
\hline
no&items \\
\hline
1&a b \\
2&a b \\
3&a b \\
4&a b c \\
\hline
\end{tabular}
}
\end{center}
\end{minipage}

\begin{minipage}{0.33\hsize}
\begin{center}
\caption{ベクトルを系列と考え、互いに隣り合う同じ要素のみを単一化する例\label{tbl:mvuniq_out2}}
\verb|vf=items -n i=in.csv| \\
{\small
\begin{tabular}{ll}
\hline
no&items \\
\hline
1&b a \\
2&a b \\
3&a b a \\
4&a b c \\
\hline
\end{tabular}
}
\end{center}
\end{minipage}

\end{tabular}
\end{center}
\end{table}

\subsection*{書式}
\verb|mvuniq vf= [-A] [-n]|
\hyperref[sect:option_i]{[i=]}
\hyperref[sect:option_o]{[o=]}
\hyperref[sect:option_delim]{[delim=]} 
\hyperref[sect:option_assert_diffSize]{[-assert\_diffSize]}
\hyperref[sect:option_assert_nullin]{[-assert\_nullin]}
\hyperref[sect:option_assert_nullout]{[-assert\_nullout]}
\hyperref[sect:option_nfn]{[-nfn]} 
\hyperref[sect:option_nfno]{[-nfno]}  
\hyperref[sect:option_x]{[-x]}
\hyperref[sect:option_option_tmppath]{[tmpPath=]}
\hyperref[sect:option_precision]{[precision=]}
\verb|[-params]|
\verb|[--help]|
\verb|[--helpl]|
\verb|[--version]|\\

\begin{table}[htbp]
%\begin{center}
{\small
\begin{tabular}{ll}
\verb|i=|    & 入力ファイル名を指定する。\\
\verb|o=|    & 出力ファイル名を指定する。\\
\verb|vf=| & 単一化する対象項目名を指定する。複数項目指定可能。 \\
           & 結果の項目名を変更したいときは、:(コロン)に続けて新項目名を指定する。\\
\verb|-A|  & \verb|vf=|で:(コロン)に続けて指定した項目名で、新たな項目が追加される。\\
           & なお\verb|-A|オプションを指定した場合、\verb|vf=|パラメータで指定するすべての\\
           & 項目に新項目名を指定しなければならない。\\
\verb|-n|  & ベクトルを系列と考え隣接する同一要素のみ単一化する \\
\verb|delim=| & ベクトル型データの区切り文字を指定する。\\
\end{tabular}
}
\end{table} 

\subsection*{利用例}
\subsubsection*{Example 1: Merges vector elements in multiple fields}



\begin{Verbatim}[baselinestretch=0.7,frame=single]
$ more dat1.csv
items1,items2
b a c,1 1
c c,2 2 3
e a a,3 1
$ mvuniq vf=items1,items2 i=dat1.csv o=rsl1.csv
#END# kgvuniq i=dat1.csv o=rsl1.csv vf=items1,items2
$ more rsl1.csv
items1,items2
a b c,1
c,2 3
a e,1 3
\end{Verbatim}

\subsection*{関連コマンド}

%\end{document}
