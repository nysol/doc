
%\documentclass[a4paper]{jsbook}
%\usepackage{mcmd_jp}
%\begin{document}

\section{mmvstats 移動窓の統計量の計算\label{sect:mmvstats}}
\index{mmvstats@mmvstats}

移動窓を設定し、各種統計量(1変量)を計算する。
\hyperref[sect:mstats]{mstats}コマンドの移動窓バージョンとして考えればよい。

\subsection*{書式}
\verb|mmvstats [s=] [k=] f= [t=] c= [skip=] -n |
\hyperref[sect:option_i]{[i=]}
\hyperref[sect:option_o]{[o=]}
\hyperref[sect:option_assert_diffSize]{[-assert\_diffSize]}
\hyperref[sect:option_assert_nullkey]{[-assert\_nullkey]}
\hyperref[sect:option_assert_nullin]{[-assert\_nullin]}
\hyperref[sect:option_assert_nullout]{[-assert\_nullout]}
\hyperref[sect:option_nfn]{[-nfn]} 
\hyperref[sect:option_nfno]{[-nfno]}  
\hyperref[sect:option_x]{[-x]}
\hyperref[sect:option_q]{[-q]}
\hyperref[sect:option_option_tmppath]{[tmpPath=]}
\hyperref[sect:option_precision]{[precision=]}
\verb|[-params]|
\verb|[--help]|
\verb|[--helpl]|
\verb|[--version]|\\

\subsection*{パラメータ}
\begin{table}[htbp]
%\begin{center}
{\small
\begin{tabular}{ll}
\verb|i=|    & 入力ファイル名を指定する。\\
\verb|o=|    & 出力ファイル名を指定する。\\
\verb|s=|    & ここで指定した項目(複数項目指定可)で並べ替えられた後、各種統計量が計算される。\\
             & \verb|-q|オプションを指定しないとき、\verb|s=|パラメータは必須。\\
\verb|k=|    & ここで指定された項目(複数項目指定可)を単位として集計する。\\
\verb|f=|    & 集計項目名リスト(複数項目指定可)を指定する。\\
\verb|t=|    & 期間数を1以上の整数で指定する。 \\
\verb|c=|    & 統計量(以下のリストから一つだけ指定可)\\
             & \verb/sum|mean|devsq|var|uvar|sd|usd|cv|min|/\\
             & \verb/|max|range|skew|uskew|kurt|ukurt/\\
             & 詳細な定義は\hyperref[sect:mstats]{mstats}コマンドを参照のこと。\\
\verb|skip=| & 出力を抑制する最初の行数\\
\verb|-n| & 期間内にNULL値が1つでも含まれていると結果もNULL値とする。\\
\end{tabular} 
}
\end{table} 


\subsection*{利用例}
\subsubsection*{Example 1: Basic Example}

Calculate sum of sliding window.
The first row is not printed as there is less than the required nubmer of intervals for computation.


\begin{Verbatim}[baselinestretch=0.7,frame=single]
$ more dat1.csv
id,value
1,5
2,1
3,3
4,4
5,4
6,6
7,1
8,4
9,7
$ mmvstats s=id f=value t=2 c=sum i=dat1.csv o=rsl1.csv
#END# kgmvstats c=sum f=value i=dat1.csv o=rsl1.csv s=id t=2
$ more rsl1.csv
id%0,value
2,6
3,4
4,7
5,8
6,10
7,7
8,5
9,11
\end{Verbatim}

\subsection*{関連コマンド}
\hyperref[sect:mmvavg] {mmvavg} : 移動平均に限定した計算を行う。

\hyperref[sect:mwindow] {mwindow} : 動窓のデータを作成するので、そのデータを使えば\verb|mmvstats|で計算できない統計量も計算可能。

\hyperref[sect:mmvsim] {mmvsim} : 移動窓の類似度(2変量統計量)の計算を行う。

%\end{document}
