
%\begin{document}

\subsection{precision= 有効桁数\label{sect:option_precision}}
内部的にはC言語におけるsprintfの書式「\verb|"%.|$n$\verb|g"|」を用いている。
この書式は、データの桁数と指定した有効桁数によって、標準標記(整数部.小数部: ex. \verb|123.456|)と、
指数表記(仮数部e$\pm$指数部: ex. \verb|1.23456e+02|)を切り替える。
切り替えの基準であるが、データを指数表記で表したときに、指数部が指定の有効桁数を超えるか、
もしくは-5以下の場合(すなわち、小数点以下に0が4つ以上続く場合)に指数表記を採用する。

$n$は1〜16の整数が指定可能で、デフォルトは10である。
$n<1$の場合は$n=1$にセットされ、$n>16$の場合は$n=16$にセットされる。

また、環境変数\verb|KG_Precision|を設定することでも有効桁数を変更できる。
ただし、環境変数を変更すると、それ以降に実行するコマンド全てに反映されることに注意する。

\subsection*{利用例}
\subsubsection*{例1: 基本例}

id=1は指数表現で1.2345678e+08であり、指数部が有効桁数6を超えているので指数表記となり、仮数部の有効桁数が6となっている。
id=2は指数表現で1.23456789e+03であり、指数部が有効桁数7を超えていないので標準標記となり、整数部+小数部の桁数が6となっている。
id=4は指数表現で1.23456789e-04であり、指数部が-4未満ではないので標準標記となり、有効桁数が6となっている。
id=5は指数表現で1.23456789e-05であり、指数部が-4未満となるため指数表記となり、仮数部の有効桁数が6となっている。


\begin{Verbatim}[baselinestretch=0.7,frame=single]
$ more dat1.csv
id,val
1,123456789
2,1234.56789
3,0.123456789
4,0.000123456789
5,0.0000123456789
$ mcal c='${val}' a=result precision=6 i=dat1.csv o=rsl1.csv
#END# kgcal a=result c=${val} i=dat1.csv o=rsl1.csv precision=6
$ more rsl1.csv
id,val,result
1,123456789,1.23457e+08
2,1234.56789,1234.57
3,0.123456789,0.123457
4,0.000123456789,0.000123457
5,0.0000123456789,1.23457e-05
\end{Verbatim}
\subsubsection*{例2: presicion=2の場合}



\begin{Verbatim}[baselinestretch=0.7,frame=single]
$ mcal c='${val}' a=result precision=2 i=dat1.csv o=rsl2.csv
#END# kgcal a=result c=${val} i=dat1.csv o=rsl2.csv precision=2
$ more rsl2.csv
id,val,result
1,123456789,1.2e+08
2,1234.56789,1.2e+03
3,0.123456789,0.12
4,0.000123456789,0.00012
5,0.0000123456789,1.2e-05
\end{Verbatim}
\subsubsection*{例3: 環境変数による指定}

環境変数によって設定すると、それ以降全てのコマンドがその設定値を使う。


\begin{Verbatim}[baselinestretch=0.7,frame=single]
$ export KG_Precision=4
$ mcal c='${val}' a=result i=dat1.csv o=rsl3.csv
#END# kgcal a=result c=${val} i=dat1.csv o=rsl3.csv
$ more rsl3.csv
id,val,result
1,123456789,1.235e+08
2,1234.56789,1235
3,0.123456789,0.1235
4,0.000123456789,0.0001235
5,0.0000123456789,1.235e-05
\end{Verbatim}


\subsubsection*{対応コマンド}
\hyperref[sect:msum]{msum},
\hyperref[sect:mcal]{mcal}などの実数値の演算を伴うコマンド全てで利用できる。

%\end{document}

