%\documentclass[a4paper]{jsbook}
%\usepackage{mcmd_jp}
%\begin{document}

\section{m2cross 1対Nのクロス集計\label{sect:mcross}}
\index{m2cross@m2cross}
1対Nのクロス集計を行う。
\verb|s=|を指定した場合には項目の値が項目名となるように横に展開され、
\verb|f=|で指定した項目がセルとして出力される。
\verb|a=|を指定した場合(2項目指定)には
指定した値が項目名となり、
1項目に\verb|f=|で指定した項目名が、
2項目に\verb|f=|で指定した項目値がそれぞれ縦展開される
\verb|k=|が指定されていた場合には、
指定した値が行idとなり、id単位で展開される。

\subsection*{書式}
\verb/m2cross f= s=|a= [k=] [v=] [fixfld=]/
\hyperref[sect:option_i]{[i=]}
\hyperref[sect:option_o]{[o=]}
\hyperref[sect:option_assert_diffSize]{[-assert\_diffSize]}
\hyperref[sect:option_assert_nullkey]{[-assert\_nullkey]}
\hyperref[sect:option_assert_nullin]{[-assert\_nullin]}
\hyperref[sect:option_assert_nullout]{[-assert\_nullout]}
\hyperref[sect:option_nfn]{[-nfn]} 
\hyperref[sect:option_nfno]{[-nfno]}  
\hyperref[sect:option_x]{[-x]}
\hyperref[sect:option_q]{[-q]}
\hyperref[sect:option_option_tmppath]{[tmpPath=]}
\hyperref[sect:option_precision]{[precision=]}
\verb|[-params]|
\verb|[--help]|
\verb|[--helpl]|
\verb|[--version]|\\

\subsection*{パラメータ}
\begin{table}[htbp]

{\small
\begin{tabular}{ll}
\verb|i=|    & 入力ファイル名を指定する。\\
\verb|o=|    & 出力ファイル名を指定する。\\
\verb|fixfld=| & 横に展開する際、データがない場合に追加する項目名を指定する。\\
\verb|f=|    & ここで指定された項目の値がセルの値として出力される。\\
             & a=を使用するときのみ複数項目指定可。\\
\verb|s=|    & 列項目名に展開する項目を指定する。\\
             & ここで指定された項目の値が項目名として出力される。\\
\verb|a=|    & 2項目指定する。\\
             & 1項目目に\verb|f=|で指定した項目名がデータとして展開される項目名を指定する。\\
             & 2項目目に\verb|f=|で指定した項目値の項目名を指定する\\
\verb|k=|    & キー項目名リスト\\
             & ここで指定した項目を単位に展開をおこなう。\\
\verb|v=|    & NULL値置換文字列\\
             & NULL値があった場合、\verb|v=|パラメータで指定する置換文字列により、項目の値を置換する。\\
\end{tabular} 
}
\end{table} 

\subsection*{利用例}
\subsubsection*{例1: 基本例}

\verb|item|項目を単位に\verb|date|項目を横に展開し、
\verb|quantity|項目を出力する。


\begin{Verbatim}[baselinestretch=0.7,frame=single]
$ more dat1.csv
item,date,quantity
A,20081201,1
A,20081202,2
A,20081203,3
B,20081201,4
B,20081203,5
$ m2cross k=item f=quantity s=date i=dat1.csv o=rsl1.csv
#END# kg2cross f=quantity i=dat1.csv k=item o=rsl1.csv s=date
$ more rsl1.csv
item%0,20081201,20081202,20081203
A,1,2,3
B,4,,5
\end{Verbatim}
\subsubsection*{例2: 元の入力データに戻す例}

例1の出力結果を元に戻すには、同じく\verb|m2cross|を以下のよう用いればよい。


\begin{Verbatim}[baselinestretch=0.7,frame=single]
$ more rsl1.csv
item%0,20081201,20081202,20081203
A,1,2,3
B,4,,5
$ m2cross f=2008* a=date,quantity i=rsl1.csv o=rsl2.csv
#END# kg2cross a=date,quantity f=2008* i=rsl1.csv o=rsl2.csv
$ more rsl2.csv
item%0,date,quantity
A,20081201,1
A,20081202,2
A,20081203,3
B,20081201,4
B,20081202,
B,20081203,5
\end{Verbatim}
\subsubsection*{例3: 並びを逆順する例}

横に展開する項目名の並びを逆順にする。


\begin{Verbatim}[baselinestretch=0.7,frame=single]
$ m2cross k=item f=quantity s=date%r i=dat1.csv o=rsl3.csv
#END# kg2cross f=quantity i=dat1.csv k=item o=rsl3.csv s=date%r
$ more rsl3.csv
item%0,20081203,20081202,20081201
A,3,2,1
B,5,,4
\end{Verbatim}
\subsubsection*{例4: データがない場合に項目を追加する例}

横に展開する際に、データがない場合に項目を追加する"


\begin{Verbatim}[baselinestretch=0.7,frame=single]
$ more dat2.csv
item,week,quantity
A,Monday,1
A,Tuesday,2
A,Wednesday,3
B,Thursday,4
B,Friday,5
$ m2cross k=item f=quantity s=week i=dat2.csv fixfld=Sunday,Monday,Tuesday,Wednesday,Thursday,Friday,Saturday o=rsl4.csv
#END# kg2cross f=quantity fixfld=Sunday,Monday,Tuesday,Wednesday,Thursday,Friday,Saturday i=dat2.csv k=item o=rsl4.csv s=week
$ more rsl4.csv
item%0,Friday,Monday,Saturday,Sunday,Thursday,Tuesday,Wednesday
A,,1,,,,2,3
B,5,,,,4,,
\end{Verbatim}


\subsection*{関連コマンド}
\hyperref[sect:mcross]{mcross} : イメージは同じだが、\verb|mcross|はN対Nクロス集計として出力する。\\

%\end{document}
