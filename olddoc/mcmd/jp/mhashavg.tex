
%\documentclass[a4paper]{jsbook}
%\usepackage{mcmd_jp}
%\begin{document}

\section{mhashavg ハッシュ法による項目値の平均\label{sect:mhashavg}}
\index{mhashavg@mhashavg}
hash法を使って\verb|k=|パラメータで指定した項目を単位にして、\verb|f=|パラメータで指定した項目値の平均を計算する。
\hyperref[sect:avg]{mavg}との違いは、キー項目の並べ変えが必要ないため、その分処理速度が速い。
ただし、キーのサイズ(キー項目のとる値の種類数)が多い場合は処理速度が遅くなる。

\subsection*{書式}
\verb|mhashavg f= [hs=] [k=] [-n] |
\hyperref[sect:option_i]{[i=]}
\hyperref[sect:option_o]{[o=]}
\hyperref[sect:option_assert_diffSize]{[-assert\_diffSize]}
\hyperref[sect:option_assert_nullkey]{[-assert\_nullkey]}
\hyperref[sect:option_assert_nullin]{[-assert\_nullin]}
\hyperref[sect:option_assert_nullout]{[-assert\_nullout]}
\hyperref[sect:option_nfn]{[-nfn]} 
\hyperref[sect:option_nfno]{[-nfno]}  
\hyperref[sect:option_x]{[-x]}
\hyperref[sect:option_option_tmppath]{[tmpPath=]}
\hyperref[sect:option_precision]{[precision=]}
\verb|[-params]|
\verb|[--help]|
\verb|[--helpl]|
\verb|[--version]|\\

\subsection*{パラメータ}
\begin{table}[htbp]
%\begin{center}
{\small
\begin{tabular}{ll}
\verb|i=|    & 入力ファイル名を指定する。\\
\verb|o=|    & 出力ファイル名を指定する。\\
\verb|f=|    & ここで指定された項目(複数項目指定可)の平均が計算される。\\
             & :(コロン)で新項目名を指定可能。例)\verb|f=|数量:数量平均\\
\verb|k=|    & ここで指定された項目をキーとして集計する(複数項目指定可)。\\
%             & \hyperref[sect:option_k]{集計キーブレイク処理}はしないので、事前に並べ替える必要はない。\\
\verb|hs=|   & ハッシュサイズ【デフォルト値:199999】 \\
             & ハッシュサイズを指定する。【デフォルト値:199999】\\
             & 詳細に関しては\hyperref[sect:mhashsum]{mhashsum}参照 \\
\verb|-n|    & NULL値が1つでも含まれていると結果もNULL値とする。\\
\end{tabular} 
}
\end{table} 

\subsection*{利用例}
\subsubsection*{Example 1: Basic Example}

Calculate the average \verb|Quantity| and average \verb|Amount| for each \verb|Customer|.


\begin{Verbatim}[baselinestretch=0.7,frame=single]
$ more dat1.csv
Customer,Quantity,Amount
A,1,
B,,15
A,2,20
B,3,10
B,1,20
$ mhashavg k=Customer f=Quantity,Amount i=dat1.csv o=rsl1.csv
#END# kghashavg f=Quantity,Amount i=dat1.csv k=Customer o=rsl1.csv
$ more rsl1.csv
Customer,Quantity,Amount
A,1.5,20
B,2,15
\end{Verbatim}
\subsubsection*{Example 2: NULL value in output}

The output returns NULL if there NULL value is present in \verb|Quantity| and \verb|Amount|. Use \verb|-n| option to print the null value.


\begin{Verbatim}[baselinestretch=0.7,frame=single]
$ mhashavg k=Customer f=Quantity,Amount -n i=dat1.csv o=rsl2.csv
#END# kghashavg -n f=Quantity,Amount i=dat1.csv k=Customer o=rsl2.csv
$ more rsl2.csv
Customer,Quantity,Amount
A,1.5,
B,,15
\end{Verbatim}

\subsection*{備考}
動作速度に関しては、\hyperref[sect:mhashsum]{mhashsum}のページにあるベンチマークの項を参照のこと。

\subsection*{関連コマンド}
\hyperref[sect:mavg] {mavg} : 同じ機能をもつコマンドだが、内部的にキー項目の並べ替えを行う。

\hyperref[sect:mhashsum]{mhashsum} : 同じくハッシュ法を用いた合計計算。

%\end{document}
