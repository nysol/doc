
%\documentclass[a4paper]{jsbook}
%\usepackage{mcmd_jp}
%\begin{document}

\section{mcount 行数カウント\label{sect:mcount}}
\index{mcount@mcount}
行数をカウントし、\verb|a=|パラメータで指定した項目名で出力する。
\verb|k=|を指定すると、集計キー毎の件数をカウントし、
\verb|k=|を指定しなければ、全行数がカウントされる。

\subsection*{書式}
\verb|mcount a= [k=]|
\hyperref[sect:option_i]{[i=]}
\hyperref[sect:option_o]{[o=]}
\hyperref[sect:option_assert_diffSize]{[-assert\_diffSize]}
\hyperref[sect:option_assert_nullkey]{[-assert\_nullkey]}
\hyperref[sect:option_nfn]{[-nfn]} 
\hyperref[sect:option_nfno]{[-nfno]}  
\hyperref[sect:option_x]{[-x]}
\hyperref[sect:option_q]{[-q]}
\hyperref[sect:option_option_tmppath]{[tmpPath=]}
\hyperref[sect:option_precision]{[precision=]}
\verb|[-params]|
\verb|[--help]|
\verb|[--helpl]|
\verb|[--version]|\\

\subsection*{パラメータ}
\begin{table}[htbp]
%\begin{center}
{\small
\begin{tabular}{ll}
\verb|i=|    & 入力ファイル名を指定する。\\
\verb|o=|    & 出力ファイル名を指定する。\\
\verb|a=|    & 新たに追加される項目の名前を指定する。\\
             & \verb|nfn|オプション使用時は、必須ではない。\\
\verb|k=|    & キー項目名リスト(複数項目指定可)\\
             & カウントの単位となる項目名リスト。\\
\end{tabular} 
}
\end{table} 

\subsection*{利用例}
\subsubsection*{Example 1: Basic Example}

Count the number of rows by date, and save the results in a new column \verb|count|.


\begin{Verbatim}[baselinestretch=0.7,frame=single]
$ more dat1.csv
date
20090109
20090109
20090109
20090110
20090110
$ mcount k=date a=count i=dat1.csv o=rsl1.csv
#END# kgcount a=count i=dat1.csv k=date o=rsl1.csv
$ more rsl1.csv
date%0,count
20090109,3
20090110,2
\end{Verbatim}
\subsubsection*{Example 2: Count without aggregate key}

Count the number of rows without specifying the aggregate key.


\begin{Verbatim}[baselinestretch=0.7,frame=single]
$ mcount a=count i=dat1.csv o=rsl2.csv
#END# kgcount a=count i=dat1.csv o=rsl2.csv
$ more rsl2.csv
date,count
20090110,5
\end{Verbatim}


\subsection*{関連コマンド}
\hyperref[sect:mstats]{mstats} : \verb|c=count|を指定することで、NULL値でないデータ件数をカウントできる。

%\end{document}
