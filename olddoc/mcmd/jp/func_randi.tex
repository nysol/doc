
%\begin{document}

\section{randi 整数一様乱数\label{sect:randi}}
\index{randi@randi}

書式: randi(最小値, 最大値 [, 乱数の種])

最小値〜最大値の範囲で整数乱数を生成する。
単独で利用するとその結果はmrandコマンドを\verb|-int|オプションを指定して実行した結果に等しい。

同じ乱数の種を指定すれば、同じ乱数系列となる。
指定可能な乱数の種の範囲は-2147483648〜2147483647である。
乱数の種を省略すると、
時刻(1/1000秒単位)に応じた異なる乱数の種が利用される。

乱数の生成にはメルセンヌ・ツイスター法を利用している
(\href{http://www.math.sci.hiroshima-u.ac.jp/~m-mat/MT/emt.html}{原作者のページ}
, \href{http://www.boost.org/doc/libs/1_54_0/doc/html/boost_random.html}{boostライブラリ})。

\subsection*{利用例}

\subsubsection*{例1: 基本例}

100から999(3桁整数900種類)の整数乱数を生成する。
乱数の種を指定しているので、何度実行しても同じ乱数系列が生成される。

\begin{Verbatim}[baselinestretch=0.7,frame=single]
$ cat dat1.csv
id
1
2
3
4

$ mcal c='randi(100,999,1)' a=rsl i=dat1.csv o=rsl1.csv
#END# kgcal a=rsl c=randi(100,999,1) i=dat1.csv o=rsl1.csv

$ cat rsl1.csv
id,rsl
1,475
2,997
3,748
4,939
\end{Verbatim}

\subsubsection*{例2: 0,1の整数乱数}

0と1の2種類の整数乱数を生成する。
乱数の種を指定していないので、実行の度に異なる乱数系列が生成される。

\begin{Verbatim}[baselinestretch=0.7,frame=single]
$ mcal c='randi(0,1)' a=rsl i=dat1.csv o=rsl2.csv
#END# kgcal a=rsl c=randi(0,1) i=dat1.csv o=rsl2.csv

$ cat rsl2.csv
id,rsl
1,0
2,1
3,0
4,1
\end{Verbatim}


%\end{document}

