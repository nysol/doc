
%\documentclass[a4paper]{jsbook}
%\usepackage{mcmd_jp}
%\begin{document}

\section{mcommon 参照ファイルによる行選択\label{sect:mcommon}}
\index{mcommon@mcommon}
\verb|k=|パラメータで指定した入力ファイルの項目値と\verb|m=|パラメータで指定した参照ファイルの項目値を比較し、同じ値を持つ入力ファイルの行を選択する。

\subsection*{書式}
\verb/mcommon  k= [K=] [u=] [-r] m=|/
\hyperref[sect:option_i]{i=}
\hyperref[sect:option_o]{[o=]}
\hyperref[sect:option_assert_diffSize]{[-assert\_diffSize]}
\hyperref[sect:option_assert_nullkey]{[-assert\_nullkey]}
\hyperref[sect:option_nfn]{[-nfn]} 
\hyperref[sect:option_nfno]{[-nfno]}  
\hyperref[sect:option_x]{[-x]}
\hyperref[sect:option_q]{[-q]}
\hyperref[sect:option_option_tmppath]{[tmpPath=]}
\hyperref[sect:option_precision]{[precision=]}
\verb|[-params]|
\verb|[--help]|
\verb|[--helpl]|
\verb|[--version]|\\

\subsection*{パラメータ}
\begin{table}[htbp]
%\begin{center}
{\small
\begin{tabular}{ll}
\verb|i=|    & 入力ファイル名を指定する。\\
\verb|o=|    & 出力ファイル名を指定する。\\
\verb|k=|    & 入力データ上の突き合わせる項目名リスト(複数項目指定可)\\
             & ここで指定した入力データの項目と\verb|K=|パラメータで指定された参照データの項目が同じ行が選択される。\\
             & 同じ値が複数行連続していてもよい。\\
\verb|m=|    & 参照ファイル名を指定する。\\
             & またこのパラメータが省略された時には標準入力が用いられる。(\verb|i=|指定ありの場合)\\
\verb|K=|    & 参照データ上の突き合わせる項目名リスト(複数項目指定可)\\
             & ここで指定した参照データの項目と\verb|k=|パラメータで指定された入力データの項目が同じ行が選択される。\\
             & 参照データ上に\verb|k=|パラメータで指定した入力データ上の項目と同名の項目が存在する場合は指定する必要はない。\\
             & 同じ値が複数行連続していてもよい。\\
\verb|u=|    & 指定の条件に一致しない行を出力するファイル名。\\
\verb|-r|    & 条件反転\\
             & \verb|k=|パラメータで指定した入力ファイルの項目値と\\
             & \verb|m=|パラメータで指定した参照ファイルの項目値を比較し、\\
             & 同じ値を持たない入力ファイルの行を選択する。 
\end{tabular} 
}
\end{table} 

\subsection*{利用例}
\subsubsection*{例1: 基本例}

入力ファイルにある「顧客」項目と、参照ファイルにある「顧客」項目が同じ値を持つ入力ファイルの行を選択する。
それ以外のデータは\verb|oth.csv|に出力する。


\begin{Verbatim}[baselinestretch=0.7,frame=single]
$ more dat1.csv
顧客,数量
A,1
B,2
C,1
D,3
E,1
$ more ref1.csv
顧客,性別
A,女性
B,男性
E,女性
$ mcommon k=顧客 m=ref1.csv u=oth.csv i=dat1.csv o=rsl1.csv
#END# kgcommon i=dat1.csv k=顧客 m=ref1.csv o=rsl1.csv u=oth.csv
$ more rsl1.csv
顧客%0,数量
A,1
B,2
E,1
$ more oth.csv
顧客%0,数量
C,1
D,3
\end{Verbatim}
\subsubsection*{例2: 同じ値を持たない入力ファイルの行選択}

\verb|-r|オプションを付けることで、条件が逆転し、参照ファイルにない「顧客」を選択することになる。


\begin{Verbatim}[baselinestretch=0.7,frame=single]
$ mcommon k=顧客 m=ref1.csv -r i=dat1.csv o=rsl2.csv
#END# kgcommon -r i=dat1.csv k=顧客 m=ref1.csv o=rsl2.csv
$ more rsl2.csv
顧客%0,数量
C,1
D,3
\end{Verbatim}
\subsubsection*{例3: 結合キー項目名が異なる場合}

結合キーの項目名が異なる場合は、\verb|K=|で指定する。


\begin{Verbatim}[baselinestretch=0.7,frame=single]
$ more ref2.csv
顧客ID,性別
A,女性
B,男性
E,女性
$ mcommon k=顧客 K=顧客ID i=dat1.csv m=ref2.csv o=rsl3.csv
#END# kgcommon K=顧客ID i=dat1.csv k=顧客 m=ref2.csv o=rsl3.csv
$ more rsl3.csv
顧客%0,数量
A,1
B,2
E,1
\end{Verbatim}
\subsubsection*{例4: キー項目に重複行がある場合の例}

参照ファイルと入力ファイルのキー項目に重複行があっても選択可能。


\begin{Verbatim}[baselinestretch=0.7,frame=single]
$ more dat3.csv
顧客,数量
A,1
A,2
A,3
B,1
D,1
D,2
$ more ref3.csv
顧客
A
A
D
$ mcommon k=顧客 m=ref3.csv -r i=dat3.csv o=rsl4.csv
#END# kgcommon -r i=dat3.csv k=顧客 m=ref3.csv o=rsl4.csv
$ more rsl4.csv
顧客%0,数量
B,1
\end{Verbatim}

\subsection*{関連コマンド}

\hyperref[sect:mselstr]{mselstr} : 参照ファイルの結合キーの種類数が少なければこのコマンドでも対応できる。

\hyperref[sect:mnrcommon]{mnrcommon} : 参照ファイルの結合キーがユニークでなければこちらを使う。

\hyperref[sect:mjoin]{mjoin} : 選択だけでなく、項目を結合したい場合はこのコマンド。

%\end{document}
