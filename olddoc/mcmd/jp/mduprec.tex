
%\documentclass[a4paper]{jsbook}
%\usepackage{mcmd_jp}
%\begin{document}

\section{mduprec レコードの複写\label{sect:mduprec}}
\index{mduprec@mduprec}
各レコードを複写する。
複写する行数は\verb|n=|で固定値を与えるか、
もしくは\verb|f=|で指定した項目の値により与える。

\subsection*{書式}
\verb/mduprec f=|n=/
\hyperref[sect:option_i]{[i=]}
\hyperref[sect:option_o]{[o=]}
\hyperref[sect:option_assert_diffSize]{[-assert\_diffSize]}
\hyperref[sect:option_assert_nullin]{[-assert\_nullin]}
\hyperref[sect:option_nfn]{[-nfn]} 
\hyperref[sect:option_nfno]{[-nfno]}  
\hyperref[sect:option_x]{[-x]}
\hyperref[sect:option_option_tmppath]{[tmpPath=]}
\hyperref[sect:option_precision]{[precision=]}
\verb|[-params]|
\verb|[--help]|
\verb|[--helpl]|
\verb|[--version]|\\

\subsection*{パラメータ}
\begin{table}[htbp]
%\begin{center}
{\small
\begin{tabular}{ll}
\verb|i=|    & 入力ファイル名を指定する。\\
\verb|o=|    & 出力ファイル名を指定する。\\
\verb|f=|    & 複写行数をもつ項目名\\
             & ここで指定した項目の値の数分、その行を複写する。\\
\verb|n=|    & 複写行数の指定\\
             & ここで指定した値の数分、行を複写する。\\
\end{tabular} 
}
\end{table} 
\subsection*{利用例}
\subsubsection*{例1: 基本例}

「数量」項目の値の数分、データを複写し複数行のデータを生成する。
対象項目がNULL値の場合は複写しない。


\begin{Verbatim}[baselinestretch=0.7,frame=single]
$ more dat1.csv
store,val
A,2
B,
C,5
$ mduprec f=val i=dat1.csv o=rsl1.csv
#END# kgduprec f=val i=dat1.csv o=rsl1.csv
$ more rsl1.csv
store,val
A,2
A,2
C,5
C,5
C,5
C,5
C,5
\end{Verbatim}
\subsubsection*{例2: 複写行数の指定}

データを2行づつ複写した(\verb|n=2|)データを生成する。


\begin{Verbatim}[baselinestretch=0.7,frame=single]
$ mduprec n=2 i=dat1.csv o=rsl2.csv
#END# kgduprec i=dat1.csv n=2 o=rsl2.csv
$ more rsl2.csv
store,val
A,2
A,2
B,
B,
C,5
C,5
\end{Verbatim}


\subsection*{関連コマンド}
\hyperref[sect:mcount]{mcount} : \verb|mduprec|と逆の動きをする。

\hyperref[sect:mwindow]{mwindow} : 一定数のレコードをずらしながら複写する。

%\end{document}
