
%\documentclass[a4paper]{jsbook}
%\usepackage{mcmd_jp}
%\begin{document}

\section{muniq レコードの単一化\label{sect:muniq}}
\index{muniq@muniq}
値が重複した行を単一化する。

\subsection*{書式}
\verb|muniq [k=] |
\hyperref[sect:option_i]{[i=]}
\hyperref[sect:option_o]{[o=]}
\hyperref[sect:option_assert_diffSize]{[-assert\_diffSize]}
\hyperref[sect:option_assert_nullkey]{[-assert\_nullkey]}
\hyperref[sect:option_nfn]{[-nfn]} 
\hyperref[sect:option_nfno]{[-nfno]}  
\hyperref[sect:option_x]{[-x]}
\hyperref[sect:option_q]{[-q]}
\hyperref[sect:option_option_tmppath]{[tmpPath=]}
\hyperref[sect:option_precision]{[precision=]}
\verb|[-params]|
\verb|[--help]|
\verb|[--helpl]|
\verb|[--version]|\\

\subsection*{パラメータ}
\begin{table}[htbp]
%\begin{center}
{\small
\begin{tabular}{ll}
%\verb|k=|    & 行を単一化する単位となる項目名リストを指定する。【\hyperref[sect:option_k]{集計キーブレイク処理}】\\
\verb|i=|    & 入力ファイル名を指定する。\\
\verb|o=|    & 出力ファイル名を指定する。\\
\verb|k=|    & 行を単一化する単位となる項目名リストを指定する。\\
\end{tabular} 
}
\end{table} 

\subsection*{利用例}
\subsubsection*{Example 1: Basic Example}

Remove duplicate records in the \verb|date| field.


\begin{Verbatim}[baselinestretch=0.7,frame=single]
$ more dat1.csv
date,customer
20081201,A
20081202,A
20081202,B
20081202,B
20081203,C
$ muniq k=date i=dat1.csv o=rsl1.csv
#END# kguniq i=dat1.csv k=date o=rsl1.csv
$ more rsl1.csv
date%0,customer
20081201,A
20081202,B
20081203,C
\end{Verbatim}
\subsubsection*{Example 2: Delete duplicate rows in multiple columns}

Remove duplicate records based on unique values in \verb|date| and \verb|customer| field.


\begin{Verbatim}[baselinestretch=0.7,frame=single]
$ muniq k=date,customer i=dat1.csv o=rsl2.csv
#END# kguniq i=dat1.csv k=date,customer o=rsl2.csv
$ more rsl2.csv
date%0,customer%1
20081201,A
20081202,A
20081202,B
20081203,C
\end{Verbatim}

\subsection*{関連コマンド}
\hyperref[sect:mbest]{mbest} : 同一キーの中で何番目の行を選択するかを指定したい場合は\verb|mbest|コマンドを使う。

%\end{document}
