
%\documentclass[a4paper]{jsbook}
%\usepackage{mcmd_jp}
%\begin{document}

\section{mnrcommon 参照ファイルの複数範囲条件による行撰択\label{sect:mnrcommon}}
\index{mnrcommon@mnrcommon}
参照ファイルの範囲条件にマッチする入力ファイルの行を選択する。
\verb|k=|パラメータで指定した入力ファイルの項目値と\verb|K=|パラメータで指定した参照ファイルの項目値が同じ行について、
\verb|r=|でパラメータで指定した項目値が\verb|R=|パラメータで指定した2項目の値の範囲条件(項目1以上項目2未満)にマッチすれば選択する。
数値として処理したい場合は\verb|r=|パラメータの項目名のあとに\verb|%n|をつけること。

\subsection*{書式}
\verb/mnrcommon [k=] R= r= [K=] [u=] [-r] m=|/ 
\hyperref[sect:option_i]{i=}
\hyperref[sect:option_o]{[o=]}
\hyperref[sect:option_assert_diffSize]{[-assert\_diffSize]}
\hyperref[sect:option_assert_nullkey]{[-assert\_nullkey]}
\hyperref[sect:option_nfn]{[-nfn]} 
\hyperref[sect:option_nfno]{[-nfno]}  
\hyperref[sect:option_x]{[-x]}
\hyperref[sect:option_q]{[-q]}
\hyperref[sect:option_option_tmppath]{[tmpPath=]}
\hyperref[sect:option_precision]{[precision=]}
\verb|[-params]|
\verb|[--help]|
\verb|[--helpl]|
\verb|[--version]|\\

\subsection*{パラメータ}
\begin{table}[htbp]
%\begin{center}
{\small
\begin{tabular}{ll}
\verb|i=|    & 入力ファイル名を指定する。\\
\verb|o=|    & 出力ファイル名を指定する。\\
\verb|k=|    & 入力データ上の突き合わせる項目名リスト(複数項目指定可)を指定する。\\
             & ここで指定した入力データの項目と\verb|K=|パラメータで指定された参照データの項目が同じ行の項目結合が行われる。\\
\verb|m=|    & 参照ファイル名を指定する。\\
             & このパラメータが省略された時には標準入力が用いられる。(\verb|i=|指定ありの場合)\\
%\verb|R=|    & 参照ファイル上の範囲項目名(start,end)を指定する。【\hyperref[sect:option_k]{結合キーブレイク処理}】\\
\verb|R=|    & 参照ファイル上の範囲項目名(start,end)を指定する。\\
             & 第一項目のNULL値は無限小,第二項目のNULL値は無限大として扱われる。\\
%\verb|r=|    & 範囲比較される入力ファイル上の項目名を指定する。[\%{n}]【\hyperref[sect:option_k]{結合キーブレイク処理}】\\
\verb|r=|    & 範囲比較される入力ファイル上の項目名を指定する。[\%{n}]\\
             & ここで指定した参照データの項目と\verb|k=|パラメータで指定された入力データの項目が同じ行が選択される。\\
             & 数値として処理したい場合は\verb|r=|パラメータの項目名のあとに\%nをつける。\\
\verb|K=|    & 参照データ上の突き合わせる項目名リスト(複数項目指定可)\\
             & ここで指定した参照データの項目と\verb|k=|パラメータで指定された入力データの項目が同じ行の項目結合が行われる。\\
             & 参照データ上に\verb|k=|パラメータで指定した入力データ上の項目と同名の項目が存在する場合は指定する必要はない。\\
\verb|u=|    & 指定の条件に一致しない行を出力するファイル名。\\
\verb|-r|    & 条件反転\\
             & \verb|R=|パラメータで指定した行番号以外の行を選択する。\\
\end{tabular} 
}
\end{table} 

%\subsection*{並べ替え条件}
%\verb|r=,R=|の項目について事前に並べ替えておく必要がある。
%ただし、数値として範囲比較して結合するのであれば、\verb|r=,R=|で指定した何れの項目も数値昇順で並べ替えなければならない。
%\verb|k=,K=|を指定するのであれば、
%それぞれのパラメータで指定した項目リストで文字列昇順で並べ替えておく必要がある。
%例えば、パラメータを\verb|k=key K=Key r=val%n R=range i=dat.csv m=ref.csv|と指定するのであれば、
%\verb|dat.csv|データは、\verb|msortf f=key,val%n|の条件で、また
%\verb|ref.csv|データは、\verb|msortf f=Key,range%n|の条件によって並べ替えておかなければならない。

\subsection*{利用例}
\subsubsection*{例1: 基本例}

日付項目の値が\verb|20080203|で、「金額」項目の値が\verb|5|以上\verb|15|未満の行、および\verb|40|以上\verb|50|未満の行を選択する。


\begin{Verbatim}[baselinestretch=0.7,frame=single]
$ more dat1.csv
日付,金額
20080123,10
20080203,10
20080203,20
20080203,45
200804l0,50
$ more ref1.csv
日付,金額F,金額T
20080203,5,15
20080203,40,50
$ mnrcommon k=日付 m=ref1.csv R=金額F,金額T r=金額%n i=dat1.csv o=rsl1.csv
#END# kgnrcommon R=金額F,金額T i=dat1.csv k=日付 m=ref1.csv o=rsl1.csv r=金額%n
$ more rsl1.csv
日付%0,金額
20080203,10
20080203,45
\end{Verbatim}
\subsubsection*{例2: 条件反転}

\verb|-r|を付けると選択条件は反転する。


\begin{Verbatim}[baselinestretch=0.7,frame=single]
$ mnrcommon k=日付 m=ref1.csv R=金額F,金額T r=金額%n -r i=dat1.csv o=rsl2.csv
#END# kgnrcommon -r R=金額F,金額T i=dat1.csv k=日付 m=ref1.csv o=rsl2.csv r=金額%n
$ more rsl2.csv
日付%0,金額
20080123,10
20080203,20
200804l0,50
\end{Verbatim}

\subsection*{関連コマンド}
\hyperref[sect:mcommon] {mcommon} : 範囲でなく文字列マッチで選択したい場合はこのコマンドを使う。

\hyperref[sect:mnrjoin] {mnrjoin} : 選択ではなく参照ファイルの項目を結合する。

%\end{document}
