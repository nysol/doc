
%\begin{document}

\section{mnewrand 乱数データの新規生成\label{sect:mnewrand}}
\index{mnewrand@mnewrand}
0.0から1.0の範囲の実数乱数を生成する。
\verb|-int|を指定することで、整数乱数を生成することもできる。

乱数の生成にはメルセンヌ・ツイスター法を利用している
(\href{http://www.math.sci.hiroshima-u.ac.jp/~m-mat/MT/emt.html}{原作者のページ}
, \href{http://www.boost.org/doc/libs/1_54_0/doc/html/boost_random.html}{boostライブラリ})。


\subsection*{書式}
\verb|mnewrand a= [max=] [min=] [S=] [l=] [-int]|
\hyperref[sect:option_o]{[o=]}
\hyperref[sect:option_nfn]{[-nfn]} 
\hyperref[sect:option_nfno]{[-nfno]}  
\hyperref[sect:option_x]{[-x]}
\hyperref[sect:option_option_tmppath]{[tmpPath=]}
\hyperref[sect:option_precision]{[precision=]}
\verb|[-params]|
\verb|[--help]|
\verb|[--helpl]|
\verb|[--version]|\\

\subsection*{パラメータ}
\begin{table}[htbp]
%\begin{center}
{\small
\begin{tabular}{ll}
\verb|o=|      & 出力ファイル名を指定する。\\
\verb|a=|      & 新規に作成するデータの項目名を指定する。\\
               & \verb|-nfn,-nfno|オプション指定時は指定の必要はない。\\
\verb|max=|    & 乱数の最大値を指定する。【デフォルト値:INT\_MAX】\\
               & このパラメータを指定するときは\verb|-int|も指定しなければならない。\\
\verb|min=|    & 乱数の最小値を指定する。【デフォルト値:0】\\
               & このパラメータを指定するときは\verb|-int|も指定しなければならない。\\
\verb|S=|      & 乱数の種を指定する。【デフォルト値:現在時刻】\\
\verb|l=|      & 行数【デフォルト値:10】\\
               & 新規作成する乱数データの行数を指定する。\\
\verb|-int|    & 整数乱数を生成する\\
\end{tabular} 
}
\end{table} 


\subsection*{利用例}
\subsubsection*{Example 1: Basic Example}

Generate 10 rows of random integers. Use a fixed random seed so that it will always return the same sequence of random numbers.


\begin{Verbatim}[baselinestretch=0.7,frame=single]
$ mnewrand a=rand S=1 o=rsl1.csv
#END# kgnewrand S=1 a=rand o=rsl1.csv
$ more rsl1.csv
rand
0.4170219984
0.9971848081
0.7203244893
0.9325573612
0.0001143810805
0.1281244478
0.3023325677
0.9990405154
0.1467558926
0.2360889763
\end{Verbatim}
\subsubsection*{Example 2: Random Integers}

Use random seed 1 to generate 5 rows of random integers with minimum value of 10 and maximum value of 100.


\begin{Verbatim}[baselinestretch=0.7,frame=single]
$ mnewrand a=rand -int max=1000 min=0 l=5 S=1 o=rsl2.csv
#END# kgnewrand -int S=1 a=rand l=5 max=1000 min=0 o=rsl2.csv
$ more rsl2.csv
rand
417
998
721
933
0
\end{Verbatim}
\subsubsection*{Example 3: Generate Output without Header}

Specify \verb|-nfn| option to generate random number data without header.


\begin{Verbatim}[baselinestretch=0.7,frame=single]
$ mnewrand -nfn l=5 S=1 o=rsl3.csv
#END# kgnewrand -nfn S=1 l=5 o=rsl3.csv
$ more rsl3.csv
0.4170219984
0.9971848081
0.7203244893
0.9325573612
0.0001143810805
\end{Verbatim}


\subsection*{関連コマンド}
\hyperref[sect:mnewnumber] {mnewnumber} : 連番を生成する。

\hyperref[sect:mnewstr] {mnewstr} : 固定文字列を生成する。

%\end{document}
