
%\begin{document}

\section{match 検索\label{sect:match}}
\index{match@match}

書式1: match(検索文字列,$str_1,str_2,\cdots$)

書式2: matcha(検索文字列,$str_1,str_2,\cdots$)

書式3: matchs(検索文字列,$str_1,str_2,\cdots$)

書式4: matchas(検索文字列,$str_1,str_2,\cdots$)

$str_1,str_2,\cdots$から指定した検索文字列を検索し、
マッチすれば真をマッチしなければ偽を返す。

OR検索かAND検索か、そして完全一致か部分一致かにより、
表\ref{tbl:func_match_cond}に示すとおり異なる関数名を指定する。

\begin{table}[htbp]
\begin{center}
\caption{入力データ\label{tbl:func_match_cond}}
{\small
\begin{tabular}{l|ll}
\hline
& OR検索 & AND検索 \\
\hline
完全一致 & match  & matcha  \\
部分一致 & matchs & matchas \\
\hline
\end{tabular}
}
\end{center}
\end{table}

\verb|match|関数は、指定した検索文字列が、$str_1,str_2,\cdots$のいずれかに完全一致すれば真を返す。
\verb|matcha|関数は、指定した検索文字列が、$str_1,str_2,\cdots$の全てに完全一致すれば真を返す。
\verb|matchs|関数は、指定した検索文字列が、$str_1,str_2,\cdots$のいずれかに部分一致すれば真を返す。
\verb|matchas|関数は、指定した検索文字列が、$str_1,str_2,\cdots$の全てに部分一致すれば真を返す。
NULL値を含めたOR/AND論理演算の真偽値表は表\ref{tbl:mcal_and}を参照のこと。

\subsection*{利用例}
\subsubsection*{Example 1: OR exact match}

Returns true if either column \verb|f1,f2,f3| contains \verb|1|.


\begin{Verbatim}[baselinestretch=0.7,frame=single]
$ more dat1.csv
id,f1,f2,f3
1,1,1,1
2,1,0,1
3,,,
4,1,,1
$ mcal c='match("1",$s{f1},$s{f2},$s{f3})' a=rsl i=dat1.csv o=rsl1.csv
#END# kgcal a=rsl c=match("1",$s{f1},$s{f2},$s{f3}) i=dat1.csv o=rsl1.csv
$ more rsl1.csv
id,f1,f2,f3,rsl
1,1,1,1,1
2,1,0,1,1
3,,,,0
4,1,,1,1
\end{Verbatim}
\subsubsection*{Example 2: AND exact match}

Returns true if columns \verb|f1,f2,f3| contains the character \verb|"1"|.


\begin{Verbatim}[baselinestretch=0.7,frame=single]
$ mcal c='matcha("1",$s{f1},$s{f2},$s{f3})' a=rsl i=dat1.csv o=rsl2.csv
#END# kgcal a=rsl c=matcha("1",$s{f1},$s{f2},$s{f3}) i=dat1.csv o=rsl2.csv
$ more rsl2.csv
id,f1,f2,f3,rsl
1,1,1,1,1
2,1,0,1,0
3,,,,0
4,1,,1,0
\end{Verbatim}
\subsubsection*{Example 3: OR partial match}

Returns true if the character string \verb|ab| exists in either column \verb|s1,s2,s3|.


\begin{Verbatim}[baselinestretch=0.7,frame=single]
$ more dat2.csv
id,s1,s2,s3
1,ab,abx,x
2,abc,xaby,xxab
3,,,
4,#ac,x,x
$ mcal c='matchs("ab",$s{s1},$s{s2},$s{s3})' a=rsl i=dat2.csv o=rsl3.csv
#END# kgcal a=rsl c=matchs("ab",$s{s1},$s{s2},$s{s3}) i=dat2.csv o=rsl3.csv
$ more rsl3.csv
id,s1,s2,s3,rsl
1,ab,abx,x,1
2,abc,xaby,xxab,1
3,,,,0
4,#ac,x,x,0
\end{Verbatim}
\subsubsection*{Example 4: AND partial match}

Returns true if the character string \verb|ab| exists in columns \verb|s1,s2,s3|.


\begin{Verbatim}[baselinestretch=0.7,frame=single]
$ mcal c='matchas("ab",$s{s1},$s{s2},$s{s3})' a=rsl i=dat2.csv o=rsl4.csv
#END# kgcal a=rsl c=matchas("ab",$s{s1},$s{s2},$s{s3}) i=dat2.csv o=rsl4.csv
$ more rsl4.csv
id,s1,s2,s3,rsl
1,ab,abx,x,0
2,abc,xaby,xxab,1
3,,,,0
4,#ac,x,x,0
\end{Verbatim}
\subsubsection*{Example 5: Search for NULL value}

Return true if \verb|str| column contains NULL value.


\begin{Verbatim}[baselinestretch=0.7,frame=single]
$ mcal c='match(nulls(),$s{s1},$s{s2},$s{s3})' a=rsl i=dat2.csv o=rsl5.csv
#END# kgcal a=rsl c=match(nulls(),$s{s1},$s{s2},$s{s3}) i=dat2.csv o=rsl5.csv
$ more rsl5.csv
id,s1,s2,s3,rsl
1,ab,abx,x,0
2,abc,xaby,xxab,0
3,,,,1
4,#ac,x,x,0
\end{Verbatim}


%\end{document}

