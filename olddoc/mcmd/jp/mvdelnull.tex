
%\documentclass[a4paper]{jsbook}
%\usepackage{mcmd_jp}
%\begin{document}

\section{mvdelnull ベクトルのNULL要素の削除\label{sect:mvdelnull}}
\index{mvreplace@mvreplace}
ベクトル要素でNULLの要素を全て削除する。
ベクトル要素がNULLであれば、要素の区切り文字が連続する。
以下に示したベクトルは全てNULLを含む。
ただし、わかりやすさのためにベクトルの末尾に\verb|`\n'|を記している。
上から順番に、3番目、1番目、4番目の要素がNULLである。
\begin{Verbatim}[baselinestretch=0.7,frame=single]
a b  c\n
 a b\n
a b c \n
\end{Verbatim}

\subsection*{書式}
\verb/mvdelnull vf= [-A] /
\hyperref[sect:option_i]{i=}
\hyperref[sect:option_o]{[o=]}
\hyperref[sect:option_delim]{[delim=]} 
\hyperref[sect:option_assert_diffSize]{[-assert\_diffSize]}
\hyperref[sect:option_assert_nullin]{[-assert\_nullin]}
\hyperref[sect:option_assert_nullout]{[-assert\_nullout]}
\hyperref[sect:option_nfn]{[-nfn]} 
\hyperref[sect:option_nfno]{[-nfno]}  
\hyperref[sect:option_x]{[-x]}
\hyperref[sect:option_option_tmppath]{[tmpPath=]}
\hyperref[sect:option_precision]{[precision=]}
\verb|[-params]|
\verb|[--help]|
\verb|[--helpl]|
\verb|[--version]|\\

\subsection*{パラメータ}
\begin{table}[htbp]
%\begin{center}
{\small
\begin{tabular}{ll}
\verb|i=|    & 入力ファイル名を指定する。\\
\verb|o=|    & 出力ファイル名を指定する。\\
\verb|vf=| & NULL要素を削除する対象となる項目名(\verb|i=|ファイル上)を指定する。\\
           & 複数項目指定可能。\\
		   & 結果の項目名を変更したいときは、:(コロン)に続けて新項目名を指定する。\\
\verb|-A|  & \verb|vf=|で:(コロン)に続けて指定した項目名で、新たな項目が追加される。\\
           & なお\verb|-A|オプションを指定した場合、\verb|vf=|パラメータで指定するすべての\\
           & 項目に新項目名を指定しなければならない。\\
\verb|delim=| & ベクトル型データの区切り文字を指定する。\\
\end{tabular}
}
\end{table} 

\subsection*{利用例}
\subsubsection*{Example 1: Example 1: Basic example of removing null characters}



\begin{Verbatim}[baselinestretch=0.7,frame=single]
$ more dat1.csv
items
b a  c
 c c
e a   b 
$ mvdelnull vf=items i=dat1.csv o=rsl1.csv
#END# kgvdelnull i=dat1.csv o=rsl1.csv vf=items
$ more rsl1.csv
items
b a c
c c
e a b
\end{Verbatim}
\subsubsection*{Example 2: Example 2: Example of using .(dot) as delimiting character}



\begin{Verbatim}[baselinestretch=0.7,frame=single]
$ more dat2.csv
items
b.a..c
.c.c
e.a...b.
$ mvdelnull vf=items delim=. i=dat2.csv o=rsl2.csv
#END# kgvdelnull delim=. i=dat2.csv o=rsl2.csv vf=items
$ more rsl2.csv
items
b.a.c
c.c
e.a.b
\end{Verbatim}
\subsubsection*{Example 3: Example 3: Change field name and output }



\begin{Verbatim}[baselinestretch=0.7,frame=single]
$ mvdelnull vf=items:new i=dat1.csv o=rsl3.csv
#END# kgvdelnull i=dat1.csv o=rsl3.csv vf=items:new
$ more rsl3.csv
new
b a c
c c
e a b
\end{Verbatim}
\subsubsection*{Example 4: Example 3: Add output as an new field by specifying -A}



\begin{Verbatim}[baselinestretch=0.7,frame=single]
$ mvdelnull vf=items:new -A i=dat1.csv o=rsl4.csv
#END# kgvdelnull -A i=dat1.csv o=rsl4.csv vf=items:new
$ more rsl4.csv
items,new
b a  c,b a c
 c c,c c
e a   b ,e a b
\end{Verbatim}


\subsection*{関連コマンド}
\hyperref[sect:mvnullto]{mvnullto} : NULL要素を任意の値に置換する。

%\end{document}
