
%\documentclass[a4paper]{jsbook}
%\usepackage{mcmd_jp}
%\begin{document}

\section{mselstr 文字列による行選択\label{sect:mselstr}}
\index{mselstr@mselstr}
\verb|f=|で指定した項目の値が、\verb|v=|で指定した文字列に一致すれば、その行を選択する。

\if0 #no help# following sentences will not apear on the help document. \fi
典型例を表\ref{tbl:mselstr_input}〜\ref{tbl:mselstr_out2}に示す。
表\ref{tbl:mselstr_out1}では\verb|key|に関係なく\verb|val|が\verb|"y"|である行を選択する。
表\ref{tbl:mselstr_out2}では、\verb|val|が\verb|"x"|の行を含んでいる同一キーの行全て、
すなわち\verb|key|項目が\verb|"a"|である行全てを選択する。
すなわち\verb|key|項目が\verb|"b"|の行はいずれも\verb|"x"|を含んでいないので選択されない。

\begin{table}[htbp]
\begin{center}
\begin{tabular}{ccc}

\begin{minipage}{0.18\hsize}
\begin{center}
\caption{入力データ\label{tbl:mselstr_input}}
{\small
\begin{tabular}{cc}
\hline
key&val \\
\hline
a&x \\
a&y \\
b&y \\
b&z \\
\hline

\end{tabular}
}
\end{center}
\end{minipage}

\begin{minipage}{0.28\hsize}
\begin{center}
\caption{f=val v=y\label{tbl:mselstr_out1}}
{\small
\begin{tabular}{ccc}
\hline
key&val \\
\hline
a&y \\
b&y \\
\hline
\end{tabular}
}
\end{center}
\end{minipage}

\begin{minipage}{0.28\hsize}
\begin{center}
\caption{k=key f=val v=x\label{tbl:mselstr_out2}}
{\small
\begin{tabular}{ccc}
\hline
key&val \\
\hline
a&x \\
a&y \\
\hline
\end{tabular}
}
\end{center}
\end{minipage}

\end{tabular}
\end{center}
\end{table}

また、以下に示すように多様な選択条件を指定することも可能である。
このコマンドで指定できない複雑な条件(例えば正規表現など)を設定するのであれば
\hyperref[sect:msel]{msel}コマンドを利用すればよい。

\begin{itemize}
\item \verb|v=|に複数の文字列を指定すれば、いずれかの文字列にマッチすれば選択される。
\item \verb|f=|に複数項目を指定すれば、いずれかの項目の値がマッチすれば選択される。
\item 複数項目のマッチ条件をAND条件とすることも可能(\verb|-F|オプション)。
\item 完全一致だけでなく、先頭一致、末尾一致、部分一致の指定も可能(\verb|-head,-tail,-sub|オプション)。
\item \verb|k=|を指定することでキー単位で選択することが可能。
\item キー単位選択の場合、複数レコードのAND条件を指定可能(\verb|-R|オプション)。
\end{itemize}

いま同じキーのデータとして2項目2行からなるデータ(表\ref{tbl:mselstr_input2})に対して、
\begin{Verbatim}[baselinestretch=0.7,frame=single,fontsize=\small]
mselstr k=key f=fld1,fld2 v=s1,s2
\end{Verbatim}
を実行した場合、
オプション\verb|-R,-F|の指定の有無によるマッチ条件を表\ref{tbl:mselstr_cond}に示す。

\begin{table}[htbp]
\begin{center}

\caption{入力データ\label{tbl:mselstr_input2}}
{\small
\begin{tabular}{ccc}
\hline
$\verb|key|$ & $\verb|fld1|$ & $\verb|fld2|$ \\
\hline
k & $v_{a1}$ & $v_{a2}$ \\
k & $v_{b1}$ & $v_{b2}$ \\
\hline

\end{tabular}
}
\end{center}
\end{table}

\begin{table}[htbp]
\begin{center}
\caption{表\ref{tbl:mselstr_input2}で示されるデータに、
mselstr k=key f=fld1,fld2 v=v1,v2を実行した時の、
-R,-Fオプションの指定の有無によるマッチ条件の違い。
条件にマッチすれば全行(2行)出力され、アンマッチなら
1行も出力されない。\label{tbl:mselstr_cond}}
{\footnotesize
\begin{tabular}{ccl}
\hline
\verb|-F| & \verb|-R| & マッチ条件 \\
\hline
   &    &
(($v_{a1}$ \verb|==| s1 or $v_{a1}$ \verb|==| s2)  or
 ($v_{a2}$ \verb|==| s1 or $v_{a2}$ \verb|==| s2)) or
(($v_{b1}$ \verb|==| s1 or $v_{b1}$ \verb|==| s2)  or
 ($v_{b2}$ \verb|==| s1 or $v_{b2}$ \verb|==| s2)) \\
-F &    &
(($v_{a1}$ \verb|==| s1 or $v_{a1}$ \verb|==| s2)  and
 ($v_{a2}$ \verb|==| s1 or $v_{a2}$ \verb|==| s2)) or
(($v_{b1}$ \verb|==| s1 or $v_{b1}$ \verb|==| s2)  and
 ($v_{b2}$ \verb|==| s1 or $v_{b2}$ \verb|==| s2)) \\
   & -R & 
(($v_{a1}$ \verb|==| s1 or $v_{a1}$ \verb|==| s2)  or
 ($v_{a2}$ \verb|==| s1 or $v_{a2}$ \verb|==| s2)) and
(($v_{b1}$ \verb|==| s1 or $v_{b1}$ \verb|==| s2)  or
 ($v_{b2}$ \verb|==| s1 or $v_{b2}$ \verb|==| s2)) \\
-F & -R & 
(($v_{a1}$ \verb|==| s1 or $v_{a1}$ \verb|==| s2)  and
 ($v_{a2}$ \verb|==| s1 or $v_{a2}$ \verb|==| s2)) and
(($v_{b1}$ \verb|==| s1 or $v_{b1}$ \verb|==| s2)  and
 ($v_{b2}$ \verb|==| s1 or $v_{b2}$ \verb|==| s2)) \\
\hline
\end{tabular}
}

\end{center}
\end{table}

\subsection*{書式}
\verb|mselstr f= v= [k=]  [u=] [-F] [-r] [-R] [-sub] [-head] [-tail] [-W]|
\hyperref[sect:option_i]{[i=]}
\hyperref[sect:option_o]{[o=]}
\hyperref[sect:option_bufcount]{[bufcount=]} 
\hyperref[sect:option_assert_diffSize]{[-assert\_diffSize]}
\hyperref[sect:option_assert_nullkey]{[-assert\_nullkey]}
\hyperref[sect:option_assert_nullin]{[-assert\_nullin]}
\hyperref[sect:option_nfn]{[-nfn]} 
\hyperref[sect:option_nfno]{[-nfno]}  
\hyperref[sect:option_x]{[-x]}
\hyperref[sect:option_q]{[-q]}
\hyperref[sect:option_option_tmppath]{[tmpPath=]}
\hyperref[sect:option_precision]{[precision=]}
\verb|[-params]|
\verb|[--help]|
\verb|[--helpl]|
\verb|[--version]|\\

\subsection*{パラメータ}
\begin{table}[htbp]
%\begin{center}
{\small
\begin{tabular}{ll}
\verb|i=|    & 入力ファイル名を指定する。\\
\verb|f=|    & 検索対象となる項目名リスト(複数項目指定可)を指定する。\\
\verb|v=|    & \verb|f=|パラメータで指定した項目の値が、ここで指定した文字列リスト(複数項目指定可)の1つにマッチすれば選択される。 \\
%\verb|k=|    & 選択する単位となるキー項目(複数項目指定可)を指定する。【\hyperref[sect:option_k]{集計キーブレイク処理}】\\
\verb|k=|    & 選択する単位となるキー項目(複数項目指定可)を指定する。\\
\verb|o=|    & 指定の条件に一致する行を出力するファイル名を指定する。 \\
\verb|u=|    & 指定の条件に一致しない行を出力するファイル名を指定する。\\
\verb|-F|    & \verb|f=| パラメータで複数項目を指定した場合、その全ての値がマッチする行を撰択する。\\
\verb|-r|    & 条件反転\\
             & 選択ではなく削除する。\\
\verb|-R|    & \verb|k=| パラメータを指定した場合、その全ての行がマッチすれば行を撰択する。\\
\verb|-sub|  & 検索を完全一致ではなく部分文字列マッチで比較する。\\
             & すなわち、\verb|f=|パラメータで指定した項目の値に、\\
             & \verb|v=|パラメータで指定の文字列が部分文字列として含まれていればその行を撰択する。\\
\verb|-head| & 先頭文字列マッチオプション\\
\verb|-tail| & 末尾文字列マッチオプション\\
\verb|-W|    & \verb|-sub|,\verb|-head|,\verb|-tail|オプションが指定されているときにワイド文字として部分文字列マッチをおこなう。\\
\verb|bufcount=| & バッファのサイズ数を指定する。 \\
\end{tabular} 
}
\end{table} 

\subsection*{利用例}
\subsubsection*{Example 1: Basic example}

Select records matching \verb|apple| and \verb|orange| in the Product field , print matching results to \verb|rsl1.csv| file. Unmatched records such as \verb|pineapplejuice| will be saved to other.csv file using the parameter \verb|u=oth1.csv|.


\begin{Verbatim}[baselinestretch=0.7,frame=single]
$ more dat1.csv
Product,Amount
apple,100
milk,350
orange,100
pineapplejuice,500
wine,1000
$ mselstr f=Product v=apple,orange u=oth1.csv i=dat1.csv o=rsl1.csv
#END# kgselstr f=Product i=dat1.csv o=rsl1.csv u=oth1.csv v=apple,orange
$ more rsl1.csv
Product,Amount
apple,100
orange,100
$ more oth1.csv
Product,Amount
milk,350
pineapplejuice,500
wine,1000
\end{Verbatim}
\subsubsection*{Example 2: Remove records}

Contrary to example 1, remove records matching keywords \verb|apple| and \verb|orange| using the \verb|-r| option, the output is saved to \verb|rsl2.csv| file.


\begin{Verbatim}[baselinestretch=0.7,frame=single]
$ mselstr f=Product  v=apple,orange -r i=dat1.csv o=rsl2.csv
#END# kgselstr -r f=Product i=dat1.csv o=rsl2.csv v=apple,orange
$ more rsl2.csv
Product,Amount
milk,350
pineapplejuice,500
wine,1000
\end{Verbatim}
\subsubsection*{Example 3: Select based on the key unit}

Select all records of customer who have purchased oranges by specifying \verb|Customer| at the \verb|k=| parameter. Save unmatched records to \verb|oth2.csv|.


\begin{Verbatim}[baselinestretch=0.7,frame=single]
$ more dat2.csv
Customer,Product,Amount
A,apple,100
A,milk,350
B,orange,100
B,orange,100
B,pineapple,500
B,wine,1000
C,apple,100
C,orange,100
$ mselstr k=Customer f=Product v=orange u=oth2.csv i=dat2.csv o=rsl3.csv
#END# kgselstr f=Product i=dat2.csv k=Customer o=rsl3.csv u=oth2.csv v=orange
$ more rsl3.csv
Customer%0,Product,Amount
B,orange,100
B,orange,100
B,pineapple,500
B,wine,1000
C,apple,100
C,orange,100
$ more oth2.csv
Customer%0,Product,Amount
A,apple,100
A,milk,350
\end{Verbatim}
\subsubsection*{Example 4: Partial match}

Select records where the \verb|Product| field contain the keyword \verb|apple|, and save the output to a file named  \verb|rsl4.csv|. Records with partial match such as \verb|pine(apple)juice| will also be saved in the output file \verb|rsl4.csv|.


\begin{Verbatim}[baselinestretch=0.7,frame=single]
$ mselstr f=Product v=apple -sub i=dat1.csv o=rsl4.csv
#END# kgselstr -sub f=Product i=dat1.csv o=rsl4.csv v=apple
$ more rsl4.csv
Product,Amount
apple,100
pineapplejuice,500
\end{Verbatim}
\subsubsection*{Example 5: Wide character substring match}

Select records where the \verb|Product| field contains wide characters  "柿", "桃", and "葡萄".

Matching maybe based on single byte character if the query string includes wide character, the query string maybe interpreted as multibyte character for matching. Therefore, it is necessary indicate wide character in the query string with \verb|-W| option.


\begin{Verbatim}[baselinestretch=0.7,frame=single]
$ more dat3.csv
Product,Amount
fruit:柿,100
fruit:桃,250
fruit:葡萄,300
fruit:梨,450
fruit:苺,500
$ mselstr f=Product v=柿,桃,葡萄 -sub -W i=dat3.csv o=rsl5.csv
#END# kgselstr -W -sub f=Product i=dat3.csv o=rsl5.csv v=柿,桃,葡萄
$ more rsl5.csv
Product,Amount
fruit:柿,100
fruit:桃,250
fruit:葡萄,300
\end{Verbatim}
\subsubsection*{Example 6: Select product(s) with consecutive purchases in 2013.}

Use the \verb|-F| option to select transactions where the date of purchase and the previous date of purchase for the product both took place in 2013.  Save the query results to an output file \verb|rsl6.csv|. Save unmatched records to \verb|oth3.csv|.



\begin{Verbatim}[baselinestretch=0.7,frame=single]
$ more dat4.csv
Customer,Product,Amount,Gender,Date,PreviousDate
A,apple,100,1,2013/01/04,2013/01/01
A,milk,350,1,2013/04/04,2011/05/06
B,orange,100,2,2012/11/11,2011/12/12
B,orange,100,2,2013/05/30,2012/11/11
B,pineapple,500,2,2013/04/15,2013/04/01
B,wine,1000,2,2012/12/24,2011/12/24
C,apple,100,2,2013/02/14,NULL
C,orange,100,2,2013/02/14,2013/01/31
D,orange,100,2,2011/10/28,NULL
$ mselstr f=Date,PreviousDate -F -sub v=2013 u=oth3.csv i=dat4.csv o=rsl6.csv
#END# kgselstr -F -sub f=Date,PreviousDate i=dat4.csv o=rsl6.csv u=oth3.csv v=2013
$ more rsl6.csv
Customer,Product,Amount,Gender,Date,PreviousDate
A,apple,100,1,2013/01/04,2013/01/01
B,pineapple,500,2,2013/04/15,2013/04/01
C,orange,100,2,2013/02/14,2013/01/31
$ more oth3.csv
Customer,Product,Amount,Gender,Date,PreviousDate
A,milk,350,1,2013/04/04,2011/05/06
B,orange,100,2,2012/11/11,2011/12/12
B,orange,100,2,2013/05/30,2012/11/11
B,wine,1000,2,2012/12/24,2011/12/24
C,apple,100,2,2013/02/14,NULL
D,orange,100,2,2011/10/28,NULL
\end{Verbatim}
\subsubsection*{Example 7: Extract all transactions of customers who have consecutive purchases in 2013}

Use \verb|k=| parameter to select all transactions of customers who have purchased a product with date of purchase and date of previous purchase both took place in 2013. Save unmatched records to a file \verb|oth4.csv|.


\begin{Verbatim}[baselinestretch=0.7,frame=single]
$ mselstr k=Customer f=Date,PreviousDate -F -sub v=2013 u=oth4.csv i=dat4.csv o=rsl7.csv
#END# kgselstr -F -sub f=Date,PreviousDate i=dat4.csv k=Customer o=rsl7.csv u=oth4.csv v=2013
$ more rsl7.csv
Customer%0,Product,Amount,Gender,Date,PreviousDate
A,apple,100,1,2013/01/04,2013/01/01
A,milk,350,1,2013/04/04,2011/05/06
B,orange,100,2,2012/11/11,2011/12/12
B,orange,100,2,2013/05/30,2012/11/11
B,pineapple,500,2,2013/04/15,2013/04/01
B,wine,1000,2,2012/12/24,2011/12/24
C,apple,100,2,2013/02/14,NULL
C,orange,100,2,2013/02/14,2013/01/31
$ more oth4.csv
Customer%0,Product,Amount,Gender,Date,PreviousDate
D,orange,100,2,2011/10/28,NULL
\end{Verbatim}
\subsubsection*{Example 8: Select new customer(s) who purchased in 2013}

Use the \verb|-R| option to select all transactions of new customer(s) who made their first purchase in 2013, where date of previous purchase is NULL.
Write the query results to an output file ¥verb|rsl8.csv|, and save unmatched records to \verb|oth5.csv|. 



\begin{Verbatim}[baselinestretch=0.7,frame=single]
$ mselstr k=Customer f=Date,PreviousDate -F -R -sub v=2013,NULL u=oth5.csv i=dat4.csv o=rsl8.csv
#END# kgselstr -F -R -sub f=Date,PreviousDate i=dat4.csv k=Customer o=rsl8.csv u=oth5.csv v=2013,NULL
$ more rsl8.csv
Customer%0,Product,Amount,Gender,Date,PreviousDate
C,apple,100,2,2013/02/14,NULL
C,orange,100,2,2013/02/14,2013/01/31
$ more oth5.csv
Customer%0,Product,Amount,Gender,Date,PreviousDate
A,apple,100,1,2013/01/04,2013/01/01
A,milk,350,1,2013/04/04,2011/05/06
B,orange,100,2,2012/11/11,2011/12/12
B,orange,100,2,2013/05/30,2012/11/11
B,pineapple,500,2,2013/04/15,2013/04/01
B,wine,1000,2,2012/12/24,2011/12/24
D,orange,100,2,2011/10/28,NULL
\end{Verbatim}

\subsection*{関連コマンド}
\hyperref[sect:msel] {msel} : より複雑な条件で行選択を行う。

\hyperref[sect:mcommon] {mcommon} : 選択対象となる文字列の数が多いときは、参照ファイルを用意することで\verb|mcommon|コマンドが使える。

%\end{document}
