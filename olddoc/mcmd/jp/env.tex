%\documentclass[a4paper]{book}
%\usepackage{mcmd}
%\begin{document}

\section{Environment variable}
MCMDでは、表\ref{tb:env}に示される環境変数の設定が可能で、
その値に応じてコマンドの動作を変更することが可能である。

\begin{table}[htpb]
\begin{center}
\caption{Mコマンドで設定可能な環境変数一覧\label{tb:env}}
%{\scriptsize 
{\small
\begin{tabular}{l|r|l}
\hline
変数名 & デフォルト値 & 内容\\ \hline
\verb|KG_iSize|          & 4096000 & read一回あたりのサイズ \\
                         &         & 入力バッファサイズは、\verb|KG_iSize|の4倍のメモリが確保される。\\
                         &         & ただし、kgsortfは10倍もしくはコマンドパラメータで設定した倍数確保される。\\
                         &         & また、キーブロック単位で処理するコマンドのバッファサイズは\verb|KG_BlockCount|を参照。\\
                         &         & 以下の条件を満たす必要がある。 \\
                         &         & \verb|KG_iSize|=\verb|KG_MaxRecLen|$*$i (ただし、iは1以上の整数) \\
                         &         & \verb|KG_iSize|$\ge$\verb|KG_MaxRecLen|$*$2\\
\hline
\verb|KG_oSize|          & 2048000 & write一回あたりのサイズ \\
                         &         & 出力バッファサイズは、\verb|KG_oSize|同じサイズのメモリが確保される。\\
                         &         & \verb|KG_oSize|=\verb|KG_MaxRecLen|$*$i (ただし、iは1以上の整数) \\
                         &         & \verb|KG_oSize|$\ge$\verb|KG_MaxRecLen|$*$2\\
\hline
\verb|KG_MaxRecLen|      & 1024000 & 一行あたりの最大文字数 (上限:10240000) \\
                         &         & \verb|KG_iSize|と\verb|KG_oSize|に示された条件を満たす必要がある。\\
\hline
\verb|KG_BlockCount|     &     128 & キー単位ごとに処理$^{注1)}$する際に使用するバッファ数 \\
                         &         & \verb|KG_iSize|の欄で示したバッファサイズ$*$\verb|KG_BlockCount|のバッファが確保される。\\
                         &         & \verb|KG_MaxRecLen| $*$2 + 4$*$ \verb|KG_ioSize|) $*$ \verb|KG_BlockCount| \\
\hline
\verb|KG_TmpPath|        & /tmp    & ライブラリ関数が用いるデフォルトの一時ファイル用ディレクトリ \\
\hline
\verb|KG_Precision|      & 10      & 有効桁数 \\
\hline
\verb|KG_VerboseLevel|   &       4 & Mコマンドのエラーメッセージ出力レベル \\
                         &         & 0: メッセージを一切出力しない \\
                         &         & 1: + errorメッセージ出力 \\
                         &         & 2: + warningメッセージ出力 \\
                         &         & 3: + endメッセージ出力 \\
                         &         & 4: + msgメッセージ出力(デフォルト) \\
\hline
\verb|KG_msgTimebyfsec|  & false   & false: 終了メッセージの時刻を秒単位で表示する \\
                         &         & ture: 終了メッセージの時刻をマイクロ秒単位で表示する \\
\hline
\end{tabular}  
}
注1) キー単位ごとの処理とは、\verb|mnjoin|の様に、同一キーのデータを一旦全てメモリに読み込む処理のこと。
詳細は個々のコマンドのマニュアルを参照のこと。
\end{center}
\end{table}  

%\hline
%\verb|KG_MaxFileNameLen| & 256 & ファイル名(パス込)の最大長 \\
%\hline
%\verb|KG_DEFAULT_ENC|    & \verb|ja_JP.UTF-8| & KGMODが内部で扱うデフォルトの文字列encoding \\
以下、コマンドのメッセージを制御する\verb|KG_VerboseLevel|の設定を変更する例を示す。

\subsection*{利用例}
\subsubsection*{Example 1: End message}

The default setting of output message is set as \verb|KG_Verbose=4| which displays a message when the program finished processing normally or terminates upon an error.


\begin{Verbatim}[baselinestretch=0.7,frame=single]
$ more dat.csv
k,v
A,1
B,2
$ mcut f=k,v i=dat.csv o=out.csv
#END# kgcut f=k,v i=dat.csv o=out.csv
$ mcut x=k,v i=dat.csv o=out.csv
#ERROR# unknown parameter x= (kgcut)
\end{Verbatim}
\subsubsection*{Example 2: Only display error messages}

If \verb|KG_Verbose=1|, the stop error message is displayed, but not the normal exit message.


\begin{Verbatim}[baselinestretch=0.7,frame=single]
$ export KG_VerboseLevel=1
$ mcut f=k,v i=dat.csv o=out.csv
$ mcut x=k,v i=dat.csv o=out.csv
#ERROR# unknown parameter x= (kgcut)
\end{Verbatim}
\subsubsection*{Example 3: Do not display any messages}

If \verb|KG_Verbose=0|, no message will be displayed. 


\begin{Verbatim}[baselinestretch=0.7,frame=single]
$ export KG_VerboseLevel=0
$ mcut f=k,v i=dat.csv o=out.csv
$ mcut x=k,v i=dat.csv o=out.csv
\end{Verbatim}


%\end{document}

