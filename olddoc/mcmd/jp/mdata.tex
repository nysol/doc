
%\documentclass{jarticle}
%\begin{document}

\section{mdata データセットの出力\label{sect:mdata}}
\index{mdata@mdata}
様々なデータセットを生成する。データセットの詳細は以下を参照のこと。

\subsection*{書式}
\verb/mdata [O=] -iris|-man0|-man0_en|-man1|-man1_en|-tutorial_en|-tutorial_jp|-yakiniku_en|-yakiniku_jp/
\verb|[-params]|
\verb|[--help]|
\verb|[--helpl]|
\verb|[--version]|\\

\subsection*{パラメータ}
\begin{table}[htbp]
%\begin{center}
{\small
\begin{tabular}{ll}
\verb|O=|    & 出力ファイル名。省略時は標準出力に出力される。\\
             & \verb|-tutorial_jp|もしくは\verb|-tutorial_en|が指定された場合はディレクトリ名となり、\\
             & 省略した場合は、\verb|O=tutorial_jp|もしくは\verb|O=tutorial_en|が指定されたことになる。\\
\verb|-iris| & 萼片と花びらの大きさによって、アヤメの種類の分類モデルの構築を目的に構成されるデータセット \\
             & 項目名:SepalLength,SepalWidth,PetalLength,PetalWidth,Species(ガク長,ガク幅,花びら長,花びら幅,種)\\
             & \href{http://archive.ics.uci.edu/ml/datasets/Iris?ref=datanews.io}{http://archive.ics.uci.edu/ml/datasets/Iris?ref=datanews.io} \\
\verb|-man0|  & 本マニュアルの図\ref{fig:abstract0_1}で使われている5行データ\\
             & 項目名: 顧客,金額\\
\verb|-man0_en|  & 本マニュアルの図\ref{fig:abstract0_1}で使われている5行データの英語版\\
             & 項目名: customer,amount\\
\verb|-man1|  & 本マニュアルの図\ref{fig:abstract2_1}で使われている8行データ\\
             & 項目名: 顧客,日付,商品\\
\verb|-man1_en|  & 本マニュアルの図\ref{fig:abstract2_1}で使われている8行データの英語版\\
             & 項目名: customer,date,item\\
\verb|-yakiniku_jp| & 焼肉店の販売データ(\href{http://okayamafs.com/}{岡山フードサービス株式会社}より提供された焼肉店
\href{https://r.gnavi.co.jp/c032802/}{福牛}の販売データ)。 \\
                   & 項目名: 日付,時間,レシート,商品,単価,数量,金額 \\
                   & 注1) 時間は注文時刻で、同じレシート内でも追加注文時はその時の時刻が記録されている。\\
                   & 注2) 金額=単価×数量 \\
\verb|-yakiniku_en| & 焼肉店の注文データの英語版 \\
                   & 項目名: date,time,receipt,item,price,quantity,totalAmount \\
\verb|-tutorial_jp| & チュートリアルで利用されるスーパーマーケットの擬似購買データ。\\
                   & \verb|dat.csv|: 購買履歴データ\\
                   &   項目名:店,日付,時間,レシート,顧客,商品,大分類,中分類,小分類,細分類,メーカー,ブランド,\\
                   & 仕入単価,単価,数量,金額,仕入金額,粗利金額 \\
                   & \verb|syo.csv|: 商品マスター\\
                   &   項目名: 商品,商品名,大分類,中分類,小分類,細分類,メーカー,ブランド,仕入単価 \\
                   & \verb|cust.csv|: 顧客マスター\\
                   &   項目名: 顧客,生年月日,性別 \\
                   & \verb|jicfs1,2,4,6.csv|: 商品分類マスター\\
                   &   項目名: 大分類,大分類名(中分類,中分類名)(小分類,小分類名)(細分類,細分類名) \\
\verb|-tutorial_en| & \verb|tutorial_jp|データセットの英語版\\
                   & \verb|dat.csv|: 購買履歴データ\\
                   & 項目名:shop,date,time,receipt,customer,product,CategoryCode1,CategoryCode2,CategoryCode4,\\
                   & CategoryCode6,manufacturer,brand,unitCost,unitPrice,quantity,amount,costAmount,profit\\
                   & \verb|syo.csv|: 商品マスター\\
                   & 項目名: product,productName,CategoryCode1,CategoryCode2,CategoryCode4,CategoryCode6,\\
                   & manufacturer,brand,unitCost \\
                   & \verb|cust.csv|: 顧客マスター\\
                   &   項目名: customer,dob,gender \\
                   & \verb|jicfs1,2,4,6.csv|: 商品分類マスター\\
                   &   項目名: CategoryCode1,Category1(CategoryCode2,Category2)(CategoryCode4,Category4)\\
                   & (CategoryCode6,Category6) \\
\end{tabular} 
}
\end{table} 

%データセット名とそれに対するパラメータを"/"で区切ることで指定する。
%データセットの一覧と内容は表\ref{tbl:mdata_dataset}に示すとおりである。
%パラメータの与え方はそれぞれのデータセットによって異なり、その方法も同表に示されている。

%\begin{table}[hbt]
%\begin{center}
%\caption{データセット名とその内容\label{tbl:mdata_dataset}}
%{\small
%\begin{tabular}{l|l|l}
%\hline
%データセット名 & 内容 & パラメータ \\ \hline \hline
%\verb|iris| & 萼片と花びらの大きさによって、アヤメの種類の   & なし \\
%            & 分類モデルの構築を目的に構成されるデータセット &\\ \hline
%\verb|man0| & 本マニュアルの図\ref{fig:abstract0_1}で使われているデータ & なし \\ \hline
%\verb|man1| & 本マニュアルの図\ref{fig:abstract2_1}で使われているデータ & なし \\ \hline
%\verb|yakiniku_jp| & 焼肉店の注文データ & なし \\ \hline
%\verb|yakiniku_en| & 焼肉店の注文データの英語版 & なし \\ \hline
%\verb|tutorial_jp| & チュートリアルで利用されるスーパーマーケットの & データ名を指定すると各データが標準出力に出力される。\\
%                   & 擬似購買データ。顧客マスターや商品マスターなど & 指定しないと全てのファイルが\verb|tutorial_jp|\\
%                   & 複数のデータファイルから構成される。           & ディレクトリの下に生成される。\\
%                   &                                                & データ名とその内容は以下のとおり。\\
%                   &                                                & \verb|dat|:購買データ \\
%                   &                                                & \verb|syo|:商品マスター \\
%                   &                                                & \verb|cust|:顧客マスター \\
%                   &                                                & \verb|jicfs1,jicfs2,jicfs4,jicfs6|:商品分類マスター \\ \hline
%\verb|tutorial_en| & \verb|tutorial_jp|データセットの英語版 & \verb|tutorial_jp|に同じ \\ \hline
%\end{tabular}
%}
%\end{center}
%\end{table}

\subsection*{利用例}
\subsubsection*{例1 irisデータセットの出力}
irisデータセットを標準出力に出力する。

\begin{Verbatim}[baselinestretch=0.7,frame=single,fontsize=\small]
$ mdata -iris
SepalLength,SepalWidth,PetalLength,PetalWidth,Species
5.1,3.5,1.4,0.2,setosa
4.9,3,1.4,0.2,setosa
4.7,3.2,1.3,0.2,setosa
4.6,3.1,1.5,0.2,setosa
         :
\end{Verbatim}

\subsubsection*{例2 チュートリアルデータセットの出力}
チュートリアルデータセットを全てファイル出力する。

\begin{Verbatim}[baselinestretch=0.7,frame=single,fontsize=\small]
$ mdata -tutorial_en
#END# mdata -tutorial_en

$ ls -l tutorial_en
total 4704
-rw-r--r--  1 nysol  staff    20673  8 22 08:14 cust.csv
-rw-r--r--  1 nysol  staff  2281312  8 22 08:14 dat.csv
-rw-r--r--  1 nysol  staff      128  8 22 08:14 jicfs1.csv
-rw-r--r--  1 nysol  staff      529  8 22 08:14 jicfs2.csv
-rw-r--r--  1 nysol  staff     6630  8 22 08:14 jicfs4.csv
-rw-r--r--  1 nysol  staff    36400  8 22 08:14 jicfs6.csv
-rw-r--r--  1 nysol  staff    46466  8 22 08:14 syo.csv

$ more tutorial_en/dat.csv
customer,dob,gender
00000A,19711107,female
00000B,19461025,female
00000C,19660307,female
         :
\end{Verbatim}

\subsubsection*{例3 焼肉データを出力}

\begin{Verbatim}[baselinestretch=0.7,frame=single,fontsize=\small]
$ mdata -yakiniku_jp
日付,時間,レシート,商品,単価,数量,金額
20070701,1123,10000,焼肉ヘルシーセット,1410,1,1410
20070701,1152,10001,和牛焼肉弁当,1240,1,1240
20070701,1202,10002,ランチコーヒー,130,2,260
             :
$ mdata -yakiniku_en
date,time,receipt,item,price,quantity,totalAmount
20070701,1123,10000,Low-fat BBQ set,1410,1,1410
20070701,1152,10001,Japanese grilled beef lunch box,1240,1,1240
20070701,1202,10002,Lunchtime coffee,130,2,260
         :
\end{Verbatim}

%\end{document}
