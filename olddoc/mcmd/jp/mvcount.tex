
%\documentclass[a4paper]{jsbook}
%\usepackage{mcmd_jp}
%\begin{document}

\section{mvcount ベクトルサイズの計算\label{sect:mvcount}}
\index{mvcount@mvcount}
ベクトルのサイズ(ベクトルの要素数)を計算する。

\if0 #no help# following sentences will not apear on the help document. \fi
典型的な例を表\ref{tbl:mvcount_input}〜\ref{tbl:mvcount_out3}に示す。

\begin{table}[htbp]
\begin{center}
\begin{tabular}{ccc}

\begin{minipage}{0.3\hsize}
\begin{center}
\caption{入力データ\label{tbl:mvcount_input}}
\verb|in.csv| \\
{\small
\begin{tabular}{cl}
\hline
no&items \\
\hline
1&a b c \\
2&a d \\
3&b f e f \\
4& \\
\hline

\end{tabular}
}
\end{center}
\end{minipage}

\begin{minipage}{0.5\hsize}
\begin{center}
\caption{基本例\label{tbl:mvcount_out1}}
\verb|mvcount vf=items:size i=in.csv| \\
{\small
\begin{tabular}{lll}
\hline
no&items&size \\
\hline
1&a b c&3 \\
2&a d&2 \\
3&b f e f&4 \\
4& &0\\
\hline

\end{tabular}
}
\end{center}
\end{minipage}

\end{tabular}
\end{center}
\end{table}

\subsection*{書式}
\verb|mvcount vf=|
\hyperref[sect:option_i]{[i=]}
\hyperref[sect:option_o]{[o=]}
\hyperref[sect:option_delim]{[delim=]} 
\hyperref[sect:option_assert_diffSize]{[-assert\_diffSize]}
\hyperref[sect:option_assert_nullin]{[-assert\_nullin]}
\hyperref[sect:option_nfn]{[-nfn]} 
\hyperref[sect:option_nfno]{[-nfno]}  
\hyperref[sect:option_x]{[-x]}
\hyperref[sect:option_option_tmppath]{[tmpPath=]}
\hyperref[sect:option_precision]{[precision=]}
\verb|[-params]|
\verb|[--help]|
\verb|[--helpl]|
\verb|[--version]|\\

\subsection*{パラメータ}
\begin{table}[htbp]
%\begin{center}
{\small
\begin{tabular}{ll}
\verb|i=|    & 入力ファイル名を指定する。\\
\verb|o=|    & 出力ファイル名を指定する。\\
\verb|vf=| & 要素数をカウントするベクトルの項目名(\verb|i=|ファイル上)を指定する。\\
           & 結果項目を\verb|:|に続けて複数項目指定可能。\\
           & 複数項目指定可能。\\
\verb|delim=| & ベクトル型データの区切り文字を指定する。\\
\end{tabular}
}
\end{table} 

\subsection*{利用例}
\subsubsection*{Example 1: Count multiple vectors}



\begin{Verbatim}[baselinestretch=0.7,frame=single]
$ more dat1.csv
items1,items2
b a c,b
c c,
e a a,a a a
$ mvcount vf=items1:size1,items2:size2 i=dat1.csv o=rsl1.csv
#END# kgvcount i=dat1.csv o=rsl1.csv vf=items1:size1,items2:size2
$ more rsl1.csv
items1,items2,size1,size2
b a c,b,3,1
c c,,2,0
e a a,a a a,3,3
\end{Verbatim}

\subsection*{関連コマンド}

%\end{document}
