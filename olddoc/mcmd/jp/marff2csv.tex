
%\documentclass[a4paper]{jsbook}
%\usepackage{mcmd_jp}
%\begin{document}

\section{marff2csv arff形式からcsv形式への変換\label{sect:marff2csv}}
\index{marff2csv@marff2csv}
arff形式(WEKA用のデータフォーマット)のデータからcsv形式のデータへ変換する。

\subsubsection*{arff形式データ}
以下arff形式データのフォーマットを記載する。
\begin{Verbatim}[baselinestretch=0.7,frame=single,fontsize=\small]
@RELATION       タイトル

@ATTRIBUTE      項目名    string(文字列)
@ATTRIBUTE      項目名    date(日時 フォーマット:フォーマットは省略可能。
                                省略した場合は、"yyyy-MM-dd'T'HH:mm:ss")
@ATTRIBUTE      数量    numeric(数字)
@ATTRIBUTE      商品    {A,B}(カテゴリ型項目)

@DATA(実データ)
No.1,20081201,1,10,A
No.2,20081202,2,20,A
No.3,20081203,3,30,A
No.4,20081201,4,40,B
No.5,20081203,5,50,B
\end{Verbatim}

\subsection*{書式}
%marff2csv [\href{run:option.pdf}{i=}] [\href{run:option.pdf}{-nfn}] [\href{run:option.pdf}{-nfno}] [\href{run:option.pdf}{o=}] [--help]\\
\verb|marff2csv|
\hyperref[sect:option_i]{[i=]}
\hyperref[sect:option_o]{[o=]}
\hyperref[sect:option_assert_nullout]{[-assert\_nullout]}
\hyperref[sect:option_nfn]{[-nfn]} 
\hyperref[sect:option_nfno]{[-nfno]}
\hyperref[sect:option_x]{[-x]}
\hyperref[sect:option_q]{[-q]}
\hyperref[sect:option_option_tmppath]{[tmpPath=]}
\hyperref[sect:option_precision]{[precision=]}
\verb|[-params]|
\verb|[--help]|
\verb|[--helpl]|
\verb|[--version]|\\

\subsection*{パラメータ}
\begin{table}[htbp]

{\small
\begin{tabular}{ll}
\verb|i=|    & 入力ファイル名を指定する。\\
\verb|o=|    & 出力ファイル名を指定する。\\
\end{tabular}
}
\end{table}

\subsection*{利用例}
\subsubsection*{例1: 基本例}

arff形式の顧客購買データをcsv形式のデータへ変換する。


\begin{Verbatim}[baselinestretch=0.7,frame=single]
$ more dat1.arff
@RELATION       顧客購買データ

@ATTRIBUTE      顧客    string
@ATTRIBUTE      日付    date yyyyMMdd
@ATTRIBUTE      数量    numeric
@ATTRIBUTE      金額    numeric
@ATTRIBUTE      商品    {A,B}

@DATA
No.1,20081201,1,10,A
No.2,20081202,2,20,A
No.3,20081203,3,30,A
No.4,20081201,4,40,B
No.5,20081203,5,50,B
$ marff2csv i=dat1.arff  o=rsl1.csv
#END# kgarff2csv i=dat1.arff o=rsl1.csv
$ more rsl1.csv
顧客,日付,数量,金額,商品
No.1,20081201,1,10,A
No.2,20081202,2,20,A
No.3,20081203,3,30,A
No.4,20081201,4,40,B
No.5,20081203,5,50,B
\end{Verbatim}

\subsection*{関連コマンド}

\hyperref[sect:mcsv2arff] {mcsv2arff}

\subsection*{参考資料}
\href{http://weka.wikispaces.com/ARFF}{http://weka.wikispaces.com/ARFF}

%\end{document}
