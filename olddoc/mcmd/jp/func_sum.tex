
%\begin{document}

\section{sum 合計\label{sect:sum}}
\index{sum@sum}

書式1: sum($num_1,num_2,\cdots$)

書式2: sum($num_1,num_2,\cdots,str$)

$num_i$で与えられた数値を全て合計する。
書式1では、NULL値は無視されるが、全てがNULL値であれば結果もNULLとなる。

書式2において最後の引数として"-n"を与えると、
NULL値に対する扱いが変わり、
項目値に一つでもNULL値がある場合は、結果もNULL値となる。

\subsection*{利用例}
\subsubsection*{Example 1: Basic Example}



\begin{Verbatim}[baselinestretch=0.7,frame=single]
$ more dat1.csv
id,v1,v2,v3
1,1,2,3
2,-5,2,1
3,1,,3
4,,,
$ mcal c='sum(${v1},${v2},${v3})' a=rsl i=dat1.csv o=rsl1.csv
#END# kgcal a=rsl c=sum(${v1},${v2},${v3}) i=dat1.csv o=rsl1.csv
$ more rsl1.csv
id,v1,v2,v3,rsl
1,1,2,3,6
2,-5,2,1,-2
3,1,,3,4
4,,,,
\end{Verbatim}
\subsubsection*{Example 2: Example using wildcard}

Specify fields starting with \verb|v| (\verb|v1,v2,v3|) by using wildcard "\verb|v*|".


\begin{Verbatim}[baselinestretch=0.7,frame=single]
$ mcal c='sum(${v*})' a=rsl i=dat1.csv o=rsl2.csv
#END# kgcal a=rsl c=sum(${v*}) i=dat1.csv o=rsl2.csv
$ more rsl2.csv
id,v1,v2,v3,rsl
1,1,2,3,6
2,-5,2,1,-2
3,1,,3,4
4,,,,
\end{Verbatim}


%\end{document}
