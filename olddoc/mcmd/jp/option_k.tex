
%\begin{document}

\subsection{k= キー項目名\label{sect:option_k}}



キー項目を指定する(複数項目指定可)。
キー項目とは、集計の単位として指定したり、
またファイルの結合時に2ファイル間の共通項目として指定したりする項目である。

たとえば\verb|msum|コマンドでは、同一キーごとに合計処理を実施する(集計キーブレイク処理)。
またmjoinコマンドでは、2つのデータファイルについて、キー項目の大小を見比べて
結合処理を実施する(結合キーブレイク処理)。

\verb|k=|パラメータが指定されたとき、コマンドはまずその項目を文字列昇順で並べ替えた上で、
それぞれの処理を実行する(ただし、\hyperref[sect:mhashsum]{mhashsum}コマンドのような例外もある)。

なおキーブレイク処理の詳細は、\hyperref[sect:keybreak]{キーブレイク処理}を参照のこと。
項目の並べ替えが頻繁に発生するとパフォーマンスの低下を招くため、
キーブレイク処理の内容と必要性を理解した上で、並べ替えの回数を少なくする
スクリプトを記述することが望ましい。

\subsection*{利用例}
\subsubsection*{Example 1: Basic Example}

Compute sum on \verb|val| column by \verb|id|.


\begin{Verbatim}[baselinestretch=0.7,frame=single]
$ more dat1.csv
id,val
A,1
B,1
B,2
A,2
B,3
$ msum i=dat1.csv k=id f=val o=rsl1.csv
#END# kgsum f=val i=dat1.csv k=id o=rsl1.csv
$ more rsl1.csv
id%0,val
A,3
B,6
\end{Verbatim}
\subsubsection*{Example 2: Join Process}

Use the join key “id” from \verb|dat1.csv|, and join the field “name” from \verb|ref1.csv|.


\begin{Verbatim}[baselinestretch=0.7,frame=single]
$ more dat1.csv
id,val
A,1
B,1
B,2
A,2
B,3
$ more ref1.csv
id,name
A,nysol
B,mcmd
$ mjoin k=id i=dat1.csv m=ref1.csv f=name o=rsl4.csv
#END# kgjoin f=name i=dat1.csv k=id m=ref1.csv o=rsl4.csv
$ more rsl4.csv
id%0,val,name
A,1,nysol
A,2,nysol
B,1,mcmd
B,2,mcmd
B,3,mcmd
\end{Verbatim}


\subsubsection*{対応コマンド}
\hyperref[sect:msum]{msum},
\hyperref[sect:mslide]{mslide},
\hyperref[sect:mjoin]{mjoin},
\hyperref[sect:mrjoin]{mrjoin},
\hyperref[sect:mcommon]{mcommon}など

%\end{document}

