
%\begin{document}

\subsection{i= 入力ファイル名\label{sect:option_i}}
入力ファイル名を指定する。
多くのコマンドでは単一のファイルのみ指定可能で、例外として\verb|mcat|コマンドは複数のファイル名をカンマで区切って指定できる。
また入力データを必要としないコマンド、例えば、\verb|mnewrand|や\verb|mnewnumber|などもある。

このパラメータが省略された時には標準入力からデータを読み込む。
この機能があるために、パイプラインによる接続が可能となる。
例えば、以下の例では、\verb|msum|で\verb|i=|を指定していないが、
これは\verb|msortf|の結果がパイプラインを介して標準入力としてデータが入力されるためである。
\begin{Verbatim}[baselinestretch=0.7,frame=single]
$ msortf f=a i=dat.csv | msum k=a f=b o=rsl.csv
\end{Verbatim}

また、上記と同様の処理を行うに当たって、気づきにくい間違いを以下に示そう。
上記との違いは\verb|msum|に\verb|i=|パラメータが指定されている点である。
この例は残念ながらエラーとはならない。
\verb|msortf| の結果は標準出力に出力され、\verb|msum|は\verb|dat.csv|から読み込んで処理を行う。
よって、\verb|msortf|を実行している意味が全くなくなっており、得られる結果も異なったものとなるであろう。

\begin{Verbatim}[baselinestretch=0.7,frame=single]
$ msortf f=a i=dat.csv | msum k=a f=b i=dat.csv o=rsl.csv
\end{Verbatim}

\subsection*{利用例}
\subsubsection*{Example 1: Basic Example}

Run \verb|mcut| using \verb|dat1.csv| as input data.


\begin{Verbatim}[baselinestretch=0.7,frame=single]
$ more dat1.csv
customer,quantity,amount
A,1,10
A,2,20
$ mcut f=customer,amount i=dat1.csv o=rsl1.csv
#END# kgcut f=customer,amount i=dat1.csv o=rsl1.csv
$ more rsl1.csv
customer,amount
A,10
A,20
\end{Verbatim}
\subsubsection*{Example 2: Specify output field name}

Read standard input using redirection ("\verb|"<"|").


\begin{Verbatim}[baselinestretch=0.7,frame=single]
$ mcut f= customer, amount o=rsl2.csv <dat1.csv
#ERROR# invalid argument: customer, (kgcut)
$ more rsl2.csv
rsl2.csv: No such file or directory
\end{Verbatim}


\subsubsection*{対応コマンド}
\verb|mnewnumber|,\verb|mnewrand|など一部のコマンドを除いて全てのコマンドで利用できる。

%\end{document}

