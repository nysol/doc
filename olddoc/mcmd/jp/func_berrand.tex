
%\begin{document}

\section{berrand ベルヌーイ乱数\label{sect:berrand}}
\index{berrand@berrand}

書式: berrand(確率[, 乱数の種])

ベルヌーイ分布$f(x)=p^x(1-p)^{1-x} x=0,1$について、指定された確率$p$における$x$の乱数を生成する。
本関数では、$x=0$をfalse、$x=1$をtrueとした論理型として値を返す。
また確率$p$は定数としてのみ指定可能で、項目名を指定することはできない。

同じ乱数の種を指定すれば、同じ乱数系列となる。
指定可能な乱数の種の範囲は-2147483648〜2147483647である。
乱数の種を省略すると、
時刻(1/1000秒単位)に応じた異なる乱数の種が利用される。

乱数の生成にはメルセンヌ・ツイスター法を利用している
(\href{http://www.math.sci.hiroshima-u.ac.jp/~m-mat/MT/emt.html}{原作者のページ}
, \href{http://www.boost.org/doc/libs/1_54_0/doc/html/boost_random.html}{boostライブラリ})。

\subsection*{利用例}
\input{examples/tex/func_berrand_ex}

%\end{document}

