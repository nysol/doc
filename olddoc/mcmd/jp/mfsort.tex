
%\documentclass[a4paper]{jsbook}
%\usepackage{mcmd_jp}
%\begin{document}

\section{mfsort 項目ソート\label{sect:mfsort}}
\index{mfsort@mfsort}
各行で\verb|f=|で指定した複数項目の値を並べ替え(デフォルトでは文字列昇順)、その順序で出力する。
項目名の並びは変化しないことに注意する。

\subsection*{書式}
\verb|mfsort f= [-r] [-n]| 
\hyperref[sect:option_i]{[i=]}
\hyperref[sect:option_o]{[o=]}
\hyperref[sect:option_assert_diffSize]{[-assert\_diffSize]}
\hyperref[sect:option_assert_nullin]{[-assert\_nullin]}
\hyperref[sect:option_nfn]{[-nfn]} 
\hyperref[sect:option_nfno]{[-nfno]}  
\hyperref[sect:option_x]{[-x]}
\hyperref[sect:option_option_tmppath]{[tmpPath=]}
\hyperref[sect:option_precision]{[precision=]}
\verb|[-params]|
\verb|[--help]|
\verb|[--helpl]|
\verb|[--version]|\\

\subsection*{パラメータ}
\begin{table}[htbp]
%\begin{center}
{\small
\begin{tabular}{ll}
\verb|i=|    & 入力ファイル名を指定する。\\
\verb|o=|    & 出力ファイル名を指定する。\\
\verb|f=| & ソート対象となる項目を複数指定する。単一の項目を指定してもよいが、結果は変わらない。\\
\verb|-n| & 数値順に並べる。\\
\verb|-r| & 逆順に並べる。\\
\end{tabular} 
}
\end{table} 

\subsection*{利用例}
\subsubsection*{Example 1: Basic Example}

Arrange the values in \verb|v1,v2,v3| in ascending order for each record, and output the data items in sequential order corresponding to fields \verb|v1,v2,v3|.


\begin{Verbatim}[baselinestretch=0.7,frame=single]
$ more dat1.csv
id,v1,v2,v3
1,b,a,c
2,a,b,a
3,b,,e
$ mfsort f=v* i=dat1.csv o=rsl1.csv
#END# kgfsort f=v* i=dat1.csv o=rsl1.csv
$ more rsl1.csv
id,v1,v2,v3
1,a,b,c
2,a,a,b
3,,b,e
\end{Verbatim}
\subsubsection*{Example 2: Descending Order}

Add \verb|-r| to arrange in descending order.


\begin{Verbatim}[baselinestretch=0.7,frame=single]
$ mfsort f=v* -r i=dat1.csv o=rsl2.csv
#END# kgfsort -r f=v* i=dat1.csv o=rsl2.csv
$ more rsl2.csv
id,v1,v2,v3
1,c,b,a
2,b,a,a
3,e,b,
\end{Verbatim}


\subsection*{関連コマンド}

%\end{document}
