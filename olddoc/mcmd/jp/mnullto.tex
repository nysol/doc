
%\documentclass[a4paper]{jsbook}
%\usepackage{mcmd_jp}
%\begin{document}

\section{mnullto NULL値の置換\label{sect:mnullto}}
\index{mnullto@mnullto}
\verb|f=|パラメータで指定した項目について
NULL値を\verb|v=|パラメータで指定した文字列に置換する。 

\subsection*{書式}
\verb/mnullto f= [k=] [s=] rv=|-p] [O=] [-A]/
\hyperref[sect:option_i]{[i=]}
\hyperref[sect:option_o]{[o=]}
\hyperref[sect:option_assert_diffSize]{[-assert\_diffSize]}
\hyperref[sect:option_nfn]{[-nfn]} 
\hyperref[sect:option_nfno]{[-nfno]}  
\hyperref[sect:option_x]{[-x]}
\hyperref[sect:option_q]{[-q]}
\hyperref[sect:option_option_tmppath]{[tmpPath=]}
\hyperref[sect:option_precision]{[precision=]}
\verb|[-params]|
\verb|[--help]|
\verb|[--helpl]|
\verb|[--version]|\\

\subsection*{パラメータ}
\begin{table}[htbp]
%\begin{center}
{\small
\begin{tabular}{ll}
\verb|i=|    & 入力ファイル名を指定する。\\
\verb|o=|    & 出力ファイル名を指定する。\\
\verb|f=|  & ここで指定した項目(複数項目指定可)のNULL値が置換される。\\
\verb|v=|  & ここで指定した文字列にNULL値を置換する。\\
\verb|-p|  & 前の行の値で置換する。\\
           & \verb|v=|パラメータと同時に指定できない。\\
\verb|k=|  & \verb|-p|を指定した時にのみ意味があり、ここで指定した項目値の単位で置換処理を行なう。\\
\verb|s=|  & \verb|-p|を指定した時にのみ意味があり、\verb|k=|項目内での並び順を指定する。\\
\verb|O=|  & NULL値以外を置換したい場合は、ここで値を指定する。
             指定しなければNULL値以外は置換しない。\\
\verb|-A|  & このオプションにより、指定した項目を置き換えるのではなく、
             新たに項目が追加される。\\
           & \verb|-A|オプションを指定した場合は必ず、\\
           & :(コロン)で新項目名を指定する必要がある。例)f=数量:置換後の項目名\\
\end{tabular} 
}
\end{table} 

\subsection*{利用例}
\subsubsection*{Example 1: Basic Example}

Replace NULL values in the ¥verb|birthday| field with the string \verb|“no value”|.


\begin{Verbatim}[baselinestretch=0.7,frame=single]
$ more dat1.csv
customer,birthday
A,19690103
B,
C,19500501
D,
E,
$ mnullto f=birthday v="no value" i=dat1.csv o=rsl1.csv
#END# kgnullto f=birthday i=dat1.csv o=rsl1.csv v=no value
$ more rsl1.csv
customer,birthday
A,19690103
B,no value
C,19500501
D,no value
E,no value
\end{Verbatim}
\subsubsection*{Example 2: Replace non-NULL values}

Replace Null values in the \verb|birthday| field with the string \verb|"no value"| and change non-null values to the string ¥verb|"value"|, and rename the output column as \verb|entry|.


\begin{Verbatim}[baselinestretch=0.7,frame=single]
$ mnullto f=birthday:entry v="no value" O="value" i=dat1.csv o=rsl2.csv
#END# kgnullto O=value f=birthday:entry i=dat1.csv o=rsl2.csv v=no value
$ more rsl2.csv
customer,entry
A,value
B,no value
C,value
D,no value
E,no value
\end{Verbatim}
\subsubsection*{Example 3: Add new column}

Replace Null values in the \verb|birthday| field with the string \verb|"no value"| and change non-null values to the string \verb|"value"|. Output the replacement strings in a new column named \verb|entry|.


\begin{Verbatim}[baselinestretch=0.7,frame=single]
$ mnullto f=birthday:entry v="no value" O="value" -A i=dat1.csv o=rsl3.csv
#END# kgnullto -A O=value f=birthday:entry i=dat1.csv o=rsl3.csv v=no value
$ more rsl3.csv
customer,birthday,entry
A,19690103,value
B,,no value
C,19500501,value
D,,no value
E,,no value
\end{Verbatim}
\subsubsection*{Example 4: Replace values in previous row}



\begin{Verbatim}[baselinestretch=0.7,frame=single]
$ more dat2.csv
id,date
A,19690103
B,
C,19500501
D,
E,
$ mnullto f=date -p i=dat2.csv o=rsl4.csv
#END# kgnullto -p f=date i=dat2.csv o=rsl4.csv
$ more rsl4.csv
id,date
A,19690103
B,19690103
C,19500501
D,19500501
E,19500501
\end{Verbatim}

\subsection*{関連コマンド}
\hyperref[sect:mdelnull]{mdelnull} : 置換ではなく、行を削除したい場合はこちら。

\hyperref[sect:mchgstr]{mchgstr} : NULL値でなく文字列を置換したい場合に使用する。

%\end{document}
