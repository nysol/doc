
%\begin{document}

\section{自動並べ替え機能の追加\label{sect:changes3}}
%\section{GitHuB自動並べ替え機能の追加\label{sect:changes}}

MCMDをNYSOLパッケージから独立させGitHuBに登録したことにより、
バージョンを2.xから3.0へ上げた。
コマンド仕様の大きな変更はなく、以下に示す細かな点が改善されている。

\subsection{-paramsパラメータの追加}
全コマンド共通で-paramsパラメータを追加した。
主にシステム利用を目的としたもので、指定できるパラメータの一覧を標準出力に出力する。

\begin{Verbatim}[baselinestretch=0.7,frame=single]
$ mcut -params
-assert_diffSize
-assert_nullin
-nfn
-nfni
-nfno
-params
-r
-x
f=
i=
o=
precision=
tmpPath=
\end{Verbatim}

\if0
 ・環境変数 KG_msgTimebyfsec追加
   => この環境変数を設定するとEND等の時間が
         マイクロ秒(ミリ秒)まで表示されます

○ mchkcsv
  ・BOM対応
    => UTF BOMが除去されます
  ・diag英語版
   => -diagの内容が英語で出力されます
   => 日本語は-diaglで出力されます(いままでのもの)

○ mcal
  ・複数項目対応
         => c=に複数指定できるように変更
            (a=に指定した分の項目名が必要)
  ・マイクロ秒対応
       => $t{},#t{},0tでマイクロ秒までを扱えるように変更
          20120101121212.123456 のような小数
          (小数点以下6桁まで)を含むものを扱えます
          それに伴って
          diffusecond(T,T)
          tuseconds(T)
          usecond(T)
          useconds(T)
          unow()
          の関数が追加されています
          (機能は"u"を除いた関数と同じで
          マイクロ秒(ミリ秒)まで計算されます)
  ・if(B,B,B)
         => マニュアルには書いてあるが使えていなかったので修正

○ mcat
  ・-skip_zero
      =>-nfnを指定していない場合でも0バイトファイルでエラーにならない
  ・flist=
      =>ファイルリストが保存されているCSVファイルを指定してmcatする
        flist=xxx:fldで指定する fldはファイル名のある項目名

  ・kv=
      => kv=1,2のように指定し、パス名自体からkey-value抜き出し
      データとして付加する
           mcat i=./k_1_f_1/v_1/dat.csv kv=1,2の場合だと
           v,k,f
           1,1,1
           が元のデータに付加される
           (_がkeyとvalueの区切りになります)

\fi
%\end{document}
