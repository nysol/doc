
%\documentclass[a4paper]{jsbook}
%\usepackage{mcmd_jp}
%\begin{document}

\section{mdformat 日付時刻抽出\label{sect:mdformat}}
\index{mdformat@mdformat}

他のシステムからエクスポートしたCSVデータでは、
日付時刻項目にスラッシュ記号やコロン記号が入っていることが多く、
また月日や時が1桁で格納されている場合もある
(例:\verb|2014/7/18 1:57|)。
このような項目をそのままMCMDで扱おうとすると、
日付計算や並べ替え、範囲検索がうまくいかない。

\verb|mdformat|コマンドを使うことで、
\verb|f=|パラメータで指定した項目から、
\verb|c=|パラメータで指定したフォーマットに従って
年月日・時分秒を抽出し、MCMDで扱うことが可能な
\hyperref[sect:datetime]{日付型や時刻型}
に変換することができる。

\subsection*{書式}
\verb|mdformat c= f= [-A] | 
\hyperref[sect:option_i]{[i=]}
\hyperref[sect:option_o]{[o=]}
\hyperref[sect:option_assert_diffSize]{[-assert\_diffSize]}
\hyperref[sect:option_assert_nullin]{[-assert\_nullin]}
\hyperref[sect:option_assert_nullout]{[-assert\_nullout]}
\hyperref[sect:option_nfn]{[-nfn]} 
\hyperref[sect:option_nfno]{[-nfno]}  
\hyperref[sect:option_x]{[-x]}
\hyperref[sect:option_option_tmppath]{[tmpPath=]}
\hyperref[sect:option_precision]{[precision=]}
\verb|[-params]|
\verb|[--help]|
\verb|[--helpl]|
\verb|[--version]|\\

\subsection*{パラメータ}
\begin{table}[htbp]
%\begin{center}
{\small
\begin{tabular}{ll}
\verb|i=|    & 入力ファイル名を指定する。\\
\verb|o=|    & 出力ファイル名を指定する。\\
\verb|f=|    & 抽出対象となる項目名リスト(複数項目指定可)を指定する。\\
\verb|c=|    & 文字列のフォーマットを指定する。フォーマットの変換指定文字参照 \\
\verb|-A|    & このオプションにより、指定した項目を置き換えるのではなく、新たに項目が追加される。\\
\end{tabular} 
}
\end{table} 

\subsection*{フォーマットの変換指定文字}
\verb|c=|パラメータで利用可能な変換指定文字を表\ref{tbl:mdformat}に示す。

\begin{table}[htbp]
\begin{center}
\caption{変換指定文字\label{tbl:mdformat}}
{\small
\renewcommand{\arraystretch}{1.5}
\begin{tabular}{llll}
\hline
変換指定文字 & 意味 \\
\hline
\%Y           & 西暦を表す数(4桁以内) \\
\%y           & 西暦を表す数(2桁以内)  \\
\%m           & 月を表す数字(2桁以内)  \\
\%d           & 日を表す数字(2桁以内)  \\
\%H           & 時間(2桁以内) \\
\%M           & 分(2桁以内)   \\
\%S           & 秒(2桁以内) \\
\%s           & 秒.マイクロ秒 \\
\hline \\
\end{tabular}
}      
\end{center}
\end{table}


\subsection*{利用例}
\subsubsection*{例1: 基本例}

\verb|fld|項目から日付・時刻を抽出し変換する。
\verb|fld|項目には「date:年月日 time:時分秒.マイクロ秒」の形式で日付・時刻が格納されているので、
\verb|c=|パラメータには「\verb|date:%Y%m%d time:%H%M%s|」と指定している。


\begin{Verbatim}[baselinestretch=0.7,frame=single]
$ more dat1.csv
fld
date:20120304 time:121212
date:20101204 time:011309.1234
$ mdformat f=fld c='date:%Y%m%d time:%H%M%s' i=dat1.csv o=rsl1.csv
#END# kgdformat c=date:%Y%m%d time:%H%M%s f=fld i=dat1.csv o=rsl1.csv
$ more rsl1.csv
fld
20120304121212
20101204011309.1234
\end{Verbatim}
\subsubsection*{例2: 項目の追加}

\verb|fld1|項目、\verb|fld2|項目には「年/月/日」形式で日付が格納されているので、
\verb|c=|パラメータには「\verb|%Y/%m/%d|」と指定している。
\verb|-A|オプションを使用し、変換結果を新たな\verb|f1|、\verb|f2|項目に抽出する。


\begin{Verbatim}[baselinestretch=0.7,frame=single]
$ more dat2.csv
fld,fld2
2010/11/20,2010/11/21
2010/1/1,2010/1/2
2011/01/01,2010/01/02
2010/1/01,2010/1/02
$ mdformat f=fld:f1,fld2:f2 c=%Y/%m/%d i=dat2.csv -A o=rsl2.csv
#END# kgdformat -A c=%Y/%m/%d f=fld:f1,fld2:f2 i=dat2.csv o=rsl2.csv
$ more rsl2.csv
fld,fld2,f1,f2
2010/11/20,2010/11/21,20101120,20101121
2010/1/1,2010/1/2,20100101,20100102
2011/01/01,2010/01/02,20110101,20100102
2010/1/01,2010/1/02,20100101,20100102
\end{Verbatim}
\subsubsection*{例3: 抽出がうまくいかない例}

\verb|fld|項目には「年 月 日 時:分:秒」形式で日付が格納されているので、
\verb|c=|パラメータには「\verb|%Y %m %d %H:%M:%S|」と指定している。
しかし形式が異なる行は抽出に失敗している。


\begin{Verbatim}[baselinestretch=0.7,frame=single]
$ more dat3.csv
fld
2010 11 20 12:34:56
2011 01 01 23:34:56
2010  1 01 123455
$ mdformat f=fld:f1 c='%Y %m %d %H:%M:%S' i=dat3.csv -A o=rsl3.csv
#END# kgdformat -A c=%Y %m %d %H:%M:%S f=fld:f1 i=dat3.csv o=rsl3.csv
$ more rsl3.csv
fld,f1
2010 11 20 12:34:56,20101120123456
2011 01 01 23:34:56,20110101233456
2010  1 01 123455,
\end{Verbatim}

\subsection*{関連コマンド}

%\end{document}
