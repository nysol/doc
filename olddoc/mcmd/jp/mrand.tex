
%\documentclass[a4paper]{jsbook}
%\usepackage{mcmd_jp}
%\begin{document}

\section{mrand 擬似乱数\label{sect:mrand}}
\index{mrand@mrand}
0.0から1.0の範囲の実数の擬似乱数、もしくは範囲指定による整数の擬似乱数を生成し、\verb|a=|パラメータで指定した項目名で出力する。

乱数の生成にはメルセンヌ・ツイスター法を利用している
(\href{http://www.math.sci.hiroshima-u.ac.jp/~m-mat/MT/emt.html}{原作者のページ}
, \href{http://www.boost.org/doc/libs/1_54_0/doc/html/boost_random.html}{boostライブラリ})。


\subsection*{書式}
\verb/mrand [k=] a= [max=] [min=] [S=] [-int]/
\hyperref[sect:option_i]{[i=]}
\hyperref[sect:option_o]{[o=]}
\hyperref[sect:option_assert_diffSize]{[-assert\_diffSize]}
\hyperref[sect:option_assert_nullkey]{[-assert\_nullkey]}
\hyperref[sect:option_nfn]{[-nfn]} 
\hyperref[sect:option_nfno]{[-nfno]}  
\hyperref[sect:option_x]{[-x]}
\hyperref[sect:option_q]{[-q]}  
\hyperref[sect:option_option_tmppath]{[tmpPath=]}
\hyperref[sect:option_precision]{[precision=]}
\verb|[-params]|
\verb|[--help]|
\verb|[--helpl]|
\verb|[--version]|\\

\subsection*{パラメータ}
\begin{table}[htbp]
%\begin{center}
{\small
\begin{tabular}{ll}
%\verb|k=|   & 指定したキー項目について、同じキー値には同じ乱数値が振られる。【\hyperref[sect:option_k]{集計キーブレイク処理}】\\
\verb|i=|    & 入力ファイル名を指定する。\\
\verb|o=|    & 出力ファイル名を指定する。\\
\verb|k=|   & 指定したキー項目について、同じキー値には同じ乱数値が振られる。\\
\verb|a=|   & 新たに追加される項目の名前を指定する。【但し、-nfn,-nfnoオプション指定時は必要なし】\\
\verb|max=| & 乱数の最大値【デフォルト値:INT\_MAX】\\
            & 0〜$2^{32}$(約21億)の範囲の整数が指定可能\\
            & このパラメータを指定するときは\verb|-int|も指定しなければならない。\\
\verb|min=| & 整数乱数の最小値【デフォルト値:0】\\
            & 0〜$2^{32}$(約21億)の範囲の整数が指定可能\\
            & このパラメータを指定するときは\verb|-int|も指定しなければならない。\\
\verb|S=|   & 乱数の種【デフォルト値:現在時刻】\\
            & 同じ乱数の種を指定すれば、同じ乱数系列となる。\\
            & \verb|S=|を指定しなければ、時刻(ミリ(1/1000秒単位)に応じた異なる乱数の種が利用される。\\
            & 指定可能な乱数の種(-2147483648〜2147483647)\\
\verb|-int| & 整数乱数を生成\\
\end{tabular} 
}
\end{table} 

\subsection*{利用例}
\subsubsection*{例1: 基本例}

0.0から1.0の範囲の実数乱数を生成する。


\begin{Verbatim}[baselinestretch=0.7,frame=single]
$ more dat1.csv
顧客
A
B
C
D
E
$ mrand a=rand i=dat1.csv o=rsl1.csv
#END# kgrand a=rand i=dat1.csv o=rsl1.csv
$ more rsl1.csv
顧客,rand
A,0.5477480278
B,0.3656605978
C,0.04199989187
D,0.9302980842
E,0.3048081712
\end{Verbatim}
\subsubsection*{例2: 基本例2}

-intで整数乱数


\begin{Verbatim}[baselinestretch=0.7,frame=single]
$ mrand a=rand -int i=dat1.csv o=rsl2.csv
#END# kgrand -int a=rand i=dat1.csv o=rsl2.csv
$ more rsl2.csv
顧客,rand
A,1083858011
B,549225333
C,2044140724
D,114682354
E,65869181
\end{Verbatim}
\subsubsection*{例3: 最小値、最大値を決めた乱数の生成}

最小値が10、最大値が100の整数の乱数を生成し、\verb|rand|という項目名で出力する。


\begin{Verbatim}[baselinestretch=0.7,frame=single]
$ mrand a=rand -int min=10 max=100 S=1 i=dat1.csv o=rsl3.csv
#END# kgrand -int S=1 a=rand i=dat1.csv max=100 min=10 o=rsl3.csv
$ more rsl3.csv
顧客,rand
A,47
B,100
C,75
D,94
E,10
\end{Verbatim}
\subsubsection*{例4: キー単位の乱数生成}

以下の例は、顧客\verb|A,B,C,D|の4人について同じ顧客には同じ乱数値を振る。


\begin{Verbatim}[baselinestretch=0.7,frame=single]
$ more dat2.csv
顧客
A
A
A
B
B
C
D
D
D
$ mrand k=顧客 -int min=0 max=1 a=rand i=dat2.csv o=rsl4.csv
#END# kgrand -int a=rand i=dat2.csv k=顧客 max=1 min=0 o=rsl4.csv
$ more rsl4.csv
顧客%0,rand
A,1
A,1
A,1
B,0
B,0
C,0
D,1
D,1
D,1
\end{Verbatim}


\subsection*{関連コマンド}
\hyperref[sect:mselrand] {mselrand} : ランダムに行を選択する。

\hyperref[sect:mnewrand] {mnewrand} : 入力ファイルなしに、乱数データを新たに生成する。


%\end{document}
