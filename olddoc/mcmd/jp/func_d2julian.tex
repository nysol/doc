
%\begin{document}

\section{d2julian: date-ユリウス通日変更\label{sect:d2julian}}
\index{d2julian@d2julian}

書式: d2julian($date$)

日付からユリウス通日を求める 
\begin{verbatim}
ユリウス通日は紀元前4713年1月1日正午(世界標準時による)からの日数である。
ただしグリゴリオ歴は1400-1-1~9999-12-31が有効範囲のため、
それに対応するユリウス通日の有効値範囲は2232300~5373484となる。
その範囲外ではNULL出力となる。
\end{verbatim}

\subsection*{利用方法}
\begin{verbatim}
c=d2julian(0d20080822)
\end{verbatim}

\subsection*{利用例}

\begin{table}[hbt]
\begin{center}
 \caption{入力データ}
  \begin{tabular}{|c|c|c|} \hline
日付&時間&数\\ \hline\hline
20020824&20020824145408&10660\\ \hline
20020622&20020622173449&22740\\ \hline
20020824&20020824145408&14800\\ \hline
20021009&20021009095743&54510\\ \hline
20020121&20020121173449&18750\\ \hline
  \end{tabular}
  \end{center}
\end{table}

上記のデータを用いて、それぞれの場合の利用例を以下に示す。

\subsubsection*{実行例1)}
"日付"項目を入力としてユリウス通日に変換し、
新たに"ユリウス通日"をいう項目を作成して出力している。
\begin{verbatim}
------------------------------------------------
mcal c='d2julian($d{日付})' a="ユリウス通日" i=date.csv o=od2julian.csv
------------------------------------------------
\end{verbatim}

\begin{table}[hbt]
\begin{center}
 \caption{出力ファイル(od2julian.csv)}
  \begin{tabular}{|c|c|c|c|} \hline
日付&時間&数&ユリウス通日\\ \hline\hline
20020824&20020824145408&10660&2452511\\ \hline
20020622&20020622173449&22740&2452448\\ \hline
20020824&20020824145408&14800&2452511\\ \hline
20021009&20021009095743&54510&2452557\\ \hline
20020121&20020121173449&18750&2452296\\ \hline
  \end{tabular}
  \end{center}
\end{table}

\href{run:hizuke.pdf}{mcal 日付時間関連へ戻る}\\
%\end{document}

