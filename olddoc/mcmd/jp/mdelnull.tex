
%\documentclass[a4paper]{jsbook}
%\usepackage{mcmd_jp}
%\begin{document}

\section{mdelnull NULL値行の削除\label{sect:mdelnull}}
\index{mdelnull@mdelnull}
\verb|f=|パラメータで指定した項目について、NULL値が含まれる行を削除(撰択)する。\\

\subsection*{書式}
\verb|mdelnull f= [k=] [u=] [-F] [-r] [-R]| 
\hyperref[sect:option_i]{[i=]}
\hyperref[sect:option_o]{[o=]}
\hyperref[sect:option_bufcount]{[bufcount=]} 
\hyperref[sect:option_assert_diffSize]{[-assert\_diffSize]}
\hyperref[sect:option_assert_nullkey]{[-assert\_nullkey]}
\hyperref[sect:option_nfn]{[-nfn]} 
\hyperref[sect:option_nfno]{[-nfno]}  
\hyperref[sect:option_x]{[-x]}
\hyperref[sect:option_q]{[-q]}
\hyperref[sect:option_option_tmppath]{[tmpPath=]}
\hyperref[sect:option_precision]{[precision=]}
\verb|[-params]|
\verb|[--help]|
\verb|[--helpl]|
\verb|[--version]|\\

\subsection*{パラメータ}
\begin{table}[htbp]
%\begin{center}
{\small
\begin{tabular}{ll}
\verb|i=|    & 入力ファイル名を指定する。\\
\verb|o=|    & 出力ファイル名を指定する。\\
\verb|f=|    & NULL値の検索対象となる項目名(複数項目指定可)を指定する。 \\
\verb|k=|    & 削除(撰択)する単位となるキー項目名(複数項目指定可)を指定する。\\
\verb|u=|    & 不一致データ出力ファイル名を指定する。\\
\verb|bufcount=| & バッファのサイズ数を指定する。 \\
\verb|-F|    & 項目間AND条件\\
             & \verb|f=|パラメータで複数項目を指定した場合、その全ての値がNULL値の行を削除(撰択)する。\\
\verb|-r|    & 条件反転\\
             & 削除ではなく選択する。\\
\verb|-R|    & レコード間AND条件\\
             & \verb|k=|パラメータを指定した場合、その全ての行がNULL値の行を削除(撰択)する。\\
\end{tabular} 
}
\end{table} 

\subsection*{利用例}
\subsubsection*{Example 1: Basic Example}

Remove records where \verb|Quantity| and \verb|Amount| contain null values. Save records with null values to a separate file \verb|oth.csv|.


\begin{Verbatim}[baselinestretch=0.7,frame=single]
$ more dat1.csv
Customer,Quantity,Amount
A,1,10
A,,20
B,1,15
B,3,
C,1,20
$ mdelnull f=Quantity,Amount u=oth.csv i=dat1.csv o=rsl1.csv
#END# kgdelnull f=Quantity,Amount i=dat1.csv o=rsl1.csv u=oth.csv
$ more rsl1.csv
Customer,Quantity,Amount
A,1,10
B,1,15
C,1,20
$ more oth.csv
Customer,Quantity,Amount
A,,20
B,3,
\end{Verbatim}
\subsubsection*{Example 2: Select rows with NULL values}

Select records with NULL values by specifying \verb|-r|.


\begin{Verbatim}[baselinestretch=0.7,frame=single]
$ mdelnull f=Quantity,Amount -r i=dat1.csv o=rsl2.csv
#END# kgdelnull -r f=Quantity,Amount i=dat1.csv o=rsl2.csv
$ more rsl2.csv
Customer,Quantity,Amount
A,,20
B,3,
\end{Verbatim}
\subsubsection*{Example 3: Remove records with the same key if any record contains NULL values}

Remove based on the aggregate key specified at \verb|k=|.
Remove records where \verb|Quantity| and \verb|Amount| contain null values for each customer.


\begin{Verbatim}[baselinestretch=0.7,frame=single]
$ mdelnull k=Customer f=Quantity,Amount i=dat1.csv o=rsl3.csv
#END# kgdelnull f=Quantity,Amount i=dat1.csv k=Customer o=rsl3.csv
$ more rsl3.csv
Customer%0,Quantity,Amount
C,1,20
\end{Verbatim}
\subsubsection*{Example 4: AND condition between fields}

Remove the record where both \verb|Quantity| and \verb|Amount| fields contain null values.


\begin{Verbatim}[baselinestretch=0.7,frame=single]
$ more dat2.csv
Customer,Quantity,Amount
A,1,10
A,,
B,1,15
B,3,
C,1,20
$ mdelnull f=Quantity,Amount -F i=dat2.csv o=rsl4.csv
#END# kgdelnull -F f=Quantity,Amount i=dat2.csv o=rsl4.csv
$ more rsl4.csv
Customer,Quantity,Amount
A,1,10
B,1,15
B,3,
C,1,20
\end{Verbatim}
\subsubsection*{Example 5: AND condition between records}

Remove the \verb|Customer| record if all values in \verb|Quantity| is NULL.


\begin{Verbatim}[baselinestretch=0.7,frame=single]
$ mdelnull k=Customer f=Quantity -R i=dat1.csv o=rsl5.csv
#END# kgdelnull -R f=Quantity i=dat1.csv k=Customer o=rsl5.csv
$ more rsl5.csv
Customer%0,Quantity,Amount
A,1,10
A,,20
B,1,15
B,3,
C,1,20
\end{Verbatim}

\subsection*{関連コマンド}
\hyperref[sect:mnullto]{mnullto} : NULL値を含む行を削除するのではなく、NULL値を指定の文字列に変換する。

%\end{document}
