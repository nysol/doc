
%\documentclass[a4paper]{jsbook}
%\usepackage{mcmd_jp}
%\begin{document}

\section{mcross クロス集計\label{sect:mcross}}
\index{mcross@mcross}
クロス集計を行う。
\verb|s=|で指定した項目の値が項目名となるように横に展開され、
\verb|k=|で指定した値が行idとなり、
\verb|f=|で指定した項目がセルとして出力される。

\subsection*{書式}
\verb|mcross f= s= [a=] [k=] [v=] |
\hyperref[sect:option_i]{[i=]}
\hyperref[sect:option_o]{[o=]}
\hyperref[sect:option_assert_diffSize]{[-assert\_diffSize]}
\hyperref[sect:option_assert_nullkey]{[-assert\_nullkey]}
\hyperref[sect:option_assert_nullin]{[-assert\_nullin]}
\hyperref[sect:option_assert_nullout]{[-assert\_nullout]}
\hyperref[sect:option_nfn]{[-nfn]} 
\hyperref[sect:option_nfno]{[-nfno]}  
\hyperref[sect:option_x]{[-x]}
\hyperref[sect:option_q]{[-q]}
\hyperref[sect:option_option_tmppath]{[tmpPath=]}
\hyperref[sect:option_precision]{[precision=]}
\verb|[-params]|
\verb|[--help]|
\verb|[--helpl]|
\verb|[--version]|\\

\subsection*{パラメータ}
\begin{table}[htbp]
%\begin{center}
{\small
\begin{tabular}{ll}
\verb|i=|    & 入力ファイル名を指定する。\\
\verb|o=|    & 出力ファイル名を指定する。\\
\verb|f=|    & ここで指定された項目の値がセルの値として出力される。\\
             & 複数項目指定すると、複数行に展開される。\\
             & それら複数行を識別するための項目として\verb|fld|項目が出力され、\\
             & \verb|f=|で指定した項目名が値として出力される。 \\
             & この\verb|fld|という項目名を変更したい場合は\verb|a=|パラメータを使う。\\
\verb|s=|    & 列項目名に展開する項目を指定する。\\
             & ここで指定された項目の値が項目名として出力される。\\
             %& ここで指定する項目の値は単一化(ユニーク)されていなければならない。\\
             %& \verb|-r|オプションを指定した場合は、複数の列項目名を指定し、\\
             %& それらの項目名が\verb|fld|という項目名のデータとして出力される。\\
             %& また、\verb|fld|項目の項目値名を変更したい場合は:(コロン)で項目値名を変更することができる。\\
             %& 例)\verb|s="2008*:日付&"| \\
\verb|a=|    & \verb|f=|で指定した項目名がデータとして展開される項目名を指定する。\\
             & 省略した場合は\verb|fld|という項目名で出力される。\\
\verb|k=|    & キー項目名リスト\\
             & ここで指定した項目を単位に横展開をおこなう。\\
\verb|v=|    & NULL値置換文字列\\
             & NULL値があった場合、\verb|v=|パラメータで指定する置換文字列により、項目の値を置換する。\\
\end{tabular} 
}
\end{table} 

\subsection*{利用例}
\subsubsection*{Example 1: Basic Example}

Expand the array of \verb|date| horizontally and itemize \verb|quantity|
to the corresponding \verb|item|.


\begin{Verbatim}[baselinestretch=0.7,frame=single]
$ more dat1.csv
item,date,quantity,price
A,20081201,1,10
A,20081202,2,20
A,20081203,3,30
B,20081201,4,40
B,20081203,5,50
$ mcross k=item f=quantity s=date i=dat1.csv o=rsl1.csv
#END# kgcross f=quantity i=dat1.csv k=item o=rsl1.csv s=date
$ more rsl1.csv
item%0,fld,20081201,20081202,20081203
A,quantity,1,2,3
B,quantity,4,,5
\end{Verbatim}
\subsubsection*{Example 2: Restore the original input data}

Restore the output from Example 1 to the original input data with \verb|mcross|.


\begin{Verbatim}[baselinestretch=0.7,frame=single]
$ more rsl1.csv
item%0,fld,20081201,20081202,20081203
A,quantity,1,2,3
B,quantity,4,,5
$ mcross k=item f=2008* s=fld a=date i=rsl1.csv o=rsl2.csv
#END# kgcross a=date f=2008* i=rsl1.csv k=item o=rsl2.csv s=fld
$ more rsl2.csv
item%0,date,quantity
A,20081201,1
A,20081202,2
A,20081203,3
B,20081201,4
B,20081202,
B,20081203,5
\end{Verbatim}
\subsubsection*{Example 3: Crosstab with multiple fields}

Display crosstab results on two fields \verb|quantity,price|.


\begin{Verbatim}[baselinestretch=0.7,frame=single]
$ mcross k=item f=quantity,price s=date i=dat1.csv o=rsl3.csv
#END# kgcross f=quantity,price i=dat1.csv k=item o=rsl3.csv s=date
$ more rsl3.csv
item%0,fld,20081201,20081202,20081203
A,quantity,1,2,3
A,price,10,20,30
B,quantity,4,,5
B,price,40,,50
\end{Verbatim}
\subsubsection*{Example 4: Reverse data sequence}

Restore the sequence of the items that was expanded horizontally.


\begin{Verbatim}[baselinestretch=0.7,frame=single]
$ mcross k=item f=quantity,price s=date%r i=dat1.csv o=rsl4.csv
#END# kgcross f=quantity,price i=dat1.csv k=item o=rsl4.csv s=date%r
$ more rsl4.csv
item%0,fld,20081203,20081202,20081201
A,quantity,3,2,1
A,price,30,20,10
B,quantity,5,,4
B,price,50,,40
\end{Verbatim}


\subsection*{関連コマンド}
\hyperref[sect:mtra]{mtra} : 横展開するイメージは同じだが、\verb|mtra|は1つのベクトル項目として出力する。

%\end{document}
