%\begin{document}

%\section{書式について\label{sect:format}}
全コマンドに共通して示される書式の意味について解説する。
全てのコマンドリファレンスにおいて、書式は以下の例のように示される。

\begin{table}[htbp]
\begin{center}
\begin{tabular}{l}
\hline
{\large 書式} \\
\verb/  mjoin k= [f=] [K=] [-n] [-N] m=|/ 
\hyperref[sect:option_i]{i=}
\hyperref[sect:option_o]{[o=]}
\hyperref[sect:option_nfn]{[-nfn]} 
\hyperref[sect:option_nfno]{[-nfno]}  
\hyperref[sect:option_x]{[-x]}
\verb|[--help]|
\verb|[--helpl]|
\verb|[--version]|\\

\\
{\large パラメータ} \\
\verb|  k=  ここで指定した入力データの項目とK=パラメータで指定された|\\
\verb|      参照データの項目が同じ行の項目結合が行われる。|\\
\verb|      NULL値は,参照ファイルのK=で指定した項目のどの値にもマッチしない値として扱われる。|\\
\verb|  f=  結合する参照ファイル上の項目名リストを指定する。|\\
\verb|      省略するとキー項目を除いた全ての項目が結合される。|\\
\verb|  :                  :|\\
\hline
\end{tabular} 
\end{center}
\end{table} 

コマンド名に続いて、そのコマンドに指定可能なパラメータおよびオプションが列挙されている。
多くのコマンドで共通する\verb|i=,-nfn|などのパラメータやオプションについては、別の節にてまとめて記述されており、
その節へのリンクが張られている。
そして、各パラメータの説明が、書式の下に記述される。

\verb|[f=]|のように四角括弧で囲われたパラメータは省略可能であることを意味する。
一方で、\verb|k=|のように四角括弧で囲われていないパラメータは必須であることを意味する。
また\verb/[to=|size=]/のように縦棒で区切られたパラメータは、
いずれか一つのパラメータしか指定できないことを意味し、括弧で囲われていることで省略することも可能という意味になる
(例えば\hyperref[sect:mbest]{mbestコマンド})。
一方で\verb/m=|i=/のように四角括弧で囲われていなければ、必ずいずれかのパラメータは指定しなければならない、
すなわち選択必須のパラメータであることを意味する(例えば、上記のmjoinコマンド)。

また、あるオプションを指定していたときのみ必須となるパラメータなど、より条件が複雑なパラメータもあるが、
それらは、各パラメータの説明欄にて解説している。

%\end{document}
