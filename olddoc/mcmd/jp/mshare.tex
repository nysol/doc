
%\documentclass[a4paper]{jsbook}
%\usepackage{mcmd_jp}
%\begin{document}

\section{mshare 構成比の計算\label{sect:mshare}}
\index{mshare@mshare}
\verb|f=|パラメータで指定した項目の構成比を計算し、新しい項目として追加する。\\

\subsection*{書式}
\verb|mshare f= [k=] |
\hyperref[sect:option_i]{[i=]}
\hyperref[sect:option_o]{[o=]}
\hyperref[sect:option_assert_diffSize]{[-assert\_diffSize]}
\hyperref[sect:option_assert_nullkey]{[-assert\_nullkey]}
\hyperref[sect:option_assert_nullin]{[-assert\_nullin]}
\hyperref[sect:option_assert_nullout]{[-assert\_nullout]}
\hyperref[sect:option_nfn]{[-nfn]} 
\hyperref[sect:option_nfno]{[-nfno]}  
\hyperref[sect:option_x]{[-x]}
\hyperref[sect:option_x]{[-q]}
\hyperref[sect:option_option_tmppath]{[tmpPath=]}
\hyperref[sect:option_precision]{[precision=]}
\verb|[-params]|
\verb|[--help]|
\verb|[--helpl]|
\verb|[--version]|\\

\subsection*{パラメータ}
\begin{table}[htbp]
%\begin{center}
{\small
\begin{tabular}{ll}
\verb|i=|    & 入力ファイル名を指定する。\\
\verb|o=|    & 出力ファイル名を指定する。\\ 
\verb|f=|    & ここで指定された項目(複数項目指定可)の値のシェアが計算される。 \\
             & :(コロン)で新項目名を指定する必要がある。例)f=数量:数量シェア\\
%\verb|k=|    & シェア計算の単位となる項目名リスト(複数項目指定可)を指定する。【\hyperref[sect:option_k]{集計キーブレイク処理}】\\
\verb|k=|    & シェア計算の単位となる項目名リスト(複数項目指定可)を指定する。\\
             & 省略すると全行同じキーの値として処理される。\\
\end{tabular} 
}
\end{table} 

\subsection*{利用例}
\subsubsection*{Example 1: Basic Example}

Calculate the share of "quantity" and "amount" fields for each "customer". Save the output in columns "volume share" and "share amount”.


\begin{Verbatim}[baselinestretch=0.7,frame=single]
$ more dat1.csv
customer,quantity,amount
A,1,10
A,2,20
B,1,15
B,3,10
B,1,20
$ mshare k=customer f=quantity:qtyShare,amount:amountShare i=dat1.csv o=rsl1.csv
#END# kgshare f=quantity:qtyShare,amount:amountShare i=dat1.csv k=customer o=rsl1.csv
$ more rsl1.csv
customer%0,quantity,amount,qtyShare,amountShare
A,1,10,0.3333333333,0.3333333333
A,2,20,0.6666666667,0.6666666667
B,1,15,0.2,0.3333333333
B,3,10,0.6,0.2222222222
B,1,20,0.2,0.4444444444
\end{Verbatim}

\subsection*{関連コマンド}

%\end{document}
