
%\documentclass[a4paper]{jsbook}
%\usepackage{mcmd_jp}
%\begin{document}

\section{mslide 行ずらし\label{sect:mslide}}
\index{mslide@mslide}
指定した項目の値を指定した行数ずらして出力する。
例えば、日別の株価データにおいて収益率(当日の株価/前日の株価)を計算するなど
レコード間の演算を行いたい場合に利用する。

\if0 #no help# following sentences will not apear on the help document. \fi
典型的な例を表\ref{tbl:mslide_input}〜\ref{tbl:mslide_out3}に示す。

\begin{table}[htbp]
\begin{center}
\begin{tabular}{ccc}

\begin{minipage}{0.18\hsize}
\begin{center}
\caption{入力データ\label{tbl:mslide_input}}
{\small
\begin{tabular}{cc}
\hline
date&val \\
\hline
4/6&1 \\
4/7&2 \\
4/8&3 \\
4/9&4 \\
\hline

\end{tabular}
}
\end{center}
\end{minipage}

\begin{minipage}{0.23\hsize}
\begin{center}
\caption{f=val:nextVal\label{tbl:mslide_out1}}
{\small
\begin{tabular}{ccc}
\hline
date&val&nextVal \\
\hline
4/6&1&2 \\
4/7&2&3 \\
4/8&3&4 \\
\hline
\end{tabular}
}
\end{center}
\end{minipage}

\begin{minipage}{0.23\hsize}
\begin{center}
\caption{f=val:nextVal -r\label{tbl:mslide_out2}}
{\small
\begin{tabular}{ccc}
\hline
date&val&nextVal \\
\hline
4/7&2&1 \\
4/8&3&2 \\
4/9&4&3 \\
\hline
\end{tabular}
}
\end{center}
\end{minipage}

\begin{minipage}{0.28\hsize}
\begin{center}
\caption{f=val t=2\label{tbl:mslide_out3}}
{\small
\begin{tabular}{cccc}
\hline
date&val&val1&val2 \\
\hline
4/6&1&2&3 \\
4/7&2&3&4 \\
\hline
\end{tabular}
}
\end{center}
\end{minipage}

\end{tabular}
\end{center}
\end{table}

表\ref{tbl:mslide_input}に示される入力データは日別の集計値が4日分示されており、
スーパーの売上推移や株価推移と考えればよい。
この入力データについて、
日々の増加率(ここでは簡単のために「増加率=翌日の値/当日の値」とする)
を計算することを考える。
入力データに示される日付4/6〜4/9について、
それぞれの日の値(val)を1行上にずらし、
新しい項目(newVal)として出力した結果が表\ref{tbl:mslide_out1}に示されている。
この出力結果に対してmcalコマンドでnextVal/valを計算すれば増加率が求められる。
ちなみに、4/9の行が消えているのは、4/9の行の次の行が存在しないからである。
存在しない時も-nオプションを指定することでNULL値を出力することができる。

表\ref{tbl:mslide_out1}は、下の行の値を上にずらしたが、-rオプションを指定することで、
逆に(上の行の値を下に)ずらすことも可能である(表\ref{tbl:mslide_out2})。
さらに、t=を指定することで、スライドの回数を指定することもできる。
t=2で実行した結果を表\ref{tbl:mslide_out3}に示している。
これは "\verb/mslide f=val:val1 | mslide f=val1:val2/"のように、
mslideを複数回実行するのと同じ効果がある。
なお、\verb|t=|を指定した場合、新たに出力される項目名は、
\verb|f=|で指定した項目名に、1から始まる連番が付与されたものとなる。
また\verb|t=|と\verb|-l|を併用することで、最後にずらした結果のみを出力することも可能である。

mslideの機能はmwindowによく似ている。
mslideはレコード間の演算を項目演算として実現し、
一方で、mwindowはレコード間演算を行集計として実現している。
よって、mslideした後の演算はmcalやmselが主役となり、
一方でmwindowしたのちはmsumやmavgなどのデータ集約のコマンドが主役となる。

\subsection*{書式}
\verb|mslide f= [s=] [k=] [t=] [-r] [-n] [-l] [i=] [o=]|
\hyperref[sect:option_assert_diffSize]{[-assert\_diffSize]}
\hyperref[sect:option_assert_nullkey]{[-assert\_nullkey]}
\hyperref[sect:option_assert_nullin]{[-assert\_nullin]}
\hyperref[sect:option_assert_nullout]{[-assert\_nullout]}
\hyperref[sect:option_nfn]{[-nfn]} 
\hyperref[sect:option_nfno]{[-nfno]}  
\hyperref[sect:option_x]{[-x]}
\hyperref[sect:option_q]{[-q]}
\hyperref[sect:option_option_tmppath]{[tmpPath=]}
\hyperref[sect:option_precision]{[precision=]}
\verb|[-params]|
\verb|[--help]|
\verb|[--helpl]|
\verb|[--version]|\\

\begin{table}[htbp]
%\begin{center}
{\small
\begin{tabular}{ll}
\verb|i=|    & 入力ファイル名を指定する。\\
\verb|o=|    & 出力ファイル名を指定する。\\ 
\verb|f=|    & ずらす対象となる項目名を指定する。複数項目指定可能。 \\
             & \verb|t=|を指定しないときは、コロンに続いて窓キーの項目名を指定しなければならない。\\
\verb|s=|    & ここで指定した項目(複数項目指定可)で並べ替えられた後、行をずらす。 \\
             & \verb|-q|オプションを指定しないとき、\verb|s=|パラメータは必須。\\
%\verb|k=|    & ここで指定された項目の値を単位に処理する。【\hyperref[sect:option_k]{集計キーブレイク処理}】\\
\verb|k=|    & ここで指定された項目の値を単位に処理する。\\
\verb|t=|    & ずらす回数を指定する。省略すれば\verb|t=1|が設定される。 \\
\verb|-r|    & 逆方向に(上の値を下に)ずらす。\\
\verb|-n|    & 次(前)の行がなくてもNULL値を出力する。\\
\verb|-l|    & 最後にずらした結果のみを出力する。\\
\end{tabular} 
}
\end{table} 

\subsection*{利用例}
\subsubsection*{Example 1: Basic Example}



\begin{Verbatim}[baselinestretch=0.7,frame=single]
$ more dat1.csv
date,val
20130406,1
20130407,2
20130408,3
20130409,4
$ mslide s=date f=val:newVal i=dat1.csv o=rsl1.csv
#END# kgslide f=val:newVal i=dat1.csv o=rsl1.csv s=date
$ more rsl1.csv
date%0,val,newVal
20130406,1,2
20130407,2,3
20130408,3,4
\end{Verbatim}
\subsubsection*{Example 2: Slide rows in reverse direction}



\begin{Verbatim}[baselinestretch=0.7,frame=single]
$ mslide s=date f=val:newVal -r i=dat1.csv o=rsl2.csv
#END# kgslide -r f=val:newVal i=dat1.csv o=rsl2.csv s=date
$ more rsl2.csv
date%0,val,newVal
20130407,2,1
20130408,3,2
20130409,4,3
\end{Verbatim}
\subsubsection*{Example 3: Slide records more than once}



\begin{Verbatim}[baselinestretch=0.7,frame=single]
$ mslide s=date f=val t=2 i=dat1.csv o=rsl3.csv
#END# kgslide f=val i=dat1.csv o=rsl3.csv s=date t=2
$ more rsl3.csv
date%0,val,val1,val2
20130406,1,2,3
20130407,2,3,4
\end{Verbatim}
\subsubsection*{Example 4: Display output from the last column shifted}



\begin{Verbatim}[baselinestretch=0.7,frame=single]
$ mslide s=date f=val t=2 -l i=dat1.csv o=rsl4.csv
#END# kgslide -l f=val i=dat1.csv o=rsl4.csv s=date t=2
$ more rsl4.csv
date%0,val,val2
20130406,1,3
20130407,2,4
\end{Verbatim}
\subsubsection*{Example 5: Change multiple field names}



\begin{Verbatim}[baselinestretch=0.7,frame=single]
$ mslide s=date f=date:d_,val:v_ t=2 i=dat1.csv o=rsl5.csv
#END# kgslide f=date:d_,val:v_ i=dat1.csv o=rsl5.csv s=date t=2
$ more rsl5.csv
date%0,val,d_1,d_2,v_1,v_2
20130406,1,20130407,20130408,2,3
20130407,2,20130408,20130409,3,4
\end{Verbatim}

\subsection*{関連コマンド}

%\end{document}
