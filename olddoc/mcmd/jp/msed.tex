
%\documentclass[a4paper]{jsbook}
%\usepackage{mcmd_jp}
%\begin{document}

\section{msed 正規表現による文字列置換\label{sect:msed}}
\index{msed@msed}
\verb|f=|パラメータで指定した項目について、
\verb|c=|パラメータで指定した正規表現に一致する内容を
\verb|v=|パラメータ指定した文字列で置換する。

\subsection*{書式}
\verb|msed c= f= v= [-A] [-g] [-W] |     
\hyperref[sect:option_i]{[i=]}
\hyperref[sect:option_o]{[o=]}
\hyperref[sect:option_assert_diffSize]{[-assert\_diffSize]}
\hyperref[sect:option_assert_nullin]{[-assert\_nullin]}
\hyperref[sect:option_assert_nullout]{[-assert\_nullout]}
\hyperref[sect:option_nfn]{[-nfn]} 
\hyperref[sect:option_nfno]{[-nfno]}  
\hyperref[sect:option_x]{[-x]}
\hyperref[sect:option_option_tmppath]{[tmpPath=]}
\hyperref[sect:option_precision]{[precision=]}
\verb|[-params]|
\verb|[--help]|
\verb|[--helpl]|
\verb|[--version]|\\

\subsection*{パラメータ}
\begin{table}[htbp]
%\begin{center}
{\small
\begin{tabular}{ll}
\verb|i=|  & 入力ファイル名を指定する。\\
\verb|o=|  & 出力ファイル名を指定する。\\
\verb|f=|  & 置換対象となる項目名リスト(複数項目指定可)を指定する。\\
\verb|c=|  & 置換したい文字列についての正規表現を指定する。\\
           & 正規表現の使用方法参照\\
\verb|v=|  & \verb|c=|パラメータで指定した正規表現にマッチした部分文字列が、\\
           & ここで指定した文字列に置換される。\\
           & マッチ結果を用いた置換も可能で、指定方法は以下の通り。\\
           & \verb|$&| : マッチした文字列\\
           & \verb|$`| : 置換対象文字列の先頭から、マッチした文字列の直前までの文字列\\
           & \verb|$'| : マッチした文字列の直後から、置換対象文字列の最後までの文字列\\
           & \verb|$N| : N番目の部分マッチ文字列(\verb|N>=1|)\\
\verb|-A|  & このオプションにより、指定した項目を置き換えるのではなく、\\
           & 新たに項目が追加される。\\
\verb|-g|  & 正規表現にマッチする全ての部分文字列を置換対象とする。\\
\verb|-W|  & ワイド文字として正規表現による文字列置換を行う。\\
\end{tabular} 
}
\end{table} 


\subsection*{正規表現の使用方法}
c=パラメータで指定する正規表現を表\ref{tbl:msed_regex1}から表\ref{tbl:msed_regex4}に示す。

\begin{table}[htbp]
\begin{center}
{\small
\caption{正規表現1文字マッチ\label{tbl:msed_regex1}}
\begin{tabular}{l|l|l|l|l}
\hline
正規表現      & 意味                                      & 値例             & \verb|c=,v=|例         & 結果例\\
\hline
\verb|.|      & 任意の一文字                               & \verb|abbbcc|   & \verb|c=. v=X -g|      & \verb|XXXXXX| \\
\verb|[abc]|	& \verb|a,b,c|のいずれか一文字               & \verb|abbbcc|   & \verb|c=[ac] v=X -g|   & \verb|XbbbXX| \\
\verb|[^abc]| & \verb|a,b,c|以外の任意の一文字             & \verb|abbbcc|   & \verb|c=[^ac] v=X -g|  & \verb|aXXXcc| \\
\verb|[a-z]|  & \verb|a|から\verb|z|の範囲の任意の一文字   & \verb|abbbcc|   & \verb|c=[a-b] v=X -g|  & \verb|XXXXcc| \\
\verb|[^a-z]| & \verb|a|から\verb|z|の範囲外の任意の一文字 & \verb|abbbcc|   & \verb|c=[^a-b] v=X -g| & \verb|abbbXX| \\
\verb|\t|     & タブ文字                                   &                 &                        & \\
\verb|\w|     & 単語構成文字(\verb|[0-9a-zA-Z_]|)          & \verb|ab#cd&ef| & \verb|c=\w v=X -g|     & \verb|XX#XX&XX| \\
\verb|\W|     & 単語構成文字以外                           & \verb|ab#cd&ef| & \verb|c=\w v=X -g|     & \verb|abXcdXef| \\
\verb|\s|     & 空白文字(\verb|[ \t]|)                     & \verb|ab cd ef| & \verb|c=\s v=X -g|     & \verb|abXcdXef|\\
\verb|\S|     & 空白文字以外                               & \verb|ab cd ef| & \verb|c=\s v=X -g|     & \verb|XX XX XX|\\
\verb|\d|     & 数字構成文字文字(\verb|[0-9]|)             & \verb|ab12c0|   & \verb|c=\d v=X -g|     & \verb|abXXcX|\\
\verb|\D|     & 数字構成文字文字以外                       & \verb|ab12c0|   & \verb|c=\d v=X -g|     & \verb|XX12X0|\\
\hline
\end{tabular} 
}
\end{center}
\end{table} 


\begin{table}[htbp]
\begin{center}
{\small
\caption{正規表現繰り返し\label{tbl:msed_regex2}}
\begin{tabular}{l|l|l|l|l}
\hline
正規表現      & 意味                                 & 値例            & \verb|c=,v=|例          & 結果例\\
\hline
\verb|a*|     & \verb|a|の0個以上の繰り返し          & \verb|abbbcc|   & \verb|c=ab* v=X|        & \verb|Xcc|   \\
\verb|a+|     & \verb|a|の1個以上の繰り返し          & \verb|abbbcc|   & \verb|c=ab+ v=X|        & \verb|Xcc|   \\
\verb|a?|     & \verb|a|の0個または1個の出現         & \verb|abbbcc|   & \verb|c=ab? v=X|        & \verb|Xbbcc| \\
\verb|a{M,N}| & \verb|a|のM個以上N個以下の繰り返し   & \verb|abbbbbcc| & \verb|c=ab{3,4} v=X|    & \verb|Xbcc|  \\
\verb|a{M}|   & \verb|a|のM個以上の繰り返し          & \verb|abbbbbcc| & \verb|c=ab{3} v=X|      & \verb|Xbbcc| \\
\verb/a|b/   & \verb|a|または\verb|b|               & \verb|abbbc|    & \verb/c=(ab)|(bc) v=X/ & \verb|XbX|   \\
\verb|?|      & 繰り返し記号の後ろに付けて最短マッチ & \verb|abbbc|    & \verb|c=ab*? v=X|       & \verb|Xbbbc| \\
\hline
\end{tabular} 
}
\end{center}
\end{table} 

\begin{table}[htbp]
\begin{center}
{\small
\caption{正規表現位置指定\label{tbl:msed_regex3}}
\begin{tabular}{l|l|l|l|l}
\hline
正規表現  & 意味                       & 値例                 & \verb|c=,v=|例      & 結果例\\
\hline
\verb|^|  & 行頭にマッチする           & \verb|abac|          & \verb|c=^a v=X -g|  & \verb|Xbac|\\
\verb|$|  & 行末にマッチする           & \verb|acac|          & \verb|c=c$ v=X -g|  & \verb|acaX|\\
\verb|\b| & 単語頭または単語末にマッチ & \verb|aac ba ac bac| & \verb|c=\ba v=X -g| & \verb|Xac bX Xc bac|\\
\verb|\B| & 単語中にマッチ             & \verb|aac ba ac bac| & \verb|c=\Ba v=X -g| & \verb|aXc ba ac bXc|\\
\hline
\end{tabular} 
}
\end{center}
\end{table} 

\begin{table}[htbp]
\begin{center}
{\small
\caption{その他\label{tbl:msed_regex4}}
\begin{tabular}{l|l|l|l|l}
\hline
正規表現        & 意味                                      & 値例             & \verb|c=,v=|例         & 結果例\\
\hline
(expr)          & グループ化                                &&& \\			
\verb|\1,..,\9| & 後方参照                                  & \verb|abbcababc| &\verb|c=(ab)(bc)\1 v=x| & \verb|Xabc| \\
\verb|(?=expr)| & \verb|expr|にマッチする直前位置にマッチ   &&& \\
\verb|(?!expr)| & \verb|expr|にマッチしない直前位置にマッチ &&& \\
\hline
\end{tabular} 
}
\end{center}
\end{table} 


\subsection*{利用例}
\subsubsection*{例1: 基本例}

\verb|zipCode|項目の値が00から始まる4桁文字列を\verb|####|に置換する。


\begin{Verbatim}[baselinestretch=0.7,frame=single]
$ more dat1.csv
customer,zipCode
A,6230041
B,6240053
C,6330032
D,6230087
E,6530095
$ msed f=zipCode c=00.. v=#### i=dat1.csv o=rsl1.csv
#END# kgsed c=00.. f=zipCode i=dat1.csv o=rsl1.csv v=####
$ more rsl1.csv
customer,zipCode
A,623####
B,624####
C,633####
D,623####
E,653####
\end{Verbatim}
\subsubsection*{例2: 項目名指定}

\verb|zipCode|の値が00から始まる4桁の数字を\verb|####|に置換し、\verb|zipCode4|という項目名で出力する。


\begin{Verbatim}[baselinestretch=0.7,frame=single]
$ msed f=zipCode:zipCode4 c='00\d\d' v=#### i=dat1.csv o=rsl2.csv
#END# kgsed c=00\d\d f=zipCode:zipCode4 i=dat1.csv o=rsl2.csv v=####
$ more rsl2.csv
customer,zipCode4
A,623####
B,624####
C,633####
D,623####
E,653####
\end{Verbatim}
\subsubsection*{例3: グローバル置換}

\verb|zipCode|の値が\verb|0|を全て\verb|-|にグローバル置換する。


\begin{Verbatim}[baselinestretch=0.7,frame=single]
$ msed f=zipCode c=0 v=- -g i=dat1.csv o=rsl3.csv
#END# kgsed -g c=0 f=zipCode i=dat1.csv o=rsl3.csv v=-
$ more rsl3.csv
customer,zipCode
A,623--41
B,624--53
C,633--32
D,623--87
E,653--95
\end{Verbatim}
\subsubsection*{例4: 部分置換}

\verb|item|の先頭の\verb|fruit|を削除する。
先頭一致(\verb|^|)を指定しているので、最後の行の\verb|grapefruit|は削除されていないことに注意する。


\begin{Verbatim}[baselinestretch=0.7,frame=single]
$ more dat2.csv
item,price
fruit:apple,100
fruit:peach,250
fruit:pineapple,300
fruit:orange,450
fruit:grapefruit,500
$ msed f=item c='^fruit' v= -g i=dat2.csv o=rsl4.csv
#END# kgsed -g c=^fruit f=item i=dat2.csv o=rsl4.csv v=
$ more rsl4.csv
item,price
:apple,100
:peach,250
:pineapple,300
:orange,450
:grapefruit,500
\end{Verbatim}
\subsubsection*{例5: マッチ結果を用いた置換}

\verb|v=|の中で\verb|$&|を用いれば、マッチした文字列(連続した\verb|b|)に置き換わる。


\begin{Verbatim}[baselinestretch=0.7,frame=single]
$ more dat3.csv
str1
abc
abbc
ac
$ msed f=str1 c='b+' v='#$&#' i=dat3.csv o=rsl5.csv
#END# kgsed c=b+ f=str1 i=dat3.csv o=rsl5.csv v=#$&#
$ more rsl5.csv
str1
a#b#c
a#bb#c
ac
\end{Verbatim}
\subsubsection*{例6: グローバルマッチとの組み合せ}

グローバルマッチにすると、個々のマッチ毎に\verb|v=|の内容が評価される。


\begin{Verbatim}[baselinestretch=0.7,frame=single]
$ msed f=str1 c=b v='#$&#' -g i=dat3.csv o=rsl6.csv
#END# kgsed -g c=b f=str1 i=dat3.csv o=rsl6.csv v=#$&#
$ more rsl6.csv
str1
a#b#c
a#b##b#c
ac
\end{Verbatim}
\subsubsection*{例7: プレフィックス置換}

\verb|$`|にて、マッチした箇所の前の文字列(プレフィックス)に置換される。


\begin{Verbatim}[baselinestretch=0.7,frame=single]
$ msed f=str1 c=b v='#$`#' i=dat3.csv o=rsl7.csv
#END# kgsed c=b f=str1 i=dat3.csv o=rsl7.csv v=#$`#
$ more rsl7.csv
str1
a#a#c
a#a#bc
ac
\end{Verbatim}
\subsubsection*{例8: サフィックス置換}

\verb|$'|にて、マッチした箇所の後の文字列(サフィックス)に置換される。


\begin{Verbatim}[baselinestretch=0.7,frame=single]
$ msed f=str1 c=b v="#$'#" i=dat3.csv o=rsl8.csv
#END# kgsed c=b f=str1 i=dat3.csv o=rsl8.csv v=#$'#
$ more rsl8.csv
str1
a#c#c
a#bc#bc
ac
\end{Verbatim}

\subsection*{関連コマンド}
\hyperref[sect:mchgstr] {mchgstr} : 単純な文字列マッチによる置換であればこのコマンドを利用する。

\hyperref[sect:mcal] {mcal} : 正規表現を扱う関数がいくつか用意されている。

%\end{document}
