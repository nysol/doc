
%\documentclass[a4paper]{jsbook}
%\usepackage{mcmd_jp}
%\begin{document}

\section{mkeybreak キーブレイク箇所\label{sect:mkeybreak}}
\index{mkeybreak@mkeybreak}
\verb|k=|パラメータで指定した項目をキー項目について、先頭と終端に印を付ける。
先頭は\verb|top|項目に、終端は\verb|bot|項目に\verb|1|を出力する。
先頭/終端でない行はNULL値を出力する。

\subsection*{書式}
\verb|mkeybreak k= [s=] [a=]|
\hyperref[sect:option_i]{[i=]}
\hyperref[sect:option_o]{[o=]}
\hyperref[sect:option_assert_diffSize]{[-assert\_diffSize]}
\hyperref[sect:option_assert_nullkey]{[-assert\_nullkey]}
\hyperref[sect:option_assert_nullout]{[-assert\_nullout]}
\hyperref[sect:option_nfn]{[-nfn]} 
\hyperref[sect:option_nfno]{[-nfno]}  
\hyperref[sect:option_x]{[-x]}
\hyperref[sect:option_q]{[-q]}
\hyperref[sect:option_option_tmppath]{[tmpPath=]}
\hyperref[sect:option_precision]{[precision=]}
\verb|[-params]|
\verb|[--help]|
\verb|[--helpl]|
\verb|[--version]|\\

\subsection*{パラメータ}
\begin{table}[htbp]
%\begin{center}
{\small
\begin{tabular}{ll}
\verb|i=|    & 入力ファイル名を指定する。\\
\verb|o=|    & 出力ファイル名を指定する。\\
\verb|k=|    & 集計キーとなる項目名リスト(複数項目指定可)を指定する。\\
\verb|s=|    & ここで指定した項目(複数項目指定可)で並べ替えた後、先頭・終端に印を付ける。\\
\verb|a=|    & 先頭と終端の印を出力する項目名を指定する。【デフォルト値:top,bot】\\
\end{tabular} 
}
\end{table} 

\subsection*{利用例}
\subsubsection*{例1: 基本例}

\verb|k1|項目で並べ替えた後、\verb|k1|キー項目の先頭(\verb|top|項目)と終端(\verb|bottom|項目)に印(\verb|1|)をつける。


\begin{Verbatim}[baselinestretch=0.7,frame=single]
$ more dat1.csv
id,k1,k2,val
1,A,a,1
2,A,b,2
3,A,b,3
4,B,a,4
5,B,a,5
$ mkeybreak k=k1 i=dat1.csv o=rsl1.csv
#END# kgkeybreak i=dat1.csv k=k1 o=rsl1.csv
$ more rsl1.csv
id,k1%0,k2,val,top,bot
1,A,a,1,1,
2,A,b,2,,
3,A,b,3,,1
4,B,a,4,1,
5,B,a,5,,1
\end{Verbatim}
\subsubsection*{例2: 2項目キー}

\verb|k1|・\verb|k2|項目で並べ替えた後、\verb|k1|キー項目の先頭(\verb|top|項目)と終端(\verb|bottom|項目)に印(\verb|1|)をつける。


\begin{Verbatim}[baselinestretch=0.7,frame=single]
$ mkeybreak s=k1,k2 k=k1 i=dat1.csv o=rsl2.csv
#END# kgkeybreak i=dat1.csv k=k1 o=rsl2.csv s=k1,k2
$ more rsl2.csv
id,k1,k2,val,top,bot
1,A,a,1,1,
2,A,b,2,,
3,A,b,3,,1
4,B,a,4,1,
5,B,a,5,,1
\end{Verbatim}


\subsection*{関連コマンド}

%\end{document}
