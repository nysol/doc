
%\documentclass[a4paper]{jsbook}
%\usepackage{mcmd_jp}
%\begin{document}

\section{mdsp 画面表示\label{sect:mdsp}}
\index{mdsp@mdsp}
\underline{注)本コマンドは開発バージョンであり、仕様が変更される可能性があります。}

座標\verb|x=,y=|で指定したターミナル上の位置に\verb|str=|で指定した文字列を表示する。
\verb|i=|でファイル名を指定すると、その内容を表示する。
\verb|str=|と\verb|i=|の両方が指定されればi=が優先される。
\verb|i=|が複数行ある場合、全ての行が\verb|x=|で指定した座標位置から表示される。

\verb|fc=|で文字列の色を指定でき、また\verb|bg=|で文字列の背景の色を指定できる。
指定できる色は、black, red, green, yellow, blue, magenda, cyan, white、の8色である。
\verb|fc=|のデフォルトはblackで、\verb|bg=|のデフォルトはwhiteである。
また\verb|-bold|を指定することで強調表示が可能である。

\subsection*{書式}
\verb/mdsp x= y= str=|i= [fc=] [bg=] [-bold] /
\hyperref[sect:option_nfn]{[-nfn]}
\hyperref[sect:option_nfno]{[-nfno]}
\hyperref[sect:option_x]{[-x]}
\hyperref[sect:option_option_tmppath]{[tmpPath=]}
\hyperref[sect:option_precision]{[precision=]}
\verb|[-params]|
\verb|[--help]|
\verb|[--helpl]|
\verb|[--version]|\\

\subsection*{パラメータ}
\begin{table}[htbp]
%\begin{center}
{\small
\begin{tabular}{ll}
\verb|x=|   & x軸(左から右への横方向)表示開始位置(1以上の値)を指定する。\\
\verb|y=|   & y軸(上から下への縦方向)表示開始位置(1以上の値)を指定する。\\
\verb|str=| & 表示する文字列 \\
\verb|i=|   & 表示する内容のファイル名 \\
\verb|fc=|  & 文字色\\
\verb|bg=|  & 背景色\\
\verb|-bold|& 強調表示\\
\end{tabular} 
}
\end{table} 


\subsection*{利用例}

\subsubsection*{例1: 基本例}

ターミナルの\verb|x=10,y=5|の位置に文字列\verb|abcd|を表示する。

\begin{Verbatim}[baselinestretch=0.7,frame=single]
$ mdsp x=10 y=5 str=abcd

以下、画面イメージ
+--------------------------------------
|
|
|
|
|          abcd
|
|
\end{Verbatim}

\subsubsection*{例2: ファイルで指定する場合}

ターミナルの\verb|x=10,y=5|の位置に\verb|dat.txt|の内容を表示する。

\begin{Verbatim}[baselinestretch=0.7,frame=single]
$ more dat.txt
abcd
efg
$ mdsp x=10 y=5 i=dat.txt

以下、画面イメージ
+--------------------------------------
|
|
|
|
|          abcd
|          efg
|
\end{Verbatim}

\subsubsection*{例3: 色をつける例}

ターミナルの\verb|x=10,y=5|の位置に文字列\verb|abcd|を、
文字の色を赤、背景を青にして表示する。

\begin{Verbatim}[baselinestretch=0.7,frame=single,commandchars=\\\{\}]
$ mdsp x=10 y=5 str=abcd fc=red bc=blue

以下、画面イメージ
+--------------------------------------
|
|
|
|
|          \textColor{blue}{red}{abcd}
|
|
\end{Verbatim}

\subsection*{関連コマンド}
\hyperref[sect:minput] {minput} :入力画面を表示する。

\hyperref[sect:mminput] {mminput} : 複数入力枠による入力画面を表示する。

\hyperref[sect:mseldsp] {mseldsp} : 画面に単一選択入力窓を表示する。

\hyperref[sect:mmseldsp] {mmseldsp} : 画面に複数選択入力窓を表示する。

%\end{document}
