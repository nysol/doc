
%\documentclass[a4paper]{jsbook}
%\usepackage{mcmd_jp}
%\begin{document}

\section{msetstr 文字列項目の追加\label{sect:msetstr}}
\index{msetstr@msetstr}
指定した文字列を項目として全行に追加する。複数項目の追加も可能。

\subsection*{書式}
\verb|msetstr v= a= |
\hyperref[sect:option_i]{[i=]}
\hyperref[sect:option_o]{[o=]}
\hyperref[sect:option_assert_diffSize]{[-assert\_diffSize]}
\hyperref[sect:option_nfn]{[-nfn]} 
\hyperref[sect:option_nfno]{[-nfno]}  
\hyperref[sect:option_x]{[-x]}
\hyperref[sect:option_option_tmppath]{[tmpPath=]}
\hyperref[sect:option_precision]{[precision=]}
\verb|[-params]|
\verb|[--help]|
\verb|[--helpl]|
\verb|[--version]|\\

\subsection*{パラメータ}
\begin{table}[htbp]
%\begin{center}
{\small
\begin{tabular}{ll}
\verb|i=| & 入力ファイル名を指定する。\\
\verb|o=| & 出力ファイル名を指定する。\\ 
\verb|v=| & 追加する文字列リスト。\\
          & 値を何も指定しないとNULL値が追加される。\\
\verb|a=| & 追加する項目名。\\
          & \verb|v=|で指定した文字列の個数と同数の項目名を指定しなければならない。\\
\end{tabular}
}
\end{table}


\subsection*{利用例}
\subsubsection*{Example 1: Basic Example}

Calculate the date by setting a reference date  (defined as January 01, 2007)  and add the string “\verb|20070101|” in all lines and save the output as a new column named “ReferenceDate”.


\begin{Verbatim}[baselinestretch=0.7,frame=single]
$ more dat1.csv
customer,date
A,20081202
A,20081204
B,20081203
$ msetstr v=20070101 a=ReferenceDate i=dat1.csv o=rsl1.csv
#END# kgsetstr a=ReferenceDate i=dat1.csv o=rsl1.csv v=20070101
$ more rsl1.csv
customer,date,ReferenceDate
A,20081202,20070101
A,20081204,20070101
B,20081203,20070101
\end{Verbatim}
\subsubsection*{Example 2: Add multiple fields}



\begin{Verbatim}[baselinestretch=0.7,frame=single]
$ msetstr v=20070101,20070201 a=RefDate1,RefDate2 i=dat1.csv o=rsl2.csv
#END# kgsetstr a=RefDate1,RefDate2 i=dat1.csv o=rsl2.csv v=20070101,20070201
$ more rsl2.csv
customer,date,RefDate1,RefDate2
A,20081202,20070101,20070201
A,20081204,20070101,20070201
B,20081203,20070101,20070201
\end{Verbatim}
\subsubsection*{Example 3: Add column with null values}



\begin{Verbatim}[baselinestretch=0.7,frame=single]
$ msetstr v= a=NewColumn i=dat1.csv o=rsl3.csv
#END# kgsetstr a=NewColumn i=dat1.csv o=rsl3.csv v=
$ more rsl3.csv
customer,date,NewColumn
A,20081202,
A,20081204,
B,20081203,
\end{Verbatim}

\subsection*{関連コマンド}
\hyperref[sect:mcal]{mcal} : \verb|if|関数を使えば、行ごとに条件を判定して異なる固定文字列を追加できる。

%\end{document}

