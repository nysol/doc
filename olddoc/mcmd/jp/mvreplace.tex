
%\documentclass[a4paper]{jsbook}
%\usepackage{mcmd_jp}
%\begin{document}

\section{mvreplace ベクトル要素の参照置換\label{sect:mvreplace}}
\index{mvreplace@mvreplace}
ベクトル要素をキーにして参照ファイル上のベクトル型データで置換する。
ベクトル型の項目とは、表\ref{tbl:mvreplace_input}のitems項目のように、
スペースで区切られた複数の文字列を値として持つ項目である。

\if0 #no help# following sentences will not apear on the help document. \fi
典型的な例を表\ref{tbl:mvreplace_input}〜\ref{tbl:mvreplace_out3}に示す。

\begin{table}[htbp]
\begin{center}
\begin{tabular}{ccc}

\begin{minipage}{0.3\hsize}
\begin{center}
\caption{入力データ\label{tbl:mvreplace_input}}
\verb|in.csv| \\
{\small
\begin{tabular}{cl}
\hline
no&items \\
\hline
1&a b c \\
2&a d \\
3&b f e f \\
4&f c d \\
\hline

\end{tabular}
}
\end{center}
\end{minipage}

\begin{minipage}{0.3\hsize}
\begin{center}
\caption{参照ファイル\label{tbl:mvreplace_ref}}
\verb|ref.csv| \\
{\small
\begin{tabular}{cc}
\hline
item&taxo \\
\hline
a&X \\
b&Y \\
c&Z \\
e&X \\
f&Z \\
\hline
\end{tabular}
}
\end{center}
\end{minipage}
\end{tabular}
\end{center}
\end{table}

\begin{table}[htbp]
\begin{center}
\begin{tabular}{cc}

\begin{minipage}{0.48\hsize}
\begin{center}
\caption{基本例\label{tbl:mvreplace_out2}}
\verb|vf=items m=ref.csv K=item f=taxo| \\
{\small
\begin{tabular}{ll}
\hline
no&items \\
\hline
1&X Y Z \\
2&X d \\
3&Y Z X Z \\
4&Z Z d \\
\hline

\end{tabular}
}
\end{center}
\end{minipage}

\begin{minipage}{0.48\hsize}
\begin{center}
\caption{アンマッチ要素の指定例\label{tbl:mvreplace_out3}}
\verb|vf=items m=ref.csv K=item f=taxo n=* | \\
{\small
\begin{tabular}{ll}
\hline
no&items \\
\hline
1&X Y Z \\
2&X * \\
3&Y Z X Z \\
4&Z Z * \\
\hline
\end{tabular}
}
\end{center}
\end{minipage}

\end{tabular}
\end{center}
\end{table}


\verb|mvreplace|コマンドは、参照ファイルデータを一旦全てメモリにセットするので、
巨大な参照ファイルを指定した場合はメモリを使い果たす可能性があることに注意する。

\subsection*{書式}
\verb/mvreplace vf= K= f= [n=] m=| [-A] /
\hyperref[sect:option_i]{i=}
\hyperref[sect:option_o]{[o=]}
\hyperref[sect:option_delim]{[delim=]} 
\hyperref[sect:option_assert_diffSize]{[-assert\_diffSize]}
\hyperref[sect:option_assert_nullin]{[-assert\_nullin]}
\hyperref[sect:option_assert_nullout]{[-assert\_nullout]}
\hyperref[sect:option_nfn]{[-nfn]} 
\hyperref[sect:option_nfno]{[-nfno]}  
\hyperref[sect:option_x]{[-x]}
\hyperref[sect:option_option_tmppath]{[tmpPath=]}
\hyperref[sect:option_precision]{[precision=]}
\verb|[-params]|
\verb|[--help]|
\verb|[--helpl]|
\verb|[--version]|\\

\subsection*{パラメータ}
\begin{table}[htbp]
%\begin{center}
{\small
\begin{tabular}{ll}
\verb|i=|  & 入力ファイル名を指定する。\\
\verb|o=|  & 出力ファイル名を指定する。\\
\verb|vf=| & 結合キーとなるベクトルの項目名(\verb|i=|ファイル上)を指定する。\\
           & 複数項目指定可能。ベクトル要素はソーティングされている必要はない。 \\
		   & 結果の項目名を変更したいときは、:(コロン)に続けて新項目名を指定する。\\
\verb|-A|  & \verb|vf=|で:(コロン)に続けて指定した項目名で、新たな項目が追加される。\\
           & なお\verb|-A|オプションを指定した場合、\verb|vf=|パラメータで指定するすべての\\
           & 項目に新項目名を指定しなければならない。\\
\verb|m=|  & 参照ファイルを指定する。\\
\verb|K=|  & 参照ファイル(\verb|m=|)上の結合キーとなるベクトル要素の項目名を指定する。\\
           & 並べ変わっている必要はないが、ベクトル要素は単一化されていなければならない。\\
           & 単一化されていない時の動作は不定である。\\
\verb|f=|  & 結合するベクトル(要素)項目名を指定する。 \\
\verb|n=|  & \verb|vf=|と\verb|K=|のベクトル要素がマッチしなかった場合に結合する文字列を指定する。 \\
           & 省略した場合、対象のベクトル(要素)の結合は行われない。 \\
\verb|delim=| & ベクトル型データの区切り文字を指定する。\\
\end{tabular}
}
\end{table} 

\subsection*{利用例}
\subsubsection*{Example 1: Replace elements in a vector}



\begin{Verbatim}[baselinestretch=0.7,frame=single]
$ more dat1.csv
items
b a c
c c
e a a
$ more ref1.csv
item,taxo
a,X Y
b,X
c,Z Z
$ mvreplace vf=items K=item m=ref1.csv f=taxo i=dat1.csv o=rsl1.csv
#END# kgvreplace K=item f=taxo i=dat1.csv m=ref1.csv o=rsl1.csv vf=items
$ more rsl1.csv
items
X X Y Z Z
Z Z Z Z
e X Y X Y
\end{Verbatim}
\subsubsection*{Example 2: Replace character in multiple elements}



\begin{Verbatim}[baselinestretch=0.7,frame=single]
$ more dat2.csv
items1,items2
b a c,b b
c c,a d
e a a,a a
$ more ref2.csv
item,taxo
a,X
b,X
c,Y
d,Y
$ mvreplace vf=items1,items2 K=item m=ref2.csv f=taxo i=dat2.csv o=rsl2.csv
#END# kgvreplace K=item f=taxo i=dat2.csv m=ref2.csv o=rsl2.csv vf=items1,items2
$ more rsl2.csv
items1,items2
X X Y,X X
Y Y,X Y
e X X,X X
\end{Verbatim}

\subsection*{関連コマンド}
\hyperref[sect:mvjoin]{mvjoin} : 要素の置換ではなく、結合であれば\verb|mvjoin|を使う。

%\end{document}
