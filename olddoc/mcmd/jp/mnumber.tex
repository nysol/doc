
%\documentclass[a4paper]{jsbook}
%\usepackage{mcmd_jp}
%\begin{document}

\section{mnumber 連番\label{sect:mnumber}}
\index{mnumber@mnumber}
数字連番もしくはアルファベット連番(A,B,...,Z,AA,AB,...,AZ,BA,BB,...,ZZ,AAA,AAB,...)ををふり、\verb|a=|パラメータで指定した項目名で出力する。

\subsection*{書式}
\verb|mnumber a= [e=] [I=] [k=] [s=] [S=] [-B]|
\hyperref[sect:option_i]{[i=]}
\hyperref[sect:option_o]{[o=]}
\hyperref[sect:option_assert_diffSize]{[-assert\_diffSize]}
\hyperref[sect:option_assert_nullkey]{[-assert\_nullkey]}
\hyperref[sect:option_nfn]{[-nfn]} 
\hyperref[sect:option_nfno]{[-nfno]}  
\hyperref[sect:option_x]{[-x]}
\hyperref[sect:option_q]{[-q]}
\hyperref[sect:option_option_tmppath]{[tmpPath=]}
\hyperref[sect:option_precision]{[precision=]}
\verb|[-params]|
\verb|[--help]|
\verb|[--helpl]|
\verb|[--version]|\\

\subsection*{パラメータ}
\begin{table}[htbp]
%\begin{center}
{\small
\begin{tabular}{ll}
\verb|i=|    & 入力ファイル名を指定する。\\
\verb|o=|    & 出力ファイル名を指定する。\\
\verb|a=|    & 新たに追加される項目の名前を指定する。【但し、-nfn,-nfnoオプション指定時は必要なし】\\
\verb|e=|    & 同Rankの処理方法 \\
             & 同一キー同一ソート項目値への処理方法を指定する。\\
             & 指定しない場合は、デフォルトは\verb|e=seq|である。\\
             & \verb|seq:|同Rankの場合は各行に連番を振るが、その順序は不定である。\\
             & \verb|same:|同Rankの場合は同じNoもしくはアルファベットを付け加える。\\
             & \verb|skip:|同Rankの場合は同じNoを振り、\\
             & 次のNoはスキップするようにNoもしくはアルファベット連番を付け加える。\\
             & (注意)\verb|e={same/skip}|を指定する場合は、\verb|s=|パラメータを同時に指定しなければならない。\\
\verb|I=|    & 連番の間隔を指定する。 \\
\verb|k=|    & 連番もしくは連文字をふる単位となる項目名リスト(複数項目指定可)を指定する。【\hyperref[sect:option_k]{集計キーブレイク処理}】\\
             & (注意)指定する場合は事前に\verb|k=|パラメータで指定する連番、\\
             & もしくは連文字をふる単位となる項目順に並べ替えておく必要がある。\\
\verb|s=|    & ここで指定した項目(複数項目指定可)で並べ替えられた後、連番が追加される。\\
             & \verb|-B|オプション指定時以外は必須。\\
\verb|S=|    & 開始No \\
             & 連番の開始Noを指定する。 \\
             & 大文字のアルファベットが指定された場合はアルファベット連番となる。\\
             & ただし、アルファベット連番の場合、間隔(\verb|I=|)に負の値は指定できない。 \\
 \verb|-B|   & キー毎に連番もしくはアルファベット連番をふる。 \\
             & あるキー内では全行同じNoもしくはアルファベットがふられる。 \\
\end{tabular} 
}
\end{table} 

\subsection*{利用例}
\subsubsection*{Example 1: Sequential numbers}

Generate sequential numbers for each value in ascending order in the \verb|Customer| column. Name the sequence as \verb|No| in a new column.


\begin{Verbatim}[baselinestretch=0.7,frame=single]
$ more dat1.csv
Customer,Val,Sum
A,29,300
B,35,250
C,15,200
D,23,150
E,10,100
$ mnumber s=Customer a=No i=dat1.csv o=rsl1.csv
#END# kgnumber a=No i=dat1.csv o=rsl1.csv s=Customer
$ more rsl1.csv
Customer%0,Val,Sum,No
A,29,300,0
B,35,250,1
C,15,200,2
D,23,150,3
E,10,100,4
\end{Verbatim}
\subsubsection*{Example 2: Serialize the Date column}

Sequentially number items in the \verb|Date| column according to earliest date to latest date. Use same sequence number (\verb|No|) for same \verb|Date|. Save the sequence in a new column named \verb|"No"|.


\begin{Verbatim}[baselinestretch=0.7,frame=single]
$ more dat2.csv
Date
20090101
20090101
20090102
20090103
20090103
$ mnumber k=Date a=No -B i=dat2.csv o=rsl2.csv
#END# kgnumber -B a=No i=dat2.csv k=Date o=rsl2.csv
$ more rsl2.csv
Date%0,No
20090101,0
20090101,0
20090102,1
20090103,2
20090103,2
\end{Verbatim}
\subsubsection*{Example 3: Serialize the Sum column (use same alphabet for same Rank order)}

Create a alphabetical sequence according to the \verb|Sum| column which is arranged in descending order. Save the sequence in a new column named \verb|“Rank”|. Assign the same alphabet character to items with the same values.


\begin{Verbatim}[baselinestretch=0.7,frame=single]
$ more dat3.csv
Customer,Val,Sum
A,3,300
B,1,250
C,2,250
D,1,150
E,1,100
$ mnumber a=Rank e=same s=Sum%nr S=A  i=dat3.csv o=rsl3.csv
#END# kgnumber S=A a=Rank e=same i=dat3.csv o=rsl3.csv s=Sum%nr
$ more rsl3.csv
Customer,Val,Sum%0nr,Rank
A,3,300,A
B,1,250,B
C,2,250,B
D,1,150,C
E,1,100,D
\end{Verbatim}
\subsubsection*{Example 4: Serialize the Sum column (sequential numbers for same Rank order)}

Number records sequentially according to \verb|Sum| column (sum arranged in descending order), and save serials in the \verb|"Rank"| column. For items with same rank order, assign sequential numbers according to sort order.


\begin{Verbatim}[baselinestretch=0.7,frame=single]
$ mnumber a=Rank e=seq s=Sum%nr i=dat3.csv o=rsl4.csv
#END# kgnumber a=Rank e=seq i=dat3.csv o=rsl4.csv s=Sum%nr
$ more rsl4.csv
Customer,Val,Sum%0nr,Rank
A,3,300,0
B,1,250,1
C,2,250,2
D,1,150,3
E,1,100,4
\end{Verbatim}
\subsubsection*{Example 5: Serialize the Sum column (Same No for same Rank)}

Number records sequentially according to \verb|Sum| column (sum arranged in descending order), and save the numbers in the \verb|“Rank”| column. Assign the same No to records with the same Rank order.


\begin{Verbatim}[baselinestretch=0.7,frame=single]
$ mnumber a=Rank e=same s=Sum%nr i=dat3.csv o=rsl5.csv
#END# kgnumber a=Rank e=same i=dat3.csv o=rsl5.csv s=Sum%nr
$ more rsl5.csv
Customer,Val,Sum%0nr,Rank
A,3,300,0
B,1,250,1
C,2,250,1
D,1,150,2
E,1,100,3
\end{Verbatim}
\subsubsection*{Example 6: Serialize the Sum column (duplicate numbers for same Rank and skip number for next record)}

Number records sequentially according to \verb|Sum| column (sum arranged in descending order), and save the numbers is the “Rank” column. Assign same \verb|RankNo| number to records with same rank order, subsequent No is skipped for the following record.


\begin{Verbatim}[baselinestretch=0.7,frame=single]
$ mnumber a=Rank e=skip s=Sum%nr i=dat3.csv o=rsl6.csv
#END# kgnumber a=Rank e=skip i=dat3.csv o=rsl6.csv s=Sum%nr
$ more rsl6.csv
Customer,Val,Sum%0nr,Rank
A,3,300,0
B,1,250,1
C,2,250,1
D,1,150,3
E,1,100,4
\end{Verbatim}
\subsubsection*{Example 7: Number sequence starting from 10}

Serialize the \verb|Sum| column sequentially from 10 with items, where values of sum is arranged in ascending order. Save the serials in the \verb|"Score"| column. Assign same RankNo to records with same Rank order , subsequent No is skipped for the following record.


\begin{Verbatim}[baselinestretch=0.7,frame=single]
$ more dat4.csv
Customer,Val,Sum
A,1,100
B,1,150
C,1,250
D,2,250
E,3,300
$ mnumber a=Score e=same s=Sum%n S=10 i=dat4.csv o=rsl7.csv
#END# kgnumber S=10 a=Score e=same i=dat4.csv o=rsl7.csv s=Sum%n
$ more rsl7.csv
Customer,Val,Sum%0n,Score
A,1,100,10
B,1,150,11
C,1,250,12
D,2,250,12
E,3,300,13
\end{Verbatim}
\subsubsection*{Example 8: Start sequence from 10 with an interval of 5}

Number the \verb|Sum| column sequentially from 10 at an interval of 5, where values of sum is arranged in ascending order. Save the serials in the \verb|“Score”| column. Assign the same number to records with the same Rank order.


\begin{Verbatim}[baselinestretch=0.7,frame=single]
$ mnumber a=Score e=same s=Sum%n S=10 I=5 i=dat4.csv o=rsl8.csv
#END# kgnumber I=5 S=10 a=Score e=same i=dat4.csv o=rsl8.csv s=Sum%n
$ more rsl8.csv
Customer,Val,Sum%0n,Score
A,1,100,10
B,1,150,15
C,1,250,20
D,2,250,20
E,3,300,25
\end{Verbatim}


\subsection*{関連コマンド}
\hyperref[sect:mnewnumber]{mnewnumber} : 新たに連番データを生成する場合に使う。

\hyperref[sect:mbest]{mbest} : 行番号による選択であれば、\verb|mnumber|を使わずともこのコマンドで。

%\end{document}
