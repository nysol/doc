
%\documentclass[a4paper]{jsbook}
%\usepackage{mcmd_jp}
%\begin{document}

\section{mstats 一変数の統計量算出\label{sect:mstats}}
\index{mstats@mstats}
\verb|f=|パラメータで指定した数値項目について
\verb|c=|パラメータで指定した統計量の計算をする。
\verb|k=|を指定することで、キー単位で集計することができる。
\verb|f=|で指定した項目のNULL値は無視される。
ただし、全行がNULL値であればNULL値が出力される。
\verb|(注)|k=とf=パラメータで指定した項目以外については、どの行が出力されるか>は不定であることに注意してください。\\

\subsection*{書式}
\verb|mstats c= f= [k=] [-n] | 
\hyperref[sect:option_i]{[i=]}
\hyperref[sect:option_o]{[o=]}
\hyperref[sect:option_assert_diffSize]{[-assert\_diffSize]}
\hyperref[sect:option_assert_nullkey]{[-assert\_nullkey]}
\hyperref[sect:option_assert_nullin]{[-assert\_nullin]}
\hyperref[sect:option_assert_nullout]{[-assert\_nullout]}
\hyperref[sect:option_nfn]{[-nfn]} 
\hyperref[sect:option_nfno]{[-nfno]}  
\hyperref[sect:option_x]{[-x]}
\hyperref[sect:option_q]{[-q]}
\hyperref[sect:option_option_tmppath]{[tmpPath=]}
\hyperref[sect:option_precision]{[precision=]}
\verb|[-params]|
\verb|[--help]|
\verb|[--helpl]|
\verb|[--version]|\\

\subsection*{パラメータ}
\begin{table}[htbp]
%\begin{center}
{\small
\begin{tabular}{ll}
%\verb|k=|    & ここで指定された項目(複数項目指定可)を単位として集計する。【\hyperref[sect:option_k]{集計キーブレイク処理}】 \\
%             & 集計の単位となる項目順に並べ替えておく必要がある。\\
\verb|i=|    & 入力ファイル名を指定する。\\
\verb|o=|    & 出力ファイル名を指定する。\\ 
\verb|k=|    & ここで指定された項目(複数項目指定可)を単位として集計する。 \\
\verb|f=|    & ここで指定された項目(複数項目指定可)の値が集計される。\\
\verb|c=|    & 統計量(以下のリストから一つだけ指定可)\\
             & \verb/sum|mean|count|ucount|devsq|var|uvar|sd|usd|USD|cv|min|qtile1|/\\
             & \verb/median|qtile3|max|range|qrange|mode|skew|uskew|kurt|ukurt/\\
\verb|-n|    & NULL値が1つでも含まれていると結果もNULL値とする。\\
\end{tabular} 
}
\end{table} 

\subsection*{統計量リスト}
\begin{table}[htbp]
%\begin{center}
{\small
\renewcommand{\arraystretch}{1.5}
\begin{tabular}{llll}
\hline
\verb|c=|の値 & 内容 & 式 & 備考 \\
\hline
count  & 件数(NULL値以外) & $n$: NULL値以外の件数 & 文字列項目に対しては適用できない。\\
ucount & ユニーク件数     & $un$: 重複値を省いた件数 &  文字列項目に対しては適用できない。\\
sum    & 合計             & $sum=\sum_{i=1}^n x_i$ & \\
mean   & 算術平均         & $m=\frac{1}{n}\sum_{i=1}^n x_i$ & \\
devsq  & 偏差平方和       & $S=\sum_{i=1}^n(x_i-m)^2$ & \\
var    & 分散             & $s^2=\frac{1}{n}S$ & \\
uvar   & 分散(不偏推定値) & $u^2=\frac{1}{n-1}S$ & \\
sd     & 標準偏差         & $s=\sqrt{s^2}$ & \\
usd    & 標準偏差(不偏分散のsqrt) & $u=\sqrt{u^2}$ & 一般的によく使われる標準偏差 \\
USD    & 不偏標準偏差     & 省略               & 正確な不偏推定 \\
cv     & 変動係数         & $cv=s/m×100\%$ & \\
mode   & 最頻値           & $mode$: 最頻出の値& 全ての値が異なる場合はNULLを、同頻度\\
       &                  &                   & の場合はより小さい方の値を出力する。\\
min    & 最小値           & $min=\min_i x_i$ & \\
max    & 最大値           & $max=\max_i x_i$ & \\
range  & 範囲             & $r=max-min$  & \\
median & 中央値           & $Q2=昇順に並べた時の第2四文位点$ & \\
qtile1 & 第1四分位点      & $Q1=昇順に並べた時の第1四文位点$ & \\
%qtile1 & 第1四分位点      & $Q1$ 第1四分位順位が自然数でない場合:$Q1=(\lceil t\rceil -t)x\lfloor t\rfloor +(t-\lfloor t \rfloor)x\lceil t\rceil$ただし,$t=1+0.25(n-1)$ & \\
qtile3 & 第1四分位点      & $Q3=昇順に並べた時の第3四文位点$ & \\
% 第3四分位順位が自然数でない場合:$Q3=(\lceil t\rceil -t)x\lfloor t\rfloor +(t-\lfloor t \rfloor)x\lceil t\rceil$ただし,$t=1+0.75(n-1)$ \\
qrange & 四分位範囲       & $rq=Q3-Q1$ & \\
skew   & 歪度             & $\frac{\frac{1}{n}\sum_{i=1}^n (x_i-m)^3}{s^3}$ & \\
uskew  & 歪度(不偏推定値) & 省略 & \\
kurt   & 尖度             & $\frac{\frac{1}{n}\sum_{i=1}^n (x_i-m)^4}{s^4}-3.0$ & \\
ukurt  & 尖度(不偏推定値) & 省略 & \\
\hline \\
\end{tabular}
}
\end{table}

\subsection*{利用例}
\subsubsection*{例1: 基本例}

「顧客」項目を単位に「数量」と「金額」項目の
各統計量合計値を計算する。


\begin{Verbatim}[baselinestretch=0.7,frame=single]
$ more dat1.csv
顧客,数量,金額
A,1,10
B,5,20
B,2,10
C,1,15
C,3,10
C,1,21
$ mstats k=顧客 f=数量,金額 c=sum i=dat1.csv o=rsl1.csv
#END# kgstats c=sum f=数量,金額 i=dat1.csv k=顧客 o=rsl1.csv
$ more rsl1.csv
顧客%0,数量,金額
A,1,10
B,7,30
C,5,46
\end{Verbatim}
\subsubsection*{例2: 基本例2}

各統計量最大値を計算する。


\begin{Verbatim}[baselinestretch=0.7,frame=single]
$ mstats k=顧客 f=数量,金額 c=max i=dat1.csv o=rsl2.csv
#END# kgstats c=max f=数量,金額 i=dat1.csv k=顧客 o=rsl2.csv
$ more rsl2.csv
顧客%0,数量,金額
A,1,10
B,5,20
C,3,21
\end{Verbatim}

\subsection*{関連コマンド}
\hyperref[sect:msim]{msim} : 2変量の統計量を求める。

\hyperref[sect:mavg] {mavg} : \verb|c=avg|に特化したコマンド。

\hyperref[sect:msum] {msum} : \verb|c=sum|に特化したコマンド。

\hyperref[sect:mcount] {mcount} : \verb|c=count|と異なり、集計キーの行数をカウントする。

%\end{document}
