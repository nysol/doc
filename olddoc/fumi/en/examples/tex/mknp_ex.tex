\subsubsection*{Example 1: Basic Example }

File \verb|test.txt| is contained in \verb|text| directory for parsing. 
The resuls is saved in \verb|xml| directory.  

\begin{Verbatim}[baselinestretch=0.7,frame=single]
$ more text/test.txt
子どもはリンゴがすきです。
望遠鏡で泳ぐ少女を見た。
$ mknp.rb I=text O=xml
#MSG# KNP: reading text/test.txt
#MSG# KNP: MP-2 aid=test.txt sid=0 (sentences:1/2, articles:1/1)
#MSG# KNP: MP-2 aid=test.txt sid=1 (sentences:2/2, articles:1/1)
#MSG# KNP2XML 1/1
#MSG# Elapse: 0.149sec, # of sentences=2, # of articles=1
#MSG#   0.075sec/sentence, 0.149sec/article
#MSG#   mpCount=2, poolSize=1000
#MSG#   maxLen=512Byte, maxSec=30sec, sizeLimit=2000MB
#END# /Users/maegawa/.rvm/rubies/ruby-2.0.0-p247/bin/mknp.rb I=text O=xml
$ head -n20 xml/test.txt
<?xml version='1.0' encoding='UTF-8'?>
<article id='test.txt'>
  <sentence id='0' text='子どもはリンゴがすきです。'>
    <chunk id='0' link='2' phraseType='格助詞句' caseType='ガ2格' phrase='子供' phraseTok='
      <token id='0' class1='名詞' class2='普通名詞' word='子ども' orgWord='子ども' daiWord='
      <token id='1' class1='助詞' class2='副助詞' word='は' orgWord='は'/>
    </chunk>
    <chunk id='1' link='2' phraseType='格助詞句' caseType='ガ格' phrase='林檎' phraseTok='リ
      <token id='2' class1='名詞' class2='普通名詞' word='リンゴ' orgWord='リンゴ' daiWord='
      <token id='3' class1='助詞' class2='格助詞' word='が' orgWord='が'/>
    </chunk>
    <chunk id='2' link='-1' phraseType='用言句' phraseTok='すきだ' rawPhrase='すきです。' ph
      <token id='4' class1='形容詞' class3='ナ形容詞' class4='デス列基本形' word='すきだ' or
      <token id='5' class1='特殊' class2='句点' word='。' orgWord='。'/>
    </chunk>
  </sentence>
  <sentence id='1' text='望遠鏡で泳ぐ少女を見た。'>
    <chunk id='0' link='3' phraseType='格助詞句' caseType='デ格' phrase='望遠鏡' phraseTok='
      <token id='0' class1='名詞' class2='普通名詞' word='望遠' orgWord='望遠' daiWord='望遠
      <token id='1' class1='名詞' class2='普通名詞' word='鏡' orgWord='鏡' daiWord='鏡' cate
\end{Verbatim}
\subsubsection*{Example 2: Example of output results of KNP (original). }

Resuls of KNP (original) is saved in \verb|knp| directory. 


\begin{Verbatim}[baselinestretch=0.7,frame=single]
$ more text/test.txt
子どもはリンゴがすきです。
望遠鏡で泳ぐ少女を見た。
$ mknp.rb I=text O=xml P=knp
#MSG# KNP: reading text/test.txt
#MSG# KNP: MP-2 aid=test.txt sid=0 (sentences:1/2, articles:1/1)
#MSG# KNP: MP-2 aid=test.txt sid=1 (sentences:2/2, articles:1/1)
#MSG# KNP2XML 1/1
#MSG# Elapse: 0.147sec, # of sentences=2, # of articles=1
#MSG#   0.074sec/sentence, 0.147sec/article
#MSG#   mpCount=2, poolSize=1000
#MSG#   maxLen=512Byte, maxSec=30sec, sizeLimit=2000MB
#END# /Users/maegawa/.rvm/rubies/ruby-2.0.0-p247/bin/mknp.rb I=text O=xml P=knp
$ head knp/test.txt
# S-ID:1 KNP:4.11-CF1.1 DATE:2014/07/28 SCORE:-21.86138
* 2D <文頭><SM-主体><SM-人><ハ><助詞><体言><係:未格><提題><区切:3-5><主題表現><格要素><連用
+ 2D <文頭><SM-主体><SM-人><ハ><助詞><体言><係:未格><提題><区切:3-5><主題表現><格要素><連用
子ども こども 子ども 名詞 6 普通名詞 1 * 0 * 0 "代表表記:子供/こども カテゴリ:人" <代表表記:
は は は 助詞 9 副助詞 2 * 0 * 0 NIL <かな漢字><ひらがな><付属>
* 2D <ガ><助詞><体言><係:ガ格><区切:0-0><格要素><連用要素><正規化代表表記:林檎/りんご><主辞
+ 2D <ガ><助詞><体言><係:ガ格><区切:0-0><格要素><連用要素><名詞項候補><先行詞候補><正規化代
リンゴ りんご リンゴ 名詞 6 普通名詞 1 * 0 * 0 "代表表記:林檎/りんご カテゴリ:植物
が が が 助詞 9 格助詞 1 * 0 * 0 NIL <かな漢字><ひらがな><付属>
* -1D <文末><句点><用言:形><レベル:C><区切:5-5><ID:(文末)><係:文末><提題受:30><主節><格要
\end{Verbatim}
