\subsubsection*{Example 1: Basic Example}



\begin{Verbatim}[baselinestretch=0.7,frame=single]
$ head tweets.txt
3年ぶりにウォークマンを買ったけど、育休中はあまり活躍の余地がないですね。
待機児童解消の方がいい気がするけど。
読売テレビ(日本テレビ系列)ウェークアップ!ぷらすに蓮舫ネクスト規制改革担当大臣が生出演!
この学生さんは、仕事に不利じゃなかったら、3年育休取れるのも良いな、って思ってるよね。
今の人事制度のまま育休3年とか、前以上に女性が締め出されるだけでは。
女子大生でも分かる、3年間の育児休暇が最悪な結果をもたらす理由。(中嶋よしふみ)
保育園を中心に期間とか決めるの、おかしいよな\UTF{FF5E}。
女性が必ず子育てしなきゃならない社会なら結婚絶対したくない…
育休とかの前に、母親に育児に専念させるなら女性の雇用よりもまず、男性の雇用、給料なんだよね。
安倍総理きた! 育児休暇三年は…女としては嬉しいけど、会社に申し訳ないよねえ
$ mnewdic.rb i=tweets.txt O=newdic
#MSG# start to parse each line...
#MSG# working at line 0
#MSG# working at line 100
#MSG# working at line 200
#MSG# working at line 300
#MSG# working at line 400
#MSG# working at line 500
#MSG# working at line 600
#MSG# working at line 700
#MSG# working at line 800
#MSG# working at line 900
#MSG# working at line 1000
#END# /Users/maegawa/.rvm/rubies/ruby-2.0.0-p247/bin/mnewdic.rb i=tweets.txt O=newdic
$ ls newdic
corpus.csv
corpus_sjis.csv
words.csv
words_sjis.csv
$ head newdic/words.csv
見出し語,品詞,読み,カテゴリ,ドメイン,pid,word1,word2,freq
職場復帰,,,,,0,職場,復帰,31
授業参観,,,,,1,授業,参観,28
会議参加,,,,,2,会議,参加,26
育休延長,,,,,3,育休,延長,19
子育て支援,,,,,4,子育て,支援,18
育児休暇3,,,,,5,育児休暇,3,18
待機児童ゼロ,,,,,6,待機児童,ゼロ,17
規制緩和,,,,,7,規制,緩和,16
給付金,,,,,8,給付,金,15
\end{Verbatim}
