

%\documentclass[a4paper]{jarticle}


\section{mnewdic.rb Output adjacent word pair candidate from corpus \label{sect:mnewdic}}
\index{mnewdic@mnewdic}

From text corpus  (large set of text files),  returns the candidate information of adjacent word pair to be registered in the dictionary. 

JUMAN contains a standard dictionary with  common words, for text mining in specific fields, interpretation of 1 phrase  may be divided into multiple words. 
    
\verb|mnewdic.rb| command analyze the given corpus based on the frequency of phrases with words pairs that appears at the same time, and output the list in CSV file. 

\verb|mjumandic.rb| command simply dictionary registration in the process of text mining.    


\subsection{Format}
\begin{verbatim}
mnewdic.rb [i=] [O=] [S=] [n=] [seed=] [-dai] [-mcmdenv] [--help]
\end{verbatim}

\begin{table}[htbp]
{\small
\begin{tabular}{ll}
\verb|i=|       & : Corpus file name  \\
\verb|O=|       & : Output directory name  \\
\verb|S=|       & : Minimum value of word pair appearance \\
\verb|n=|       & : Output sentence number for word pair  \\
\verb|seed=|    & : Seed of random number  \\
\verb|-dai|     & : Use a symbol to represent headwords  \\
\verb|-mcmdenv| & : Display MCMD message containing environment variables.  \\
                &    Default returns warning and error message  (KG\_VerboseLevel=2). \\
\verb|--help|   & : Display help   \\
\end{tabular} 
}
\end{table} 

\subsubsection*{Example of input file }

Given the corpus file specified at \verb|i=| parameter, one row corresponds to one sentence in text file. 


\begin{Verbatim}[baselinestretch=0.7,frame=single]
3年ぶりにウォークマンを買ったけど、育休中はあまり活躍の余地がないですね。
待機児童解消の方がいい気がするけど。
:
\end{Verbatim}

\subsubsection*{Example of Output File  }

In the directory specified at \verb|O=| parameter, 
the results are saved in \verb|words.csv| file and  \verb|corpus.csv| file  (when \verb|nkf| command is installed,  the character code converted from Shift JIS is also saved in the output in both files. 
 
\begin{Verbatim}[baselinestretch=0.7,frame=single]
見出し語,品詞,読み,カテゴリ,ドメイン,pid,word1,word2,freq
職場復帰,,,,,0,職場,復帰,31
授業参観,,,,,1,授業,参観,28
会議参加,,,,,2,会議,参加,26
:
\end{Verbatim}

\newpage

The description of each item in \verb|words.csv| file is shown below.  

\begin{table}[htbp]
{\small
\begin{tabular}{ll}
\verb|見出し語| & : 見出し語 \\  
\verb|品詞|     & : 品詞 \\  
\verb|読み|     & : 読み \\  
\verb|カテゴリ| & : カテゴリ \\  
\verb|ドメイン| & : ドメイン \\  
\verb|pid|      & : pid \\  
\verb|word1|    & : 語1 \\  
\verb|word2|    & : 語2 \\  
\verb|freq|     & : 出現頻度 \\  
\end{tabular} 
}
\end{table} 

Users can refer to \verb|corpus.csv| to check whether the word of registered candidates appeared in the text.  


\begin{Verbatim}[baselinestretch=0.7,frame=single]
pid,id,text
0,52,"神戸で始めての、育休後職場復帰セミナーを開催しました。"
0,317,"僕の知り合いは2人子どもを産んで、立て続けに産休+育休を取って、職場復帰した。"
:
\end{Verbatim}




\subsection{Examples}
\subsubsection*{例1: 基本例}



\begin{Verbatim}[baselinestretch=0.7,frame=single]
$ head tweets.txt
3年ぶりにウォークマンを買ったけど、育休中はあまり活躍の余地がないですね。
待機児童解消の方がいい気がするけど。
読売テレビ(日本テレビ系列)ウェークアップ!ぷらすに蓮舫ネクスト規制改革担当大臣が生出演!
この学生さんは、仕事に不利じゃなかったら、3年育休取れるのも良いな、って思ってるよね。
今の人事制度のまま育休3年とか、前以上に女性が締め出されるだけでは。
女子大生でも分かる、3年間の育児休暇が最悪な結果をもたらす理由。(中嶋よしふみ)
保育園を中心に期間とか決めるの、おかしいよな~。
女性が必ず子育てしなきゃならない社会なら結婚絶対したくない…
育休とかの前に、母親に育児に専念させるなら女性の雇用よりもまず、男性の雇用、給料なんだよね。
安倍総理きた! 育児休暇三年は…女としては嬉しいけど、会社に申し訳ないよねえ
$ mnewdic.rb i=tweets.txt O=newdic
#MSG# start to parse each line...
#MSG# working at line 0
#MSG# working at line 100
#MSG# working at line 200
#MSG# working at line 300
#MSG# working at line 400
#MSG# working at line 500
#MSG# working at line 600
#MSG# working at line 700
#MSG# working at line 800
#MSG# working at line 900
#MSG# working at line 1000
#END# /Users/maegawa/.rvm/rubies/ruby-2.0.0-p247/bin/mnewdic.rb i=tweets.txt O=newdic
$ ls newdic
corpus.csv
corpus_sjis.csv
words.csv
words_sjis.csv
$ head newdic/words.csv
見出し語,品詞,読み,カテゴリ,ドメイン,pid,word1,word2,freq
職場復帰,,,,,0,職場,復帰,31
授業参観,,,,,1,授業,参観,28
会議参加,,,,,2,会議,参加,26
育休延長,,,,,3,育休,延長,19
子育て支援,,,,,4,子育て,支援,18
育児休暇3,,,,,5,育児休暇,3,18
待機児童ゼロ,,,,,6,待機児童,ゼロ,17
規制緩和,,,,,7,規制,緩和,16
給付金,,,,,8,給付,金,15
\end{Verbatim}


%\end{document}

