

%\documentclass[a4paper]{jarticle}


\section{mjumandic.rb Conversion from CSV to JUMAN Dictionary \label{sect:mjumandic}}
\index{mjumandic@mjumandic}

Convert CSV dictionary data to JUMAN dictionary format.  

The output from \verb|mnewdic.rb| command is also in CSV format. 


\subsection{Format}
\begin{verbatim}
mjumandic.rb [i=] [O=] [exe=] [-mcmdenv] [--help]
\end{verbatim}

\begin{table}[htbp]
{\small
\begin{tabular}{ll}
\verb|i=|       & :  Dictionary file name of CSV file.   \\
\verb|O=|       & :  Directory name that contains the dictionary of JUMAN. \\
\verb|exe=|     & :  Command path such as makein (Default:  /usr/local/bin) \\
                &    This is not required when JUMAN is installed with default setting.   \\
\verb|-mcmdenv| & :  Display MCMD message containing environment variables.  \\
                &    Default returns warning and error message  (KG\_VerboseLevel=2).  \\
\verb|--help|   & : Display help \\
\end{tabular} 
}
\end{table} 

\subsubsection*{Example of input file}

An example of dictionary file defined at  \verb|i=| parameter is shown below. 
The 5 items includes \verb|headword|, \verb|read|, \verb|part of speech|, \verb|category|, \verb|domain|. 

\begin{Verbatim}[baselinestretch=0.7,frame=single]
id,見出し語,読み,品詞,カテゴリ,ドメイン
1,連結営業利益,れんけつえいぎょうりえき,普通名詞,抽象物,ビジネス
2,米国債,べいこくさい,,抽象物,ビジネス
3,上方修正,じょうほうしゅうせい,サ変名詞,抽象物,ビジネス
4,日本航空,にほんこうくう,組織名,,
5,夏目漱石,なつめそうせき,人名,日本,姓
6,安倍首相 安倍晋太郎 安倍晋太郎首相,あべしゅしょう,人名,日本,姓名
\end{Verbatim}

The meaning of each item is as follows.  

\begin{description}
\item[見出し語]
見出し語には、表記ゆれなどの複数の見出し語を半角空白で区切って列挙できる。
見出し語がないとエラーとなる。
\item[品詞]
品詞は名詞のみ対応しており、以下に示す名詞の下位の品詞を「品詞」項目に登録する。\\
普通名詞,サ変名詞,時相名詞,数詞,副詞的名詞,固有名詞,人名,組織名,地名\\
品詞が省略されると、「普通名詞」が指定されたものとする。
品詞の体系は以下のURLを参照のこと。
\url{http://www.unixuser.org/~euske/doc/postag/}
\item[読み]
読みがないとエラーとなる。
\item[カテゴリ]
カテゴリは以下の22種(省略可能)\\
人,組織・団体,動物,植物,動物-部位,植物-部位,人工物-食べ物,人工物-衣類,人工物-乗り物\\
人工物-金銭,人工物-その他,自然物,場所-施設,場所-施設部位,場所-自然,場所-機能\\
場所-その他,抽象物,形・模様,色,数量,時間
\item[ドメイン]
ドメインは以下の12種(省略可能)\\
文化・芸術,交通,レクリエーション,教育・学習,スポーツ,科学・技術,健康・医学\\
ビジネス,家庭・暮らし,メディア,料理・食事,政治\\
\end{description}

Category and domain is a valid item to common noun and word formed by adding suru to noun. 
 
 Please refer to the following URL for references on category and domain. 

\url{http://nlp.ist.i.kyoto-u.ac.jp/DLcounter/lime.cgi?down=http://nlp.ist.i.kyoto-u.ac.jp/nl-resource/knp/20090930-juman-knp.ppt&name=20090930-juman-knp.ppt}






\subsection{Examples}
\subsubsection*{例1: 基本例}



\begin{Verbatim}[baselinestretch=0.7,frame=single]
$ more dic1.csv
id,見出し語,読み,品詞,カテゴリ,ドメイン
1,連結営業利益,れんけつえいぎょうりえき,普通名詞,抽象物,ビジネス
2,米国債,べいこくさい,,抽象物,ビジネス
3,上方修正,じょうほうしゅうせい,サ変名詞,抽象物,ビジネス
4,日本航空,にほんこうくう,組織名,,
5,夏目漱石,なつめそうせき,人名,日本,姓
6,安倍首相 安倍晋太郎 安倍晋太郎首相,あべしゅしょう,人名,日本,姓名
7,2ちゃんねる にちゃんねる,にちゃんねる,,,
$ mjumandic.rb i=dic1.csv O=jumandic
#END# kgcut f=品詞,見出し語,読み i=dic1.csv
#END# kgdelnull f=見出し語,読み
#END# kgsortf f=見出し語
#END# kguniq k=見出し語 o=/tmp/__MTEMP_68157_70357348549040_0
#END# mcsvin i=/tmp/__MTEMP_68157_70357348549040_0
Mon Jul 28 01:38:37 2014
/usr/local/share/juman/dic/JUMAN.grammar parsing... done.

Mon Jul 28 01:38:37 2014
/usr/local/share/juman/dic/JUMAN.katuyou parsing... done.

Mon Jul 28 01:38:37 2014
/usr/local/share/juman/dic/jumandic.tab parsing... done.

jumandic.dic parsing... done.

execution time:    0.000s
processor time:    0.000s
File Name "/Users/maegawa/git/nysol/nysol/doc/fumi/jp/examples/jumandic/jumandic.dat"

## 10 entry  814 th char
Saving pat-tree "/Users/maegawa/git/nysol/nysol/doc/fumi/jp/examples/jumandic/jumandic.pat" 
QUIT
#MSG# jumandic内のjumandic.dat,jumandic.patの2つのファイルがユーザ辞書として必要となる。
#MSG# ~/.jumanrcファイルを編集し、これらのファイルが格納されたパス名を以下のように追加登録す
#MSG# (辞書ファイル
#MSG#         /usr/local/share/juman/dic
#MSG#         /usr/local/share/juman/autodic
#MSG#         /usr/local/share/juman/wikipediadic
#MSG#         /Users/maegawa/git/nysol/nysol/doc/fumi/jp/examples/jumandic
#MSG# )
#END# /Users/maegawa/.rvm/rubies/ruby-2.0.0-p247/bin/mjumandic.rb i=dic1.csv O=jumandic
$ ls jumandic
jumandic.dat
jumandic.dic
jumandic.int
jumandic.pat
$ more jumandic/jumandic.dic
(名詞 (サ変名詞 ((読み じょうほうしゅうせい) (見出し語 上方修正) (意味情報 "代表表記:上方修
(名詞 (人名 ((読み なつめそうせき) (見出し語 夏目漱石) (意味情報 ""))))
(名詞 (人名 ((読み あべしゅしょう) (見出し語 安倍首相 安倍晋太郎 安倍晋太郎首相) (意味情報 "
(名詞 (組織名 ((読み にほんこうくう) (見出し語 日本航空) (意味情報 "代表表記:日本航空/にほん
(名詞 (普通名詞 ((読み べいこくさい) (見出し語 米国債) (意味情報 "代表表記:米国債/べいこくさ
(名詞 (普通名詞 ((読み れんけつえいぎょうりえき) (見出し語 連結営業利益) (意味情報 "代表表記
(名詞 (普通名詞 ((読み にちゃんねる) (見出し語 2ちゃんねる にちゃんねる) (意味情報 "代表表
\end{Verbatim}


%\end{document}

