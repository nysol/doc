\subsubsection*{例1: 基本例}



\begin{Verbatim}[baselinestretch=0.7,frame=single]
$ more dic1.csv
id,見出し語,読み,品詞,カテゴリ,ドメイン
1,連結営業利益,れんけつえいぎょうりえき,普通名詞,抽象物,ビジネス
2,米国債,べいこくさい,,抽象物,ビジネス
3,上方修正,じょうほうしゅうせい,サ変名詞,抽象物,ビジネス
4,日本航空,にほんこうくう,組織名,,
5,夏目漱石,なつめそうせき,人名,日本,姓
6,安倍首相 安倍晋太郎 安倍晋太郎首相,あべしゅしょう,人名,日本,姓名
7,2ちゃんねる にちゃんねる,にちゃんねる,,,
$ mjumandic.rb i=dic1.csv O=jumandic
#END# kgcut f=品詞,見出し語,読み i=dic1.csv
#END# kgdelnull f=見出し語,読み
#END# kgsortf f=見出し語
#END# kguniq k=見出し語 o=/tmp/__MTEMP_68157_70357348549040_0
#END# mcsvin i=/tmp/__MTEMP_68157_70357348549040_0
Mon Jul 28 01:38:37 2014
/usr/local/share/juman/dic/JUMAN.grammar parsing... done.

Mon Jul 28 01:38:37 2014
/usr/local/share/juman/dic/JUMAN.katuyou parsing... done.

Mon Jul 28 01:38:37 2014
/usr/local/share/juman/dic/jumandic.tab parsing... done.

jumandic.dic parsing... done.

execution time:    0.000s
processor time:    0.000s
File Name "/Users/maegawa/git/nysol/nysol/doc/fumi/jp/examples/jumandic/jumandic.dat"

## 10 entry  814 th char
Saving pat-tree "/Users/maegawa/git/nysol/nysol/doc/fumi/jp/examples/jumandic/jumandic.pat" 
QUIT
#MSG# jumandic内のjumandic.dat,jumandic.patの2つのファイルがユーザ辞書として必要となる。
#MSG# ~/.jumanrcファイルを編集し、これらのファイルが格納されたパス名を以下のように追加登録す
#MSG# (辞書ファイル
#MSG#         /usr/local/share/juman/dic
#MSG#         /usr/local/share/juman/autodic
#MSG#         /usr/local/share/juman/wikipediadic
#MSG#         /Users/maegawa/git/nysol/nysol/doc/fumi/jp/examples/jumandic
#MSG# )
#END# /Users/maegawa/.rvm/rubies/ruby-2.0.0-p247/bin/mjumandic.rb i=dic1.csv O=jumandic
$ ls jumandic
jumandic.dat
jumandic.dic
jumandic.int
jumandic.pat
$ more jumandic/jumandic.dic
(名詞 (サ変名詞 ((読み じょうほうしゅうせい) (見出し語 上方修正) (意味情報 "代表表記:上方修
(名詞 (人名 ((読み なつめそうせき) (見出し語 夏目漱石) (意味情報 ""))))
(名詞 (人名 ((読み あべしゅしょう) (見出し語 安倍首相 安倍晋太郎 安倍晋太郎首相) (意味情報 "
(名詞 (組織名 ((読み にほんこうくう) (見出し語 日本航空) (意味情報 "代表表記:日本航空/にほん
(名詞 (普通名詞 ((読み べいこくさい) (見出し語 米国債) (意味情報 "代表表記:米国債/べいこくさ
(名詞 (普通名詞 ((読み れんけつえいぎょうりえき) (見出し語 連結営業利益) (意味情報 "代表表記
(名詞 (普通名詞 ((読み にちゃんねる) (見出し語 2ちゃんねる にちゃんねる) (意味情報 "代表表
\end{Verbatim}
