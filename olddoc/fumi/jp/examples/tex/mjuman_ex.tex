\subsubsection*{例1: 基本例}

\verb|text|ディレクトリに文書ファイル\verb|test.txt|を置き、
形態素解析を実行する。結果は\verb|csv|ディレクトリに出力する。


\begin{Verbatim}[baselinestretch=0.7,frame=single]
$ more text/test.txt
子どもはリンゴがすきです。
望遠鏡で泳ぐ少女を見た。
$ mjuman.rb I=text O=csv
#MSG# KNP: reading text/test.txt
#MSG# JUMAN: MP-2 aid=test.txt sid=0 (sentences:1/2, articles:1/1)
#MSG# JUMAN: MP-2 aid=test.txt sid=1 (sentences:2/2, articles:1/1)
#MSG# JUM2CSV 1/1
#MSG# Elapse: 0.048sec, # of sentences=2, # of articles=1
#MSG#   0.024sec/sentence, 0.048sec/article
#MSG#   mpCount=2, poolSize=1000
#MSG#   maxLen=512Byte, maxSec=30sec, sizeLimit=2000MB
#END# /Users/maegawa/.rvm/rubies/ruby-2.0.0-p247/bin/mjuman.rb I=text O=csv
$ more csv/test.txt
aid,sid,tid,word,orgWord,daiWord,yomi,class1,class2,class3,class4,annotation
test.txt,0,0,子ども,子ども,子供,こども,名詞,普通名詞,,,代表表記:子供/こども カテゴリ:人
test.txt,0,1,は,は,,は,助詞,副助詞,,,
test.txt,0,2,リンゴ,リンゴ,林檎,りんご,名詞,普通名詞,,,代表表記:林檎/りんご カテゴリ:植物
test.txt,0,3,が,が,,が,助詞,格助詞,,,
test.txt,0,4,すきだ,すきです,好きだ,すきです,形容詞,,ナ形容詞,デス列基本形,代表表記:好きだ/
test.txt,0,5,。,。,,。,特殊,句点,,,
test.txt,1,0,望遠,望遠,望遠,ぼうえん,名詞,普通名詞,,,代表表記:望遠/ぼうえん カテゴリ:抽象物
test.txt,1,1,鏡,鏡,鏡,かがみ,名詞,普通名詞,,,代表表記:鏡/かがみ 漢字読み:訓 カテゴリ:人工物-
test.txt,1,2,で,で,,で,助詞,格助詞,,,
test.txt,1,3,泳ぐ,泳ぐ,泳ぐ,およぐ,動詞,,子音動詞ガ行,基本形,代表表記:泳ぐ/およぐ
test.txt,1,4,少女,少女,少女,しょうじょ,名詞,普通名詞,,,代表表記:少女/しょうじょ カテゴリ:人
test.txt,1,5,を,を,,を,助詞,格助詞,,,
test.txt,1,6,見る,見た,見る,みた,動詞,,母音動詞,タ形,代表表記:見る/みる 補文ト 自他動詞:自:
test.txt,1,7,。,。,,。,特殊,句点,,,
\end{Verbatim}
\subsubsection*{例2: JUMANの結果(オリジナル)も出力する例}

JUMANの結果(オリジナル)も\verb|juman|ディレクトリに出力しておく。


\begin{Verbatim}[baselinestretch=0.7,frame=single]
$ more text/test.txt
子どもはリンゴがすきです。
望遠鏡で泳ぐ少女を見た。
$ mjuman.rb I=text O=csv P=juman
#MSG# KNP: reading text/test.txt
#MSG# JUMAN: MP-2 aid=test.txt sid=0 (sentences:1/2, articles:1/1)
#MSG# JUMAN: MP-2 aid=test.txt sid=1 (sentences:2/2, articles:1/1)
#MSG# JUM2CSV 1/1
#MSG# Elapse: 0.054sec, # of sentences=2, # of articles=1
#MSG#   0.027sec/sentence, 0.054sec/article
#MSG#   mpCount=2, poolSize=1000
#MSG#   maxLen=512Byte, maxSec=30sec, sizeLimit=2000MB
#END# /Users/maegawa/.rvm/rubies/ruby-2.0.0-p247/bin/mjuman.rb I=text O=csv P=juman
$ more juman/test.txt
子ども こども 子ども 名詞 6 普通名詞 1 * 0 * 0 "代表表記:子供/こども カテゴリ:人"
は は は 助詞 9 副助詞 2 * 0 * 0 NIL
リンゴ りんご リンゴ 名詞 6 普通名詞 1 * 0 * 0 "代表表記:林檎/りんご カテゴリ:植物
が が が 助詞 9 格助詞 1 * 0 * 0 NIL
すきです すきです すきだ 形容詞 3 * 0 ナ形容詞 21 デス列基本形 29 "代表表記:好きだ/すきだ 反
。 。 。 特殊 1 句点 1 * 0 * 0 NIL
EOS
望遠 ぼうえん 望遠 名詞 6 普通名詞 1 * 0 * 0 "代表表記:望遠/ぼうえん カテゴリ:抽象物"
鏡 かがみ 鏡 名詞 6 普通名詞 1 * 0 * 0 "代表表記:鏡/かがみ 漢字読み:訓 カテゴリ:人工物-その
で で で 助詞 9 格助詞 1 * 0 * 0 NIL
泳ぐ およぐ 泳ぐ 動詞 2 * 0 子音動詞ガ行 4 基本形 2 "代表表記:泳ぐ/およぐ"
少女 しょうじょ 少女 名詞 6 普通名詞 1 * 0 * 0 "代表表記:少女/しょうじょ カテゴリ:人"
を を を 助詞 9 格助詞 1 * 0 * 0 NIL
見た みた 見る 動詞 2 * 0 母音動詞 1 タ形 10 "代表表記:見る/みる 補文ト 自他動詞:自:見える/
。 。 。 特殊 1 句点 1 * 0 * 0 NIL
EOS
\end{Verbatim}
