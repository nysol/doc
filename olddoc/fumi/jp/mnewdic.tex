

%\documentclass[a4paper]{jarticle}


\section{mnewdic.rb コーパスからの隣接単語ペア候補出力\label{sect:mnewdic}}
\index{mnewdic@mnewdic}

コーパス(大量の文例からなるテキストファイル)から、
辞書に登録すべき隣接単語ペアの候補情報を出力する。

JUMANが標準で搭載する辞書は一般的なもののため、
特定分野のテキストをマイニングするには、
1つの語として解釈(形態素解析)して欲しい語が
複数の語に分割されてしまうことがある。
\verb|mnewdic.rb|コマンドは
与えられたコーパスを分析し、並んで出現する頻度が高く
1つの語として解釈すべき可能性が高い語のペアを
リストアップしてCSVに出力する。

このCSVを参考に登録の要否を人目で判断し、
\verb|mjumandic.rb|コマンドに与えることで
JUMANへの辞書登録が容易になる。


\subsection{書式}
\begin{verbatim}
mnewdic.rb [i=] [O=] [S=] [n=] [seed=] [-dai] [-mcmdenv] [--help]
\end{verbatim}

\begin{table}[htbp]
{\small
\begin{tabular}{ll}
\verb|i=|       & : コーパスファイル名 \\
\verb|O=|       & : 出力ディレクトリ名 \\
\verb|S=|       & : 単語ペア出現件数最小値 \\
\verb|n=|       & : 単語ペアごとに出力する文例数 \\
\verb|seed=|    & : 乱数の種 \\
\verb|-dai|     & : 見出し語として代表表記を使う \\
\verb|-mcmdenv| & : 内部で利用しているMCMDのメッセージ出力レベルを環境変数に任せる。 \\
                &   省略時は警告とエラーメッセージのみ出力(KG\_VerboseLevel=2)。 \\
\verb|--help|   & : ヘルプメッセージの表示 \\
\end{tabular} 
}
\end{table} 

\subsubsection*{入力ファイル例}

\verb|i=|パラメータで指定するコーパスファイルには、
1行1文にしたテキストファイルを与える。

\begin{Verbatim}[baselinestretch=0.7,frame=single]
3年ぶりにウォークマンを買ったけど、育休中はあまり活躍の余地がないですね。
待機児童解消の方がいい気がするけど。
:
\end{Verbatim}

\subsubsection*{出力ファイル例}

\verb|O=|パラメータで指定したディレクトリには、
\verb|words.csv|ファイルと\verb|corpus.csv|ファイルが出力される
(\verb|nkf|コマンドがインストールされている場合、
両ファイルの文字コードをShift JISに変換したファイルも同時に出力される)。

\begin{Verbatim}[baselinestretch=0.7,frame=single]
見出し語,品詞,読み,カテゴリ,ドメイン,pid,word1,word2,freq
職場復帰,,,,,0,職場,復帰,31
授業参観,,,,,1,授業,参観,28
会議参加,,,,,2,会議,参加,26
:
\end{Verbatim}

\newpage

\verb|words.csv|ファイルの各項目の意味を以下に示す。

\begin{table}[htbp]
{\small
\begin{tabular}{ll}
\verb|見出し語| & : 見出し語 \\  
\verb|品詞|     & : 品詞 \\  
\verb|読み|     & : 読み \\  
\verb|カテゴリ| & : カテゴリ \\  
\verb|ドメイン| & : ドメイン \\  
\verb|pid|      & : pid \\  
\verb|word1|    & : 語1 \\  
\verb|word2|    & : 語2 \\  
\verb|freq|     & : 出現頻度 \\  
\end{tabular} 
}
\end{table} 

\verb|corpus.csv|は、登録候補の語がどのようなテキストに出現したのかを
確認するために参照すればよい。

\begin{Verbatim}[baselinestretch=0.7,frame=single]
pid,id,text
0,52,"神戸で始めての、育休後職場復帰セミナーを開催しました。"
0,317,"僕の知り合いは2人子どもを産んで、立て続けに産休+育休を取って、職場復帰した。"
:
\end{Verbatim}




\subsection{利用例}
\subsubsection*{例1: 基本例}



\begin{Verbatim}[baselinestretch=0.7,frame=single]
$ head tweets.txt
3年ぶりにウォークマンを買ったけど、育休中はあまり活躍の余地がないですね。
待機児童解消の方がいい気がするけど。
読売テレビ(日本テレビ系列)ウェークアップ!ぷらすに蓮舫ネクスト規制改革担当大臣が生出演!
この学生さんは、仕事に不利じゃなかったら、3年育休取れるのも良いな、って思ってるよね。
今の人事制度のまま育休3年とか、前以上に女性が締め出されるだけでは。
女子大生でも分かる、3年間の育児休暇が最悪な結果をもたらす理由。(中嶋よしふみ)
保育園を中心に期間とか決めるの、おかしいよな~。
女性が必ず子育てしなきゃならない社会なら結婚絶対したくない…
育休とかの前に、母親に育児に専念させるなら女性の雇用よりもまず、男性の雇用、給料なんだよね。
安倍総理きた! 育児休暇三年は…女としては嬉しいけど、会社に申し訳ないよねえ
$ mnewdic.rb i=tweets.txt O=newdic
#MSG# start to parse each line...
#MSG# working at line 0
#MSG# working at line 100
#MSG# working at line 200
#MSG# working at line 300
#MSG# working at line 400
#MSG# working at line 500
#MSG# working at line 600
#MSG# working at line 700
#MSG# working at line 800
#MSG# working at line 900
#MSG# working at line 1000
#END# /Users/maegawa/.rvm/rubies/ruby-2.0.0-p247/bin/mnewdic.rb i=tweets.txt O=newdic
$ ls newdic
corpus.csv
corpus_sjis.csv
words.csv
words_sjis.csv
$ head newdic/words.csv
見出し語,品詞,読み,カテゴリ,ドメイン,pid,word1,word2,freq
職場復帰,,,,,0,職場,復帰,31
授業参観,,,,,1,授業,参観,28
会議参加,,,,,2,会議,参加,26
育休延長,,,,,3,育休,延長,19
子育て支援,,,,,4,子育て,支援,18
育児休暇3,,,,,5,育児休暇,3,18
待機児童ゼロ,,,,,6,待機児童,ゼロ,17
規制緩和,,,,,7,規制,緩和,16
給付金,,,,,8,給付,金,15
\end{Verbatim}


%\end{document}

