

%\documentclass[a4paper]{jarticle}


\section{mcaseframe.rb 格フレームの抽出\label{sect:mcaseframe}}
\index{mcaseframe@mcaseframe}

KNPの解析結果から、格フレームを抽出し出力する。

格フレームとは、用言とそれに係る格助詞句からなる組をいい、
「リンゴ(が)」+「好き」、「望遠鏡(で)」+「見る」のように表現される。
本コマンドは、\verb|mknp.rb|コマンドが出力した構文解析結果
(XML)を読み込み、格フレームを抽出してCSVに出力する。

\subsection{書式}
\begin{verbatim}
mcaseframe.rb I= o= [-key] [-mcmdenv] [--help]
\end{verbatim}

\begin{table}[htbp]
{\small
\begin{tabular}{ll}
\verb|I=|        & : mknp.rbでparsingした結果xmlファイルが格納されたパス名 \\
\verb|o=|        & : 出力する格フレームファイル名 \\
\verb|-key|      & : key型フォーマットで出力する。 \\
\verb|-mcmdenv|  & : 内部で利用しているMCMDのメッセージ出力レベルを環境変数に任せる。 \\
                 &   省略時は警告とエラーメッセージのみ出力(KG\_VerboseLevel=2)。 \\
\verb|--help|    & : ヘルプメッセージの表示 \\
\end{tabular} 
}
\end{table} 

\subsubsection{格フレームの抽出}

\verb|mknp.rb|コマンドの出力するXMLは次のようになっている(抜粋)。

\begin{Verbatim}[baselinestretch=0.7,frame=single]
 <sentence id='0' text='子どもはリンゴがすきです。'>
  <chunk id='0' link='2' phraseType='格助詞句' caseType='ガ2格' phrase='子供' phraseTok='子
   <token id='0' class1='名詞' class2='普通名詞' word='子ども' orgWord='子ども' daiWord='子供
   <token id='1' class1='助詞' class2='副助詞' word='は' orgWord='は'/>
  </chunk>
  <chunk id='1' link='2' phraseType='格助詞句' caseType='ガ格' phrase='林檎' phraseTok='リン
   <token id='2' class1='名詞' class2='普通名詞' word='リンゴ' orgWord='リンゴ' daiWord='林檎
   <token id='3' class1='助詞' class2='格助詞' word='が' orgWord='が'/>
  </chunk>
  <chunk id='2' link='-1' phraseType='用言句' phraseTok='すきだ' rawPhrase='すきです。' phrase
   <token id='4' class1='形容詞' class3='ナ形容詞' class4='デス列基本形' word='すきだ' orgWord
   <token id='5' class1='特殊' class2='句点' word='。' orgWord='。'/>
  </chunk>
 </sentence>
\end{Verbatim}

上の例だと、
chunk id='0'「子どもは」はlink='2'に、
chunk id='1'「リンゴが」もlink='2'になっており、
いずれもchunk id='2'「すきです」に係っていることがわかる。
図にすると次のような係り受け関係である。

\begin{Verbatim}[baselinestretch=0.7,frame=single]
子どもは──┐ 
リンゴが──┤ 
      すきです。
\end{Verbatim}

本コマンドを利用すると、係り受け関係は次のようなCSVとして抽出される。

\begin{Verbatim}[baselinestretch=0.7,frame=single]
aid,sid,cid,contrastConj,denial,declinableWord,lid,caseWord,case
test.txt,0,2,,,すきだ,0,子ども,ガ2
test.txt,0,2,,,すきだ,1,リンゴ,ガ
\end{Verbatim}

CSVの各項目の意味を以下に示す。

\begin{table}[htbp]
{\small
\begin{tabular}{ll}
\verb|aid|            & : 入力ファイル名 \\  
\verb|sid|            & : 行番号(センテンスID) \\  
\verb|cid|            & : チャンクID \\  
\verb|contrastConj|   & : 逆接接続詞 \\  
\verb|denial|         & : 否定語を伴うチャンクのとき1 \\  
\verb|declinableWord| & : 用言句 \\  
\verb|lid|            & : 格助詞句のチャンクID \\  
\verb|caseWord|       & : 格助詞句 \\  
\verb|case|           & : 格助詞句の種類 \\  
\end{tabular} 
}
\end{table} 

\vspace{3cm}





\subsection{利用例}
\subsubsection*{Example 1: Basic example}

Example used in the previous section. One line has become one case frame. 


\begin{Verbatim}[baselinestretch=0.7,frame=single]
$ more xml/test.txt
<?xml version='1.0' encoding='UTF-8'?>
<article id='test.txt'>
  <sentence id='0' text='子どもはリンゴがすきです。'>
    <chunk id='0' link='2' phraseType='格助詞句' caseType='ガ2格' phrase='子供' phraseTok='
      <token id='0' class1='名詞' class2='普通名詞' word='子ども' orgWord='子ども' daiWord='
      <token id='1' class1='助詞' class2='副助詞' word='は' orgWord='は'/>
    </chunk>
    <chunk id='1' link='2' phraseType='格助詞句' caseType='ガ格' phrase='林檎' phraseTok='リ
      <token id='2' class1='名詞' class2='普通名詞' word='リンゴ' orgWord='リンゴ' daiWord='
      <token id='3' class1='助詞' class2='格助詞' word='が' orgWord='が'/>
    </chunk>
    <chunk id='2' link='-1' phraseType='用言句' phraseTok='すきだ' rawPhrase='すきです。' ph
      <token id='4' class1='形容詞' class3='ナ形容詞' class4='デス列基本形' word='すきだ' or
      <token id='5' class1='特殊' class2='句点' word='。' orgWord='。'/>
    </chunk>
  </sentence>
  <sentence id='1' text='望遠鏡で泳ぐ少女を見た。'>
    <chunk id='0' link='3' phraseType='格助詞句' caseType='デ格' phrase='望遠鏡' phraseTok='
      <token id='0' class1='名詞' class2='普通名詞' word='望遠' orgWord='望遠' daiWord='望遠
      <token id='1' class1='名詞' class2='普通名詞' word='鏡' orgWord='鏡' daiWord='鏡' cate
      <token id='2' class1='助詞' class2='格助詞' word='で' orgWord='で'/>
    </chunk>
    <chunk id='1' link='2' phraseType='用言句' phrase='泳ぐ' phraseTok='泳ぐ' rawPhrase='泳
      <token id='3' class1='動詞' class3='子音動詞ガ行' class4='基本形' word='泳ぐ' orgWord=
    </chunk>
    <chunk id='2' link='3' phraseType='格助詞句' caseType='ヲ格' phrase='少女' phraseTok='少
      <token id='4' class1='名詞' class2='普通名詞' word='少女' orgWord='少女' daiWord='少女
      <token id='5' class1='助詞' class2='格助詞' word='を' orgWord='を'/>
    </chunk>
    <chunk id='3' link='-1' phraseType='用言句' phraseTok='見る' rawPhrase='見た。' phrase='
      <token id='6' class1='動詞' class3='母音動詞' class4='タ形' word='見る' orgWord='見た'
      <token id='7' class1='特殊' class2='句点' word='。' orgWord='。'/>
    </chunk>
  </sentence>
</article>mcaseframe.rb I=xml o=caseframe.csv
#END# /Users/maegawa/.rvm/rubies/ruby-2.0.0-p247/bin/mcaseframe.rb I=xml o=caseframe.csv
more caseframe.csv
aid,sid,cid,contrastConj,denial,declinableWord,lid,caseWord,case
test.txt,0,2,,,すきだ,0,子ども,ガ2
test.txt,0,2,,,すきだ,1,リンゴ,ガ
test.txt,1,3,,,見る,0,望遠鏡,デ
test.txt,1,3,,,見る,2,少女,ヲ
\end{Verbatim}
\subsubsection*{Example 2: Output of key type format}

When executing by adding the option \verb|-key|, 
case particle influencing inflectable word from the line is saved in output. 


\begin{Verbatim}[baselinestretch=0.7,frame=single]
$ mcaseframe.rb -key I=xml o=caseframe2.csv
#END# /Users/maegawa/.rvm/rubies/ruby-2.0.0-p247/bin/mcaseframe.rb -key I=xml o=caseframe2.c
$ more caseframe2.csv
aid,sid,cid,contrastConj,denial,lid,word,type
test.txt,0,2,,,2,すきだ,用言
test.txt,0,2,,,0,子ども,ガ2
test.txt,0,2,,,1,リンゴ,ガ
test.txt,1,1,,,1,泳ぐ,用言
test.txt,1,3,,,3,見る,用言
test.txt,1,3,,,0,望遠鏡,デ
test.txt,1,3,,,2,少女,ヲ
\end{Verbatim}


%\end{document}

