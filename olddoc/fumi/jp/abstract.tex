
\section{概要}

本「Fumi(文)」パッケージは、日本語のテキストマイニングに関する
複数のコマンドから構成される。

データマイニングはこれまで、
数値情報を中心とした定量的・定型的なデータを主に扱ってきた。
しかし近年、コンピュータの性能向上やデータ分析技術の発展などにより、
非定型的な情報を扱うことも可能となってきた。
非定型情報の代表格が、「人が書いた文章」(自然言語)である。

本パッケージを用いると、日本語の形態素解析・構文解析を
容易に実行することができる。その結果はCSVとして出力されるので、
Mコマンドによる各種の処理ができるため、さまざまな分析モデルに
投入することが可能となる。

なお本パッケージでは、京都大学情報学研究科 黒橋・河原研究室が開発する
形態素解析システムJUMAN、構文解析システムKNPを用いている。
JUMANおよびKNPについての詳細は、以下の公式ページを参照のこと。

\begin{itemize}
\item 形態素解析システムJUMAN \url{http://nlp.ist.i.kyoto-u.ac.jp/index.php?JUMAN}
\item 構文解析システムKNP \url{http://nlp.ist.i.kyoto-u.ac.jp/index.php?KNP}
\end{itemize}



%本「View(眺)」パッケージは、複数のデータ視覚化用コマンドから構成されている。
%視覚化とは、データをグラフやチャートとして描画することをいい、
%データを概観したり、資料に掲載したりする際には重要な役割を果たす。

%本パッケージには、
%円グラフを描画する\verb|mpie.rb|コマンド、
%Sankeyダイアグラム(流量グラフ)を描画する\verb|msankey.rb|コマンド、
%グラフデータを汎用的な形式に変換する\verb|mgv.rb|コマンド
%が含まれている。

%CSV形式のデータを入力として与え(Ruby拡張ライブラリを用いている)、
%D3ライブラリ(Data-Driven Documents)で作成された視覚化アプリケーションsankey diagram
%(\url{http://bost.ocks.org/mike/sankey/})を利用した、Sankeyダイアグラムや、
%円グラフを組み込んだ単体のhtmlを出力する。

