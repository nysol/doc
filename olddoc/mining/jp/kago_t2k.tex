
% how to compile: platex xxx.tex ; dvipdfmx xxx.dvi

\documentclass[a4paper]{jarticle}

%--余白の設定
\setlength{\topmargin}{20mm}
\addtolength{\topmargin}{-1in}
\setlength{\oddsidemargin}{20mm}
\addtolength{\oddsidemargin}{-1in}
\setlength{\evensidemargin}{15mm}
\addtolength{\evensidemargin}{-1in}
\setlength{\textwidth}{170mm}
\setlength{\textheight}{254mm}
\setlength{\headsep}{0mm}
\setlength{\headheight}{0mm}
\setlength{\topskip}{0mm}

%--ハイバーリンクを可能にするパッケージ
\usepackage[dvipdfmx,%
 bookmarks=true,%
 bookmarksnumbered=true,%
 colorlinks=true,%
 setpagesize=false,%
 pdftitle={mcat},%
 pdfauthor={BMRC},%
 pdfkeywords={TeX; dvipdfmx; hyperref; color;}]{hyperref}


\begin{document}

\setlength{\baselineskip}{4mm}

\section*{kago\_t2k.rb コマンド}
籠(kago)で使うトランザクションデータは図\ref{tbl:key}に例示されるような
トランザクションキーとアイテムのペアをレコードとするCSVデータである(以下、key型CSVと呼ぶ)。
本コマンドは図\ref{tbl:tra}に示されるような、アイテムをスペースで区切って一つの項目とした
フォーマット(tra型CSVと呼ぶ)をkey型CSVに変換する。
アイテムに重複があれば単一化される。

クラス項目を含むデータ(図\ref{tbl:traClass})であれば、
クラス項目名と出力するクラスファイル名を指定することで、
key型CSVファイルとクラスファイルに分割出力できる。
詳細は利用例を参照のこと。

さらに、アイテム系列のデータに変換することも可能で、
その場合、tra型データのアイテムの出現順序に応じて系列番号を出力する。
詳細は利用例を参照のこと。

\begin{table}[htbp]
\begin{center}
\begin{tabular}{ccc}

\begin{minipage}{0.3\hsize}
\begin{center}
\caption{key型データ\label{tbl:key}}
{\small
\begin{tabular}{ll}
\hline
key&item \\
\hline
a&P \\
a&Q \\
b&R \\
b&Q \\
b&S \\ \hline
\end{tabular} 
}
\end{center}
\end{minipage}

\begin{minipage}{0.3\hsize}
\begin{center}
\caption{tra型データ\label{tbl:tra}}
{\small
\begin{tabular}{ll}
\hline
id&item \\
\hline
a&P Q \\
b&R Q S \\
\hline
\\
\\
\\
\end{tabular} 
}
\end{center}
\end{minipage}

\begin{minipage}{0.4\hsize}
\begin{center}
\caption{クラス項目を持つtra型データ\label{tbl:traClass}}
{\small
\begin{tabular}{lll}
\hline
id&class&item \\
\hline
a&c1&P Q \\
b&c2&R Q S \\
\hline
\\
\\
\\
\end{tabular} 
}
\end{center}
\end{minipage}

%\begin{minipage}{0.4\hsize}
%\begin{center}
%\caption{matrix型データ\label{tbl:matrix}}
%{\small
%\begin{tabular}{lllll}
%\hline
%id&apple&orange&grape&pine \\
%\hline
%a&1&1&&\\
%b&&1&1&1\\
%\hline
%\\
%\\
%\\
%\end{tabular} 
%}
%\end{center}
%\end{minipage}

\end{tabular} 
\end{center}
\end{table} 



\subsection*{書式}
\begin{verbatim}
kago\_t2k.rb tid= item= [seq=] [class=] i= o= [c=] [--help]
\end{verbatim}

\begin{description}
\setlength{\itemindent}{0mm}
\item[i=    ] tra型データファイル名【必須】
\item[tid=  ] トランザクションID項目名【必須】
\item[item= ] アイテム項目名【必須】
\item[seq=  ] 新たに生成される系列項目名【オプション】
\item[class=] クラス項目名【c=を指定した時のみ必須】
\item[o=    ] 出力ファイル名【必須】
\item[c=    ] 出力クラスファイル名【オプション】

\end{description}

\subsection*{備考}
\begin{itemize}
\item 
\end{itemize}

\subsection*{利用例}
\subsubsection*{例1 基本例}

\begin{verbatim}
------------------------------------------------
# tra1.csv
user,fruit
Tom,apple orange
Mary,orange grape pine
Jim,pine apple

$ kago_t2k.rb tid=user item=fruit i=tra1.csv o=tra1out.csv

# tra1out.csv
user,fruit
Jim,apple
Jim,pine
Mary,grape
Mary,orange
Mary,pine
Tom,apple
Tom,orange
------------------------------------------------
\end{verbatim}

\subsubsection*{例2 クラス項目を持ったtra型データからkey型ファイルとclassデータに分割する例}

\begin{verbatim}
------------------------------------------------
# tra2.csv
user,gender,fruit
Tom,male,apple orange
Mary,female,orange grape pine
Jim,male,pine apple

$ kago_t2k.rb tid=user item=fruit class=gender i=tra2.csv o=dat1.csv c=tra2out.csv

# tra2out.csv
user,fruit
Jim,apple
Jim,pine
Mary,grape
Mary,orange
Mary,pine
Tom,apple
Tom,orange

# tra2class.csv
user,gender
Jim,male
Mary,female
Tom,male
------------------------------------------------
\end{verbatim}

\subsubsection*{例3 系列データとして出力する例}

\begin{verbatim}
------------------------------------------------
# tra3.csv
user,fruit
Tom,apple orange apple
Mary,orange grape pine
Jim,pine apple

$ kago_t2k.rb tid=user item=fruit seq=time i=tra3.csv o=tra3out.csv

# tra3out.csv
user,time,fruit
Jim,1,pine
Jim,2,apple
Mary,1,orange
Mary,2,grape
Mary,3,pine
Tom,1,apple
Tom,2,orange
Tom,3,apple
------------------------------------------------
\end{verbatim}

\subsection*{関連コマンド}

\hyperlink{kago\_m2k.pdf}{kago\_m2k} : matrix型データの変換

\end{document}

