
%----------------------------------------------------------------------------------------
% MCMD Basic Techniques 	
% Lessons on Basic Commands  
%----------------------------------------------------------------------------------------

%\documentclass[a4paper]{jarticle}
%\begin{document}

%\section*{基本コマンド}

\section{レッスン 2: ソート(並べ替え)}

\verb|msortf| コマンドは、指定された項目に従ってレコード(行)を並べ替えます。

%\section{msortfコマンドの使い方}

\subsection{ステップ 1: スクリプトを作成する}

テキストエディタを使って、スクリプトファイル \verb|msortf.sh| を作成しましょう。作成したら、ファイルに実行権限を与えるのを忘れないようにしてください。\\

このレッスンでは、トランザクションデータ(スーパーのレシートデータ)から日別の数量合計・金額合計を求め、最近の日付から順に表示されるレポート(一覧表)を作成します。\\

そのためには、まず日付、数量、金額の3項目を \verb|mcut| コマンドを使って取り出し、それをパイプ(|)を介して \verb|msortf| コマンドに渡します。\verb|msortf| コマンドはデータを日付の降順に並べ替え、やはりパイプを通して \verb|msum| コマンドに渡します。\verb|msum| コマンドは日付ごとに数量と金額を合計し、最後は \verb|msortfout.csv| ファイルに書き出します。\\

 \subsection{ステップ 2: パラメータを指定する}

{\setlength{\parindent}{0cm}

\verb|msortf| コマンドに与えるパラメータは次のようになります。\\

項目: 		\verb|f=日付%r| \\
説明: データは「日付」項目の値でソートされます。項目の指定には、「\%r」「\%n」「\%nr」のいずれかを付け加えることができます。\%rは、ソート順を逆順(降順;大→小)にします。\%nは、項目の値を数値と見なして大小関係を判定します(\%nをつけないと、文字列とみなして大小関係を判定します)。\%nrは、\%nと\%rの両方を指定したことになります。
なおf=パラメータでは、「,」で区切って複数の項目を指定することもできます。} \\

このレッスンでは、レッスン1で作成した \verb|dat.csv| (スーパーマーケットのレシートデータ)を使います。

\begin{verbatim}
#!/bin/bash
#=========================================
# MCMD bash script - Lesson 2: Sort 

# Variables
inPath="tutorial_jp"

# Command 
mcut f=日付,数量,金額 i=${inPath}/dat.csv |
msortf f=日付%r |       
msum k=日付 f=数量:数量合計,金額:金額合計 o=outdat/msortfout.csv

#==========================================
\end{verbatim}

\subsection{ステップ 3: シェルスクリプトを実行する}

\begin{verbatim}
$ ./msortf.sh 
#END# kgcut f=日付,数量,金額 i=tutorial_jp/dat.csv; IN=24737 OUT=24737; 2013/08/14 18:01:51
#END# kgsortf f=日付%r; IN=24737 OUT=24737; 2013/08/14 18:01:51
#END# kgsum f=数量:数量合計,金額:金額合計 k=日付 o=outdat/msortfout.csv; IN=24737 OUT=324;
2013/08/14 18:01:51
\end{verbatim}

\noindent
結果は次の通りです。日付の降順に並んでいることを確認しましょう。

\begin{verbatim}
日付,数量合計,金額合計
20011230,93,58331
20011229,300,151488
20011228,292,162534
20011227,112,61542
20011226,73,34360
20011225,202,109219
20011224,305,141597
20011222,121,71352
20011221,156,93305
20011220,153,77793
20011219,168,91842
20011218,251,125010
20011217,19,8687
20011216,154,76426
20011215,160,74503
20011214,71,32422
20011213,124,67817
20011212,80,32682
20011211,251,133119
20011210,55,31919
20011209,47,23439
20011208,442,218180
20011207,285,134496
20011206,70,39913
20011205,175,85918
20011204,26,10297
20011203,99,45827
20011202,285,129702
20011201,117,73060
...
..
\end{verbatim}

\subsection{そのほかの例}

\noindent

msortfコマンドを使ったソートの例をいくつか紹介します。\\

例 1: 「金額」の小さい順(昇順)にソートしたいときは、項目に\%nを加えます。
\begin{verbatim}
msortf f=金額%n
\end{verbatim}

例 2: 「金額」の大きい順(降順)にソートしたいときは、項目に\%nrを加えます。
\begin{verbatim}
msortf f=金額%nr
\end{verbatim}

例 3: 「日付」の大きい順(新しい日付から古い日付の順)にソートし、各日の中では「金額」の小さい順でソートしたいときは、次のように指定します。
\begin{verbatim}
msortf f=日付%nr,金額%n
\end{verbatim}

\newpage

\subsection{練習問題}

msortfコマンドの練習をしましょう。出力ファイルとスクリプトを確認してください。

\begin{table}[htbp]
%\begin{center}
{\small
\begin{tabular}{ l | c || r }
\hline
\textbf{レポートの内容}   & \textbf{スクリプト名} & \textbf{出力ファイル名}  \\
\hline
1. 日別の数量合計と金額合計を求めたい。結果は数量の昇順でソートする。  & \href{exercise/msortf1.sh}{msortf1.sh} & \href{exercise/outdat/msortfout1.csv}{msortfout1.csv} \\
2. 日別の数量合計と金額合計を求めたい。結果は数量の降順でソートする。  & \href{exercise/msortf2.sh}{msortf2.sh} & \href{exercise/outdat/msortfout2.csv}{msortfout2.csv} \\
3. 日別かつ商品の小分類(4桁の商品カテゴリコード)別に、数量合計と\\金額合計を求めたい。結果は日付と合計数量の昇順でソートする。& \href{exercise/msortf3.sh}{msortf3.sh} & \href{exercise/outdat/msortfout3.csv}{msortfout3.csv} \\
4. 日別かつ商品の小分類(4桁の商品カテゴリコード)別に、数量合計と\\金額合計を求めたい。結果は日付と合計数量の降順でソートする。 & \href{exercise/msortf4.sh}{msortft4.sh} & \href{exercise/outdat/msortfout4.csv}{msortfout4.csv} \\

\hline
\end{tabular} 
}
\end{table} 


%\end{document}
