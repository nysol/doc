
%----------------------------------------------------------------------------------------
% MCMD Basic Techniques 	
% Lessons on Basic Commands  
%----------------------------------------------------------------------------------------

%\documentclass[a4paper]{jarticle}
%\begin{document}

\section{レッスン 6: mselコマンドでレコードを選択する}

大規模なデータを扱う際には、まず特定の条件に従うデータだけを取り出してみることがよくあります。\verb|msel| コマンドを使うと、数値や文字列を条件に指定して、特定のレコードを選択することができます。

\subsection{ステップ 1: スクリプトを作成する}

テキストエディタを使って、スクリプトファイル \verb|msel.sh| を作成しましょう。作成したら、ファイルに実行権限を与えるのを忘れないようにしてください。\\

このレッスンでは、2001年10月20日のトランザクションデータだけを選択します。\\

そのために、まず「日付」「数量」「金額」項目を \verb|mcut| コマンドで取り出し、パイプを介して \verb|msel| コマンドに渡します。\verb|msel| コマンドで「日付」列が「20011020」のレコードを選択し、結果を \verb|mselout.csv| ファイルに書き出します。\\

このレッスンでは、レッスン1で作成したdat.csv(スーパーマーケットのレシートデータ)を使います。

\subsection{ステップ 2: mselコマンドを使う}
 
{\setlength{\parindent}{0cm}
\verb|msel| コマンドに与えるパラメータは次のようになります。\\

条件式: \verb|c='${日付}==20011020'| \\
説明: レコード選択の条件はc=パラメータで指定します。ここでは、「日付」項目の値が20011020に等しい、という条件を式で表しています(=ではなく==、すなわち等号が2つ必要なことに注意してください)。\\
}

\begin{verbatim}
#!/bin/bash
#=========================================
# MCMD bash script - Lesson 6: Select Records 
#=========================================
# Variables
inPath="tutorial_jp"

# Command 
mcut f=日付,数量,金額 i=${inPath}/dat.csv |
msel c='${日付}==20011020' o=outdat/mselout.csv
#==========================================
\end{verbatim}

\begin{verbatim}
$ ./msel.sh 
#END# kgcut f=日付,数量,金額 i=tutorial_jp/dat.csv; IN=24737 OUT=24737; 2013/08/17 19:18:03
#END# kgsel c=${日付}==20011020 o=outdat/mselout.csv; IN=24737 OUT=106; 2013/08/17 19:18:03
\end{verbatim}

\noindent
結果は次のようになります。

\begin{verbatim}
日付,数量,金額
20011020,1,541
20011020,1,273
20011020,1,506
20011020,1,441
20011020,1,634
20011020,1,411
20011020,1,561
20011020,4,1940
20011020,3,1515
20011020,1,515
20011020,1,467
20011020,1,515
20011020,1,1077
20011020,1,357
...
..
\end{verbatim}

\if0
\subsection*{ステップ 3: mselstrコマンドを使う}

{\setlength{\parindent}{0cm}

レコードの選択条件が単純(ある項目がある値のとき、など)である場合は、\verb|mselstr| コマンドが使えます。\verb|mselstr| コマンドに与えるパラメータは次のようになります。\\

項目: \verb|f=日付| \\
説明: 選択条件となる項目はf=パラメータで指定します。ここでは、「日付」項目を指定します。\\

値: \verb|v=20011020| \\
説明: 選択条件となる値はv=パラメータで指定します。

}

\begin{verbatim}
#!/bin/bash
#=========================================
# MCMD bash script - Lesson 6: Select Records 
#=========================================
# Variables
inPath="tutorial_jp"

# Command 
mcut f=日付,数量,金額 i=${inPath}/dat.csv |
mselstr f=日付 v=20011020 o=outdat/mselout.csv
#==========================================
\end{verbatim}

%\subsubsection*{シェルスクリプトを実行する}

\begin{verbatim}
$ ./msel.sh
#END# kgcut f=日付,数量,金額i=tutorial_jp/dat.csv; IN=24737 OUT=24737; 2013/08/17 19:07:25
#END# kgselstr f=日付 o=outdat/mselout.csv v=20011020; IN=24737 OUT=106; 2013/08/17 19:07:25
\end{verbatim}

\noindent
結果は \verb|msel| コマンドを使ったときと同じで、106レコードからなるファイルが出力されます。
\fi

\subsection{ステップ 3: 複数の選択条件を指定する}

\verb|msel| コマンドを使う場合は、条件式を "\&\&" でつなげることで複数の選択条件を指定することができます。たとえば、数量が5を超え、かつ金額が1000以上のレコードを選択するには、次のようにします。

\begin{verbatim}
mcut f=date,quantity,amount i=${inPath}/dat.csv |
msel c='${数量}>5 && ${金額}>=1000' o=outdat/mselout0.csv
\end{verbatim}

複数の条件式を $\mid \mid$ でつなげると、複数の条件のいずれかを満たすレコードを選択することができます。また条件の優先順位をつけるために、括弧を使うこともできます。たとえば、2001年10月15日以降(「日付」項目が20011015を超える値)で、数量が5を超えるもしくは金額が1000以上のレコードを選択するには、次のようにします。

\begin{verbatim}
msel c='${日付}>20011015 && (${数量}>5 || ${金額}>=1000)'
\end{verbatim}

\subsection{ステップ 4: 算術演算子を使う}

\verb|mcal| コマンドを使う場合は、選択条件式に四則演算などの算術演算子を使うことができます。\\

金額=単価$\times$数量だとして、単価が100以下のレコードを選択するには次のようにします。

\begin{verbatim}
mcut f=日付,数量,金額 i=${inPath}/dat.csv |
msel c='(${金額}/${数量})<=100' o=outdat/mselout0.csv
\end{verbatim}

結果は次のようになります。

\begin{verbatim}
日付,数量,金額
20010326,1,91
20010509,5,455
20010616,1,91
20010619,1,91
20010702,1,91
20010720,1,91
20010730,5,455
20010909,1,91
20010326,1,91
20010502,1,91
20010509,5,455
20010616,1,91
20010619,1,91
20010702,1,91
20010720,1,91
20010730,1,91
20010909,1,91
20010911,1,91
20011117,1,91
20010326,1,91
20010509,6,546
20010326,1,91
20010509,5,455
20010616,5,455
\end{verbatim}

{\setlength{\parindent}{0cm}
\textbf{文字列と数値の違い}\\

\fbox{
  \parbox{\textwidth}{

CSVファイルの「金額」項目に「20」という値が記録されていたとしても、それが数値なのか文字列なのかという情報はCSVファイルには含まれていません。
しかしMコマンドでは、数値と文字列では扱いが異なります。
数値として比較すると20より100が大きいのは当然ですが、文字列として比較すると "100" より"20" の方が大きくなります。
mcalコマンドやmselコマンドにおいて、項目を数値として扱うには\$\{金額\}のように\$\{\}で囲み、文字列として扱うには\$s\{金額\}のように\$s\{\}で囲みます。
また20もただ「20」と書くと数値として扱われ、"20" と書くと文字列として扱われます。

  }
} 

}

\newpage

\subsection{練習問題}

\verb|msel|コマンドを使って、次のレポートを作成してみましょう。作成したら、出力ファイルとスクリプトを確認してください。

\begin{table}[htbp]
%\begin{center}
{\small
\begin{tabular}{ l | c || r }
\hline
\textbf{レポートの内容}   & \textbf{スクリプト名} & \textbf{出力ファイル}  \\
\hline
1. 2001年11月20日以降で、かつ金額が1000以上のレコードの数量合計と金額合計 & \href{exercise/msel1.sh}{msel1.sh} & \href{exercise/outdat/mselout1.csv}{mselout1.csv} \\
2. 17時以降で、かつ単価が200以下のレコードについて、日別の数量合計と金額合計 
 & \href{exercise/msel2.sh}{msel2.sh} & \href{exercise/outdat/mselout2.csv}{mselout2.csv} \\
3. メーカーが「0300」のレコードについて、日別の数量合計と金額合計 & \href{exercise/msel3.sh}{msel3.sh} & \href{exercise/outdat/mselout3.csv}{mselout3.csv} \\
4. メーカーが「0300」もしくは「0320」のレコードについて、日別の数量合計と金額合計 & \href{exercise/msel4.sh}{msel4.sh} & \href{exercise/outdat/mselout4.csv}{mselout4.csv} \\
\hline
\end{tabular} 
}
\end{table} 


%\end{document}
