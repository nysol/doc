
%----------------------------------------------------------------------------------------
% MCMD Basic Techniques 	
% Lessons on Basic Commands  
%----------------------------------------------------------------------------------------

%\documentclass[a4paper]{jarticle}
%\begin{document}

\section{レッスン 10: maccumコマンドで累計を求める}

\verb|maccum| コマンドを使うと、指定した項目の累計を求めることができます。

\subsection{ステップ 1: スクリプトを作成する}

テキストエディタを使って、スクリプトファイル \verb|maccum.sh| を作成しましょう。作成したら、ファイルに実行権限を与えるのを忘れないようにしてください。\\

このレッスンでは、日別の数量および金額について、年初からの累計を求めます。\\

そのために、まず「日付」「数量」「金額」の3項目を \verb|mcut| コマンドで取り出し、\verb|msortf| コマンドで日付順にソートしたのち \verb|msum| コマンドで日別の数量合計と金額合計を求めます。最後に \verb|maccum| コマンドで年初からの日別累計を求め、結果は \verb|maccumout.csv| ファイルに書き出します。\\

このレッスンでは、レッスン1で作成した \verb|dat.csv|(スーパーマーケットのレシートデータ)を使います。

 \subsection{Step 2: パラメータを指定する}

{\setlength{\parindent}{0cm}

\verb|maccum| コマンドに与えるパラメータは次のようになります。\\

キー項目: 			\verb|k=| \\
説明: 入力ファイル全体を通して累計を求める場合は、k=パラメータは省略できます。ある項目内で累計を求めたいようなときk=パラメータを使います。\\

項目: \verb|f=数量合計:数量累計,金額合計:金額累計| \\
説明: 累計を求めたい項目を指定します。カンマで区切って複数の項目を指定することもできます。コロン(:)を使って、累計の項目名を指定することができます。\\
}

\begin{verbatim}
#!/bin/bash
#=========================================
# MCMD bash script - Lesson 10: Calculation among rows with maccum
#=========================================
# Variables
inPath="tutorial_jp"

# Command 
mcut f=日付,数量,金額 i=${inPath}/dat.csv |
msortf f=日付 |   
msum k=日付 f=数量:数量合計,金額:金額合計 |   
maccum f=数量合計:数量累計,金額合計:金額累計 o=outdat/maccumout.csv
#==========================================
\end{verbatim}

\subsection{ステップ 3: シェルスクリプトを実行する}

\begin{verbatim}
$ ./maccum.sh
#END# kgcut f=日付,数量,金額 i=tutorial_en/dat.csv; IN=24737 OUT=24737; 2013/08/19 16:58:07
#END# kgsortf f=日付; IN=24737 OUT=24737; 2013/08/19 16:58:07
#END# kgsum f=数量:数量合計,金額:金額合計 k=日付; IN=24737 OUT=324; 2013/08/19 16:58:07
#END# kgaccum f=数量合計:数量累計,金額合計:金額累計 o=outdat/maccumout.csv; 2013/08/19 16:58:07
\end{verbatim}

結果は次のようになります。

\begin{verbatim}
日付,数量合計,金額合計,数量累計,金額累計
20010108,21,8680,21,8680
20010110,67,23495,88,32175
20010111,94,38261,182,70436
20010116,65,28333,247,98769
20010119,210,93586,457,192355
20010123,54,19476,511,211831
20010125,43,14516,554,226347
20010128,43,15448,597,241795
20010130,157,60577,754,302372
20010131,14,1806,768,304178
20010201,8,1024,776,305202
20010203,13,2236,789,307438
20010204,132,52936,921,360374
20010205,62,24132,983,384506
20010206,82,29950,1065,414456
20010208,78,24906,1143,439362
20010209,50,21776,1193,461138
20010211,58,23650,1251,484788
20010212,82,35368,1333,520156
20010213,65,24974,1398,545130
20010216,245,89800,1643,634930
...
..
\end{verbatim}

\subsection{ステップ 4: 日付の降順で数量累計と金額累計を求める}

\verb|maccum| コマンドを実行する前に、「日付」の降順でデータをソートしておきます。シェルスクリプトを次のように変更してみましょう。

\begin{verbatim}
mcut f=日付,数量,金額 i=${inPath}/dat.csv |
msortf f=日付%r |   
msum k=日付 f=数量:数量合計,金額:金額合計 |   
maccum f=数量合計:数量累計,金額合計:金額累計 o=outdat/maccumout.csv
\end{verbatim}

シェルスクリプトを保存したら、実行しましょう。結果は次のようになります。

\begin{verbatim}
日付,数量合計,金額合計,数量累計,金額累計
20011230,93,58331,93,58331
20011229,300,151488,393,209819
20011228,292,162534,685,372353
20011227,112,61542,797,433895
20011226,73,34360,870,468255
20011225,202,109219,1072,577474
20011224,305,141597,1377,719071
20011222,121,71352,1498,790423
20011221,156,93305,1654,883728
20011220,153,77793,1807,961521
20011219,168,91842,1975,1053363
20011218,251,125010,2226,1178373
20011217,19,8687,2245,1187060
20011216,154,76426,2399,1263486
20011215,160,74503,2559,1337989
...
..
\end{verbatim}

\subsection{ステップ 5: 商品の「中分類」内で、「小分類」の数量累計と金額累計を求める}

\verb|maccum| コマンドの実行前に、\verb|msortf| コマンドで「中分類」「小分類(降順)」でソートします。次に \verb|msum| コマンドで「小分類」単位で数量と金額の合計を求め、最後に \verb|maccum| コマンドで「中分類」内の累計を求めます。シェルスクリプトを次のように変更してみましょう。

\begin{verbatim}
mcut f=中分類,小分類,数量,金額  i=${inPath}/dat.csv |
msortf f=中分類,小分類%r |   
msum k=中分類,小分類 f=数量,金額 |   
maccum k=中分類 f=数量:数量累計,金額:金額累計 o=outdat/maccumout0.csv
\end{verbatim}

シェルスクリプトを保存したら、実行しましょう。結果は次のようになります。

\begin{verbatim}
中分類,小分類,数量,金額,数量累計,金額累計
11,1197,332,173474,332,173474
11,1121,2430,1122457,2762,1295931
11,1120,960,411380,3722,1707311
11,1119,2791,1262170,6513,2969481
11,1118,1125,435023,7638,3404504
11,1117,938,489566,8576,3894070
11,1116,1144,542858,9720,4436928
11,1115,2194,1106249,11914,5543177
11,1114,212,80084,12126,5623261
11,1113,1150,445836,13276,6069097
11,1112,1450,732630,14726,6801727
11,1111,1768,744427,16494,7546154
11,1110,1478,577858,17972,8124012
11,1108,1157,410367,19129,8534379
11,1107,1946,911273,21075,9445652
11,1106,1100,484361,22175,9930013
11,1105,2847,1611468,25022,11541481
11,1104,2019,1024216,27041,12565697
11,1103,1160,500887,28201,13066584
11,1102,1698,732146,29899,13798730
11,1101,3758,1526511,33657,15325241
12,1297,160,83802,160,83802
12,1203,10,4495,170,88297
...
..
\end{verbatim}

\vspace {5mm}

{\setlength{\parindent}{0cm}

\textbf{ソートの重要性}\\

\fbox{
  \parbox{\textwidth}{
    
msumやmcount、maccumなど、データをあらかじめソートしてから実行する必要のあるコマンドがいくつもあります。データのレコード数が多くなるほどソートにも時間がかかるようになるので、効率的にソートする、無駄なソートはしないなどの工夫をすることで、処理全体にかかる時間を短縮することができます。\\

  }
}
}


\subsection{練習問題}

\verb|maccum| コマンドを使って、次のレポートを作成してみましょう。作成したら、出力ファイルとスクリプトを確認してください。

\begin{table}[htbp]
%\begin{center}
{\small
\begin{tabular}{ l | c || r }
\hline
\textbf{レポートの内容}   & \textbf{スクリプト名} & \textbf{出力ファイル名}  \\
\hline
1. 「メーカー」の昇順で、数量累計と金額累計 & \href{exercise/maccum1.sh}{maccum1.sh} & \href{exercise/outdat/maccumout1.csv}{maccumout1.csv} \\
2. 「メーカー」の降順で、「中分類」ごとの数量累計と金額累計 & \href{exercise/maccum2.sh}{maccum2.sh} & \href{exercise/outdat/maccumout2.csv}{maccumout2.csv} \\


\hline
\end{tabular} 
}
\end{table} 


%\end{document}
