
%----------------------------------------------------------------------------------------
% MCMD Basic Techniques 	
% Lessons on Basic Commands  
%----------------------------------------------------------------------------------------

%\documentclass[a4paper]{jarticle}
%\begin{document}

\section{レッスン 4: 集計(その2)}

\verb|mcount| コマンドを使うと、レコード数(行数)を数えることができます。集計の単位となる項目の指定もできますので、日別、分類別、商品別のレコード数を数えたりするのによく使われます。

\subsection{ステップ 1: スクリプトを作成する}

テキストエディタを使って、スクリプトファイル \verb|mcount.sh| を作成しましょう。作成したら、ファイルに実行権限を与えるのを忘れないようにしてください。\\

このレッスンでは、トランザクションデータ(スーパーのレシートデータ)から日別のレコード件数を数えます。\\

そのためには、まず「日付」項目を \verb|mcut| コマンドで取り出し、ソートした上で \verb|mcount| コマンドに渡します。\verb|mcount| コマンドは日付単位でレコード数を数え、その結果を \verb|mcountout.csv| ファイルに書き出します。

 \subsection{ステップ 2: パラメータを指定する}

{\setlength{\parindent}{0cm}

\verb|mcount| コマンドに与えるパラメータは次のようになります。\\

キー項目: 		\verb|k=日付| \\
説明: 集計単位となる列を指定します。このレッスンでは日別に集計を行うので、「日付」列を指定します。\\

結果を格納する項目: \verb|a=行数| \\
説明: 結果を格納するための新しい項目名(列名)は、a=で指定します。この例では、日別にいくつの商品がレジを通ったかがカウントされ「行数」列に格納されます。\\

このレッスンでは、レッスン1で作成した \verb|dat.csv| (スーパーマーケットのレシートデータ)を使います。
}

\begin{verbatim}
#!/bin/bash
#=========================================
# MCMD bash script - Lesson 4: Aggregate functions II
#=========================================

# Variables
inPath="tutorial_jp"

# Command 
mcut f=日付 i=${inPath}/dat.csv |
msortf f=日付 |   
mcount k=日付 a=行数 o=outdat/mcountout.csv

#==========================================
\end{verbatim}

\subsection{ステップ 3: シェルスクリプトを実行する}

\begin{verbatim}
$ ./mcount.sh
#END# kgcut f=日付 i=tutorial_jp/dat.csv; IN=24737 OUT=24737; 2013/08/23 11:51:14
#END# kgsortf f=日付; IN=24737 OUT=24737; 2013/08/23 11:51:14
#END# kgcount a=行数 k=日付 o=outdat/mcountout.csv; IN=24737 OUT=324; 2013/08/23 11:51:14
\end{verbatim}

\noindent
結果は次のようになります。

\begin{verbatim}
日付,行数
20010108,12
20010110,28
20010111,48
20010116,28
20010119,84
20010123,24
20010125,20
20010128,20
20010130,96
20010131,4
20010201,4
20010203,4
20010204,71
20010205,32
20010206,36
20010208,52
20010209,20
20010211,36
20010212,36
20010213,36
20010216,122
20010217,36
...
..
\end{verbatim}

\subsection{ステップ 4: パラメータを変更する}

\noindent
集計単位を指定するk=パラメータは省略することができます。その場合、入力ファイル全体のレコード数がカウントされます。

\begin{verbatim}
#!/bin/bash
#=========================================
# MCMD bash script - Lesson 3: Aggregate functions II
#=========================================

# Variables
inPath="tutorial_jp"

# Command 
mcut f=日付 i=${inPath}/dat.csv |
mcount a=行数 o=outdat/mcountout.csv

#==========================================
\end{verbatim}

\noindent
dat.csvファイルには24,737レコードあることがわかります。

\begin{verbatim}
日付,行数
20011230,24737
\end{verbatim}

(日付に20011230と表示されていますが、k=パラメータを指定していない場合、この値に特に意味はありません)

\newpage 

\subsection{練習問題 }

\verb|mcount| コマンドを使って、次のレポートを作成してみましょう。作成したら、出力ファイルとスクリプトを確認してください。

\begin{table}[htbp]
%\begin{center}
{\small
\begin{tabular}{ l | c || r }
\hline
\textbf{レポートの内容}   & \textbf{スクリプト名} & \textbf{出力ファイル名}  \\
\hline
1. メーカー別のレコード数 & \href{exercise/mcount1.sh}{mcount1.sh} & \href{exercise/outdat/mcountout1.csv}{mcountout1.csv} \\
2. ブランド別のレコード数 & \href{exercise/mcount2.sh}{mcount2.sh} & \href{exercise/outdat/mcountout2.csv}{mcountout2.csv} \\
3. 大分類別のレコード数 & \href{exercise/mcount3.sh}{mcount3.sh} & \href{exercise/outdat/mcountout3.csv}{mcountout3.csv} \\
4. 中分類別のレコード数 & \href{exercise/mcount4.sh}{mcount4.sh} & \href{exercise/outdat/mcountout4.csv}{mcountout4.csv} \\
5. 小分類別のレコード数 & \href{exercise/mcount5.sh}{mcount5.sh} & \href{exercise/outdat/mcountout5.csv}{mcountout5.csv} \\
6. メーカー別・中分類別のレコード数 & \href{exercise/mcount6.sh}{mcount6.sh} & \href{exercise/outdat/mcountout6.csv}{mcountout6.csv} \\

\hline
\end{tabular} 
}
\end{table} 


%\end{document}
