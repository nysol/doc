
%----------------------------------------------------------------------------------------
% MCMD Basic Techniques 	
% Lessons on Basic Commands  
%----------------------------------------------------------------------------------------

%\documentclass[a4paper]{jarticle}
%\begin{document}

\section{レッスン 13: ファイルを連結する}

レッスン12ではファイルの分割を学びましたが、それとは逆に、日付や地域で分割されたファイルを1つにまとめたいこともあります。\verb|mcat| コマンドを使うと、複数のファイルを1つのファイルに連結することができます。

\subsection{パラメータを指定する}

テキストエディタを使って、スクリプトファイル \verb|mcat.sh| を作成しましょう。作成したら、ファイルに実行権限を与えるのを忘れないようにしてください。\\

このレッスンでは、分割されたトランザクションデータ(スーパーのレシートデータ)を1つに連結します。レッスン12の後半で作成した、年月単位に分割された12のファイルをmcatコマンドで連結し、\verb|mcatout.csv| ファイルとして書き出しましょう。


{\setlength{\parindent}{0cm}

\verb|mcat|コマンドに与えるパラメータは次のようになります。\\

入力ファイル: \verb|i=${inPath}/dat2001*.csv| \\
説明: 連結したいファイル群をi=パラメータで指定します。複数のファイルをカンマで区切って指定することもできますが、ファイル数が多くなるとその方法では大変です。そのため、一般的にはワイルドカード(?および*)を用いてファイル群を指定します。
}

\begin{verbatim}
#!/bin/bash
#=========================================
# MCMD bash script - Lesson 13: Merge files
#=========================================
# Variables
inPath="tutorial_jp"

# Command 
mcat i=${inPath}/dat2001*.csv o=outdat/mcatout.csv
#==========================================
\end{verbatim}

\verb|tutorial_jp| ディレクトリ内にある、ファイル名が\verb|dat2001|で始まるすべてのファイルが連結され、\verb|outdat|ディレクトリ内に\verb|mcatout.csv|ファイルが作成されます。\\

では次に、2001年のすべてのレシートデータについて、数量合計と金額合計を求めてみましょう。ここでも、年月単位に分割された12のファイルを使いますが、先ほどの例と異なるのは、mcatコマンドでファイルを連結する段階で、「日付」「数量」「金額」項目のみを取り出しておく点です。そのあとはこれまでどおり、日付でソートしたあとmsumコマンドで日別に数量と金額を合計します。結果は\verb|mcatout.csv|に書き出しましょう。\\

\verb|mcat| コマンドで項目を指定するには、次のパラメータを与えます。\\

項目: \verb|f=日付,数量,金額| \\
説明: \verb|mcat|コマンドが複数のファイルを連結する際に、取り出す項目名はf=パラメータで指定します。カンマで区切ることで複数の項目を指定できます。

\subsection{スクリプトの変更と実行}
 
\begin{verbatim}
#!/bin/bash
#=========================================
# MCMD bash script - Lesson 13: Merge files
#=========================================
# Variables
inPath="tutorial_jp"

# Command 
mcat f=日付,数量,金額 i=${inPath}/dat2001*.csv | 
msortf f=日付%n |   
msum k=日付 f=数量:数量合計,金額:金額合計 o=outdat/mcatout.csv
#==========================================
\end{verbatim}
 
シェルスクリプトを保存したら、実行しましょう。実行すると、次のようなメッセージが表示されます。

\begin{verbatim}
$ ./mcat.sh 
#END# kgcat f=日付,数量,金額 i=tutorial_jp/dat2001*.csv; IN=24737 OUT=24737; 2013/08/24 10:48:52
#END# kgsortf f=日付%n; IN=24737 OUT=24737; 2013/08/24 10:48:53
#END# kgsum f=数量:数量合計,金額:金額合計 k=日付 o=outdat/mcatout.csv; IN=24737 OUT=324; 2013/08/24
10:48:53
\end{verbatim}

また、結果は次のようになります。

 \begin{verbatim}
日付,数量合計,金額合計
20010108,21,8680
20010110,67,23495
20010111,94,38261
20010116,65,28333
20010119,210,93586
20010123,54,19476
20010125,43,14516
20010128,43,15448
20010130,157,60577
20010131,14,1806
20010201,8,1024
20010203,13,2236
20010204,132,52936
20010205,62,24132
20010206,82,29950
20010208,78,24906
20010209,50,21776
20010211,58,23650
20010212,82,35368
20010213,65,24974
20010216,245,89800
20010217,64,21223
20010220,30,9774
20010222,78,32783
20010223,118,45828
20010224,182,82685
20010226,107,40550
20010227,252,96736
...
..
\end{verbatim}


\subsection{ディレクトリの効率的な使い方}

ディレクトリ構造をうまく使うと、ファイルの管理が効率的になります。

このレッスンでは年月で分割されたファイルを連結して使いましたが、たとえば店舗Aの、2001年1月のファイルは\verb|a/dat20011.csv|(\verb|a|ディレクトリ内の\verb|dat20011.csv|というファイル)のようにしておくと、必要なデータを効率的に探し出すことができます。

ディレクトリ名やファイル名の指定にはワイルドカードを使うことができます。たとえば10月、11月、12月のファイルを結合するには\verb|20011*/dat.csv|のように指定します。


\subsection{練習問題}

\verb|mcat|コマンドを使って、次のレポートを作成してみましょう。作成したら、出力ファイルとスクリプトを確認してください。

\begin{table}[htbp]
%\begin{center}
{\small
\begin{tabular}{ l | c || r }
\hline
\textbf{レポートの内容}   & \textbf{スクリプト名} & \textbf{出力ファイル名}  \\
\hline
4月・5月・7月について、日別の数量合計と金額合計 & \href{exercise/mcat1.sh}{mcat1.sh} & \href{exercise/outdat/mcatout1.csv}{mcatout1.csv} \\
4月および10月〜12月について、日別の数量合計と金額合計 & \href{exercise/mcat2.sh}{mcat2.sh} & \href{exercise/outdat/mcatout2.csv}{mcatout2.csv} \\

\hline
\end{tabular} 
}
\end{table} 


%\end{document}
