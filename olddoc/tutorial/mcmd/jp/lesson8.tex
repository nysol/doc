
%----------------------------------------------------------------------------------------
% MCMD Basic Techniques 	
% Lessons on Basic Commands  
%----------------------------------------------------------------------------------------

%\documentclass[a4paper]{jarticle}
%\begin{document}

\section{レッスン 8: 重複するレコードを削除する}

\verb|muniq| コマンドを使うと、指定した項目の重複を取り除くことができます。データからマスタ表を作るときによく使われるコマンドです。

\subsection{ステップ 1: スクリプトを作成する}

テキストエディタを使って、スクリプトファイル \verb|muniq.sh| を作成しましょう。作成したら、ファイルに実行権限を与えるのを忘れないようにしてください。\\

このレッスンでは、日付の一覧表を作成します。\\

そのために、まず「日付」列を \verb|mcut| コマンドで取り出し、パイプを介して \verb|muniq| コマンドに渡します。\verb|muniq| コマンドは重複する日付を1レコードにまとめ、結果を \verb|muniqout.csv| に書き出します。\\

このレッスンでは、レッスン1で作成したdat.csv(スーパーマーケットのレシートデータ)を使います。

\subsection{ステップ 2: パラメータを指定する}

{\setlength{\parindent}{0cm}

\verb|muniq| コマンドに与えるパラメータは次のようになります。\\

キー項目: \verb|k=日付| \\
説明: 重複を取り除きたい項目はk=パラメータで指定します。複数の項目を指定することもできます。
}

\begin{verbatim}
#!/bin/bash
#=========================================
# MCMD bash script - Lesson 8: Remove Duplicate Rows
#=========================================
# Variables
inPath="tutorial_jp"

# Command 
mcut f=日付 i=${inPath}/dat.csv |
msortf f=日付 |   
muniq k=日付 o=outdat/muniqout.csv
#==========================================
\end{verbatim}

\subsection*{ステップ 3: シェルスクリプトを実行する}

\begin{verbatim}
$ ./muniq.sh 
#END# kgcut f=日付 i=tutorial_en/dat.csv; IN=24737 OUT=24737; 2013/08/19 13:39:12
#END# kgsortf f=日付; IN=24737 OUT=24737; 2013/08/19 13:39:12
#END# kguniq k=日付 o=outdat/muniqout.csv; IN=24737 OUT=324; 2013/08/19 13:39:12
\end{verbatim}

\noindent
結果は次のようになります。

\begin{verbatim}
日付
20010108
20010110
20010111
20010116
20010119
20010123
20010125
20010128
20010130
20010131
20010201
20010203
20010204
20010205
20010206
20010208
20010209
20010211
...
..
\end{verbatim}



\subsection{練習問題}

\verb|muniq| コマンドを使って、次のレポートを作成してみましょう。作成したら、出力ファイルとスクリプトを確認してください。

\begin{table}[htbp]
%\begin{center}
{\small
\begin{tabular}{ l | c || r }
\hline
\textbf{レポートの内容}   & \textbf{スクリプト名} & \textbf{出力ファイル名}  \\
\hline
1. 「ブランド」項目のマスタファイル & \href{exercise/muniq1.sh}{muniq1.sh} & \href{exercise/outdat/muniqout1.csv}{muniqout1.csv} \\
2. 「メーカー」項目のマスタファイル & \href{exercise/muniq2.sh}{muniq2.sh} & \href{exercise/outdat/muniqout2.csv}{muniqout2.csv} \\


\hline
\end{tabular} 
}
\end{table} 


%\end{document}
