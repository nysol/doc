
%----------------------------------------------------------------------------------------
% MCMD Basic Techniques 	
% Lessons on Basic Commands  
%----------------------------------------------------------------------------------------

%\documentclass[a4paper]{jarticle}
%\begin{document}

\section{レッスン 9: mshareコマンドでシェアを計算する}

\verb|mshare| コマンドを使うと、たとえば商品の売上シェア(売上全体に対する構成比)を簡単に求めることができます。

\subsection{ステップ 1: スクリプトを作成する}

テキストエディタを使って、スクリプトファイル \verb|mshare.sh| を作成しましょう。作成したら、ファイルに実行権限を与えるのを忘れないようにしてください。\\

このレッスンでは、日別の数量シェアと金額(売上)シェアを求めます。\\

そのために、まず「日付」「数量」「金額」列をmcutコマンドで取り出し、パイプを介してmsumコマンドに渡します。Msumコマンドで日別に数量と金額を合計し、続くmshareコマンドで日別の数量シェアと金額シェアを求めます。結果は \verb|mshareout.csv| ファイルに書き出します。\\

このレッスンでは、レッスン1で作成した \verb|dat.csv|(スーパーマーケットのレシートデータ)を使います。

\subsection{ステップ 2: パラメータを指定する}

{\setlength{\parindent}{0cm}

\verb|mshare| コマンドに与えるパラメータは次のようになります。\\

項目: 		\verb|k=| \\
説明: 入力ファイル全体に対するシェアを求める場合、k=パラメータは省略できます。商品分類内でのシェアを求めたいようなときk=パラメータを使います。\\

項目: \verb|f=数量:数量シェア,金額:金額シェア| \\
説明: シェアを求めたい項目を指定します。カンマで区切って複数の項目を指定することもできます。コロン(:)を使って、シェアの項目名を指定することができます。
}

\begin{verbatim}
#!/bin/bash
#=========================================
# MCMD bash script - Lesson 9: Calculation among rows with mshare
#=========================================
# Variables
inPath="tutorial_jp"

# Command 
mcut f=日付,数量,金額 i=${inPath}/dat.csv |
msortf f=日付 |   
msum k=日付 f=数量,金額 |   
mshare f=数量:数量シェア,金額:金額シェア o=outdat/mshareout.csv
#==========================================
\end{verbatim}

\subsection{ステップ 3: シェルスクリプトを実行する}

\begin{verbatim}
$ ./mshare.sh 
#END# kgcut f=日付,数量,金額 i=tutorial_jp/dat.csv; IN=24737 OUT=24737; 2013/08/19 14:42:10
#END# kgsortf f=日付; IN=24737 OUT=24737; 2013/08/19 14:42:10
#END# kgsum f=数量,金額 k=日付; IN=24737 OUT=324; 2013/08/19 14:42:10
#END# kgshare f=数量:数量シェア,金額:金額シェア o=outdat/mshareout.csv; IN=324 OUT=324; 2013/08/19
\end{verbatim}

\noindent
結果は次のようになります。

\begin{verbatim}
日付,数量,金額,数量シェア,金額シェア
20010108,21,8680,0.0004130687071,0.0003729484345
20010110,67,23495,0.001317885875,0.001009495791
20010111,94,38261,0.001848974213,0.001643937794
20010116,65,28333,0.001278545998,0.001217367281
20010119,210,93586,0.004130687071,0.0040210544
20010123,54,19476,0.001062176675,0.0008368137915
20010125,43,14516,0.0008458073526,0.0006237004004
20010128,43,15448,0.0008458073526,0.000663745094
20010130,157,60577,0.003088180334,0.002602776189
20010131,14,1806,0.0002753791381,7.759733557e-05
20010201,8,1024,0.0001573595075,4.399760333e-05...
..
\end{verbatim}

\subsection{ステップ 4: シェルスクリプトを修正する}

次は、商品の分類単位で数量シェア、金額シェアを求めてみましょう。商品の小分類ごとに、中分類内でのシェアを求めてみます。\\

そのためには、まず「中分類」「小分類」「数量」「金額」の4項目を、\verb|mcut| コマンドを使って取り出します。\verb|msortf| コマンドで「中分類」「小分類」項目のソートをしたあとに、\verb|msum| コマンドで数量・金額を合計し、最後に\verb|mshare| コマンドで小分類別のシェア(中分類全体に対する構成比)を求めます。結果は \verb|mshareout0.csv| ファイルに書き出します。\\

シェルスクリプトを次のように変更してみましょう。\\

\begin{verbatim}
mcut f=中分類,小分類,数量,金額 i=${inPath}/dat.csv |
msortf f=中分類,小分類 |
msum k=中分類,小分類 f=数量,金額 |
mshare k=中分類 f=数量:数量シェア,金額:金額シェア o=outdat/mshareout0.csv
\end{verbatim}

シェルスクリプトを保存したら、実行しましょう。結果は次のようになります。

\begin{verbatim}
中分類,小分類,数量,金額,数量シェア,金額シェア
11,1101,3758,1526511,0.111655822,0.09960763423
11,1102,1698,732146,0.05045012925,0.04777386535
11,1103,1160,500887,0.03446534153,0.03268379271
11,1104,2019,1024216,0.05998752117,0.06683196695
11,1105,2847,1611468,0.08458864426,0.1051512338
11,1106,1100,484361,0.03268265145,0.03160544098
11,1107,1946,911273,0.05781858157,0.05946222966
11,1108,1157,410367,0.03437620703,0.02677719717
11,1110,1478,577858,0.04391359895,0.03770629121
11,1111,1768,744427,0.05252993434,0.04857522306
11,1112,1450,732630,0.04308167692,0.04780544724
11,1113,1150,445836,0.03416822652,0.02909161429
11,1114,212,80084,0.00629883828,0.005225627447
11,1115,2194,1106249,0.0651870339,0.07218477021
11,1116,1144,542858,0.03398995751,0.03542247721
11,1117,938,489566,0.02786938824,0.03194507675
\end{verbatim}

\vspace {5mm}

{\setlength{\parindent}{0cm}

\textbf{中間結果を確認する}\\

\fbox{
  \parbox{\textwidth}{

シェルスクリプトによる実行結果が思っていたものと異なるとき、その処理過程を確認したいと思うことがあります。しかしパイプを使って複数のコマンドをつなげていると、中間結果はファイルとして残らないため、確認することができません。そのようなときは、teeコマンドを使いましょう。パイプの途中にteeコマンドを挟み込むと、その時点でパイプを流れるデータ(中間結果)をファイルに出力することができます。以下の例では、mcutコマンドからmsortfコマンドに渡されるデータをcheckdat.csvというファイルに書き出してくれます。\\

  }
}
}
\begin{verbatim}
mcut f=日付,数量,金額 i=${inPath}/dat.csv | tee checkdat.csv |
msortf f=日付 |
msum k=日付
...
..         
\end{verbatim}

\subsection{練習問題}

\verb|mshare| コマンドを使って、次のレポートを作成してみましょう。作成したら、出力ファイルとスクリプトを確認してください。

\begin{table}[htbp]
%\begin{center}
{\small
\begin{tabular}{ l | c || r }
\hline
\textbf{レポートの内容}   & \textbf{スクリプト名} & \textbf{出力ファイル名}  \\
\hline
1. 「メーカー」別の数量シェア、金額シェア & \href{exercise/mshare1.sh}{mshare1.sh} & \href{exercise/outdat/mshareout1.csv}{mshareout1.csv} \\
2. 「中分類」の中での「メーカー」別の数量シェア、金額シェア & \href{exercise/mshare2.sh}{mshare2.sh} & \href{exercise/outdat/mshareout2.csv}{mshareout2.csv} \\


\hline
\end{tabular} 
}
\end{table} 


%\end{document}
