
%----------------------------------------------------------------------------------------
% MCMD Basic Techniques 	
% Lessons on Basic Commands  
%----------------------------------------------------------------------------------------

%\documentclass[a4paper]{jarticle}
%\begin{document}

\section{レッスン 3: 集計(その1)}

\verb|msum、mavg、mstats| コマンドを使うと、数値の合計や平均、最小値や最大値を求めることができます。

\subsection{ステップ 1: スクリプトを作成する}

テキストエディタを使って、スクリプトファイル \verb|msum.sh| を作成しましょう。作成したら、ファイルに実行権限を与えるのを忘れないようにしてください。\\

このレッスンでは、トランザクションデータ(スーパーのレシートデータ)から日別の数量合計・金額合計を求めます。\\

そのためには、まず日付、数量、金額の3項目を\verb|mcut| コマンドを使って取り出し、その結果はパイプ(|)を介して \verb|msortf| コマンドに渡します。パイプを使うと、コマンドの出力を次のコマンドに引き渡すことができます。\\
集計コマンドを使うには、あらかじめ集計のキー項目(今回の例では日付)でレコードをソートしておく必要があります。最後に \verb|msum| コマンドが数量と金額を日別で合計し、その結果を \verb|msum.csv| ファイルに書き出します。

\subsection{ステップ 2: パラメータを指定する}

{\setlength{\parindent}{0cm}

\verb|msum| コマンドに与えるパラメータは次のようになります。

キー項目: 		\verb|k=日付| \\
説明: 集計単位となる列を指定します(複数の列を指定することもできます)。このレッスンでは日別に集計を行うので、「日付」列を指定します。\\

項目: 	\verb|f=数量,金額| \\
説明: どの列の値を合計するのかを指定します。ここでは数量合計と金額合計を求めたいので、「数量」「金額」の2列を指定します。\\

このレッスンでは、レッスン1で作成した \verb|dat.csv| (スーパーマーケットのレシートデータ)を使います。
}

\begin{verbatim}
#!/bin/bash
#========================================
# MCMD bash script - Lesson 3: Aggregate functions I
#========================================
# Variables
inPath="tutorial_jp"

# Command 
mcut f=日付,数量,金額 i=${inPath}/dat.csv |
msortf k=日付 |
msum k=日付 f=数量,金額 o=outdat/msumout.csv

#========================================
\end{verbatim}

\subsection{ステップ 3: シェルスクリプトを実行する}

\begin{verbatim}
$ ./msum.sh
#END# kgcut f=日付,数量,金額 i=tutorial_jp/dat.csv; IN=24737 OUT=24737; 2013/08/13 12:20:37
#END# kgsortf f=日付; IN=24737 OUT=24737; 2013/08/13 12:20:37
#END# kgsum f=数量,金額 k=日付 o=outdat/msumout.csv; IN=24737 OUT=324; 2013/08/13 12:20:37
\end{verbatim}

\noindent
結果は次の通りです。

\begin{verbatim}
日付,数量,金額
20010108,21,8680
20010110,67,23495
20010111,94,38261
20010116,65,28333
20010119,210,93586
20010123,54,19476
20010125,43,14516
20010128,43,15448
20010130,157,60577
20010131,14,1806
20010201,8,1024
20010203,13,2236
20010204,132,52936
20010205,62,24132
20010206,82,29950
20010208,78,24906
20010209,50,21776
20010211,58,23650
20010212,82,35368
20010213,65,24974
20010216,245,89800
20010217,64,21223
20010220,30,9774
20010222,78,32783
20010223,118,45828
20010224,182,82685
20010226,107,40550
20010227,252,96736
20010228,137,58384
20010301,108,40948
..
.
\end{verbatim}

\subsection{ステップ 4: 項目の名前を変更する}

\verb|msum、mavg、mstats| コマンドは、集計した列の名前を変更することができます。たとえば、数量を合計した列は「数量合計」、金額を合計した列は「金額合計」という名前にすることができます。
具体的には、これらのコマンドでf=パラメータを指定する際、コロン(:)で区切ってf=項目名:新しい項目名のように指定します。:の前後に空白を開けてはいけません。\\

\begin{verbatim}
#!/bin/bash
#========================================
# MCMD bash script - Lesson 3: Aggregate Functions
#========================================
# Variables
inPath="tutorial_jp"

# Command 
mcut f=日付,数量,金額 i=${inPath}/dat.csv |
msortf k=日付 |
msum k=日付 f=数量:数量合計,金額:金額合計 o=outdat/msumout.csv

#========================================
\end{verbatim}

出力結果の見出し行を確認しておきましょう。

\begin{verbatim}
日付,数量合計,金額合計
20010108,21,8680
20010110,67,23495
20010111,94,38261
20010116,65,28333
20010119,210,93586
20010123,54,19476
20010125,43,14516
20010128,43,15448
20010130,157,60577
...
..
\end{verbatim}

\subsection{数値データの整形} 
    
データに含まれる数値が、たとえば\$99,111,000のように貨幣記号やカンマを含んでいる場合があります。\verb|msum| コマンドなどを使って数値の集計を行うには、事前にこれら貨幣記号やカンマを取り除いておく必要があります。\verb|msed| コマンドは指定した文字を簡単に取り除くことができるので、\$99,111,000から99111000への変換が可能です。\\

\verb|msed| コマンドでは、まず変換したい項目をf=、どのような文字列を探したいかをc=、そしてどのような文字に変換したいかをv=で指定します(今回は特定の文字を取り除きたいので、v=には何も指定しません)。-gオプションを使うと、正規表現にマッチした文字をすべて変換します(-gを付けないと、最初にマッチした文字列のみ変換します)。\\

以下の例は、金額列の変換をしたのちその合計を求めています。\\

 \begin{verbatim}
# データセット (金額は US$)
 日付,商品,金額
 20010101,laptop,"$4,000"
 20010110,keyboard,"$50"
 20010310,lcd,"$10,000"
 
# コマンドの例
msed f=金額 v= c=[$,] -g  i=dat.csv  |
msum  f=金額:金額合計   |
mcut f=金額合計 o=outdat.csv

# 出力結果
金額合計
14050
 \end{verbatim}

{\setlength{\parindent}{0cm}

\subsection{平均を求める}

\verb|mavg| コマンドを使うと、数値の平均を求めることができます。パラメータの与え方は \verb|msum| コマンドと同じです。以下の例は、日別の数量平均を求めています。\\

\begin{verbatim}
mcut f=日付,数量,金額 i=${inPath}/dat.csv |
msort k=日付 |
mavg k=日付 f=数量:数量平均o=outdat/mavgout.csv
\end{verbatim}


\subsection{mstatsコマンドの使い方}

\verb|mstats| コマンドを使うと、合計や平均に加えて、レコード件数、最小値、最大値、分散、標準偏差を求めることができます。mstatsコマンドを用いて数量と金額の日別合計を求めるスクリプトは、次のようになります。

\begin{verbatim}
mcut f=日付,数量,金額 i=${inPath}/dat.csv |
msort k=日付                     |
mstats k=日付 f=数量:数量合計,金額:金額合計 c=sum o=outdat/mstatsout.csv
\end{verbatim}

\verb|mstats| コマンドのc=パラメータで指定できる統計量は1つのみです。次の例は、数量と金額の日別標準偏差を求めています。
\begin{verbatim}
mcut f=日付,数量,金額 i=${inPath}/dat.csv |
msort k=日付 |
mstats k=日付 f=数量:数量標準偏差,金額:金額標準偏差 c=sd o=outdat/mstatsout.csv
\end{verbatim}


\subsection{練習問題}

\verb|msum、mavg、mstats| コマンドを使って、次の集計をしてみましょう。出力ファイルとスクリプトを確認してください。

\begin{table}[htbp]
%\begin{center}
{\small
\begin{tabular}{ l | c || r }
\hline
\textbf{レポートの内容}   & \textbf{スクリプト名} & \textbf{出力ファイル名}  \\
\hline
1. 日別の数量平均と金額平均 & \href{exercise/agg1.sh}{agg1.sh} & \href{exercise/outdat/aggout1.csv}{aggout1.csv} \\
2. 日別の数量最大値と金額最大値 & \href{exercise/agg2.sh}{agg2.sh} & \href{exercise/outdat/aggout2.csv}{aggout2.csv} \\
3. 日別の数量最小値と金額最小値 & \href{exercise/agg3.sh}{agg3.sh} & \href{exercise/outdat/aggout3.csv}{aggout3.csv} \\
4. 中分類で集計した数量合計と金額合計 & \href{exercise/agg4.sh}{agg4.sh} & \href{exercise/outdat/aggout4.csv}{aggout4.csv} \\
5. 中分類で集計した数量平均と金額平均 & \href{exercise/agg5.sh}{agg5.sh} & \href{exercise/outdat/aggout5.csv}{aggout5.csv} \\
6. メーカーと小分類で集計した数量合計と金額合計
 & \href{exercise/agg6.sh}{agg6.sh} & \href{exercise/outdat/aggout6.csv}{aggout6.csv} \\
7. メーカーと小分類で集計した数量平均と金額平均
 & \href{exercise/agg7.sh}{agg7.sh} & \href{exercise/outdat/aggout7.csv}{aggout7.csv} \\
\hline
\end{tabular} 
}
\end{table} 


%\end{document}
