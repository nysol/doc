
%----------------------------------------------------------------------------------------
% MCMD Basic Techniques 	
% Lessons on Basic Commands  
%----------------------------------------------------------------------------------------

%\documentclass[a4paper]{jarticle}
%\begin{document}

\section{レッスン 11: mcalコマンドで項目間の計算をおこなう}

\verb|mcal| は大変強力なコマンドで、Excelのようにさまざまな計算を行うことができます。このレッスンでは、四則演算をはじめとする基本的な演算、日付に関する演算、数値の四捨五入など、\verb|mcal| コマンドの持つ基本的な機能のいくつかを学びます。

\subsection{基本的な演算}

テキストエディタを使って、スクリプトファイル \verb|mcal.sh| を作成しましょう。作成したら、ファイルに実行権限を与えるのを忘れないようにしてください。\\

ここでは、レシートデータから商品の単価一覧を作ります。\\

そのために、「数量」「金額」の2項目を \verb|mcut| コマンドで取り出し、\verb|mcal| コマンドで「金額」÷「数量」の割り算を行い「単価」を求めます。「単価」項目のみを再度 \verb|mcut| コマンドで取り出したのち、\verb|msortf| コマンドで単価の昇順にソートし、\verb|muniq| コマンドで重複を取り除きます。結果は \verb|mcalout.csv| ファイルに書き出します。\\

このレッスンでは、レッスン1で作成した \verb|dat.csv|(スーパーマーケットのレシートデータ)を使います。

\subsection{パラメータを指定する}

{\setlength{\parindent}{0cm}

\verb|mcal| コマンドに与えるパラメータは次のようになります。\\

計算式: \verb|c='${金額}/${数量}'| \\
説明: 計算式はc=パラメータで指定します。項目を数値として扱いたいので、項目名は\$\{\}で囲むようにします。c=オプション全体を引用符「'」で囲むのも忘れないようにしてください。\verb|mcal| コマンドで使える演算子はあとで紹介します。\\

項目: \verb|a=単価| \\
説明: 計算結果を格納する項目名をa=パラメータで指定します。\\
}

\begin{verbatim}
#!/bin/bash
#=========================================
# MCMD bash script - Lesson 11: Create calculated fields
#=========================================
# Variables
inPath="tutorial_jp"

# Command 
mcut f=数量,金額 i=${inPath}/dat.csv | 
mcal c='${金額}/${数量}' a=単価 |   
mcut f=単価 |   
msortf f=単価%n |   
muniq k=単価 o=outdat/mcalout.csv
#==========================================
\end{verbatim}

\subsection{シェルスクリプトを実行する}
 
シェルスクリプトを保存したら、実行しましょう。実行すると、次のようなメッセージが表示されます。\\

\begin{verbatim}
$ ./mcal.sh 
#END# kgcut f=数量,金額 i=tutorial_jp/dat.csv; IN=24737 OUT=24737; 2013/08/20 17:10:24
#END# kgcal a=単価 c=${金額}/${数量}; IN=24737 OUT=24737; 2013/08/20 17:10:24
#END# kgcut f=単価; IN=24737 OUT=24737; 2013/08/20 17:10:24
#END# kgsortf f=単価%n; IN=24737 OUT=24737; 2013/08/20 17:10:24
#END# kguniq k=単価 o=outdat/mcalout.csv; IN=24737 OUT=839; 2013/08/20 17:10:24
\end{verbatim}

mcalout.csvファイルへの出力は次のようになります。

\begin{verbatim}
単価
91
101
104
106
112
114
116
119
121
122
123
124
126
128
129
130
...
..
\end{verbatim}


\begin{table}[htbp]
%\begin{center}

\begin{tabular}{ l | c | l }
\hline
\textbf{種類}   & \textbf{演算子} & \textbf{説明}  \\
\hline \hline
算術演算子 & + & 加算 \\ \hline
& - & 減算 \\ \hline
& * & 乗算 \\ \hline
& / & 除算 \\ \hline
& \% & 剰余\\ \hline\hline

比較演算子 & == & 左右の値が等しければtrue \\ \hline
& \textless     & 左の値が右の値より小さければtrue(数値/文字列) \\ \hline
& \textless=    & 左の値が右の値より小さいか等しければtrue(数値/文字列) \\ \hline
& \textgreater  & 左の値が右の値より大きければtrue(数値/文字列) \\ \hline
& \textgreater= & 左の値が右の値より大きいか等しければtrue(数値/文字列)\\ \hline
& !=            & 左右の値が等しくなければtrue(数値/文字列) \\ \hline\hline

論理演算子 & \textbar\textbar & OR演算子; 左右の式のどちらかがtrueならばtrue \\ \hline
& \&\& & AND演算子; 左右の式のどちらもがtrueならばtrue \\ \hline
\hline
\end{tabular} 

\end{table} 

算術演算子は左右の値で計算を行い、その結果を返します。比較演算子は左右の値で比較を行い、指定した関係が成立すれば(trueのとき)1、成立しなければ0を返します。論理演算子は左右の式を評価し、1または0を返します。

\subsection{日付に関する演算}
\verb|mcal| コマンドは、日付や時刻を扱う関数をいくつも備えています。このチュートリアルでは、日付を扱う代表的な関数として \verb|diffday| 関数および \verb|today| 関数を紹介します。\\

例1: スクリプト \verb|mcalday.sh| を作成しましょう。このスーパーが2001年1月2日にオープンしたと仮定して、そのレシートデータがオープン何日後のものかを計算します(これにより、顧客の来店頻度を求めるようなこともできますね)。レシートデータから「顧客」「日付」の2項目を取り出し、2001年1月2日と「日付」との差を \verb|mcal| コマンドで計算します。\verb|mcal| コマンドで使える \verb|diffday| 関数には、差を求めたい2つの日付をパラメータとして指定します。結果は \verb|mcaldayout.csv| ファイルに書き出しましょう。\\

スクリプトは次のようになります。スクリプトを実行し、結果を確かめてみましょう。

\begin{verbatim}
#!/bin/bash
#=========================================
# MCMD bash script - Lesson 11: Calculated fields
#=========================================
# Variables
inPath="tutorial_jp"

# Command 
mcut f=顧客,日付 i=${inPath}/dat.csv |
msortf f=顧客,日付 |   
mcal c='diffday($d{日付},0d20010102)' a=日数 o=outdat/mcaldayout.csv
#==========================================
\end{verbatim}

結果は次の通りです。

\begin{verbatim}
顧客,日付,日数
00000A,20011209,341
00000A,20011209,341
00000A,20011209,341
00000A,20011209,341
00000A,20011217,349
00000A,20011217,349
00000A,20011217,349
00000B,20011209,341
00000B,20011209,341
00000B,20011209,341
00000B,20011209,341
00000B,20011217,349
00000B,20011217,349
00000B,20011217,349
00000C,20011209,341
00000C,20011209,341
00000C,20011209,341
00000C,20011209,341
00000C,20011217,349
00000C,20011217,349
00000C,20011217,349
00000D,20011209,341
00000D,20011209,341
00000D,20011209,341
00000D,20011209,341
00000D,20011217,349
00000D,20011217,349
00000D,20011217,349
00001A,20011124,326
00001A,20011124,326
...
..
\end{verbatim}

では次は、「日付」と今日との差を求めてみましょう。今日の日付を求めるには、today関数を使います。

\begin{verbatim}
#!/bin/bash
#=========================================
# MCMD bash script - Lesson 11: Calculated fields
#=========================================
# Variables
inPath="tutorial_jp"

# Command 
mcut f=顧客,日付 i=${inPath}/dat.csv |
msortf f=顧客,日付 |   
mcal c='diffday(today(),$d{日付})' a=日数 o=outdat/mcaldayout.csv
#==========================================
\end{verbatim}

結果は次のようになります(実行した日によって、「日数」列の値は異なることに注意してください)。

\begin{verbatim}
顧客,日付,日数
00000A,20011209,4272
00000A,20011209,4272
00000A,20011209,4272
00000A,20011209,4272
00000A,20011217,4264
00000A,20011217,4264
00000A,20011217,4264
00000B,20011209,4272
00000B,20011209,4272
00000B,20011209,4272
00000B,20011209,4272
00000B,20011217,4264
00000B,20011217,4264
00000B,20011217,4264
00000C,20011209,4272
00000C,20011209,4272
00000C,20011209,4272
00000C,20011209,4272
00000C,20011217,4264
00000C,20011217,4264
00000C,20011217,4264
00000D,20011209,4272
00000D,20011209,4272
...
..
\end{verbatim}

\subsection{数値に関する関数}

\verb|mcal| コマンドは、ほかにも多くの関数を備えています。ここでは、数値の四捨五入について紹介しましょう。レッスン9の練習問題2で作った \verb|mshare2.sh| を改良して、数量シェア・金額シェアを小数点以下第2位までのパーセント表示で求めてみます。\\

そのために、まず \verb|mshare2.sh| を \verb|mcal45.sh| という名前でコピーします。レッスン9の演習問題2では、「中分類」の中での各「メーカー」のシェアを求めましたが、その値は0から1までの実数でした。100をかけることでパーセント表示にし、\verb|round| 関数を用いて小数点以下第3位を四捨五入します。\\

\verb|round| 関数は四捨五入したい値と、どの桁までの値に丸めるか(四捨五入したい桁の1つ上の桁)を指定します。
\verb|round(156.2841, 10)| は 160(10の桁までに丸めたいので、1の桁を四捨五入)、\verb|round(156.2841, 0.1)| は 156.3(0.1の桁までに丸めたいので、0.01の桁を四捨五入)となります。\\

\begin{verbatim}
#!/bin/bash
#=========================================
# MCMD bash script - Lesson 11: Calculated fields
#=========================================
# Variables
inPath="tutorial_jp"

# Command 
mcut f=中分類,メーカー,数量,金額 i=${inPath}/dat.csv |
msortf f=中分類,メーカー |   
msum k=中分類,メーカー f=数量,金額 |   
mshare k=中分類 f=数量:数量シェア,金額:金額シェア| 
mcal c='round(${数量シェア}*100,0.01)' a="数量シェアPCT" |
mcal c='round(${金額シェア}*100,0.01)' a="金額シェアPCT" o=outdat/mcal45out.csv
#==========================================
\end{verbatim}

結果は次のようになります。

\begin{verbatim}
中分類,メーカー,数量,金額,数量シェア,金額シェア,数量シェアPCT,金額シェアPCT
11,0002,36,9990,0.001069614048,0.0006518657684,0.11,0.07
11,0011,9,1980,0.0002674035119,0.0001291986208,0.03,0.01
11,0013,318,172075,0.00944825742,0.01122820842,0.94,1.12
11,0018,68,23979,0.00202038209,0.0015646736,0.2,0.16
11,0019,25,13051,0.0007427875331,0.0008516016159,0.07,0.09
11,0023,174,98112,0.00516980123,0.006401987414,0.52,0.64
...
..
12,0293,80,31955,0.1891252955,0.156795109,18.91,15.68
12,0316,16,5568,0.0378250591,0.02732076879,3.78,2.73
12,0573,34,19890,0.08037825059,0.09759520316,8.04,9.76
12,0828,13,6903,0.03073286052,0.03387127639,3.07,3.39
12,1416,10,4495,0.02364066194,0.02205582897,2.36,2.21
12,1483,49,17815,0.1158392435,0.08741370258,11.58,8.74
12,1725,141,65328,0.3333333333,0.3205479855,33.33,32.05
12,1878,80,51847,0.1891252955,0.2544001256,18.91,25.44
13,0000,4,1248,0.0007092198582,0.0004598998981,0.07,0.05
13,0025,106,50153,0.01879432624,0.01848185864,1.88,1.85
...
..
\end{verbatim}
\verb|mcal| コマンドには、ほかにもさまざまな関数が用意されています。詳しくはマニュアルをご覧ください。

\vspace {5mm}

\subsection{練習問題}

\verb|mcal| コマンドを使って、次のレポートを作成してみましょう。作成したら、出力ファイルとスクリプトを確認してください。

\begin{table}[htbp]
%\begin{center}
{\small
\begin{tabular}{ l | c || r }
\hline
\textbf{帳票名}   & \textbf{スクリプト名} & \textbf{結果ファイル}  \\
\hline
1. 日別の数量平均と金額平均を、整数で(小数点以下を四捨五入) & \href{exercise/mcal1.sh}{mcal1.sh} & \href{exercise/outdat/mcalout1.csv}{mcalout1.csv} \\
2. 日別の金額合計に消費税(5%)を掛けた値を、整数で(同上) & \href{exercise/mcal2.sh}{mcal2.sh} & \href{exercise/outdat/mcalout2.csv}{mcalout2.csv} \\
3. レシートデータに含まれるすべての日付から重複を取り除いた上で、\\ 各日の45日後の日付 & \href{exercise/mcal3.sh}{mcal3.sh} & \href{exercise/outdat/mcalout3.csv}{mcalout3.csv} \\



\hline
\end{tabular} 
}
\end{table} 


%\end{document}
