
%----------------------------------------------------------------------------------------
% MCMD Basic Techniques 	
% Lessons on Basic Commands  
%----------------------------------------------------------------------------------------

%\documentclass[a4paper]{jarticle}
%\begin{document}

%\section*{基本コマンド}

\section{レッスン 1: mcutコマンドで列を選択する}

{Mコマンド(MCMD)とは、データを加工するための便利なコマンド群の総称です。Mコマンドは、ビッグデータの処理でその威力を発揮します。各コマンドは、UNIXやLinuxのシェルから呼び出すことで実行されます(単独のコマンドとして実行するほか、シェルスクリプトから実行することももちろん可能です)。
\\ 
\\
このレッスンではまず、Mコマンドを学習するための準備として、サンプルデータを生成します。
\\
\\
次に、簡単なシェルスクリプトの作成と実行を通して、\verb|mcut| コマンドの使い方について学びます。\verb|mcut| コマンドは、ファイルから特定の項目を切り出すコマンドです。 }

\subsection{チュートリアルのための環境設定}

レッスンを開始する前に、データやスクリプトを置くためのディレクトリを作成しましょう。\\

\begin{verbatim}
$ mkdir mcmd_exercise
\end{verbatim}

\verb|mdata| コマンドはMコマンドのひとつで、チュートリアルで使うためのデータ(スーパーマーケットでのトランザクション(購買情報)を模した1年分のサンプルデータ)を生成します。このデータセットには、以下のファイルが含まれます。\\

\begin{enumerate}
 	\item cust.csv - 顧客データ
 	\item dat.csv - スーパーマーケットのトランザクションデータ(2001年)
	\item jicfs1.csv - 大分類(1桁の商品カテゴリコード)マスタ
	\item jicfs2.csv - 中分類(2桁の商品カテゴリコード)マスタ
	\item jicfs4.csv - 小分類(4桁の商品カテゴリコード)マスタ
	\item jicfs6.csv - 細分類(6桁の商品カテゴリコード)マスタ
	\item syo.csv - 商品説明マスタ
\end{enumerate}

では、チュートリアルのためのサンプルデータを生成しましょう。\\

さきほど生成したmcmd\_exerciseディレクトリに行き、\verb|mdata| コマンドを実行します。\\

\begin{verbatim}
$ cd mcmd_exercise
$ mdata tutorial_jp
\end{verbatim}

\verb|tutorial_jp| ディレクトリが作成され、7つのファイルが作成されます。確認してみましょう。\\

\begin{verbatim}
$ cd tutorial_jp
$ ls -l 
total 4720
-rw-r--r--  1 user  staff      18247 8 22 16:24 cust.csv
-rw-r--r--  1 user  staff    2281332 8 22 16:24 dat.csv
-rw-r--r--  1 user  staff        119 8 22 16:24 jicfs1.csv
-rw-r--r--  1 user  staff        643 8 22 16:24 jicfs2.csv
-rw-r--r--  1 user  staff       7421 8 22 16:24 jicfs4.csv
-rw-r--r--  1 user  staff      43593 8 22 16:24 jicfs6.csv
-rw-r--r--  1 user  staff      49438 8 22 16:24 syo.csv
\end{verbatim}
 
次のコマンドを実行して、トランザクションデータの内容を確認してみましょう。\\

\begin{verbatim}
$ less dat.csv

店,日付,時間,レシート,顧客,商品,大分類,中分類,小分類,細分類,メーカー,ブランド,仕入単価,単価,数量,金額,仕入金額,粗利金額
manufacturer,brand,unitCost,unitPrice,quantity,amount,costAmount,profit
A,20010108,142748,1000000,00245A,0000311,1,11,1116,111603,1776,177601,339,441,1,441,339,102
A,20010108,142748,1000000,00245A,0000384,1,14,1402,140205,0556,055600,286,372,1,372,286,86
A,20010108,142748,1000000,00245A,0000304,1,11,1107,110707,1487,148704,343,446,1,446,343,103
A,20010110,121010,1000001,00228A,0000426,1,11,1106,110601,1763,176305,177,231,5,1155,885,270
A,20010110,121010,1000001,00228A,0000313,1,11,1119,111997,1378,137803,185,241,6,1446,1110,336
A,20010110,121010,1000001,00228A,0000486,1,11,1108,110801,1315,131504,208,271,1,271,208,63
A,20010110,121010,1000001,00228A,0000393,1,11,1105,110517,1889,188901,410,533,6,3198,2460,738
A,20010110,121010,1000001,00228A,0000472,1,14,1402,140205,1974,197403,267,348,1,348,267,81
A,20010110,121010,1000001,00228A,0000332,1,11,1107,110721,0550,055003,354,461,1,461,354,107
...
..
\end{verbatim}

このチュートリアルでは、次のようなディレクトリ構造を想定します。\\

\begin{table}[htbp]
%\begin{center}
{\small
\begin{tabular}{ l l l }
\hline
\textbf{ディレクトリ}   & \textbf{場所} & \textbf{概要}   \\
\hline
 & ./ & レッスンファイル (PDFファイル) \\
mcmd\_exercise & ./mcmd\_exercise/ & チュートリアルのシェルスクリプト \\
outdat & ./mcmd\_exercise/outdat/ & スクリプトの出力ファイル(CSVファイル)  \\
tutorial\_jp & ./mcmd\_exercise/tutorial\_jp/ & チュートリアルで使うデータファイル(CSVファイル)  \\

\hline
\end{tabular} 
}
\end{table} 

%\subsection{mcutコマンドの使い方}

\subsection{ステップ 1: はじめてのスクリプト}

\noindent テキストエディタを使って、スクリプトファイル \verb|mcut.sh| を作成しましょう。作成したら、ファイルに実行権限を与えるのを忘れないようにしてください。
\\
\\
スクリプトのサンプルを以下に示します。ファイルの1行目には、このスクリプトを実行するためのプログラム名を \verb|#!| に続けて指定します。また \verb|#| で始まる行はコメントですので、処理内容の説明文などを書いておくと将来スクリプトを見返すときなどに役立ちます。
\\
\\
もしスクリプトファイルとデータファイルが異なるディレクトリにあるときは、以下のサンプルのように、データファイルの置いてあるディレクトリ名をシェル変数 \verb|inPath| に指定しておくとよいでしょう。
\\
\\
また、mcutコマンドのパラメータとして示されている「\verb|f=項目名|」については次節にて解説します。

\begin{verbatim}
#!/bin/bash
#================================
# MCMD bash script - Lesson1: Select column(s) with mcut
#================================
# Variables
inPath="tutorial_jp"

# Command 

mcut f=項目名 i=${inPath}/dat.csv o=outdat/mcutout.csv
#================================

\end{verbatim}

{\setlength{\parindent}{0cm}
\textbf{i=パラメータとo=パラメータ}\\

\textbf{なぜディレクトリ名をシェル変数に入れておくのか?} \\

\fbox{
  \parbox{\textwidth}{
    
入力ファイルのディレクトリ名はシェル変数 \emph{inPath} に指定してあります。もちろん、\emph{i=} パラメータにディレクトリ名ごと指定することもできるのですが、このようにシェル変数に入れておくことをおすすめします。
\\
\\
あとのレッスンになると、スクリプトでは複数のファイルを取り扱います。そのとき、ディレクトリ名をシェル変数に入れておくと、スクリプトがぐっと読みやすくなります。ディレクトリ構造を変更して、入力ファイルの置き場所が変わるような場合でも、修正箇所が少なくてすみます。
\\
  }
}
}

\subsection{ステップ 2: パラメータを指定する}

dat.csvファイルには18もの項目(列)があります。\verb|mcut| コマンドを使って、「日付」「数量」「金額」の3項目を取り出してみましょう。\\

{\setlength{\parindent}{0cm}

パラメータは次のようになります。 \\

項目: 		\verb|f=日付,数量,金額| \\
説明: この例では、「日付」「数量」「金額」項目を取り出します。 \\

入力ファイル: 		\verb|i=${inPath}/dat.csv| \\
説明: シェル変数 \verb|inPath| には、入力ファイルdat.csvの置いてあるディレクトリ名があらかじめ格納されているものとします。\\

出力ファイル: 	\verb|o=outdat/mcutout.csv| \\
説明: コマンドの処理結果は、\verb|outdat| ディレクトリ内に \verb|mcutout.csv| という名前で書き出されます。
}

\begin{verbatim}

#!/bin/bash
#================================
# MCMD bash script - Lesson1: mcut
#================================
# Variables
inPath="tutorial_jp"

# Command 
mcut f=日付,数量,金額 i=${inPath}/dat.csv o=outdat/mcutout.csv
#================================

\end{verbatim}

\subsection{ステップ 3: シェルスクリプトを実行する}

それでは、作成したスクリプト \verb|mcut.sh| を実行してみましょう。

\begin{verbatim}
$ ./mcut.sh
\end{verbatim} 

\noindent 実行すると、次のようなメッセージが表示されます。

\begin{verbatim}
#END# kgcut f=日付,数量,金額 i=tutorial_jp/dat.csv o=outdat/mcutout.csv; IN=24737 OUT=24737;
2013/08/12 15:37:32
\end{verbatim}

\subsection{ステップ 4: うまくいかないときは}

ケース 1: いつまでたっても \verb|#END| メッセージが表示されないときは、おそらくスクリプトに誤りがあります。Ctrl+c で中断してコマンドプロンプトに戻り、スクリプトを見直してください。\\
\\
ケース 2: \verb|#ERROR| メッセージが表示されたときは、スクリプトに誤りがあります。メッセージの内容に従って、スクリプトを修正してください。
\\
\subsection{ステップ 5: 結果 }

スクリプトの実行が成功すると、\verb|outdat| ディレクトリ内に \verb|mcutout.csv| ファイルができています。ファイルの内容を確認してみましょう。日付、数量、金額の3項目が取り出されていますか?


\begin{verbatim}
日付,数量,金額
20010108,1,441
20010108,1,372
20010108,1,446
20010110,5,1155
20010110,6,1446
20010110,1,271
20010110,6,3198
20010110,1,348
20010110,1,461
20010110,1,363
20010111,1,290
20010111,5,1165
20010111,1,387
20010111,1,375
20010111,1,687
20010111,1,500
...
..
\end{verbatim}

\vspace {5mm}

{\setlength{\parindent}{0cm}
\textbf{i=パラメータとo=パラメータ}\\

\fbox{
  \parbox{\textwidth}{
    
Mコマンドでは、入力ファイル・出力ファイルの指定に2つの方法が使えます。
\begin{itemize}
\item i=パラメータ、o=パラメータ
\item UNIXの標準入力、標準出力
\end{itemize}

たとえば、次の2つのコマンドは同じ結果をもたらします。
\begin{itemize}
\item mcut f=日付 i=dat.csv o=output.csv
\item mcut f=日付 $<$ dat.csv $>$ output.csv
\end{itemize}
  }
} 

\vspace {5mm}

\textbf{ファイルの拡張子}\\

\fbox{
  \parbox{\textwidth}{
    
.csv といったファイルの拡張子は、省略してもかまいません。Mコマンドは、ファイル名の拡張子に関係なくCSVファイルとして処理します。\\

またシェルスクリプトファイルの拡張子 .sh も省略可能です。その場合、次のように実行します。\\
\$ ./mcut 
  }
}
}

\newpage
\subsection{練習問題}

指定された項目を取り出し、レポートを作成するスクリプトを書きましょう。結果はファイルに出力してください。

\begin{table}[htbp]
{\small
\begin{tabular}{ l | c || r }
\hline
\textbf{レポートの内容}   & \textbf{スクリプト名} & \textbf{出力ファイル名}  \\
\hline
1. 「メーカー」「ブランド」「粗利金額」 & \href{exercise/mcut1.sh}{mcut1.sh} & \href{exercise/outdat/mcutout1.csv}{mcutout1.csv} \\
2. 「顧客」「レシート」 & \href{exercise/mcut2.sh}{mcut2.sh} & \href{exercise/outdat/mcutout2.csv}{mcutout2.csv} \\
\hline
\end{tabular} 
}
\end{table} 


%\end{document}
