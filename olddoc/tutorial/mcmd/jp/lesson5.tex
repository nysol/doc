
%----------------------------------------------------------------------------------------
% MCMD Basic Techniques 	
% Lessons on Basic Commands  
%----------------------------------------------------------------------------------------

%\documentclass[a4paper]{jarticle}
%\begin{document}
\section{レッスン 5: 文字列の切り出し}

\verb|mcal| コマンドを使うと、さまざまな計算・加工を行うことができます。ここでは、「日付」列から「年」や「月」を取り出すのにmcalコマンドを使ってみましょう。

\subsection{ステップ 1: スクリプトを作成する}

テキストエディタを使って、スクリプトファイル \verb|substring.sh| を作成しましょう。作成したら、ファイルに実行権限を与えるのを忘れないようにしてください。\\

このレッスンでは、年月別の数量合計、金額合計を求めます。\\


「日付」項目には年月日が8桁(YYYYMMDD;年4桁+月2桁+日2桁)で格納されています。
年と月を取り出す方法は大きく2つあり、一つは文字列として左から6文字の部分文字列を切り出す方法(left関数)で、
他方は、内部的に日付型に変換した後に年(year関数)と月(month関数)を取り出す方法です。
以下のスクリプトでは、left関数で年月を切り出した後に、年月別に数量と金額の合計を計算しています。
また練習問題に示された解答では、year関数を使う方法が示されています。

%そのためには、まず「日付」「数量」「金額」項目を \verb|mcut| コマンドで取り出し、\verb|mcal| コマンドに渡します。「日付」項目には年月日が8桁(YYYYMMDD;年4桁+月2桁+日2桁)で格納されています。年と月を取り出して「年月」項目を作成し、ソートしたあとで数量と金額の合計を年月別に求めます。結果は \verb|substringout.csv| ファイルに書き出します。

 \subsection{ステップ 2: パラメータを指定する}

{\setlength{\parindent}{0cm}

\verb|mcal| コマンドに与えるパラメータは次のようになります。\\

計算式: \verb|c='left($s{日付},6)'|\\
説明: c=パラメータで計算式を指定します。ここでは、「日付」列から左側の部分文字列を切り出すleft関数を用いています。

%計算式: \verb|c='cat("",n2s(year($d{日付})),n2s(month($d{日付})))'| \\
%説明: c=パラメータで計算式を指定します。ここでは、「日付」列から年を切り出すyear関数、月を切り出すmonth関数を用いた上で、その2つを結合するcat関数に与えています。\\

結果を格納する列: \verb|a=年月| \\
計算結果を格納するための新しい項目名(列名)は、a=で指定します。この例では、c=で指定した計算結果を「年月」項目に格納します。\\

このレッスンでは、レッスン1で作成した \verb|dat.csv| (スーパーマーケットのレシートデータ)を使います。
}

\begin{verbatim}
#!/bin/bash
#=========================================
# MCMD bash script - Lesson 5: Query by substring

# Variables
inPath="tutorial_jp"

# Command 
mcal c='left($s{日付},6)' a=年月 i=${inPath}/dat.csv |
mcut f=年月,数量,金額 |
msortf f=年月 |
msum k=年月 f=数量:数量合計,金額:金額合計 o=outdat/substringout.csv
#==========================================
\end{verbatim}
%mcut f=日付,数量,金額 i=${inPath}/dat.csv |
%mcal c='cat("",n2s(year($d{日付})),n2s(month($d{日付})))' a="年月" | 
%msortf f=年月%n |   
%msum k=年月 f=数量:数量合計,金額:金額合計 |
%mcut f=日付 -r o=outdat/substringout.csv

\subsection{ステップ 3: シェルスクリプトを実行する}

\begin{verbatim}
$ ./substring.sh
#END# kgcal a=年月 c=left($s{日付},6) i=xxd; IN=24737 OUT=24737; 2014/05/01 04:35:43
#END# kgcut f=年月,数量,金額; IN=24737 OUT=24737; 2014/05/01 04:35:43
#END# kgsortf f=年月; IN=24737 OUT=24737; 2014/05/01 04:35:43
#END# kgsum f=数量:数量合計,金額:金額合計 k=年月 o=outdat/substringout.csv; IN=24737 OUT=12; 2014/05/01 04:35:43
\end{verbatim}
%#END# kgcut f=日付,数量,金額 i=tutorial_jp/dat.csv; IN=24737 OUT=24737; 2013/08/15 14:49:20
%#END# kgcal a=年月 c=cat("",n2s(year($d{日付})),n2s(month($d{日付}))); ; 2013/08/15 14:49:20
%#END# kgsortf f=年月%n; IN=24737 OUT=24737; 2013/08/15 14:49:20
%#END# kgsum f=数量:数量合計,金額:金額合計 k=年月; IN=24737 OUT=12; 2013/08/15 14:49:20
%#END# kgcut -r f=日付 o=outdat/substringout.csv; IN=12 OUT=12; 2013/08/15 14:49:20

\noindent
結果は次のようになります。

\begin{verbatim}
年月,数量合計,金額合計
200101,768,304178
200102,1843,718715
200103,3760,1497215
200104,4041,1617813
200105,5050,2091847
200106,4844,2164878
200107,5984,2722816
200108,5904,2775503
200109,5115,2444511
200110,4965,2533169
200111,3879,2006570
200112,4686,2396780
\end{verbatim}
%数量合計,金額合計,年月
%768,304178,20011
%1843,718715,20012
%3760,1497215,20013
%4041,1617813,20014
%5050,2091847,20015
%4844,2164878,20016
%5984,2722816,20017
%5904,2775503,20018
%5115,2444511,20019
%4965,2533169,200110
%3879,2006570,200111
%4686,2396780,200112

\subsection{練習問題 }

\verb|mcal| コマンドを使って、次のレポートを作成してみましょう。作成したら、出力ファイルとスクリプトを確認してください。

\begin{table}[htbp]
%\begin{center}
{\small
\begin{tabular}{ l | c || r }
\hline
\textbf{レポートの内容}   & \textbf{スクリプト名} & \textbf{出力ファイル名}  \\
\hline
1. 「月」単位に集計した合計数量、合計金額 & \href{exercise/substring1.sh}{substring1.sh} & \href{exercise/outdat/substringout1.csv}{substringout1.csv} \\
2. 「日」単位に集計した合計数量、合計金額 & \href{exercise/substring2.sh}{substring2.sh} & \href{exercise/outdat/substringout2.csv}{substringout2.csv} \\
3. 「時」単位に集計した合計数量、合計金額 & \href{exercise/substring3.sh}{substring3.sh} & \href{exercise/outdat/substringout3.csv}{substringout3.csv} \\

\hline
\end{tabular} 
}
\end{table} 


%\end{document}
