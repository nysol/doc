
%----------------------------------------------------------------------------------------
% MCMD Basic Techniques 	
% Lessons on Basic Commands  
%----------------------------------------------------------------------------------------

%\documentclass[a4paper]{jarticle}
%\begin{document}

%\section*{基本コマンド}

\section{チュートリアルのための環境設定}

レッスンを開始する前に、データやスクリプトを置くためのディレクトリを作成しましょう。\\

\begin{verbatim}
$ mkdir mcmd_exercise
\end{verbatim}

\verb|mdata| コマンドはMコマンドのひとつで、チュートリアルで使うためのデータ(スーパーマーケットでのトランザクション(購買情報)を模した1年分のサンプルデータ)を生成します。このデータセットには、以下のファイルが含まれます。\\

\begin{enumerate}
 	\item cust.csv - 顧客データ
 	\item dat.csv - スーパーマーケットのトランザクションデータ(2001年)
	\item jicfs1.csv - 大分類(1桁の商品カテゴリコード)マスタ
	\item jicfs2.csv - 中分類(2桁の商品カテゴリコード)マスタ
	\item jicfs4.csv - 小分類(4桁の商品カテゴリコード)マスタ
	\item jicfs6.csv - 細分類(6桁の商品カテゴリコード)マスタ
	\item syo.csv - 商品説明マスタ
\end{enumerate}

では、チュートリアルのためのサンプルデータを生成しましょう。\\

さきほど生成したmcmd\_exerciseディレクトリに行き、\verb|mdata| コマンドを実行します。\\

\begin{verbatim}
$ cd mcmd_exercise
$ mdata tutorial_jp
\end{verbatim}

\verb|tutorial_jp| ディレクトリが作成され、7つのファイルが作成されます。確認してみましょう。\\

\begin{verbatim}
$ cd tutorial_jp
$ ls -l 
total 4720
-rw-r--r--  1 user  staff      18247 8 22 16:24 cust.csv
-rw-r--r--  1 user  staff    2281332 8 22 16:24 dat.csv
-rw-r--r--  1 user  staff        119 8 22 16:24 jicfs1.csv
-rw-r--r--  1 user  staff        643 8 22 16:24 jicfs2.csv
-rw-r--r--  1 user  staff       7421 8 22 16:24 jicfs4.csv
-rw-r--r--  1 user  staff      43593 8 22 16:24 jicfs6.csv
-rw-r--r--  1 user  staff      49438 8 22 16:24 syo.csv
\end{verbatim}
 
次のコマンドを実行して、トランザクションデータの内容を確認してみましょう。\\

\begin{verbatim}
$ less dat.csv

店,日付,時間,レシート,顧客,商品,大分類,中分類,小分類,細分類,メーカー,ブランド,仕入単価,単価,数量,金額,仕入金額,粗利金額
manufacturer,brand,unitCost,unitPrice,quantity,amount,costAmount,profit
A,20010108,142748,1000000,00245A,0000311,1,11,1116,111603,1776,177601,339,441,1,441,339,102
A,20010108,142748,1000000,00245A,0000384,1,14,1402,140205,0556,055600,286,372,1,372,286,86
A,20010108,142748,1000000,00245A,0000304,1,11,1107,110707,1487,148704,343,446,1,446,343,103
A,20010110,121010,1000001,00228A,0000426,1,11,1106,110601,1763,176305,177,231,5,1155,885,270
A,20010110,121010,1000001,00228A,0000313,1,11,1119,111997,1378,137803,185,241,6,1446,1110,336
A,20010110,121010,1000001,00228A,0000486,1,11,1108,110801,1315,131504,208,271,1,271,208,63
A,20010110,121010,1000001,00228A,0000393,1,11,1105,110517,1889,188901,410,533,6,3198,2460,738
A,20010110,121010,1000001,00228A,0000472,1,14,1402,140205,1974,197403,267,348,1,348,267,81
A,20010110,121010,1000001,00228A,0000332,1,11,1107,110721,0550,055003,354,461,1,461,354,107
...
..
\end{verbatim}

このチュートリアルでは、次のようなディレクトリ構造を想定します。\\

\begin{table}[htbp]
%\begin{center}
{\small
\begin{tabular}{ l l l }
\hline
\textbf{ディレクトリ}   & \textbf{場所} & \textbf{概要}   \\
\hline
 & ./ & レッスンファイル (PDFファイル) \\
mcmd\_exercise & ./mcmd\_exercise/ & チュートリアルのシェルスクリプト \\
outdat & ./mcmd\_exercise/outdat/ & スクリプトの出力ファイル(CSVファイル)  \\
tutorial\_jp & ./mcmd\_exercise/tutorial\_jp/ & チュートリアルで使うデータファイル(CSVファイル)  \\

\hline
\end{tabular} 
}
\end{table} 

%\subsection{mcutコマンドの使い方}

