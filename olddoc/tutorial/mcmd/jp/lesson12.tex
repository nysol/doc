
%----------------------------------------------------------------------------------------
% MCMD Basic Techniques 	
% Lessons on Basic Commands  
%----------------------------------------------------------------------------------------

%\documentclass[a4paper]{jarticle}
%\begin{document}

\setlength{\baselineskip}{4mm}

\section{レッスン 12: ファイルを複数に分割する}

データのサイズが大きくなると、日付や地域、商品分類などを基準にファイルを分割することで、整理や分析がやりやすくなることがあります。\verb|msep| コマンドを使うと、指定した項目を基準にファイルを複数に分割することができます。

\subsection{msepコマンドの使い方}

テキストエディタを使って、スクリプトファイル \verb|msep.sh| を作成しましょう。作成したら、ファイルに実行権限を与えるのを忘れないようにしてください。\\

このレッスンでは、トランザクションデータ(スーパーのレシートデータ)を店舗単位に分割します。レシートデータにはA、B、C、Dの4店舗のレコードが含まれていますので、それぞれ\verb|shopA.csv|、\verb|shopB.csv|、\verb|shopC.csv|、\verb|shopD.csv|の4つのファイルに分割します。
 
\subsubsection{パラメータを指定する}

{\setlength{\parindent}{0cm}

\verb|msep|コマンドに与えるパラメータは次のようになります。\\

ファイル名: \verb|d=tutorial_jp/shop${店}.csv| \\
説明: 分割されたファイルのディレクトリ名およびファイル名を指定します。ファイル名の中に項目名を\$\{\}で囲んで指定することで、分割の基準となる項目が決まります。\\
}

\begin{verbatim}
#!/bin/bash
#=========================================
# MCMD bash script - Lesson 12: Separate files by shop
#=========================================
# Variables
inPath="tutorial_jp"

# Command 
msep d='tutorial_jp/shop${店}.csv' i=${inPath}/dat.csv  
#==========================================
\end{verbatim}

すべてのレコードが、\verb|tutorial_jp|ディレクトリ内の\verb|shopA.csv|から\verb|shopD.csv|のいずれかに分類され保存されます。\verb|tutorial_jp|ディレクトリの中身を\verb|ls|コマンドで確認してみましょう。\\

\begin{verbatim}
$ ls
cust.csv	jicfs2.csv	outdat		shopC.csv
dat.csv		jicfs4.csv	shopA.csv	shopD.csv
jicfs1.csv	jicfs6.csv	shopB.csv	syo.csv
\end{verbatim}

\subsection{月を基準に分割する}
 
 では、次は月を基準にファイルを分割してみましょう。\\
 
 {\setlength{\parindent}{0cm}
 
テキストエディタを使って、スクリプトファイル \verb|msep1.sh| を作成します。
レッスン5で学習したように、mcalのleft関数を使って「年月」項目を作成します。
この項目の値によって、その行が出力されるファイル名が決まることになります。
%まずは \verb|msortf| コマンドを使って「日付」項目の降順でソートし、次に \verb|mcal| コマンドで「日付」項目から年+月を取り出し、「年月」項目を作成します(レッスン5を思い出してください)。
「年月」項目でソーティングした後に、 \verb|msep| コマンドによって「年月」を基準に12のファイルに分割します
(レシートデータには2001年の1年分=12か月分のデータが含まれているためです)。
出力先のディレクトリは \verb|tutorial_jp|、ファイル名は \verb|dat${年月}.csv| としましょう。\\

\verb|msep| コマンドに与えるパラメータは次のようになります。\\

ファイル名: \verb|d=tutorial_jp/dat${年月}.csv| \\
説明: \verb|tutorial_jp|ディレクトリの中に、\verb|dat+「年月」項目+.csv| という名前でファイルを作成します。
}
 
\begin{verbatim}
 #!/bin/bash
#=========================================
# MCMD bash script - Lesson 12: Separate files by year and month
#=========================================
# Variables
inPath="tutorial_jp"

# Command 
mcal c='left($s{日付},6)' a=年月 i=${inPath}/dat.csv |
msortf f=年月 |
msep d='tutorial_jp/dat${年月}.csv'
#==========================================
\end{verbatim}
%msortf f=日付%n i=${inPath}/dat.csv |
%mcal c='cat("",n2s(year($d{日付})),n2s(month($d{日付})))' a=年月   |
%msep d='tutorial_jp/dat${年月}.csv'
 
シェルスクリプトを保存したら、実行しましょう。実行すると、次のようなメッセージが表示されます。

\begin{verbatim}
$ ./msep1.sh 
#END# kgcal a=年月 c=left($s{日付},6) i=tutorial_jp/dat.csv; IN=24737 OUT=24737; 2014/05/01 05:07:47
#END# kgsortf f=年月; IN=24737 OUT=24737; 2014/05/01 05:07:47
#END# kgsep d=tutorial_jp/dat${年月}.csv; IN=24737 OUT=2283; 2014/05/01 05:07:47
\end{verbatim}
%#END# kgsortf f=日付%n i=tutorial_jp/dat.csv; IN=24737 OUT=24737; 2013/08/24 10:48:07
%#END# kgcal a=年月 c=cat("",n2s(year($d{日付})),n2s(month($d{日付}))); IN=24737 OUT=24737;
%2013/08/24
%#END# kgsep d=tutorial_jp/dat${年月}.csv; IN=24737 OUT=2283; 2013/08/24 10:48:07
 
すべてのレコードが\verb|tutorial_jp|ディレクトリ内の12のファイルに分割して書き出されます。\verb|tutorial_jp|ディレクトリの中身を\verb|ls|コマンドで確認してみましょう。

\begin{verbatim}
$ ls
cust.csv	dat200112.csv	dat200106.csv	jicfs2.csv	shopB.csv
dat.csv		dat200102.csv	dat200107.csv	jicfs4.csv	shopC.csv
dat200101.csv	dat200103.csv	dat200108.csv	jicfs6.csv	shopD.csv
dat200110.csv	dat200104.csv	dat200109.csv	syo.csv
dat200111.csv	dat200105.csv	jicfs1.csv	shopA.csv
\end{verbatim}

200101.csvファイルの中身を確認してみましょう。次のようになっているはずです。

\begin{verbatim}
店,日付,時間,レシート,顧客,商品,大分類,中分類,小分類,細分類,メーカー,ブランド,仕入単価,単価,数量,金額,仕入金額,粗利金額,年月
A,20010108,142748,1000000,00245A,0000311,1,11,1116,111603,1776,177601,339,441,1,441,339,102,200101
B,20010130,185345,1000011,00018B,0000362,1,11,1119,111901,0171,017104,222,289,1,289,222,67,200101
D,20010119,111636,1000007,00228D,0000461,1,14,1406,140671,0894,089401,381,496,1,496,381,115,200101
B,20010131,080841,1000014,00217B,0000215,1,11,1119,111907,0143,014301,99,129,7,903,693,210,200101
D,20010119,111636,1000007,00228D,0000421,1,11,1105,110597,0616,061601,412,536,6,3216,2472,744,200101
B,20010130,185345,1000011,00018B,0000255,1,14,1406,140621,1894,189400,256,333,1,333,256,77,200101
D,20010119,111636,1000007,00228D,0000409,1,13,1301,130125,1033,103300,333,433,6,2598,1998,600,200101
B,20010130,185345,1000011,00018B,0000362,1,11,1119,111901,0171,017104,222,289,1,289,222,67,200101
D,20010119,111636,1000007,00228D,0000405,1,13,1301,130123,1310,131000,408,531,1,531,408,123,200101
B,20010130,185345,1000011,00018B,0000208,1,11,1112,111203,0777,077700,358,466,1,466,358,108,200101
D,20010119,111636,1000007,00228D,0000304,1,11,1107,110707,1487,148704,343,446,4,1784,1372,412,200101
B,20010130,185345,1000011,00018B,0000455,1,14,1406,140681,0361,036103,376,489,1,489,376,113,200101
...
..
\end{verbatim}

%\end{document}
