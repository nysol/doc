
%----------------------------------------------------------------------------------------
% MCMD Basic Techniques 	
% Lessons on Basic Commands  
%----------------------------------------------------------------------------------------

%\documentclass[a4paper]{jarticle}
%\begin{document}

\section{レッスン 14: データの結合}

多くの場合、データは複数のファイルから構成されます。そのため、データを分析するためには、あらかじめデータを結合することが重要となります。\verb|mjoin| コマンドは、項目の値に基づいてデータを結合するためのコマンドです。このレッスンでは、\verb|mjoin| コマンドを用いたデータの結合について学びます。

\subsection{パラメータを指定する}

テキストエディタを使って、スクリプトファイル\verb|mjoin.sh|を作成しましょう。作成したら、ファイルに実行権限を与えるのを忘れないようにしてください。\\

このレッスンでは、トランザクションデータ(スーパーのレシートデータ)小分類ごとの数量合計・金額合計を求め、その小分類名とともに出力します。小分類名は、小分類のマスタ(\verb|jicfs4.csv|)から結合します。\\

そのために、まず「小分類」「数量」「金額」の3項目を \verb|mcut| コマンドを使って取り出します。小分類でソートしたあと \verb|msum| コマンドで数量合計と金額合計を求めます。最後に小分類マスタから小分類名を結合し、結果を \verb|mjoinout.csv| に書き出します。\\

このレッスンでは、\verb|dat.csv|(スーパーマーケットのレシートデータ)のほか、\verb|jicfs4.csv|(小分類マスタ)を使います。これらのファイルについては、レッスン1を確認してください。


{\setlength{\parindent}{0cm}

\verb|mjoin| コマンドに与えるパラメータは次のようになります。\\

項目: \verb|k=小分類| \\
説明: 2つのファイルを結合するためのキー項目をk=パラメータで指定します。\verb|dat.csv|と\verb|jicfs4.csv|のいずれにも「小分類」項目があることを確認しましょう。また、\verb|mjoin|コマンドでファイルを結合するときは、どちらのファイルにおいてもこのk=パラメータで指定した項目でソートされていることが必要です。\\

項目: \verb|f=小分類名| \\
説明: k=パラメータで指定した項目をキーとして、連結する(マスタから取り出す)項目名はf=パラメータで指定します。カンマで区切ることで、複数の項目を指定することができます。\\

マスタファイル: \verb|m=${inPath}/jicfs4.csv| \\
説明: どのファイルから項目を結合するのか、そのファイル名はm=パラメータで指定します。

マスタファイル(ここでは\verb|jicfs4.csv|)は次のようになっています。\\

\begin{verbatim}
小分類,小分類名
1101,調味料
1102,食用油
1103,スプレッド類
1104,乳製品
1105,調理品
1106,スープ
1107,冷凍食品
1108,缶詰
1110,粉類
1111,ホームメーキング材料
1112,麺類
1113,パン・シリアル類
...
..
\end{verbatim}
}

スクリプトファイルは次のようになります。\\

\begin{verbatim}
#!/bin/bash
#=========================================
# MCMD bash script - Lesson 14: Join 
#=========================================
# Variables
inPath="tutorial_jp"

# Command 
mcut f=小分類,数量,金額 i=${inPath}/dat.csv | 
msortf f=小分類 |   
msum k=小分類 f=数量,金額 |   
mjoin k=小分類 f=小分類名 m=${inPath}/jicfs4.csv o=outdat/mjoinout.csv
#==========================================
\end{verbatim}

\subsection{シェルスクリプトを実行する}
 
シェルスクリプトを保存したら、実行しましょう。実行すると、次のようなメッセージが表示されます。\\

\begin{verbatim}
$ ./mjoin.sh
#END# kgcut f=小分類,数量,金額 i=tutorial_jp/dat.csv; IN=24737 OUT=24737; 2013/08/21 16:04:13
#END# kgsortf f=小分類; IN=24737 OUT=24737; 2013/08/21 16:04:13
#END# kgsum f=数量:数量合計,金額:金額合計 k=小分類; IN=24737 OUT=35; 2013/08/21 16:04:13
#END# kgjoin f=小分類名 k=小分類 m=tutorial_jp/jicfs4.csv o=outdat/mjoinout.csv; IN=35 OUT=35;
\end{verbatim}

結果は次のようになります。

\begin{verbatim}
小分類,数量合計,金額合計,小分類名
1101,3758,1526511,調味料
1102,1698,732146,食用油
1103,1160,500887,スプレッド類
1104,2019,1024216,乳製品
1105,2847,1611468,調理品
1106,1100,484361,スープ
1107,1946,911273,冷凍食品
1108,1157,410367,缶詰
1110,1478,577858,粉類
1111,1768,744427,ホームメイキング材料
1112,1450,732630,麺類
1113,1150,445836,パン・シリアル類
...
..
\end{verbatim}

\subsection{練習問題}

\verb|mjoin| コマンドを使って、次のレポートを作成してみましょう。作成したら、出力ファイルとスクリプトを確認してください。

\begin{table}[htbp]
%\begin{center}
{\small
\begin{tabular}{ l | c || r }
\hline
\textbf{レポートの内容}   & \textbf{スクリプト名} & \textbf{出力ファイル名}  \\
\hline
1. 中分類ごとの数量合計と金額合計(中分類名も添える) & \href{exercise/mjoin1.sh}{mjoin1.sh} & \href{exercise/outdat/mjoinout1.csv}{mjoinout1.csv} \\
2. 細分類ごとの数量合計と金額合計(細分類名も添える) & \href{exercise/mjoin2.sh}{mjoin2.sh} & \href{exercise/outdat/mjoinout2.csv}{mjoinout2.csv} \\

\hline
\end{tabular} 
}
\end{table} 


%\end{document}
