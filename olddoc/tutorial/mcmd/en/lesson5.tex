
%----------------------------------------------------------------------------------------
% MCMD Basic Techniques 	
% Lessons on Basic Commands  
%----------------------------------------------------------------------------------------

%\documentclass[a4paper]{jarticle}
%\begin{document}

\section{Lesson 5: Query by substring }

The \verb|mcal| command allows user to extract values in the data that matches the specified substring format, and stores the substring as a new field in each record.  Common usage includes extracting year from the \emph{date} attribute which includes year/date/day information.


\subsection{Step 1: Create script}

Create a new script \verb|substring.sh| with a text editor. Make sure the file permission is changed to executable permission. \\

Objective: Calculate the total quantity and total amount by month.   \\

Methodology: "Date" field contains date format in eight digit year-month-day (YYYYMMDD;Year 4 digit+Month 2 digit+Day 2 digit). 
There are two main methods to extract year and month. First is to extract the character string starting from the left to the 6th character (left function). Another method is to convert to date format data type,  then extract the year and month information using the year and month function respectively. 

In the script below, left function is used to extract the YearMonth information, afterwards, the sum of quantity and amount is calculated for each month (YearMonth). 
The following exercise demonstrates  the usage of  year function. 


 \subsection{Step 2: Define attributes and options }

{\setlength{\parindent}{0cm}

Specify the parameters in \verb|mcal| as follows: \\

Condition: 	\verb|c='left($s{date},6)'|\\
Description: 	Specify the computation formula at the \verb|c=| parameter. In this condition, the left function is used to extract the substring starting from the left in the "date" field.  \\

New attribute (count results): 	\verb|a=YearMonth| \\
Description:  		The \verb|a=| parameter defines the new column where the computed values is saved. In this example, the computation results is stored in the column "YearMonth". \\

This lesson will use the dataset \verb|dat.csv| generated from the \verb|mdata| command in lesson 1. 
}

\begin{verbatim}
#/bin/bash
#=========================================
# MCMD bash script - Lesson 5: Query by substring

# Variables
inPath="tutorial_en"

# Command 
mcal c='left($s{date},6)' a=YearMonth i=${inPath}/dat.csv |
mcut f=YearMonth,quantity,amount |
msortf f=YearMonth |
msum k=YearMonth f=quantity:totalQuantity,amount:totalAmount o=outdat/substringout.csv

#==========================================
\end{verbatim}

\subsection{Step 3: Run the shell script }

\begin{verbatim}
$ ./substring.sh 
#END# kgcal a=YearMonth c=left($s{date},6) i=xxd; IN=24737 OUT=24737; 2014/05/01 04:35:43
#END# kgcut f=YearMonth,quantity,amount; IN=24737 OUT=24737; 2014/05/01 04:35:43
#END# kgsortf f=YearMonth; IN=24737 OUT=24737; 2014/05/01 04:35:43
#END# kgsum f=quantity:totalQuantity,amount:totalAmount k=YearMonth o=outdat/substringout.csv; IN=24737 OUT=12; 2014/05/01 04:35:43
\end{verbatim}

\noindent
The result looks like the following: 

\begin{verbatim}
YearMonth,totalQuantity,totalAmount
200101,768,304178
200102,1843,718715
200103,3760,1497215
200104,4041,1617813
200105,5050,2091847
200106,4844,2164878
200107,5984,2722816
200108,5904,2775503
200109,5115,2444511
200110,4965,2533169
200111,3879,2006570
200112,4686,2396780

\end{verbatim}

\subsection{Convert month substring to English characters }

\begin{verbatim}
Condition:		c='emonth($d{date})'
\end{verbatim}

\noindent Description: 		The \verb|c=| parameter specify the conditional function for computation. 
The above function extracts the month from the \emph{date} attribute and convert to month
 in English characters. \\


\subsection{Exercise }

Let's practice the substring function on the reports below. Check your results with the scripts and output files below. 

\begin{table}[htbp]
%\begin{center}
{\small
\begin{tabular}{ l | c || r }
\hline
\textbf{Report name}   & \textbf{Script name} & \textbf{Output file}  \\
\hline
1. Total quantity and total amount for each month. & \href{exercise/substring1.sh}{substring1.sh} & \href{exercise/outdat/substringout1.csv}{substringout1.csv} \\
2. Total quantity and total amount for each day. & \href{exercise/substring2.sh}{substring2.sh} & \href{exercise/outdat/substringout2.csv}{substringout2.csv} \\
3. Total quantity and total amount for each hour. & \href{exercise/substring3.sh}{substring3.sh} & \href{exercise/outdat/substringout3.csv}{substringout3.csv} \\

\hline
\end{tabular} 
}
\end{table} 


%\end{document}
