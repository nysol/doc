
%----------------------------------------------------------------------------------------
% MCMD Basic Techniques 	
% Lessons on Basic Commands  
%----------------------------------------------------------------------------------------

%\documentclass[a4paper]{jarticle}
%\begin{document}


\section{Lesson 7: Select records with mselstr command}

Another related command for record selection is \verb|mselstr| command. It allows users to define more details on the information to extract. The command is more efficient for matching multiple values in the same column. 


\subsection{Step 1: Create script}

Create a new script \verb|mselstr.sh| with a text editor. Make sure the file permission is changed to executable permission. \\

Objective: Select transactions that occurred on January 10, 23, or 30 in 2001.     \\

Methodology: Select \emph{date, quantity, amount} by \verb|mcut|, the result is piped to \verb|mselstr|. Records that matches the date \emph{20010110, 20010123, 20010130} are extracted and output is saved to the file \verb|mselstrout.csv|. \\ 

This lesson will use the dataset \verb|dat.csv| generated from the \verb|mdata| command in lesson 1. 

 \subsection{Step 2: Define attributes and options }

{\setlength{\parindent}{0cm}

The parameters for \verb|mselstr| command are as follows: \\

Field: 		\verb|f=date| \\
Description: 	Define the field(s) that has values which match the query conditions at the \verb|f=| parameter. In this case, date will be specified as the parameter. Multiple fields can be specified. \\

Value: 		\verb|v=20010110,20010123, 20010130 | \\
Description: 	Specify the list of substring(s) for matching with the values in the field parameter at the \verb|v=| parameter. Multiple values can be specified separated by ",". 

}

\begin{verbatim}
#/bin/bash
#=========================================
# MCMD bash script - Lesson 7: Select Records 
#=========================================
# Variables
inPath="tutorial_en"

# Command 
mcut f=date,quantity,amount                      					i=${inPath}/dat.csv |
mselstr f=date v=20010110,20010123, 20010130  			o=outdat/mselstrout.csv
#==========================================
\end{verbatim}

\subsection{Step 3: Run the shell script }

\begin{verbatim}
$ ./mselstr.sh 
#END# kgcut f=date,quantity,amount i=tutorial_en/dat.csv; IN=24737 OUT=24737; 2013/08/19 11:26:13
#END# kgselstr f=date o=outdat/mselstrout.csv v=20010110,20010123,20010130; IN=24737 OUT=148; \\ 2013/08/19 11:26:13
\end{verbatim}

\noindent
The result is as follows: 

\begin{verbatim}
date,quantity,amount
20010110,5,1155
20010110,6,1446
20010110,1,271
20010110,6,3198
20010110,1,348
20010110,1,461
20010110,1,363
20010123,2,564
20010123,1,401
20010123,1,528
20010123,1,436
20010123,3,813
20010123,1,419
20010130,1,362
20010130,1,294
20010130,2,544
...
..
\end{verbatim}

The character strings are stored in a hash table, thus the order of character strings and the length of strings do not affect the efficiency of the \verb|mselstr| command. \\

{\setlength{\parindent}{0cm}
\textbf{Other functions in select command  }\\

\fbox{
  \parbox{\textwidth}{
Besides the above examples illustrated in this lesson, "msel" and "mselstr" have other ways to select records. Nevertheless, the declaration of the key value is important as it groups multiple records with the same key attribute as one group for data processing. This will be further explained in the Advanced Techniques in the tutorial.

  }
} 

}


\subsection{Exercise }

Let's practice the \verb|mselstr| command on the reports below. Check your results with the scripts and output files below. 

\begin{table}[htbp]
%\begin{center}
{\small
\begin{tabular}{ l | c || r }
\hline
\textbf{Report name}   & \textbf{Script name} & \textbf{Output file}  \\
\hline
1. Total quantity and total amount for manufacturer 0054,0085, and 1110. & \href{exercise/mselstr1.sh}{mselstr1.sh} & \href{exercise/outdat/mselstrout1.csv}{mselstrout1.csv} \\
2. Total quantity and total amount for 4-digit classification code matching \\1101,1111,1115,1401,1403, and 1406. & \href{exercise/mselstr2.sh}{mselstr2.sh} & \href{exercise/outdat/mselstrout2.csv}{mselstrout2.csv} \\
3. Report on the total quantity and total amount for transactions with amount matching \\ 199, 299 and 399. & \href{exercise/mselstr3.sh}{mselstr3.sh} & \href{exercise/outdat/mselstrout3.csv}{mselstrout3.csv} \\

\hline
\end{tabular} 
}
\end{table} 


%\end{document}
