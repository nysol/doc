
%----------------------------------------------------------------------------------------
% MCMD Basic Techniques 	
% Lessons on Basic Commands  
%----------------------------------------------------------------------------------------

%\documentclass[a4paper]{jarticle}
%\begin{document}

\section{Lesson 10: Calculation among rows with maccum command}

This command computes the cumulative value of the defined key field. 

\subsection{Step 1: Create script}

Create a new script \verb|maccum.sh| with a text editor. Make sure the file permission is changed to executable permission. \\

Example:  Compute the accumulative value of daily quantity and amount from the beginning of the year to the end of the year. \\ 

Methodology: Select  \emph{date,quantity,amount} with \verb|mcut| command, then use \verb|msum| command to compute the total quantity and total amount for each unique "date". Next, pipe the result to \verb|maccum| to calculate the cumulative value for all days. Send the output to the file \verb|maccumout.csv|. \\

This lesson will use the dataset \verb|dat.csv| generated from the \verb|mdata| command in lesson 1. 

\subsection{Step 2: Define attributes and options }

{\setlength{\parindent}{0cm}

Specify the parameters for \verb|maccum| command as follows: \\

Key: 			\verb|k=| \\
Description: 	This parameter specify the key attribute for the calculation of accumulated values. In this example, accumulative value is computed based on all dates in the dataset, thus this parameter is not required in this case. \\

Attributes: 	\verb|f=quantity:accumQuantity,amount:accumAmount| \\
Description: 	Define the attributes where the cumulative values will be computed. The cumulative values will be appended in the last column in the dataset, therefore, specify the new attribute using a colon. \\
}

\begin{verbatim}
#/bin/bash
#=========================================
# MCMD bash script - Lesson 10: Calculation among rows with maccum
#=========================================
# Variables
inPath="tutorial_en"

# Command 
mcut f=date,quantity,amount             i=${inPath}/dat.csv |
msortf f=date                           |
msum k=date f=quantity:totalQuantity,amount:totalAmount         |
maccum f=totalQuantity:accumQuantity,totalAmount:accumAmount o=outdat/maccumout.csv
#==========================================
\end{verbatim}

\subsection{Step 3: Run the shell script }

\begin{verbatim}
$ ./maccum.sh
#END# kgcut f=date,quantity,amount i=tutorial_en/dat.csv; IN=24737 OUT=24737; 
2013/08/19 16:58:07
#END# kgsortf f=date; IN=24737 OUT=24737; 2013/08/19 16:58:07
#END# kgsum f=quantity:totalQuantity,amount:totalAmount k=date; IN=24737 OUT=324; 2013/08/19 16:58:07
#END# kgaccum f=totalQuantity:accumQuantity,totalAmount:accumAmount o=outdat/maccumout.csv; ;
 2013/08/19 16:58:07

The result is as follows: 

date,totalQuantity,totalAmount,accumQuantity,accumAmount
20010108,21,8680,21,8680
20010110,67,23495,88,32175
20010111,94,38261,182,70436
20010116,65,28333,247,98769
20010119,210,93586,457,192355
20010123,54,19476,511,211831
20010125,43,14516,554,226347
20010128,43,15448,597,241795
20010130,157,60577,754,302372
20010131,14,1806,768,304178
20010201,8,1024,776,305202
20010203,13,2236,789,307438
20010204,132,52936,921,360374
20010205,62,24132,983,384506
20010206,82,29950,1065,414456
20010208,78,24906,1143,439362
20010209,50,21776,1193,461138
20010211,58,23650,1251,484788
20010212,82,35368,1333,520156
20010213,65,24974,1398,545130
20010216,245,89800,1643,634930
...
..
\end{verbatim}

\subsection{Step 4: Cumulative value of daily total quantity and total amount arranged from the latest transaction date }

Methodology: Before running \verb|maccum| command, sort all dates in descending order. The script will look as follows:

\begin{verbatim}
mcut f=date,quantity,amount             i=${inPath}/dat.csv |
msortf f=date%r                         	|
msum k=date f=quantity:totalQuantity,amount:totalAmount         |
maccum f=totalQuantity:accumQuantity,totalAmount:accumAmount o=outdat/maccumout.csv

Save and execute your script. The result should be as follows:

date,totalQuantity,totalAmount,accumQuantity,accumAmount
20011230,93,58331,93,58331
20011229,300,151488,393,209819
20011228,292,162534,685,372353
20011227,112,61542,797,433895
20011226,73,34360,870,468255
20011225,202,109219,1072,577474
20011224,305,141597,1377,719071
20011222,121,71352,1498,790423
20011221,156,93305,1654,883728
20011220,153,77793,1807,961521
20011219,168,91842,1975,1053363
20011218,251,125010,2226,1178373
20011217,19,8687,2245,1187060
20011216,154,76426,2399,1263486
20011215,160,74503,2559,1337989
...
..
\end{verbatim}

\subsection{Step 5: Calculate the cumulative value of quantity and amount for each 4-digit classification code within each 2-digit classification code }

Before running the \verb|maccum| command, sort the data by 4 digit classification codes within the respective 2 digit classification codes. Sort the 4 digit category code in descending order. The 2-digit classification code is the unit to compute cumulative values, and the cumulative values resets to zero upon a change in 2-digit classification code. The script is modified below.

\begin{verbatim}
mcut f=CategoryCode2,CategoryCode4,quantity,amount      i=${inPath}/dat.csv |
msortf f=CategoryCode2,CategoryCode4%r          |
msum k=CategoryCode2,CategoryCode4 f=quantity,amount            |
maccum k=CategoryCode2 f=quantity:accumQuantity,amount:accumAmount o=outdat/maccumout0.csv

Save and execute your script. The result should be as follows:

CategoryCode2,CategoryCode4,quantity,amount,accumQuantity,accumAmount
11,1197,332,173474,332,173474
11,1121,2430,1122457,2762,1295931
11,1120,960,411380,3722,1707311
11,1119,2791,1262170,6513,2969481
11,1118,1125,435023,7638,3404504
11,1117,938,489566,8576,3894070
11,1116,1144,542858,9720,4436928
11,1115,2194,1106249,11914,5543177
11,1114,212,80084,12126,5623261
11,1113,1150,445836,13276,6069097
11,1112,1450,732630,14726,6801727
11,1111,1768,744427,16494,7546154
11,1110,1478,577858,17972,8124012
11,1108,1157,410367,19129,8534379
11,1107,1946,911273,21075,9445652
11,1106,1100,484361,22175,9930013
11,1105,2847,1611468,25022,11541481
11,1104,2019,1024216,27041,12565697
11,1103,1160,500887,28201,13066584
11,1102,1698,732146,29899,13798730
11,1101,3758,1526511,33657,15325241
12,1297,160,83802,160,83802
12,1203,10,4495,170,88297
...
..
\end{verbatim}

\vspace {5mm}

{\setlength{\parindent}{0cm}

\textbf{Sort priority}\\


    
The sort operation optimizes the speed of search or data manipulation operations on big data by grouping the rows in specified categories.  Thus, data should be sorted in specific order before aggregation operations, as well as maccum, mbest, mslide commands. \\


}


\subsection{Exercise }

Let's practice the \verb|maccum| command on the reports below. Check your results with the scripts and output files below. 

\begin{table}[htbp]
%\begin{center}
{\small
\begin{tabular}{ l | c || r }
\hline
\textbf{Report name}   & \textbf{Script name} & \textbf{Output file}  \\
\hline
1. Cumulative values of quantity and amount by manufacturers \\ (sort manufacturer by ascending order). & \href{exercise/maccum1.sh}{maccum1.sh} & \href{exercise/outdat/maccumout1.csv}{maccumout1.csv} \\
2. Cumulative values of quantity and amount in 2-digit classification code classified \\ by manufacturers in descending order. & \href{exercise/maccum2.sh}{maccum2.sh} & \href{exercise/outdat/maccumout2.csv}{maccumout2.csv} \\


\hline
\end{tabular} 
}
\end{table} 


%\end{document}
