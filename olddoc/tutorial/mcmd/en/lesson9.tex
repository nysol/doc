
%----------------------------------------------------------------------------------------
% MCMD Basic Techniques 	
% Lessons on Basic Commands  
%----------------------------------------------------------------------------------------

%\documentclass[a4paper]{jarticle}
%\begin{document}


\section{Lesson 9: Calculation among rows with mshare command}

This command computes individual share out of the total value in the defined field.  The share is derived by: valueonerecord / valueallrecords.



\subsection{Step 1: Create script}

Create a new script \verb|mshare.sh| with a text editor. Make sure the file permission is changed to executable permission. \\

Objective:  Calculate the share of total quantity and total amount for each transaction. \\

Methodology: Select \emph{date,quantity,amount} with \verb|mcut| command, pipe the results to \verb|msum| command to compute the total quantity and total amount based on each unique date. Next, the output is forwarded to \verb|mshare| command to calculate the share of quantity and amount for each date out of the total quantity and total amount. Save the results to the file \verb|mshareout.csv|.   \\

This lesson will use the dataset \verb|dat.csv| generated from the \verb|mdata| command in lesson 1. 

 \subsection{Step 2: Define attributes and options }

{\setlength{\parindent}{0cm}

The parameters for \verb|mshare| command are as follows: \\

Key: 		\verb|k=| \\
Description: 	Since the ratio is calculated against all records in the dataset, the key parameter do not need to be specified in this case. \\

Attributes: 	\verb|f=quantity:quantityShare,amount:amountShare|\\
Description: 	Define the attributes where the percentages will be computed. For arguments with multiple attributes, use comma as delimiters to separate the field names without spaces in between. Rename the calculated fields using a colon followed by the name of new attribute. 
}

\begin{verbatim}
#/bin/bash
#=========================================
# MCMD bash script - Lesson 9: Calculation among rows with mshare
#=========================================
# Variables
inPath="tutorial_en"

# Command 
mcut f=date,quantity,amount             i=${inPath}/dat.csv |
msortf f=date                           |
msum k=date f=quantity,amount           |
mshare f=quantity:quantityShare,amount:amountShare o=outdat/mshareout.csv
#==========================================
\end{verbatim}

\subsection{Step 3: Run the shell script }

\begin{verbatim}
$ ./mshare.sh 
#END# kgcut f=date,quantity,amount i=tutorial_en/dat.csv; IN=24737 OUT=24737; 2013/08/19 14:42:10
#END# kgsortf f=date; IN=24737 OUT=24737; 2013/08/19 14:42:10
#END# kgsum f=quantity,amount k=date; IN=24737 OUT=324; 2013/08/19 14:42:10
#END# kgshare f=quantity:quantityShare,amount:amountShare o=outdat/mshareout.csv; IN=324 OUT=324; 2013/08/19 14:42:10
\end{verbatim}

\noindent
The result is as follows: 

\begin{verbatim}
date,quantity,amount,quantityShare,amountShare
20010108,21,8680,0.0004130687071,0.0003729484345
20010110,67,23495,0.001317885875,0.001009495791
20010111,94,38261,0.001848974213,0.001643937794
20010116,65,28333,0.001278545998,0.001217367281
20010119,210,93586,0.004130687071,0.0040210544
20010123,54,19476,0.001062176675,0.0008368137915
20010125,43,14516,0.0008458073526,0.0006237004004
20010128,43,15448,0.0008458073526,0.000663745094
20010130,157,60577,0.003088180334,0.002602776189
20010131,14,1806,0.0002753791381,7.759733557e-05
20010201,8,1024,0.0001573595075,4.399760333e-05...
..
\end{verbatim}

\subsection{Step 4: Modify the shell script }

After we have verified the share of quantity and amount, let's modify the script to find out the share of quantity and amount for each 4-digit classification code within every 2-digit classification code. \\

Methodology: Select \emph{CategoryCode2,CategoryCode4,quantity,amount} with \verb|mcut| command. Compute the total quantity and amount based on each unique \emph{CategoryCode2,CategoryCode4} with \verb|msum|. Use the results to calculate the share of total quantity and total amount for each 4-digit category in their parent category. Write the results to the file \verb|mshareout0.csv|. \\

Your script should look as follows:\\

\begin{verbatim}
mcut f=CategoryCode2,CategoryCode4,quantity,amount      i=${inPath}/dat.csv |
msortf f=CategoryCode2,CategoryCode4                            |       
msum k=CategoryCode2,CategoryCode4 f=quantity:totalQuantity,amount:totalAmount  |
mshare k=CategoryCode2 f=totalQuantity:quantityShare,totalAmount:amountShare o=outdat/mshareout0.csv

Save and execute your script. The result should be as follows:

Code2Desc,Code4Desc,totalQuantity,totalAmount,quantityShare,amountShare
11,1101,3758,1526511,0.111655822,0.09960763423
11,1102,1698,732146,0.05045012925,0.04777386535
11,1103,1160,500887,0.03446534153,0.03268379271
11,1104,2019,1024216,0.05998752117,0.06683196695
11,1105,2847,1611468,0.08458864426,0.1051512338
11,1106,1100,484361,0.03268265145,0.03160544098
11,1107,1946,911273,0.05781858157,0.05946222966
11,1108,1157,410367,0.03437620703,0.02677719717
11,1110,1478,577858,0.04391359895,0.03770629121
11,1111,1768,744427,0.05252993434,0.04857522306
11,1112,1450,732630,0.04308167692,0.04780544724
11,1113,1150,445836,0.03416822652,0.02909161429
11,1114,212,80084,0.00629883828,0.005225627447
11,1115,2194,1106249,0.0651870339,0.07218477021
11,1116,1144,542858,0.03398995751,0.03542247721
11,1117,938,489566,0.02786938824,0.03194507675
\end{verbatim}

\vspace {5mm}

{\setlength{\parindent}{0cm}

\textbf{Validating interim results}\\

    
You may want to check the interim results when more commands are used in the script as in the above example or when the final result is not what you expected. Alternatively, you can use the "tee" UNIX command to copy the standard input to standard output. An example is shown below:  \\

}
\begin{verbatim}
mcut f=date,quantity,amount             i=${inPath}/dat.csv | tee checkdat.csv |
msortf f=date                           		|
msum k=date...
..         
\end{verbatim}

\subsection{Exercise }

Let's practice the \verb|mshare| command on the reports below. Check your results with the scripts and output files below. 

\begin{table}[htbp]
%\begin{center}
{\small
\begin{tabular}{ l | c || r }
\hline
\textbf{Report name}   & \textbf{Script name} & \textbf{Output file}  \\
\hline
1. Shares of total quantity and total amount for each manufacturer. & \href{exercise/mshare1.sh}{mshare1.sh} & \href{exercise/outdat/mshareout1.csv}{mshareout1.csv} \\
2. Shares of total quantity and total amount for each manufacturer in each \\ 2-digit classification code. & \href{exercise/mshare2.sh}{mshare2.sh} & \href{exercise/outdat/mshareout2.csv}{mshareout2.csv} \\


\hline
\end{tabular} 
}
\end{table} 


%\end{document}
