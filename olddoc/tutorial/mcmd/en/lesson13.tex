
%----------------------------------------------------------------------------------------
% MCMD Basic Techniques 	
% Lessons on Basic Commands  
%----------------------------------------------------------------------------------------

%\documentclass[a4paper]{jarticle}
%\begin{document}


\section{Lesson 13: Concatenate records }

In data processing, we often encounter situations where large volumes of data is stored in multiple data files classified by time, region, categories, and other attributes. The \verb|mcat| command allow users to combine separate files or combine specified columns from separate files for data processing. 

\subsection{Step 1: Create script}

Create a new script \verb|mcat.sh| with a text editor. Make sure the file permission is changed to executable permission. \\

Objective:  Merge all datasets for year 2001.  \\ 

Methodology: Merge 12 datasets for year 2001 using the \verb|mcat| command. The files are located in the same input data directory "\~/tutorial\_en/".  The output dataset \verb|mcatout.csv| can be found in the tutorial\_en directory. \\


This lesson will use the monthly datasets generated from the \verb|msep| command in lesson 12. The files are stored in "\~/tutorial\_en/" directory.


 \subsection{Step 2: Define attributes and options }

{\setlength{\parindent}{0cm}

Specify the parameters as follows: \\

Input: 		\verb|i=${inPath}/dat2001*.csv| \\
Description: 	This parameter defines the input file names for merging. The file names are separated by comma but it is not efficient when there are a large number of files. Thus, we can use wildcard character to match character strings in the input file name. 

}

\begin{verbatim}
#/bin/bash
#=========================================
# MCMD bash script - Lesson 13: Merge files
#=========================================
# Variables
inPath="tutorial_en"

# Command 
mcat               i=${inPath}/dat2001*.csv  o=outdat/mcatout.csv
#==========================================
\end{verbatim}

All files in year 2001 and concatenated and saved to \verb|mcatout.csv|.  \\

Let's calculate the total quantity and total amount for all transactions in 2001. Assuming the monthly datasets has the same structure, we would only need to extract \emph{date,quantity,amount} column and concatenate the column from all 12 monthly datasets. Afterwards, compute the total sum for quantity and amount for each unique date with \verb|msum|, and save the results to the file \verb|mcatout.csv|.  \\

Define the field parameter to extract from all input datasets as follows: \\

Field: 		\verb|f=date,quantity,amount| \\
Description: 	This parameter defines the common fields to extract from all input datasets for merging. The fields names are separated by comma. 

 \subsection{Step 3: Modify and run the shell script}
 
 \begin{verbatim}
 #/bin/bash
#=========================================
# MCMD bash script - Lesson 13: Merge files
#=========================================
# Variables
inPath="tutorial_en"

# Command 
mcat f=date,quantity,amount             i=${inPath}/dat2001*.csv | 
msortf f=date%n                                   |
msum k=date f=quantity:totalQuantity,amount:totalAmount o=outdat/mcatout.csv
 #==========================================
 
Save and run the script. The run-time message is as follows: \\

$ ./mcat.sh 
#END# kgcat f=date,quantity,amount i=tutorial_en/dat2001*.csv; IN=24737 OUT=24737; 2013/08/24 10:48:52
#END# kgsortf f=date%n; IN=24737 OUT=24737; 2013/08/24 10:48:53
#END# kgsum f=quantity:totalQuantity,amount:totalAmount k=date o=outdat/mcatout.csv; IN=24737 OUT=324; 
2013/08/24 10:48:53

The result is shown below: 
date,totalQuantity,totalAmount
20010108,21,8680
20010110,67,23495
20010111,94,38261
20010116,65,28333
20010119,210,93586
20010123,54,19476
20010125,43,14516
20010128,43,15448
20010130,157,60577
20010131,14,1806
20010201,8,1024
20010203,13,2236
20010204,132,52936
20010205,62,24132
20010206,82,29950
20010208,78,24906
20010209,50,21776
20010211,58,23650
20010212,82,35368
20010213,65,24974
20010216,245,89800
20010217,64,21223
20010220,30,9774
20010222,78,32783
20010223,118,45828
20010224,182,82685
20010226,107,40550
20010227,252,96736
...
..
\end{verbatim}


\subsection{Directory Index }

Directories can be used to index multiple files in the database to increase the efficiency for data processing and data analysis. The \verb|mcat| command can retrieve certain files from the selected the directory where the file structure is indexed based on certain categories. \\

In this lesson, sales transaction data of a retail stores are organized in separate files classified by month. For instance, to generate a sales report for store A in January, 2001, it is much faster to select the transaction data file \verb|tutorial_en/a/dat20011.csv| according to the file name in the directory index. It will be a more data processing to extract  transaction data for January from all transactions in the entire dataset.  \\

Wildcard characters can be used to select datasets which matches a search string.  For example, to merge sales data from October, November, and December you can specify \verb|~/20011*/dat.csv|.\\


\subsection{Exercise }

Let's practice the \verb|mcat| command on the reports below. Check your results with the scripts and output files below. 

\begin{table}[htbp]
%\begin{center}
{\small
\begin{tabular}{ l | c || r }
\hline
\textbf{Report name}   & \textbf{Script name} & \textbf{Output file}  \\
\hline
Daily total quantity and total amount from April, May, July & \href{exercise/mcat1.sh}{mcat1.sh} & \href{exercise/outdat/mcatout1.csv}{mcatout1.csv} \\
Daily total quantity and total amount on April, October-December & \href{exercise/mcat2.sh}{mcat2.sh} & \href{exercise/outdat/mcatout2.csv}{mcatout2.csv} \\

\hline
\end{tabular} 
}
\end{table} 


%\end{document}
