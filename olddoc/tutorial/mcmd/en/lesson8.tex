
%----------------------------------------------------------------------------------------
% MCMD Basic Techniques 	
% Lessons on Basic Commands  
%----------------------------------------------------------------------------------------

%\documentclass[a4paper]{jarticle}
%\begin{document}


\section{Lesson 8: Remove duplicate rows}

This command returns distinct (unique) values from the specified field. This technique is often used to create a master file with unique keys in the dataset. 


\subsection{Step 1: Create script}

Create a new script \verb|muniq.sh| with a text editor. Make sure the file permission is changed to executable permission. \\

Objective: Create a dataset with unique dates in each row.     \\

Methodology: Select \emph{date} with \verb|mcut|, the result is sent to \verb|muniq| command which combines records with same key values into one unique record. Save the results to the file \verb|muniqout.csv|. \\

This lesson will use the dataset \verb|dat.csv| generated from the \verb|mdata| command in lesson 1.

 \subsection{Step 2: Define attributes and options }

{\setlength{\parindent}{0cm}

The parameters for \verb|muniq| command are as follows: \\

Key: 		\verb|k=date| \\
Description: 	Define the field(s) at the \verb|k=| parameter where duplicate values for each row within the defined key will be removed. Multiple fields can be specified. 

}

\begin{verbatim}
#/bin/bash
#=========================================
# MCMD bash script - Lesson 8: Remove Duplicate Rows
#=========================================
# Variables
inPath="tutorial_en"

# Command 
mcut f=date                     i=${inPath}/dat.csv |
msortf f=date                   |
muniq k=date                    o=outdat/muniqout.csv
#==========================================
\end{verbatim}

\subsection{Step 3: Run the shell script }

\begin{verbatim}
$ ./muniq.sh 
#END# kgcut f=date i=tutorial_en/dat.csv; IN=24737 OUT=24737; 2013/08/19 13:39:12
#END# kgsortf f=date; IN=24737 OUT=24737; 2013/08/19 13:39:12
#END# kguniq k=date o=outdat/muniqout.csv; IN=24737 OUT=324; 2013/08/19 13:39:12
\end{verbatim}

\noindent
The result is as follows: 

\begin{verbatim}
date
20010108
20010110
20010111
20010116
20010119
20010123
20010125
20010128
20010130
20010131
20010201
20010203
20010204
20010205
20010206
20010208
20010209
20010211
...
..
\end{verbatim}



\subsection{Exercise }

Let's practice the \verb|muniq| command on the reports below. Check your results with the scripts and output files below. 

\begin{table}[htbp]
%\begin{center}
{\small
\begin{tabular}{ l | c || r }
\hline
\textbf{Report name}   & \textbf{Script name} & \textbf{Output file}  \\
\hline
1. Create a brand code master file & \href{exercise/muniq1.sh}{muniq1.sh} & \href{exercise/outdat/muniqout1.csv}{muniqout1.csv} \\
2. Create a manufacturer master file & \href{exercise/muniq2.sh}{muniq2.sh} & \href{exercise/outdat/muniqout2.csv}{muniqout2.csv} \\


\hline
\end{tabular} 
}
\end{table} 


%\end{document}
