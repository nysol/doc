
%\documentclass[a4paper]{jarticle}
%\begin{document}

\section{meach - Execution method for simple parallel processing \label{sect:meachRB}}
Method of Array class for simple parallel processing without exclusive control. 
It is a simple implementation to initiate multiple processes asynchronously.


\subsection{Format}
{\Large
\begin{verbatim}
1) Array::meach(mpcount){|value| block}
2) Array::meach(mpcount){|value,count| block}
\end{verbatim}
} 

Process the code in parallel by \verb|block| when \verb|value| is given as the block parameter of an element in an array. 
If \verb|count| is given as a block parameter, set the array element number (integer from 0) being processed. 


\begin{description}
	\setlength{\itemindent}{-5mm}
	\item {\large \verb/mpcount /} - Number of parallel processes. 
\end{description}


\subsection{Examples}
\subsubsection*{Example 1 Print row number and value from field name}

\begin{Verbatim}[baselinestretch=0.7,frame=single]
# Process five elements from 10 to 6 (integer) in two parallel threads. 
# Processing content is as simple as displaying the value of element and the element 
number of the array. 
> [10,9,8,7,6].meach(2){|value,count|
>   puts "#{value},#{count}"
> }
10,0
9,1
8,2
7,3
6,4
\end{Verbatim}

\subsection{Related Command}

%\end{document}

