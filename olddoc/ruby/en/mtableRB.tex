
%\documentclass[a4paper]{jarticle}
%\begin{document}

\section{Mtable - Read CSV data into cell class \label{sect:mtableRB}}
This class read specified CSV data into random access memory by cell. 
Its features are as follows. 

\begin{itemize}
%\setlength{\itemindent}{0mm}
\item Access cells in random order by the specified row and column.
\item Data is read-only and cannot insert or update records. 
\item All data is read as character string. Corresponding type conversion (ex. to\_i) is necessary for other usage. 
\item Read data into available memory. The program terminates with an error when there is insufficient space.
\end{itemize} 

\subsection{Method}

{\Large
\begin{verbatim}
* MCMD::Mtable::new(arguments)
\end{verbatim}
}

Generate Mtable object. 
At the \verb|arguments| parameter, specify the following arguments in character string separated with a space.

\begin{table}[htbp]
\begin{tabular}{ll}
\verb|i=|    & Input file name (String)\\
\verb|-nfn|  & No field name in the first row. \\
\end{tabular} 
\end{table} 

{\Large
\begin{verbatim}
* MCMD::Mtable::cell(col=0, row=0) -> String
\end{verbatim}
}

Return the value of cell to the corresponding \verb|row| and \verb|col| (column). 
\verb|row| and \verb|col| is assigned by sequential row and column number. 
Both row and column number are expressed in integer starting from 0 (Column number of Mcsvin starts from 1). Returns nil if \verb|row| and \verb|col| are out of range.


\begin{description}
\setlength{\itemindent}{0mm}
\item[row ] Row number. Positive integer value including 0. Default value is 0. 
\item[col ] Column number. Positive integer value including 0. Field name cannot be specified. Default value is 0.
\end{description}

Returns the value of  col number of the column as \verb|col|. It is equivalent to specifying cell(col, 0). However, if both \verb|col| and \verb|row| is not specified, it returns 0th item of 0th row. It is equivalent to specifying cell(0,0).


{\Large
\begin{verbatim}
* MCMD::Mtable::names() -> String Array
\end{verbatim}
}

Return the field name array.

{\Large
\begin{verbatim}
* MCMD::Mtable::name2num() -> String=>Fixnum Hash
\end{verbatim}
}

Return Hash of key of field name and the the corresponding field number.


{\Large
\begin{verbatim}
* MCMD::Mtable::size() -> Fixnum
\end{verbatim}
}

Return the number of rows.

%\subsection{備考}
%\begin{itemize}
%\item k=を指定する場合は、そこで指定した項目で並べ替えておく必要はない(内部ではHash型変数で管理するため)。
%\end{itemize}

\subsection{Examples}
\subsubsection*{Example 1}

\begin{Verbatim}[baselinestretch=0.7,frame=single]
# dat1.csv
customer,date,amount
A,20081201,10
B,20081002,40

tbl=MCMD::Mtable.new("i=dat1.csv")
p tbl.names     # -> ["customer", "date", "amount"]
p tbl.name2num  # -> {"amount"=>2, "date"=>1, "customer"=>0}
p tbl.size      # -> 2
p tbl.cell(0,0) # -> "A"
p tbl.cell(0,1) # -> "B"
p tbl.cell(1,1) # -> "20081202"
p tbl.cell(1)   # -> "20081201"
p tbl.cell      # -> "A"
\end{Verbatim}

\subsubsection*{Example 2 Without column names in the first row}

\begin{Verbatim}[baselinestretch=0.7,frame=single]
# dat1.csv
customer,date,amount
A,20081201,10
B,20081002,40

tbl=MCMD::Mtable.new("i=dat1.csv -nfn") # Treats the first row as data.
p tbl.names     # -> nil
p tbl.name2num  # -> nil
p tbl.size      # -> 3
p tbl.cell(0,0) # -> "customer"
p tbl.cell(0,1) # -> "A"
p tbl.cell(1,1) # -> "20081201"
p tbl.cell(1)   # -> "date"
p tbl.cell      # -> "customer"
\end{Verbatim}

\subsection{Related Command}
\hyperref[sect:mcsvinRB]{Mcsvin} : Read CSV data. 

%\end{document}

