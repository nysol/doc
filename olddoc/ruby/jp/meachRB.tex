
%\documentclass[a4paper]{jarticle}
%\begin{document}

\section{meach 簡単な並列処理の実行メソッド\label{sect:meachRB}}
排他制御を伴わない簡単な並列処理を実行するArrayクラスのメソッド。
複数のプロセスを非同期に起動するだけの簡単な実装である。

\subsection{書式}
{\Large
\begin{verbatim}
1) Array::meach(mpcount){|value| block}
2) Array::meach(mpcount){|value,count| block}
\end{verbatim}
} 

配列の要素をブロックパラメータ\verb|value|として、\verb|block|で与えられたコードを並列に実行する。
ブロックパラメータとして\verb|count|を与えれば、実行中の配列要素番号(0から始まる整数)がセットされる。

\begin{description}
	\setlength{\itemindent}{-5mm}
	\item {\large \verb/mpcount /} 並列で実行するプロセス数。
\end{description}


\subsection{利用例}
\subsubsection*{例1 項目名の出力と行番号・値の出力}

\begin{Verbatim}[baselinestretch=0.7,frame=single]
# 10から6までの5つの要素(整数)について、2つのプロセスで並列処理する。
# 処理内容は、要素の値と配列の要素番号を表示するだけの簡単なものである。
> [10,9,8,7,6].meach(2){|value,count|
>   puts "#{value},#{count}"
> }
10,0
9,1
8,2
7,3
6,4
\end{Verbatim}

\subsection{関連コマンド}

%\end{document}

