
%\documentclass[a4paper]{jarticle}
%\begin{document}

\section{Margs 引数操作クラス\label{sect:margsRB}}
コマンドライン引数を扱うクラス。
以下のような特徴を持つ。
\begin{itemize}
	\setlength{\itemindent}{-5mm}
	\item \verb|keyword=value| および \verb|-keyword| の二つのフォーマットの引数を扱う。
	\item \verb|-keyword|はオプションで、Bool型(true/false)に変換される。
	\item \verb|value|の型としては、Rubyの原型として、String配列、Fixnum配列、Float配列を扱うことができる。
	\item その他の特殊な型として、ファイル型や項目名型を提供する。
	\item デフォルト値や値の範囲を指定することができる。
	\item 指定が正しくなければエラーメッセージを表示して終了する。
	\item \verb|--help|が指定されればhelp()を呼び出し終了する。またヘルプ関数名を指定することも可能。\verb|--help|は、一般のオプションとは違い、マイナス記号が2つであることに注意する。
\end{itemize}

\subsection{メソッド}

%%%%%%%%%%%%%%%%%%
{\Large
\begin{verbatim}
* MCMD::Margs.new(ARGV[,allKeyWords][,mandatoryKeyWords][,helpFunction)
\end{verbatim}
}

Margsオブジェクトを生成する。
「keyword=値」もしくは「-keyword」の形式で与えられたコマンドライン引数が、クラス内部のHashもしくはArray変数にセットされる。

\begin{description}
	\setlength{\itemindent}{-5mm}
	\item {\large \verb/ARGV /} RubyのARGV変数。
	\item {\large \verb/allKeyWords /} \verb/key=value/もしくは\verb/-key/による引数キーワードリスト。
	                   ここで指定した以外の引数がARGVに指定されていないことをチェックし、
	                   指定されていればエラー終了してくれる。
	                   \verb/allKeyWords/を省略した場合はこのチェックをしない。
	\item {\large \verb/mandatoryKeyWords /} \verb/key=value/による引数キーワードリスト。
	                         ここで指定した引数がコマンドラインで指定されていなければ
	                         エラー終了してくれる。
	                         \verb/mandatoryKeyWords/を省略した場合はこのチェックをしない。
	\item {\large \verb/helpFunction/} \verb/--help/が指定されたときに呼び出される関数名。

\end{description}

\begin{Verbatim}[baselinestretch=0.7,frame=single]
# コマンドライン
$ ruby test.rb i=dat.csv -abc

# test.rbの内容
args=Margs.new(ARGV, "i=,v=,-abc") # OK
args=Margs.new(ARGV, "i=,v=") # -abcは指定できないのでエラー終了
args=Margs.new(ARGV, "i=,v=,-abc","i=,v=") # v=は必須だが指定されてないのでエラー終了
\end{Verbatim}

%%%%%%%%%%%%%%%%%%
{\Large
\begin{verbatim}
* MCMD::Margs.file(keyWord,mode): ファイル名の取得
\end{verbatim}
}
\begin{description}
	\setlength{\itemindent}{-5mm}
	\item {\large \verb/keyWord/} \verb/key=/形式のキーワード(String)
	\item {\large \verb/mode/} "r"(readableのチェック)、もしくは"w"(writableのチェック)を指定する。(String)
\end{description}

keywordで指定された値を入力ファイル名とみなし、modeが"r"の場合、
そのファイルが読み取り可能であればそのファイル名を返す。
読み取りが可能でなければエラー終了する。
modeが"w"の場合は、書き込みを行うディレクトリに書き込み可能かどうかチェックし、
書き込み可能であればそのファイル名を返す。
書き込み可能でなければエラー終了する。

\begin{Verbatim}[baselinestretch=0.7,frame=single]
# コマンドライン
$ ruby test.rb i=dat.csv

# test.rbの内容
args=Margs.new(ARGV)
puts args.file("i=","r") # dat.csvがreadableであれば"dat.csv"
puts args.file("i=","w") # カレントディレクトリがwritableであれば"dat.csv"
\end{Verbatim}

%%%%%%%%%%%%%%%%%%
{\Large
\begin{verbatim}
* MCMD::Margs.field(keyWord,fileName)
\end{verbatim}
}
\verb/keyWord/で指定した項目名について、\verb/fileName/で指定されたファイルの項目
と関連付けて各種情報を返す。


\begin{description}
	\setlength{\itemindent}{-5mm}
	\item {\large \verb/keyWord /} \verb/"key="/形式のキーワード(String)。
	\item {\large \verb/fileName /} ファイル名。
\end{description}

コマンドラインでの項目名の指定は以下のフォーマットに従わなければならない。
\begin{equation}
\verb/key=/name_1\verb/[:/newName_1\verb/%/option_1\verb/],/name_2\verb/[:/newName_2\verb/%/option_2\verb/],/\ldots
\end{equation}

複数の項目名はカンマで区切って指定する。
$name_i$は、\verb/fileName/で指定されたCSVファイルの項目名でなければならない。
さもなければ\verb/"field name not found"/のエラーにて終了する。

項目名$name_i$には二つの属性$newName_i$と$option_i$を指定できる(省略可)。
それぞれ用途は自由であるが、\verb/:/と\verb/%/で区切らなければならない。

一般的な用途としては、ある項目$name_i$に対する演算結果を
新しい項目名$newName_i$として追加出力することを想定している。
そして$option_i$は処理内容を制御するオプションとして利用する。

このメソッドは、以下に示す各種情報をHashで返す。太字はHashキー。

\begin{description}
	\setlength{\itemindent}{-5mm}
	\item {\large \verb/names /} 項目名\verb/names/の配列(String Array)。
	\item {\large \verb/newNames /} 新項目名\verb/newNames/の配列(String Array)。指定がなければnil。
	\item {\large \verb/options /} オプション\verb/options/の配列(String Array)。指定がなければnil。
	\item {\large \verb/fld2csv /} \verb/"key="/で指定された項目に対応するCSVファイル(\verb/fileName/)の項目番号(0から始まる)(String Array)。
	\item {\large \verb/csv2fld /} CSVファイルの項目番号を要素番号とした\verb/"key="/で指定された項目の指定順序番号(0から始まる)(String Array)。
指定のない項目はnil。
\end{description}

\begin{Verbatim}[baselinestretch=0.7,frame=single]
# test.csvの内容
fld1,fld2,fld3
1,2,3
4,5,6

# コマンドライン
$ ruby test.rb f=fld1,fld3

# test.rbの内容
args=Margs.new(ARGV)
fld=args.field("f=","test.csv")
p fld["names"]   # -> ["fld1","fld3"]
p fld["fld2csv"] # -> [0,2] fld1,fld3はtest.csvの0番目と2番目の項目に対応
p fld["csv2fld"] # -> [0,nil,1] test.csvの0番目の項目はf=の0番目に指定された

# コマンドライン
$ ruby test.rb f=fld3:newFld3%n,fld2%nr

# test.rbの内容
args=Margs.new(ARGV)
fld=args.field("f=","test.csv")
p fld["names"]    # -> ["fld3", "fld2"]
p fld["newNames"] # -> ["newFld3", nil]
p fld["options"]  # -> ["n", "nr"]
p fld["fld2csv"]  # -> [2, 1]
p fld["csv2fld"]  # -> [nil, 1, 0]
\end{Verbatim}

%%%%%%%%%%%%%%%%%%
{\Large
\begin{verbatim}
* MCMD::Margs.str(keyWord[,default][,token1][,token2])
\end{verbatim}
}
文字列引数の取得
\begin{description}
	\setlength{\itemindent}{0mm}
	\item {\large \verb/keyWord /} \verb/"key="/形式のキーワード(String)
	\item {\large \verb/default /} 指定がなかったときのデフォルト値。省略時はnil。
	\item {\large \verb/token1 /} 複数の文字列を指定する場合の区切り文字。省略時は区切りは無いものと見なす。
	\item {\large \verb/token2 /} token1で切り出された文字列をさらにtoken2を区切りとする文字列で区切る。省略時は区切りは無いものと見なす。
\end{description}

コマンドライン上で指定された引数のうち、\verb/keyWord/とマッチする値を文字列として返す。
コマンドラインで指定されていなければ\verb/default/の文字列を返す。
この時\verb/default/が\verb/nil/であればnilを返す。

\verb/token1/が指定されていれば区切られた文字列を要素とする配列を返す。
さらに\verb/token2/が指定されていれば、配列の配列を返す。

\begin{Verbatim}[baselinestretch=0.7,frame=single]
# コマンドライン
$ ruby test.rb v=abc

# test.rbの内容
args=Margs.new(ARGV)
puts args.str("v=") # ->"abc"
puts args.str("w=") # -> nil
puts args.str("w=","xyz") # -> "xyz"
\end{Verbatim}

\begin{Verbatim}[baselinestretch=0.7,frame=single]
# コマンドライン
$ ruby test.rb v=abc,efg:xyz,hij

# test.rbの内容
args=Margs.new(ARGV)
puts args.str("v=") # ->"abc,efg:xyz,hij"
puts args.str("v=",nil,",") # ->["abc", "efg:xyz", "hij"]
puts args.str("v=",nil,",",":") # ->[["abc"], ["efg","xyz"], ["hij"]]
\end{Verbatim}


%%%%%%%%%%%%%%%%%%
{\Large
\begin{verbatim}
* MCMD::Margs.float(keyWord[,default][,from][,to]): Float型数値引数の取得
\end{verbatim}
}

\begin{description}
\item {\large \verb/keyWord /} ``key=''のキーワード(String)
\item {\large \verb/default /} 指定がなかったときのデフォルト値(Float)。省略時はnil。
\item {\large \verb/from /} 範囲チェックの下限値(Float)。省略時は下限値チェックをしない。
\item {\large \verb/to /} 範囲チェックの上限値(Float)。省略時は上限値チェックをしない。
\end{description}

コマンドライン上で指定された引数のうち、\verb/keyWord/とマッチする値をFloatに変換して返す。
コマンドラインで指定されていなければ\verb/default/の値を返す。
範囲チェックにパスしなければエラー終了する。

\begin{Verbatim}[baselinestretch=0.7,frame=single]
# コマンドライン
$ ruby test.rb v=0.12

# test.rbの内容
args=Margs.new(ARGV)
puts args.float("v=") # -> 0.12
puts args.float("v=",nil,0.2,0.3) # -> 範囲エラー
puts args.float("w=") # -> nil
puts args.float("w=",0.1) # -> 0.1
\end{Verbatim}

%%%%%%%%%%%%%%%%%%
{\Large
\begin{verbatim}
* MCMD::Margs.int(keyWord[,default][,from][,to]) Fixnum型数値引数の取得
\end{verbatim}
}

\begin{description}
\item {\large \verb/keyWord /} \verb/key=/のキーワード(String)
\item {\large \verb/default /} 指定がなかったときのデフォルト値(Float)。省略時はnil。
\item {\large \verb/from /} 範囲チェックの下限値(Float)。省略時は下限値チェックをしない。
\item {\large \verb/to /} 範囲チェックの上限値(Float)。省略時は上限値チェックをしない。
\end{description}

コマンドライン上で指定された引数のうち、\verb/keyWord/とマッチする値をFloatに変換して返す。
コマンドラインで指定されていなければ\verb/default/の値を返す。
範囲チェックにパスしなければエラー終了する。

\begin{Verbatim}[baselinestretch=0.7,frame=single]
# コマンドライン
$ ruby test.rb v=10

# test.rbの内容
args=Margs.new(ARGV)
puts args.int("v=") # -> 10
puts args.int("v=",,20,30) # -> 範囲エラー
puts args.int("w=") # -> nil
puts args.int("w=",15) # -> 15
\end{Verbatim}

%%%%%%%%%%%%%%%%%%
{\Large
\begin{verbatim}
* MCMD::Margs.bool(keyWord) Bool型引数の取得
\end{verbatim}
}

\begin{description}
\item {\large \verb/keyWord /} "-key"によるキーワード(String)
\end{description}

コマンドライン上で指定された引数のうち、\verb/keyWord/とマッチする引数があればtrueを、なければfalseを返す。

\begin{Verbatim}[baselinestretch=0.7,frame=single]
# コマンドライン
$ ruby test.rb -flag

# test.rbの内容
args=Margs.new(ARGV)
puts args.bool("-flag") # -> true
puts args.bool("-x") # -> false
\end{Verbatim}


%\subsection{備考}
%\begin{itemize}
%\item 
%\end{itemize}

\subsection{利用例}
\subsubsection*{例1}


\begin{Verbatim}[baselinestretch=0.7,frame=single]
# コマンドライン
$ ruby test.rb i=dat.csv v=value -abc

# test.rbの内容
args=Margs.new(ARGV, "i=,o=,w=,-flag,-x", "i=,w=")
iFileName = args.file("i=") # -> "dat.csv"
oFileName = args.str("o=","result.csv") # -> "result.csv"
weight    = args.float("w=",0.1,0.0,1.0) # -> 0.1
flag      = args.bool("-abc") # -> true
wFlag     = args.bool("-w") # -> false
\end{Verbatim}

\subsection{関連コマンド}

%\hyperlink{mtableRB.pdf}{Table} : 全データのメモリ読み込み
%\hypertarget{ht}
%\hyperlink{ht}{bbb}

%\end{document}

