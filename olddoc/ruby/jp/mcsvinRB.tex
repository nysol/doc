
%\documentclass[a4paper]{jarticle}
%\begin{document}

\section{Mcsvin CSVの読み込みクラス\label{sect:mcsvinRB}}
CSVデータファイルを行単位で処理するためのクラス。以下のような特徴を持つ。
\begin{itemize}
\setlength{\itemindent}{0mm}
\item C++で実装されており高速に動作する。
\item 一行目が項目名行であれば、項目名をkeyとするHashにデータを格納できる。
\item データの格納はHash/Arrayを選択可能(Arrayの方が2倍高速)。
\item キーブレイク処理を容易に扱える。
\item RFC4180にほぼ準拠
%\footnote{CSVの仕様についての詳しい情報は
%http://kgmod.jp/mcmd/の「MCMDが扱うデータ構造について」を参照。}(カンマやダブルクォーテーションのエスケープなど)。
\item 一行の項目数は固定であることを前提とする。
\end{itemize}

\subsection{メソッド}

{\Large
\begin{verbatim}
* MCMD::Mcsvin::new(arguments){block}
\end{verbatim}
}

Mcsvinオブジェクトを生成する。
\verb|arguments|に、以下の引数をスペースで区切った文字列として指定する。

\begin{table}[htbp]
%{\small
\begin{tabular}{ll}
\verb|i=|    & 入力ファイル名(String) 【必須】\\
\verb|k=|    & キーブレイクを検知する項目リスト。複数項目はカンマで区切る。\\
             & キー指定の有無によってeachメソッドのyield引数が異なることに注意する。\\
\verb|-nfn|  & 1行目を項目名と見なさない。\\
\verb|-array|& eachメソッドで各データ項目をArrayに格納する。\\
             & 指定がなければHashに格納する。\\
             & ArrayはHashに比べ、約2倍効率的である(詳細は「ベンチマーク」を参照)。\\
\verb|block| & ブロックが指定されていれば実行(yield)する。\\
\end{tabular} 
%}
\end{table} 

{\Large
\begin{verbatim}
* MCMD::Mcsvin::each{|val| block}
* MCMD::Mcsvin::each{|val,top,bot| block}
\end{verbatim}
}

CSVファイルを一行ずつ処理する。
1)の書式はキー(k=)を指定しなかった場合で、valに値が設定される。
2)の書式はキーを指定した場合で、val以外にもキーブレイク情報top,botの変数も設定される。
\begin{description}
\setlength{\itemindent}{0mm}
\item[val ] 項目名をキーとして値を格納したHash(もしくはArray)。値は全てString型で格納。
\item[top ] k=で指定したキーの先頭であればtrue、さもなければfalseがセットされる。詳細は備考を参照。
\item[bot ] k=で指定したキーの終端であればtrue、さもなければfalseがセットされる。詳細は備考を参照。
\end{description}

%\begin{verbatim}
%* Mcsvin::each_array{|val,top,bot| block}
%\end{verbatim}
%
%Mcsvin::eachと同じ機能で、valがHashではなく配列に格納される。

{\Large
\begin{verbatim}
* MCMD::Mcsvin::names()
\end{verbatim}
}

項目名配列を返す。\verb|-nfn|が指定されていればnilを返す。

\subsection{備考}
\begin{itemize}
\item \verb|-nfn|が指定された場合、データはArrayに格納される。Hashでは格納できない。
\item \verb|k=|を指定する場合は、そこで指定した項目で並べ替えておかなければならない。
\item キーブレイクのロジック:
\begin{verbatim}
MCMD::Mcsvin.new("i=input.dat k=key"){|csv|
	csv.each{|val,top,bot|
    :
  }
}
\end{verbatim}
上記のコードにおいて、bool型のブロック変数topおよびbotの設定ロジックは以下のとおり。\\
データ行を$i=1,2,\cdots,n$、$i$行目のキー項目(``key'')の値を$k_i$とし、便宜上$k_0=k_{n+1}=\phi$とする。ただし、$k_i(1\le i \le n) \ne \phi$である。

\begin{eqnarray}
\verb/top/=\left\{
\begin{array}{ll}
true,  & {\rm if}\hspace{2mm} k_i \ne k_{i-1} \\
false, & {\rm otherwise}\\
\end{array} \right.
\label{eq:retDisc}
\end{eqnarray}

\begin{eqnarray}
\verb/bot/=\left\{
\begin{array}{ll}
true,  & {\rm if}\hspace{2mm} k_i \ne k_{i+1} \\
false, & {\rm otherwise}\\
\end{array} \right.
\label{eq:retDisc}
\end{eqnarray}

\end{itemize}

\subsection{利用例}
\subsubsection*{例1 項目名の出力と行番号・値の出力}

\begin{Verbatim}[baselinestretch=0.7,frame=single]
# dat1.csv
顧客,日付,金額
A,20081201,10
B,20081002,40

MCMD::Mcsvin.new("i=dat1.csv"){|csv|
  puts csv.names.join(",")
  csv.each{|val|
    p val
  }
}
# 出力結果
顧客,日付,金額
["顧客"=>"A", "日付"=>"20081201", "金額"=>"10"]
["顧客"=>"B", "日付"=>"20081002", "金額"=>"40"]
\end{Verbatim}

\subsubsection*{例2 キーブレイク処理}

\begin{Verbatim}[baselinestretch=0.7,frame=single]
# dat1.csv
顧客,日付
A,20081201
A,20081202
B,20081003
C,20081004
C,20081005
C,20081006

csv=MCMD::Mcsvin.new("i=dat1.csv k=顧客")
csv.each{|val,top,bot|
  puts "#{val['顧客']},#{val['日付']} top=#{top} bot=#{bot}"
}
csv.close

# 出力結果
A,20081201 top=true bot=false
A,20081202 top=false bot=true
B,20081003 top=true bot=true
C,20081004 top=true bot=false
C,20081005 top=false bot=false
C,20081006 top=false bot=true
\end{Verbatim}

\subsubsection*{例3 項目名行のないデータの処理}
\verb|-nfn|を指定するとArrayに格納される。
\begin{Verbatim}[baselinestretch=0.7,frame=single]
# dat1.csv
A,20081201
A,20081202

MCMD::Mcsvin.new("i=dat1.csv k=1 -nfn"){|csv|
  puts csv.names # -> nil
  csv.each{|val|
    p val
  }
}

# 出力結果
nil
["A","20081201"]
["A","20081202"]
\end{Verbatim}

\subsubsection*{例4 Arrayに格納する例}

\begin{Verbatim}[baselinestretch=0.7,frame=single]
# dat1.csv
顧客,日付,金額
A,20081201,10
B,20081002,40

# -arrayオプションでArray格納
MCMD::Mcsvin.new("i=dat1.csv -array"){|csv|
  puts csv.names.join(",")
  csv.each{|val|
    p val
  }
}

# 出力結果
顧客,日付,金額
["A", "20081201", "10"]
["B", "20081002", "40"]
\end{Verbatim}

\subsection*{関連コマンド}
\hyperref[sect:mcsvoutRB]{Mcsvout} : CSVデータの書き込み

\hyperref[sect:mtableRB]{Mtable} : セル単位読み込み操作

\subsection{ベンチマークテスト}

CSVデータの読み込み処理について、Rubyの拡張ライブラリとして提供されている
各種ライブラリをベンチマークとして処理速度の比較を行う。
ベンチマーク対象は以下の4つのライブラリと1つのコマンドである。

\begin{description}
\setlength{\itemindent}{0mm}
\item[CSVScan] http://raa.ruby-lang.org/project/csvscan/
\item[LightCsv] http://tmtm.org/ruby/lightcsv/
\item[FasterCSV] http://www.gesource.jp/programming/ruby/database/fastercsv.html
\item[CSV] http://www.ruby-lang.org/ja/old-man/html/CSV.html
\item[mcut] 項目を切り出すMコマンド(全てC++で実装)。参考までに掲載。
\end{description}

表\ref{tb:mcsvinRB_bench1}にベンチマークテストの結果を示す。
1万行〜500万行のデータについて読み込み実験を行った。
図\ref{fig:mcsvinRB_script}には、ベンチマークテストで利用したスクリプトの抜粋が示されている。
McsvinはCSVScan(Cによる実装)とほぼ同等性能である。
その他はいずれもRubyネイティブコードによる実装なのでその差があらわれていることがわかる。
ただし、mcutとの比較においては、Mcsvinも到底及ばない。
mcutとMcsvinで採用しているCSVのparsingロジックおよびその実装は全く同じであるので、
データをArrayに格納するなどのRubyとのインターフェースにまつわるコストによって、
ここまでの差が出ていることになる。図\ref{fig:mcsvinRB_script}に、ベンチマークテストで利用したスクリプトの抜粋を示す。
\begin{table}[htpb]
\begin{center}
\caption{各種CSVライブラリの実行速度比較(単位:秒)\label{tb:mcsvinRB_bench1}}
%{\scriptsize 
\begin{tabular}{l|r|r|r|r|r|r|r}
\hline
行数      & 10K   & 100K  & 1000K & 2000K& 3000K & 4000K & 5000K\\ \hline
Mcsvin    & 0.020 & 0.196 & 1.76  & 3.51 & 5.26  & 7.02  & 8.79 \\
CSVScan   & 0.021 & 0.187 & 1.83  & 3.67 & 5.50  & 7.33  & 9.17 \\
LightCsv  & 0.155 & 1.62  & 15.99 & --   &    -- & --    & --   \\
FasterCSV & 0.196 & 1.96  & 19.50 & --   &    -- & --    & --   \\
CSV       & 1.44  & 14.3  &  --   & --   &    -- & --    & --   \\
mcut      & --    & --    & 0.095 & 0.177& 0.260 & 0.342 & 0.423\\
\hline
\end{tabular}  
{\small
\\10回実行した結果の平均値(real time)を示している。\\
-- は値が大き(小さ)過ぎるために計測していないことを意味する。\\
行数10Kは10000行の意味。データのサイズは行数が1000Kで約25Mバイト。項目数は5つ。\\
バージョン: CSVScan 0.0.20070920, FasterCSV 1.5.1, LightCsv 0.2.2  CSV(Ruby 1.8.7)\\
テスト環境: Mac Book Pro, 2.66GHz Intel Core i7, 8GB メモリ, Mac OS X 10.6.8
}
%}
\end{center}
\end{table}  

\begin{figure}[htpb]
\centering
\begin{minipage}{0.8\textwidth}
{\small
\begin{Verbatim}[baselinestretch=0.7,frame=single]
require 'rubygems'
require 'csv'
require 'fastercsv'
require 'lightcsv'
require 'csvscan'
require 'mcmd'

require 'benchmark'

puts Benchmark.measure{
  (0...10).each{|i|
    # Mcsvinの場合
    csv=MCMD::Mcsvin.new("i=data.csv -array"){|csv| csv.each{|val|}}

    # CSVScanの場合
    File.open("data.csv","r"){|fp| CSVScan.scan(fp){|row|}}

    # LightCsvの場合
    LightCsv.foreach("data.csv"){|row|}

    # FasterCSVの場合
    FasterCSV.foreach("data.csv"){|row|}

    # CSVの場合
    CSV.open("data.csv", 'r'){|row|}
  }
}
\end{Verbatim}
}
\end{minipage}
\caption{ベンチマークテストのスクリプト(抜粋)\label{fig:mcsvinRB_script}}
\end{figure}

次に、Mcsvinにおいて、キー指定の有無およびデータを格納する型による実行時間の差について見てみる(表\ref{tb:bench2})。
キー指定の有無による速度には、さほど大きな違いはないが、データの格納型については、
ArrayがHashより2倍ほど効率的である。

\begin{table}[htpb]
\begin{center}
\caption{キーの有無とデータ格納型による実行速度比較(単位:秒)\label{tb:bench2}}
%{\scriptsize 
\begin{tabular}{c|c||r|r|r|r|r}
\hline
キー & 型     & 1000K & 2000K& 3000K & 4000K & 5000K\\ \hline\hline
なし & Array  & 1.76  & 3.51 & 5.26  & 7.02  & 8.79 \\
なし & Hash   & 3.52  & 6.99 & 10.50 & 14.00 & 17.52\\
あり & Array  & 1.97  & 3.92 & 5.88  & 7.84  & 9.83 \\
あり & Hash   & 3.68  & 7.34 & 11.01 & 14.73 & 18.34\\
\hline
\end{tabular}  
{\small
\\キーのサイズは平均10行程度。\\
}
%}
\end{center}
\end{table}  

%\end{document}

