
\section{概要}
本ライブラリは、Ruby上でCSVデータを効率的に扱うための関数を提供する。
主な関数として、CSVデータのシーケンシャルな読み込み関数(Mcsvin)、
および書き込み関数(Mcsvout)、
そしてCSVデータを表とみなしたセル単位でのランダムアクセスを可能とする
関数(Mtable)がある。
いずれの関数もCSVの標準仕様であるRFC4180に概ね準拠しており
カンマや改行を含む文字列を扱うことができる。
データへのアクセスは、項目名をキーとするHashでも可能であるし、
項目番号によるArrayでも可能である。
さらに同等の機能を有する他のライブラリよりも効率的である。
図\ref{fig:mcmdRB_cp}に、CSVデータをコピーする
サンプルスクリプトを示す。

\begin{figure}[!hbt]
\begin{center}

\begin{Verbatim}[baselinestretch=0.7,frame=single]
#!/usr/bin/env ruby
require "mcmd"

# input.csv
# customer,date
# A,20081201
# A,20081202
MCMD::Mcsvin.new("i=input.csv -array"){|iCSV|
  MCMD::Mcsvout.new("o=output.csv f=customer,date"){|oCSV|
    iCSV.each{|flds|
      oCsv.write(flds)
    }
  }
}
\end{Verbatim}
\end{center}
\caption{mcmdライブラリを用いCSVファイルをコピーするRubyスクリプト\label{fig:mcmdRB_cp}}
\end{figure}

CSVデータの操作以外にも、
NYSOLで提供されるコマンドのいくつかは、
本ライブラリを利用したRubyスクリプトにより実装されている。
そこにmcmdと同等のインターフェースや基本機能を持たせるための
関数をいくつか用意している。
インターフェースとしてはコマンドライン引数を処理するためのMargs、
そして完了メッセージやエラーメッセージの出力を行う
endMsg、errorMsgなどの関数がある。
また基本機能としては、作業ファイル名の自動生成/削除を行うMtemp、
そしてmcmdと同じ環境変数の設定も可能である。

なお、Rubyのバージョンは1.9.2以降のみ動作確認している。

