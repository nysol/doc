
%\documentclass[a4paper]{jarticle}
%\begin{document}

\section{MCMDモジュール環境設定\label{sect:evnRB}}
MCMDモジュールは、KGMODライブラリを利用して構築されているので、
KGMODで設定可能なシェル変数は全て利用することができる。
以下では、Ruby MCMDモジュールを利用する中での設定方法について解説する。

\subsection{メッセージ出力}

環境変数\verb|KG_VerboseLevel|を設定することで\verb|Mcsvin|などのメソッドが標準エラー出力に出力する
メッセージを制御できる。
設定値とその内容は以下の通りである。

\begin{table}[htpb]
\begin{center}
\caption{メッセージを制御する環境変数の設定値とその内容\label{tb:bench1}}
%{\scriptsize 
\begin{tabular}{c|l}
\hline
値  & 内容\\ \hline
0 & メッセージを一切出力しない \\
1 & + errorメッセージ出力 \\
2 & + warningメッセージ出力 \\
3 & + endメッセージ出力 \\
4 & + msgメッセージ出力(デフォルト) \\
\hline
\end{tabular}  
%}
\end{center}
\end{table}  

\begin{Verbatim}[baselinestretch=0.7,frame=single]
$ irb
> require 'mcmd'

# デフォルトではKG_Verbose=4なので、正常終了時のメッセージもエラー終了メッセージも出力される。
> MCMD::Mcsvin.new("i=dat.csv"){|csv| csv.each{|flds|}}
#END# mcsvin i=dat.csv; ; 2013/08/08 15:18:52
> MCMD::Mcsvin.new("x=dat.csv"){|csv| csv.each{|flds|}}
#ERROR# unknown parameter x= (mcsvin); mcsvin x=dat.csv; ; 2013/08/08 15:18:52

# KG_Verbose=1とすると、エラー終了メッセージは表示されるが、正常終了メッセージは表示されない。
> ENV["KG_VerboseLevel"] = "1"
> MCMD::Mcsvin.new("i=dat.csv"){|csv| csv.each{|flds|}}
> MCMD::Mcsvin.new("x=dat.csv"){|csv| csv.each{|flds|}}
#ERROR# unknown parameter x= (mcsvin); mcsvin x=dat.csv; ; 2013/08/08 15:18:52

# KG_Verbose=0とすると、いずれのメッセージも表示されない。
> ENV["KG_VerboseLevel"] = "0"
> MCMD::Mcsvin.new("i=dat.csv"){|csv| csv.each{|flds|}}
> MCMD::Mcsvin.new("x=dat.csv"){|csv| csv.each{|flds|}}
\end{Verbatim}

\end{document}

\subsection{その他の変数}
メッセージの制御以外にもKGMODで設定できる環境変数は全て利用可能である。
詳細はKGMODの環境変数のドキュメントを参照されたい。


