
% how to compile: platex xxx.tex ; dvipdfmx xxx.dvi

\documentclass[a4paper]{jarticle}

%--余白の設定
\setlength{\topmargin}{20mm}
\addtolength{\topmargin}{-1in}
\setlength{\oddsidemargin}{20mm}
\addtolength{\oddsidemargin}{-1in}
\setlength{\evensidemargin}{15mm}
\addtolength{\evensidemargin}{-1in}
\setlength{\textwidth}{170mm}
\setlength{\textheight}{254mm}
\setlength{\headsep}{0mm}
\setlength{\headheight}{0mm}
\setlength{\topskip}{0mm}

%--ハイバーリンクを可能にするパッケージ
\usepackage[dvipdfmx,%
 bookmarks=true,%
 bookmarksnumbered=true,%
 colorlinks=true,%
 setpagesize=false,%
 pdftitle={Margs},%
 pdfauthor={BMRC},%
 pdfkeywords={TeX; dvipdfmx; hyperref; color;}]{hyperref}

\begin{document}

\setlength{\baselineskip}{4.7mm}

\section*{MCMD::Mhistory 処理履歴生成クラス}
Rubyスクリプトの各種履歴(処理履歴、オブジェクト、ファイル)を記録(ファイル出力)するクラスで以下のような特徴を持つ。
\begin{itemize}
	\setlength{\itemindent}{-5mm}
	\item クラスのメソッドが実行される度に、そのログを出力する。
	\item 入れ子関係にある実行はツリー形式でログが出力される。
	\item メッセージの出力とそのログへの出力が可能。
	\item オブジェクトの内容を出力する。
\end{itemize}

\subsection*{メソッド}

%%%%%%%%%%%%%%%%%%
{\large\color{blue}
\begin{verbatim}
Mtemp.new(path=nil): コンストラクタ
\end{verbatim}
}
Mhistoryオブジェクトを生成する。

\begin{description}
	\setlength{\itemindent}{-5mm}
	\item {\large \verb/path /} 一時ファイル名を作成するパス名。
一時ファイル名の命名規則は以下の通り。

\verb|"#{@path}/__MTEMP_#{@pid}_#{@oid}_#{@seq}"|
\begin{description}
	\setlength{\itemindent}{+5mm}
	\item [@pid] プロセスID (\$\$)
	\item [@oid] オブジェクトID (self.object\_id)
	\item [@seq] オブジェクト内の通し番号 (自動採番で1から始まる)
	\item [@path] 一時ファイルを作成するパス名で以下の優先順位で決まる。
\end{description}

\begin{enumerate}
	\setlength{\itemindent}{+15mm}
	\item Mtemp.newの第1引数で指定された値*
	\item KG\_TmpPath環境変数の値
	\item TMP環境変数の値
	\item TEMP環境変数の値
	\item "/tmp"
	\item "\." (カレントパス)
\end{enumerate}

注*) 第1引数でパス名を明示的に指定した場合、GC時に自動削除されない。
\end{description}

%%%%%%%%%%%%%%%%%%
{\large\color{blue}
\begin{verbatim}
Mtemp.file(name=nil): 一時ファイル名の取得
\end{verbatim}
}
\begin{description}
	\setlength{\itemindent}{-5mm}
	\item {\large \verb/name/} 一時ファイル名を直接指定する。通常はこのパラメータを指定する必要はない。
\verb/name/が指定されると、\verb/new/メソッドにて解説した一時ファイル名生成規則とは全く関係なく
\verb/name/の内容を一時ファイルとして利用する。
また\verb|name|を指定して作成されたファイルはGC時に削除されない。
デバッグでに利用する目的で用意したパラメータである。
\end{description}

%%%%%%%%%%%%%%%%%%
{\large\color{blue}
\begin{verbatim}
Mtemp.pipe(name=nil): 名前付きパイプの一時ファイルの取得
\end{verbatim}
}
\verb|file|メソッドとは違い、このメソッドが実行されると\verb|mkfifo|コマンドによってfifoファイルが作成される。

\begin{description}
	\setlength{\itemindent}{-5mm}
	\item {\large \verb/name/} \verb|file|メソッドと同様に、一時ファイル名を直接指定する。
通常はこのパラメータを指定する必要はない。詳細は\verb|file|メソッドを参照のこと。
\end{description}

%%%%%%%%%%%%%%%%%%
{\large\color{blue}
\begin{verbatim}
Mtemp.rm(): 一時ファイルを削除する。
\end{verbatim}
}

system関数から以下のコマンドを実行することで一時ファイルを全て削除する。
\verb |rm -rf \#{path}/\_\_MTEMP\_\#{@pid}\_\#{@oid}\_*|

%\subsection*{備考}
%\begin{itemize}
%\item 
%\end{itemize}

\subsection*{利用例}

\subsubsection*{例1 基本利用例}

\begin{verbatim}
------------------------------------------------------
require 'mtools'

tmp=MCMD::Mtemp.new
fName1=tmp.file
fName2=tmp.file
fName3=tmp.file("./xxa")
puts fName1 # -> /tmp/__MTEMP_60637_2152301760_0
puts fName2 # -> /tmp/__MTEMP_60637_2152301760_1
puts fName3 # -> ./xxa
File.open(fName1,"w"){|fp| fp.puts "temp1"}
File.open(fName2,"w"){|fp| fp.puts "temp2"}
File.open(fName3,"w"){|fp| fp.puts "temp3"}
# tmpがローカルのスコープを外れると
# GCが発動するタイミングで一時ファイルも自動的に削除される。
# ただし、fName3は一時ファイル名を直接指定しているの削除されない。
------------------------------------------------------
\end{verbatim}

\subsubsection*{例2 全ての一時ファイルが自動削除されない例}
\begin{verbatim}
------------------------------------------------------
require 'mtools'

# コンストラクタでパスを指定すると自動削除されない。
# (rmメソッドにより削除することはできる。)
tmp=MCMD::Mtemp.new(".")
fName=tmp.file
File.open(fName,"w"){|fp| fp.puts "temp"}
# tmpがローカルのスコープを外れGCが発動しても
# 一時ファイルは削除されない。
------------------------------------------------------
\end{verbatim}

\subsubsection*{例3 名前付きパイプの利用}
\begin{verbatim}
------------------------------------------------------
require 'mtools'

tmp=MCMD::Mtemp.new
pName=tmp.pipe
system("ls -l #{pName}") # この段階で名前付きパイプファイルが作成されている。
# -> prw-r--r--  1 user  group  0  7 19 12:20 /tmp/__MTEMP_60903_2152299720_0
system("echo 'abc' > #{pName} &") # バックグラウンド実行でnamed pipeに書き込み
system("cat <#{pName} &")         # バックグラウンド実行でnamed pipeから読み込み
# tmpがローカルのスコープを外れると
# GCが発動するタイミングで全ての一時ファイルは自動削除される。
------------------------------------------------------
\end{verbatim}

\subsection*{関連コマンド}

%\hyperlink{mtableRB.pdf}{Table} : 全データのメモリ読み込み
%\hypertarget{ht}
%\hyperlink{ht}{bbb}

\end{document}

