
%\documentclass[a4paper]{jarticle}
%\begin{document}

\section{Mcsvout CSVの書き込みクラス\label{sect:mcsvoutRB}}
CSVデータファイルに出力するためのクラス。
以下のような特徴を持つ。
\begin{itemize}
\setlength{\itemindent}{0mm}
\item C++で実装されており高速に動作する。
\item 項目名行あり/なし、いずれの形式も扱うことができる。
\item RFC4180にほぼ準拠
%\footnote{CSVの仕様についての詳しい情報は
%http://nysol.jp/mcmd/の「MCMDが扱うデータ構造について」を参照。}(カンマやダブルクォーテーションのエスケープなど)。
\item 一行の項目数は固定であることを前提とする。
\end{itemize}


\subsection{メソッド}

{\Large
\begin{verbatim}
* MCMD::Mcsvout::new(arguments){block}
\end{verbatim}
}

Mcsvoutオブジェクトを生成する。
\verb|arguments|に、以下の引数をスペースで区切った文字列として指定する。

\begin{table}[htbp]
\begin{tabular}{ll}
\verb|o=|    & 出力ファイル名(String)\\
\verb|f=|    & 出力するCSVデータのヘッダー(一行目)の項目名文字列を配列で指定する(String Array)。\\
             & \verb|f=|を省略して\verb/size=/を指定すれば、項目名なしのCSVを出力する。\\
\verb|size=| & 項目名を出力しない時には、CSV項目の数(Fixnum)を指定する。\\
\verb|precision= | & Float型変数の有効桁数を指定する。デフォルトは10桁。 \\
                   & C言語の出力書式\verb/"%.ng"/の\verb|n|の値。\\
                   & 100/3に対して有効桁数5桁であれば33.333、2桁であれば33となる。\\
\verb|bool=| & trueとfalseの出力値をカンマで区切って指定する。デフォルトは"1,0"\\
\end{tabular} 
\end{table} 

{\Large
\begin{verbatim}
* MCMD::Mcsvout::write(values)
\end{verbatim}
}

\begin{description}
\setlength{\itemindent}{0mm}
\item[values ] 配列(Array)に格納された値をCSVデータとして出力する。
配列に格納できるデータクラスはString, Fixnum, Bignum, Float, nil, true, falseである。
それ以外のクラスは全てnilとして扱われる。
配列のサイズが項目名の数より少い場合はnull値が追加出力される。
配列のサイズが項目名の数より多い場合は、超過分は出力されない。
\end{description}

\subsection{備考}
\begin{itemize}
\item 文字列にカンマが含まれていれば、その値は自動的にダブルクオーテーションで囲われ出力される。
文字列にダブルクオーテーションが含まれていれば、連続する二つのダブルクオーテーションに変換される。
\end{itemize}

\subsection{利用例}
\subsubsection*{例1 項目名行ありのCSVデータ出力例}

\begin{Verbatim}[baselinestretch=0.7,frame=single]
csv=MCMD::Mcsvout.new("o=rsl.csv f=a,b,c"){|csv|
  csv.write(["1",2,3.4])
  csv.write([1,2,3,4,5])
  csv.write([1,2])
}

# 出力結果(rsl.csv)
a,b,c
1,2,3.4
1,2,3
1,2,
\end{Verbatim}

\subsubsection*{例2 項目名行なしのCSVデータ出力例}

\begin{Verbatim}[baselinestretch=0.7,frame=single]
csv=MCMD::Mcsvout.new("o=rsl.csv size=3"){|csv|
  csv.write(["1",2,3.4])
  csv.write([true,nil,false])
  csv.write(["4\"5","","6,7"])
}

# 出力結果(rsl.csv)
1,2,3.4
1,,0
"4""5",,"6,7"
\end{Verbatim}

\subsubsection*{例3 オプション指定(有効桁数,bool値)}

\begin{Verbatim}[baselinestretch=0.7,frame=single]
MCMD::Mcsvout.new("o=rsl.csv size=3 precision=3 bools=T,F"){|csv|
  csv.write([0.123456,123456.0]) # 有効桁数の指定は小数点以下の桁数でないことに注意する
  csv.write([123456,0])  # 有効桁数の指定はFixnumには影響しない
  csv.write([true,false])
}

# 出力結果(rsl.csv)
0.123,1.23e+05
123456,0
T,F
\end{Verbatim}

\subsubsection*{例4 データコピー}

\begin{Verbatim}[baselinestretch=0.7,frame=single]
# dat1.csv
顧客,日付
A,20081201
B,20081002

MCMD::Mcsvin.new("i=dat1.csv -array"){|csvIn|
  MCMD::Mcsvout.new("o=rsl.csv f=#{csvIn.names.join(",")}){|csvOut|
    csvIn.each{|val|
      csvOut.write(val)
    }
  }
}

# rsl.csv
顧客,日付
A,20081201
B,20081002
\end{Verbatim}

\subsection{ベンチマークテスト}

CSVデータの書き込み処理について、Rubyの拡張ライブラリとして提供されている
各種ライブラリをベンチマークとして処理速度の比較を行う。
ベンチマーク対象は以下の2つのライブラリである。

\begin{description}
\setlength{\itemindent}{0mm}
\item[FasterCSV] http://www.gesource.jp/programming/ruby/database/fastercsv.html
\item[CSV] http://www.ruby-lang.org/ja/old-man/html/CSV.html
\end{description}

表\ref{tb:mcsvoutRB_bench1}にベンチマークテストの結果を示す。
1万行,10万行,100万行のデータについて書き込み実験を行った。
ただし、実ファイルとしては出力せず、nullデバイス(/dev/null)への書き込みとした。
図\ref{fig:mcsvoutRB_script}には、ベンチマークテストで利用したスクリプトの抜粋が示されている。
McsvoutはC++による実装のため、他の二つのライブラリより高速である。
Rubyネイティブコードによる実装との差があらわれているのであろう。

\begin{table}[htpb]
\begin{center}
\caption{各種CSVライブラリの実行速度比較(単位:秒)\label{tb:mcsvoutRB_bench1}}
%{\scriptsize 
\begin{tabular}{l|r|r|r}
\hline
行数      & 10K    & 100K  & 1000K \\ \hline
Mcsvout   & 0.0158 & 0.150 & 1.50  \\
FasterCSV & 0.232  & 1.90  & 20.0  \\
CSV       & 0.279  & 2.80  & 27.9  \\
\hline
\end{tabular}  
{\small
\\10回実行した結果の平均値(real time)を示している。\\
行数10Kは10000行の意味。String, Fixnum, Float, true, false, nilの6つの値を出力。\\
バージョン: FasterCSV 1.5.1 CSV(Ruby 1.8.7) \\
テスト環境: Mac Book Pro, 2.66GHz Intel Core i7, 8GB メモリ, Mac OS X 10.6.8
}
%}
\end{center}
\end{table}  

\begin{figure}[htpb]
\centering
\begin{minipage}{.7\textwidth}
{\small
\begin{Verbatim}[baselinestretch=0.7,frame=single]
require 'rubygems'
require 'csv'
require 'fastercsv'
require 'mtools'

require 'benchmark'

$data = ["12345678", 10, 1.1, true, nil, false]

puts Benchmark.measure{
  (0...10).each{|i|
    # Mcsvoutの場合
    MCMD::Mcsvout.new("o=/dev/null size=6){|csv|
      (0...10000).each{|j|
        csv.write($data)
      }
    }

    # FasterCSVの場合
    FasterCSV.open("/dev/null", 'w'){|csv|
      (0...10000).each{|j|
        csv << $data
      }
    }

    # CSVの場合
    CSV.open("/dev/null", 'w'){|csv|
      (0...10000).each{|j|
        csv << $data
      }
    }
  }
}
\end{Verbatim}
}
\end{minipage}
\caption{ベンチマークテストのスクリプト(抜粋)\label{fig:mcsvoutRB_script}}
\end{figure}


\subsection{関連コマンド}

\hyperref[sect:mcsvinRB]{Mcsvin} : CSVデータの読み込み
%\hypertarget{ht}
%\hyperlink{ht}{bbb}

