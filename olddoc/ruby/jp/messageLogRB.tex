
%\documentclass[a4paper]{jarticle}
%\begin{document}

\section{messageLog 終了ログメッセージの出力\label{sect:messageLog}}
MCMDの一般メッセージと同様のフォーマットのログメッセージを出力する。
フォーマットは以下の通りである。

\begin{Verbatim}[baselinestretch=0.7,frame=single]
#MSG# メッセージ; 日時
\end{Verbatim}

\subsection{書式}

{\Large
\begin{verbatim}
* MCMD::messageLog(msg[,fileObject])
\end{verbatim}
} 

\begin{description}
	\setlength{\itemindent}{-5mm}
	\item {\large \verb/msg /} 表示するメッセージ。
	\item {\large \verb/fileObject /} 出力するファイルオブジェクト。省略すれば標準エラー出力(STDERR)に出力される。
\end{description}

\subsection{利用例}
\subsubsection*{例1 基本例}

\begin{Verbatim}[baselinestretch=0.7,frame=single]
# 標準エラー出力に一般メッセージを表示する。
> MCMD::messageLog("mburst.rb これはメッセージです。")
#END# mburst.rb これはメッセージです。; 2013/11/01 19:09:50
\end{Verbatim}

\subsection{関連コマンド}
\hyperref[sect:errorLogRB]{errorLog} : エラーログメッセージの出力

\hyperref[sect:endRB]{endLog} : 終了ログメッセージの出力

%\end{document}

