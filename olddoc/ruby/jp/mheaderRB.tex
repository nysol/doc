
%\documentclass[a4paper]{jarticle}
%\begin{document}

\section{mheader CSVデータの項目名配列取得メソッド\label{sect:mheaderRB}}
CSVデータファイルの項目名(先頭行)を配列で返す。
先頭行がなくデータ行から始まるファイルについては、
1行目の各項目の値を配列で返す。

\subsection{書式}

{\large
\begin{verbatim}
MCMD::mheader(arguments)
\end{verbatim}
} 

\verb|arguments|に、以下の引数をスペースで区切った文字列として指定する。
\begin{table}[htbp]
\begin{tabular}{ll}
\verb|i=|    & 入力ファイル名(String)\\
\end{tabular} 
\end{table} 

\subsection{利用例}
\subsubsection*{例1}

\begin{Verbatim}[baselinestretch=0.7,frame=single]
# dat1.csv
# 顧客,日付,金額
# A,20081201,10
# B,20081002,40

> p MCMD::mheader("i=dat1.csv")
 => ["顧客", "日付", "金額"] 
\end{Verbatim}

\subsubsection*{例2}

\begin{Verbatim}[baselinestretch=0.7,frame=single]
# dat1.csv
# A,20081201,10
# B,20081002,40

> p MCMD::mheader("i=dat1.csv")
 => ["A", "20081201", "10"] 
\end{Verbatim}

\subsection{関連コマンド}
\hyperref[sect:mcsvinRB]{Mcsvin} : CSVデータの読み込みクラス

\hyperref[sect:mtableRB]{Mtable} : CSVデータのセル単位での読み込みクラス

%\end{document}

