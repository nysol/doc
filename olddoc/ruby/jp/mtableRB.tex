
%\documentclass[a4paper]{jarticle}
%\begin{document}

\section{Mtable CSVデータのセル単位での読み込み操作クラス\label{sect:mtableRB}}
指定したCSVデータ全体をメモリに読み込み、セル単位でランダムアクセスを可能としたクラス。
以下のような特徴を持つ。
\begin{itemize}
\setlength{\itemindent}{0mm}
\item 行と列の指定によりセルをランダムにアクセス可能。
\item 読み込み専用であり、データの更新や追加は一切できない。
\item データは全て文字列として読み込むので、その他の方で利用する時は適宜型変換(ex. to\_i)が必要。
\item メモリが空いている限りデータを読み込む。領域がなくなればエラー終了する。
\end{itemize}

\subsection{メソッド}

{\Large
\begin{verbatim}
* MCMD::Mtable::new(arguments)
\end{verbatim}
}

Mtableオブジェクトを生成する。
\verb|arguments|に、以下の引数をスペースで区切った文字列として指定する。

\begin{table}[htbp]
\begin{tabular}{ll}
\verb|i=|    & 入力ファイル名(String)\\
\verb|-nfn|  & 1行目を項目名と見なさない。\\
\end{tabular} 
\end{table} 

{\Large
\begin{verbatim}
* MCMD::Mtable::cell(col=0, row=0) -> String
\end{verbatim}
}

row(行),col(列)に対応するセルの値を返す。
row,colの与え方は、列番号と行番号による。
列/行番号共に0から始まる整数(Mcsvinの列番号は1から始まる)。
row,colが範囲外の場合はnilを返す。

\begin{description}
\setlength{\itemindent}{0mm}
\item[row ] 行番号で、0以上の整数を用いる。デフォルトは0。
\item[col ] 列番号で、0以上の整数を用いる。項目名は指定できない。デフォルトは0。
\end{description}

colのみ与えると0行目のcol番目の項目の値を返す。cell(col,0)を指定したのと同等。
またcol,row両方とも与えなければ0行目の0項目目の値を返す。cell(0,0)を指定したのと同等。

{\Large
\begin{verbatim}
* MCMD::Mtable::names() -> String Array
\end{verbatim}
}

項目名配列を返す。

{\Large
\begin{verbatim}
* MCMD::Mtable::name2num() -> String=>Fixnum Hash
\end{verbatim}
}

項目名をキー、対応する項目番号を値とするHashを返す。

{\Large
\begin{verbatim}
* MCMD::Mtable::size() -> Fixnum
\end{verbatim}
}

行数を返す。

%\subsection{備考}
%\begin{itemize}
%\item k=を指定する場合は、そこで指定した項目で並べ替えておく必要はない(内部ではHash型変数で管理するため)。
%\end{itemize}

\subsection{利用例}
\subsubsection*{例1}

\begin{Verbatim}[baselinestretch=0.7,frame=single]
# dat1.csv
customer,date,amount
A,20081201,10
B,20081002,40

tbl=MCMD::Mtable.new("i=dat1.csv")
p tbl.names     # -> ["customer", "date", "amount"]
p tbl.name2num  # -> {"amount"=>2, "date"=>1, "customer"=>0}
p tbl.size      # -> 2
p tbl.cell(0,0) # -> "A"
p tbl.cell(0,1) # -> "B"
p tbl.cell(1,1) # -> "20081202"
p tbl.cell(1)   # -> "20081201"
p tbl.cell      # -> "A"
\end{Verbatim}

\subsubsection*{例2 項目名行なし}

\begin{Verbatim}[baselinestretch=0.7,frame=single]
# dat1.csv
customer,date,amount
A,20081201,10
B,20081002,40

tbl=MCMD::Mtable.new("i=dat1.csv -nfn") # 一行目もデータと見なしてしまう。
p tbl.names     # -> nil
p tbl.name2num  # -> nil
p tbl.size      # -> 3
p tbl.cell(0,0) # -> "customer"
p tbl.cell(0,1) # -> "A"
p tbl.cell(1,1) # -> "20081201"
p tbl.cell(1)   # -> "date"
p tbl.cell      # -> "customer"
\end{Verbatim}

\subsection{関連コマンド}
\hyperref[sect:mcsvinRB]{Mcsvin} : CSVデータの読み込み

%\end{document}

