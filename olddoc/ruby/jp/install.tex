
%\begin{document}

\section{インストール\label{sect:install}}
Mコマンドの動作確認は、以下のOSで確認している。
\begin{itemize}
\item Mac OS X 10.7.5(Lion)
\item Ubuntu Linux 12.04(32bit, 64bit)
%\item Fedora Linux 14(32bit, 64bit)
\end{itemize}

いずれも、すぐにインストール可能なパッケージが用意されている。
上記のOSにおけるバージョンに違いがあっても、パッケージをインストールが可能であろう。
また他のOSであっても、ソースからのコンパイルを行うことでインストールが可能である。

\subsection{Mac OS X\label{sect:install_osx}}
\href{http://sourceforge.jp/projects/nysol/releases/}{http://sourceforge.jp/projects/nysol/releases/}より
最新のgemパッケージをダウンロードし、以下の手順にしたがってインストールする。
ただし、ファイル名は「\verb|mcmd-1.0-x86_64-darwin.gem|」の形式に従っており、\verb|"1.0"|は最新のバージョン番号に読み替える。

\begin{Verbatim}[baselinestretch=0.7,frame=single]
$ gem install mcmd-1.0-x86_64-darwin.gem
\end{Verbatim}

\subsection{Ubuntu Linux\label{sect:install_ubuntu}}
\href{http://sourceforge.jp/projects/nysol/releases/}{http://sourceforge.jp/projects/nysol/releases/}より
最新のgemパッケージをダウンロードし、以下の手順にしたがってインストールする。
ただし、ファイル名は以下に例示される形式に従っている(\verb|"1.0"|は最新のバージョン番号に読み替える)。
\begin{itemize}
\item 32bit環境: \verb|mcmd-1.0-x86-linux.gem|
\item 64bit環境: \verb|mcmd-1.0-x86_64-linux.gem|
\end{itemize}

\begin{Verbatim}[baselinestretch=0.7,frame=single]
$ gem install mcmd-1.0_darwin.gem
\end{Verbatim}

\subsection{ソースからのインストール\label{sect:install_source}}
プログラムソースからコンパイルする際は、以下の手順に従う。

\subsubsection{C++ boostライブラリのインストール}

C++ boostライブラリのページ(\url{http://www.boost.org/})
よりboostライブラリをダウンロード&インストールする\footnote{boost\_1.52.0で動作確認をとっている。また1.54.0ではコンパイルができないことは確認している}。
MacもしくはLinux 32bit OSの場合は以下の手順に従う。
boostライブラリのインストールには30分程度の時間がかかる。

\begin{Verbatim}[baselinestretch=0.7,frame=single]
$ wget http://sourceforge.net/projects/boost/files/boost/1.52.0/boost_1_52_0.tar.gz/download
$ tar zxvf boost_1_52_0.tar.gz
$ cd boost_1_52_0
$ ./bootstrap.sh
$ ./bjam
$ sudo ./bjam install
\end{Verbatim}

Linux 64bit OSの場合は以下の手順に従う。

\begin{Verbatim}[baselinestretch=0.7,frame=single]
$ wget http://sourceforge.net/projects/boost/files/boost/1.52.0/boost_1_52_0.tar.gz/download
$ tar zxvf boost_1_52_0.tar.gz
$ cd boost_1_52_0
$ ./bootstrap.sh
$ ./bjam cflags=-fPIC
$ sudo ./bjam install cflags=-fPIC
\end{Verbatim}

\subsubsection{mcmdとRubyMのインストール}
以下の手順に従い、MCMDのgitサーバより最新のMCMDのソースをダウンロード&インストールする。
「\verb|mcut --help|」と入力し、ヘルプメッセージが表示されることを確かめてインストール完了である。

\begin{Verbatim}[baselinestretch=0.7,frame=single]
$ git clone http://scm.sourceforge.jp/gitroot/nysol/mcmd.git
$ cd mcmd
$ ./configure
$ make
$ sudo make install
$ cd ruby
$ make
$ sudo make install
\end{Verbatim}


%\end{document}
