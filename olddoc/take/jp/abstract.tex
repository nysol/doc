

\section{概要}
本パッケージ「TAKE(竹)」は、宇野毅明氏(国立情報学研究所教授)を中心に開発された
データマイニングソフトウェア\cite{UnoWeb}を拡張し利用しやすくしたコマンド群である。
パッケージ名は開発者の名前の音からとったものである。
本パッケージが提供するコマンドの多くはパターン列挙を基本としたものであり、
その対象はアイテム集合、系列、一般グラフと多様である。

例えば、スーパーマーケットにおける買い物かごに入った商品をアイテム集合として考えれば、
多くの買い物かごに共通の商品の組合せを高速に列挙できる。
それだけでなく、クラスの概念を導入し、優良顧客に特徴的なパターンを列挙することも可能である。

またwebページの巡回ログから、多くのユーザに共通の巡回パターンを系列パターンとして列挙でき
webページの構成に参考となる知見が見つかるかもしれない。
系列パターンにおいても、クラスの概念を導入することが可能で、
男性と女性での巡回パターンの差を列挙することもできる。

さらに、一般グラフを扱うコマンドもいくつか用意されている。
企業の取引ネットワークデータやSNSのユーザネットワークデータ、
そしてアイテム間類似関係を表現した類似度グラフなどを扱うことができる。
例えば、商品の共起情報を類似度と考えれば、商品の類似度グラフを構築でき、
そのグラフから極大クリークを列挙することでお互いにつながりの強い商品クラスタを抽出できる。
抽出されたクラスタは、商品購買モデルにおける説明変数として利用することも可能である。
また、時に膨大に列挙される極大クリークの数を抑制し、
中規模サイズの少数の極大クリークを列挙するための前処理として
データ研磨手法も用意されている。

本パッケージが提供するコマンドの全てはruby言語によって記述されており、
内部でネイティブなコマンドをシェルインターフェースによって起動実行している。
ネイティブコマンドとのデータのやり取りは、基本的にはファイルによる。

\subsection{2014年10月6日リリースによる変更点}
\verb|mclique.rb,mpolishing.rb|コマンドで、枝ファイルの入出力を指定するパラメータのキーワードを、
\verb|i=|および\verb|o=|から\verb|ei=|および\verb|no=|に変更している。
これは節点ファイルを\verb|ni=|および\verb|no=|で指定できるようにした変更に伴うものである。
節点ファイルを指定することで、これまで列挙できなかった一つの節点から構成されるクリークも出力されるようになる。

\subsection{インストール}
本パッケージは全てnysolパッケージに含まれている。
nysolパッケージをインストールすれば必要なソフトウェアは全てインストールされる。
詳しくはnysolパッケージのインストールの説明(\url{http://www.nysol.jp/install})を参照のこと。

\subsection{ライセンス}
本パッケージには、宇野氏の開発したコマンドソースも含まれているが、
それらのライセンスは\cite{UnoWeb}で配布されているアーカイブに含まれる\verb|readme.txt|ファイルを参照のこと。
それ以外のソフトウェアはGNU AGPL(AFFERO GENERAL PUBLIC LICENSE: \url{http://www.gnu.org/licenses/agpl-3.0.html})のもと自由に利用可能である。

