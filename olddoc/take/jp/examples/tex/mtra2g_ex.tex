\subsubsection*{例1: 基本例}

出現件数が2件以上の類似度グラフ。
上述の解説の中で示した例。


\begin{Verbatim}[baselinestretch=0.7,frame=single]
$ more dat1.csv
tid,item
T1,C
T1,E
T2,D
T2,E
T2,F
T3,A
T3,B
T3,D
T3,F
T4,B
T4,D
T4,F
T5,A
T5,B
T5,D
T5,E
T6,A
T6,B
T6,D
T6,E
T6,F
$ mtra2g.rb  S=2 tid=tid item=item i=dat1.csv eo=edge1.csv
#MSG# converting a named item into a numbered item ...
#MSG# run lcm enumerating 2 itemset ...
#MSG# creating the edge file ...
#END# /usr/bin/mtra2g.rb S=2 tid=tid item=item i=dat1.csv eo=edge1.csv
$ more edge1.csv
node1,node2,support,void
A,B,0.5,
A,D,0.5,
A,E,0.3333333333,
A,F,0.3333333333,
B,D,0.6666666667,
E,B,0.3333333333,
E,D,0.5,
F,B,0.5,
F,D,0.6666666667,
F,E,0.3333333333,
\end{Verbatim}
\subsubsection*{例2: resemblanceを追加}

例1に加えてresemblanceが0.4以上を類似度条件に加える。
また\verb|no=|を指定することで、節点情報としてアイテム単独の出現頻度を出力する。


\begin{Verbatim}[baselinestretch=0.7,frame=single]
$ mtra2g.rb  S=2 sim=R th=0.4 tid=tid item=item i=dat1.csv eo=edge2.csv no=node2.csv
#MSG# converting a named item into a numbered item ...
#MSG# run lcm enumerating 2 itemset ...
#MSG# creating the edge file ...
#MSG# creating the node file ...
#END# /usr/bin/mtra2g.rb S=2 sim=R th=0.4 tid=tid item=item i=dat1.csv eo=edge2.csv no=nod
e2.csv
$ more node2.csv
node,support
A,0.5
B,0.6666666667
C,0.1666666667
D,0.8333333333
E,0.6666666667
F,0.6666666667
$ more edge2.csv
node1,node2,support,resemblance
A,B,0.5,0.75
A,D,0.5,0.6
A,E,0.3333333333,0.4
A,F,0.3333333333,0.4
B,D,0.6666666667,0.8
E,D,0.5,0.5
F,B,0.5,0.6
F,D,0.6666666667,0.8
\end{Verbatim}
