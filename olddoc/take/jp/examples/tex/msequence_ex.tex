\subsubsection*{例1: 基本例}

2件以上で出現する系列パターン。
入力データの項目名は、全てデフォルトのものと同じなので省略していることに注意する。


\begin{Verbatim}[baselinestretch=0.7,frame=single]
$ more dat1.csv
tid,time,item
T1,0,C
T1,2,B
T1,3,A
T1,7,C
T2,2,D
T2,3,A
T2,5,B
T2,6,C
T3,1,C
T3,2,B
T3,4,D
T3,8,E
T4,2,A
T4,5,C
T4,6,B
T5,0,B
T5,1,A
T5,2,D
T5,3,D
T5,7,C
T5,9,C
T6,0,A
T6,5,B
T6,6,D
T6,8,B
T6,9,C
$ msequence.rb  O=result1 i=dat1.csv S=2
#MSG# lcm_seq_20140215 CIf /tmp/__MTEMP_96779_70117174398360_0 2 /tmp/__MTEMP_96779_701171
74386400_0
trsact: /tmp/__MTEMP_96779_70117174398360_0 ,#transactions 6 ,#items 5 ,size 26 extracted 
database: #transactions 6 ,#items 4 ,size 25
 output to: /tmp/__MTEMP_96779_70117174386400_0
iters=20
20
1
4
10
5
#MSG# output tid-patterns ...
#MSG# the number of contrast patterns enumerated is 19
#MSG# The final results are in the directory `result1'
#END# /usr/bin/msequence.rb O=result1 i=dat1.csv S=2
$ more result1/patterns.csv
pid,pattern,size,count,total,support
1,C,1,6,6,1
4,B,1,6,6,1
11,A C,2,5,6,0.8333333333
10,A,1,5,6,0.8333333333
16,D,1,4,6,0.6666666667
7,B C,2,4,6,0.6666666667
12,A B,2,3,6,0.5
8,B D,2,3,6,0.5
2,C B,2,3,6,0.5
19,D C,2,3,6,0.5
3,C C,2,2,6,0.3333333333
18,D B C,3,2,6,0.3333333333
13,A B C,3,2,6,0.3333333333
14,A D,2,2,6,0.3333333333
15,A D C,3,2,6,0.3333333333
9,B D C,3,2,6,0.3333333333
17,D B,2,2,6,0.3333333333
6,B A C,3,2,6,0.3333333333
5,B A,2,2,6,0.3333333333
\end{Verbatim}
\subsubsection*{例2: パターン長の制限}

出現頻度が3以上の長さが2の系列パターンのみを列挙する。


\begin{Verbatim}[baselinestretch=0.7,frame=single]
$ msequence.rb  O=result2 i=dat1.csv S=3 l=2 u=2
#MSG# lcm_seq_20140215 CIf -l 2 -u 2 /tmp/__MTEMP_96838_70227485417240_0 3 /tmp/__MTEMP_96
838_70227485416360_0
trsact: /tmp/__MTEMP_96838_70227485417240_0 ,#transactions 6 ,#items 5 ,size 26 extracted 
database: #transactions 6 ,#items 4 ,size 25
 output to: /tmp/__MTEMP_96838_70227485416360_0
iters=11
6
0
0
6
#MSG# output tid-patterns ...
#MSG# the number of contrast patterns enumerated is 6
#MSG# The final results are in the directory `result2'
#END# /usr/bin/msequence.rb O=result2 i=dat1.csv S=3 l=2 u=2
$ more result2/patterns.csv
pid,pattern,size,count,total,support
3,A C,2,5,6,0.8333333333
1,B C,2,4,6,0.6666666667
0,C B,2,3,6,0.5
2,B D,2,3,6,0.5
4,A B,2,3,6,0.5
5,D C,2,3,6,0.5
$ more result2/tid_pats.csv
tid,pid
T1,0
T1,1
T1,3
T2,1
T2,3
T2,4
T2,5
T3,0
T3,2
T4,0
T4,3
T4,4
T5,1
T5,2
T5,3
T5,5
T6,1
T6,2
T6,3
T6,4
T6,5
\end{Verbatim}
\subsubsection*{例3: gap長とwindowサイズの指定}

出現頻度が2以上、長さが2以上の系列パターンのうち、gap長が2、windowサイズが4のパターンを列挙する。


\begin{Verbatim}[baselinestretch=0.7,frame=single]
$ msequence.rb  O=result3 i=dat1.csv S=2 l=2 gap=2 win=4
#MSG# lcm_seq_20140215 CIf -l 2 -g 2 -G 4 /tmp/__MTEMP_96898_70310565879140_0 2 /tmp/__MTE
MP_96898_70310565878100_0
trsact: /tmp/__MTEMP_96898_70310565879140_0 ,#transactions 6 ,#items 5 ,size 26 extracted 
database: #transactions 6 ,#items 5 ,size 26
 output to: /tmp/__MTEMP_96898_70310565878100_0
iters=15
10
0
0
9
1
#MSG# output tid-patterns ...
#MSG# the number of contrast patterns enumerated is 10
#MSG# The final results are in the directory `result3'
#END# /usr/bin/msequence.rb O=result3 i=dat1.csv S=2 l=2 gap=2 win=4
$ more result3/patterns.csv
pid,pattern,size,count,total,support
0,C B,2,3,6,0.5
2,B C,2,3,6,0.5
3,B D,2,3,6,0.5
4,A C,2,3,6,0.5
5,A B,2,3,6,0.5
1,B A,2,2,6,0.3333333333
6,A D,2,2,6,0.3333333333
7,D B,2,2,6,0.3333333333
8,D B C,3,2,6,0.3333333333
9,D C,2,2,6,0.3333333333
\end{Verbatim}
\subsubsection*{例4: paddingにより時刻を考慮する}

例3と同じ条件で、-paddingを指定することで、時間を考慮したgap長とwindowサイズ制約により系列パターンを列挙する。


\begin{Verbatim}[baselinestretch=0.7,frame=single]
$ msequence.rb  O=result4 i=dat1.csv S=2 l=2 gap=2 win=4 -padding
#MSG# lcm_seq_zero_20140215 CIf -l 2 -g 2 -G 4 /tmp/__MTEMP_96957_70107510988120_0 2 /tmp/
__MTEMP_96957_70107510987120_0
trsact: /tmp/__MTEMP_96957_70107510988120_0 ,#transactions 6 ,#items 6 ,size 46 extracted 
database: #transactions 6 ,#items 6 ,size 46
 output to: /tmp/__MTEMP_96957_70107510987120_0
iters=33
4
0
0
4
#MSG# output tid-patterns ...
#MSG# the number of contrast patterns enumerated is 4
#MSG# The final results are in the directory `result4'
#END# /usr/bin/msequence.rb O=result4 i=dat1.csv S=2 l=2 gap=2 win=4 -padding
$ more result4/patterns.csv
pid,pattern,size,count,total,support
0,C B,2,3,6,0.5
3,B D,2,3,6,0.5
1,B A,2,2,6,0.3333333333
2,B C,2,2,6,0.3333333333
\end{Verbatim}
\subsubsection*{例5: 顕在系列パターンの列挙\label{ex:ep1}}

例1と同じ条件で、クラス項目を指定することで顕在パターンを列挙する。


\begin{Verbatim}[baselinestretch=0.7,frame=single]
$ more dat2.csv
tid,time,item,class
T1,0,C,cls1
T1,2,B,cls1
T1,3,A,cls1
T1,7,C,cls1
T2,2,D,cls1
T2,3,A,cls1
T2,5,B,cls1
T2,6,C,cls1
T3,1,C,cls1
T3,2,B,cls1
T3,4,D,cls1
T3,8,E,cls1
T4,2,A,cls1
T4,5,C,cls1
T4,6,B,cls1
T5,0,B,cls2
T5,1,A,cls2
T5,2,D,cls2
T5,3,D,cls2
T5,7,C,cls2
T5,9,C,cls2
T6,0,A,cls2
T6,5,B,cls2
T6,6,D,cls2
T6,8,B,cls2
T6,9,C,cls2
$ msequence.rb  O=result5 i=dat2.csv S=2 class=class -padding
#MSG# lcm_seq_zero_20140215 CIA -w /tmp/__MTEMP_97018_70131049887220_1 /tmp/__MTEMP_97018_
70131049887220_0 1073741815 /tmp/__MTEMP_97018_70131049885920_0
trsact: /tmp/__MTEMP_97018_70131049887220_0 ,#transactions 6 ,#items 6 ,size 46 extracted 
database: #transactions 6 ,#items 5 ,size 45 ,weightfile /tmp/__MTEMP_97018_70131049887220
_1
 output to: /tmp/__MTEMP_97018_70131049885920_0
iters=33
9
1
4
4
#MSG# output patterns to CSV file ...
#MSG# the number of contrast patterns on class `cls1' enumerated is 8
#MSG# output tid-patterns ...
#MSG# lcm_seq_zero_20140215 CIA -w /tmp/__MTEMP_97018_70131049887220_2 /tmp/__MTEMP_97018_
70131049887220_0 2147483645 /tmp/__MTEMP_97018_70131049885920_2
trsact: /tmp/__MTEMP_97018_70131049887220_0 ,#transactions 6 ,#items 6 ,size 46 extracted 
database: #transactions 6 ,#items 5 ,size 45 ,weightfile /tmp/__MTEMP_97018_70131049887220
_2
 output to: /tmp/__MTEMP_97018_70131049885920_2
iters=36
5
0
0
3
2
#MSG# output patterns to CSV file ...
#MSG# the number of contrast patterns on class `cls2' enumerated is 5
#MSG# output tid-patterns ...
#MSG# the number of emerging sequence patterns enumerated is 13
#MSG# The final results are in the directory `result5'
#END# /usr/bin/msequence.rb O=result5 i=dat2.csv S=2 class=class -padding
$ more result5/patterns.csv
class,pid,pattern,size,pos,neg,posTotal,negTotal,total,support,growthRate,postProb
cls1,1,B C,2,3,0,4,2,6,0.75,inf,1
cls2,10,A D,2,2,0,2,4,6,1,inf,1
cls2,9,B C D,3,2,0,2,4,6,1,inf,1
cls2,11,A C D,3,2,0,2,4,6,1,inf,1
cls1,0,C,1,4,2,4,2,6,1,1,0.6666666667
cls1,2,B,1,4,2,4,2,6,1,1,0.6666666667
cls1,6,A B,2,2,1,4,2,6,0.5,1,0.6666666667
cls2,8,B D,2,2,1,2,4,6,1,4,0.6666666667
cls2,12,C D,2,2,1,2,4,6,1,4,0.6666666667
cls1,5,A C,2,3,2,4,2,6,0.75,0.75,0.6
cls1,4,A,1,3,2,4,2,6,0.75,0.75,0.6
cls1,7,D,1,2,2,4,2,6,0.5,0.5,0.5
cls1,3,B C,2,2,2,4,2,6,0.5,0.5,0.5
\end{Verbatim}
