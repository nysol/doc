

\chapter[Usage]{Common Usages of ZDD}

\section{Overview}

The following are some commonly used methods to learn about different usages of ZDD library. 

\subsection {Use itemset method to declare symbol} 

Items used is declared using ZDD.symbol method. When the symbol is not declared, items will be arranged according to the order of symbol defined using the itemset method.  
\begin{verbatim}
 > a=ZDD.itemset("a")
 > b=ZDD.itemset("b")
 > c=ZDD.itemset("c")
 > ZDD.symbol # If there are no arguments in the itemset method, items will be declared in 
 ascending order. 
 a b c

 > a=ZDD.itemset("a")
 > c=ZDD.itemset("c")
 > b=ZDD.itemset("b")
 > ZDD.symbol
 a c b
\end{verbatim}

\subsection {Create a row of itemset using ZDD object} 

Use ZDD.itemset method to create a list of item names separated with space delimiter assigned to a ZDD object. 
 \begin{verbatim}
 # ZDD object is represented by item set {a,b,c} consisting of 3 items `a',`b',`c'. The object 
 is assigned to a Ruby variable 'a'.
 > a=ZDD.itemset("a b c")
 > a.show
  a b c
\end{verbatim}
 
 \subsection{Define weight}
 
Use constant method to define an integer constant weight (weight of empty item set).
 \begin{verbatim}
 > a=ZDD.itemset("a b c")
 > q=ZDD.constant(4) # Weight of 4 is assigned to the empty item set ZDD object 
 and stored in Ruby variable 'q'. 
 > (q*a).show
  4 a b c
 \end{verbatim}
 
 \subsection{Operations on ZDD Object}
 
 ZDD object with one or more operators in the argument overloads the ZDD operator. 
 \begin{verbatim}
 > a=ZDD.itemset("a b c")  # Assign ZDD object to Ruby variable `a'.
 > (4*a).show              # Since '*" is ZDD operator function, Ruby integer '4' is automatically converted to a ZDD object. 
  4 a b c                  # (ZDD.constant(4)*a).show is processed in the background.
 > (a+"a b").show          # Automatic conversion of Ruby string "a b" to a ZDD object. 
  a b c + a b              # (a+ZDD.itemset("a b")).show is processed in the background.
 > a="a b"+"c d"           # This operator combines two arguments as a string of Ruby string object. 
 \end{verbatim}
 
\subsection{Conditional statements for combinations}

Conditional statements for combinations can be written as follows: 
 \begin{verbatim}
 > z=ZDD.constant(0)
 > ["a","b","c"].each{|item|
 >    z += item  # Since 'z' is a ZDD object, '+' is used as a ZDD operator.
 > }
 > z.show
  a + b + c

 > z=ZDD.constant(0)
 > (1..5).each{|i|
 >   z += "item_#{i}"
 > }
 > z.show
  item_1 + item_2 + item_3 + item_4 + item_5
 \end{verbatim}




