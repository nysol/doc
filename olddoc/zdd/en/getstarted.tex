

\chapter[Getting Started]{Introduction}

\section{Summary}

 ZDD(Zero-suppressed Binary Decision Diagrams) is used for efficient manipulation of weighted item combinations based on reduction rules. ZDD VSOP (Valued-Sum-Of-Products calculator) is implemented as a Ruby Extension Library to compute item combinations.  
 
 VSOP is a combinatorial item set (sum-of-products format) where each product term has a weight (or coefficient). 
  
 There are many operands for ZDD objects (e.g. '+','-','=='), thus, the implementation of ZDD with Ruby enables seamless integration of Boolean set operations and arithmetic operations.  
 
 This manual is complied from RDoc and can be retrieved from the command line. 
 
\begin{Verbatim}[frame=single]
 $ ri ZDD # Display manual of all ZDD modules
 $ ri ZDD.symbol # Display symbol method 
\end{Verbatim}

 Note: The modules are under development and this document is subjected to changes without prior notice. 

Japan Science and Technology Agency  ERATO Minato Discrete Structure Manipulation System Project  Kansai Satellite Lab.


\section[Installation]{Installation}

The ZDD Ruby package is tested on: 
\begin{itemize}
	\item Mac OS X 
\end{itemize}

The installation on Linux based platforms can be complied from source code. 

 This package requires Ruby 1.9.3. You may check the installed version of Ruby and the location of RubyGem using:
 
\begin{Verbatim}[frame=single]
 ruby -v
 gem environment
 \end{Verbatim}
 
 Afterwards, add RubyGems path in \.bash\_profile:
\begin{Verbatim}[frame=single]
 ruby -v
 export PATH=~/.rvm/gems/ruby-1.9.3-p448/bin:$PATH
\end{Verbatim}
 
 Download the ZDD Ruby package from sourceforge.com. 
 \# link and file name to be updated \#
 
 Extract the tar archive file to desired location on your machine as follows:
\begin{Verbatim}[frame=single]
 tar xvzf vsop_ruby.tar.gz
\end{Verbatim}
 
 Go to the vsop\_ruby directory, create a makefile and install the program. 
\begin{Verbatim}[frame=single]
 cd vsop_ruby
 make
 sudo make install
\end{Verbatim}
  
  \section{Environment Variables}
  
 Edit the environment variable to change the default behavior 

 \textbf{ZDDLimitNode}

    Maximum number of nodes Default value:1000000
    
 \textbf{ZDDWarning}

    Warning message : Send a warning message when the item exceeds a specific value. 
    
    
