
\section{restrict : 上位集合の選択\label{sect:restrict}}
\subsection*{書式}
$obj$.restrict($zdd1$) $\rightarrow$ $zdd2$

\subsection*{説明}
$obj$に含まれるアイテム集合について、$zdd1$中の少なくとも1つのアイテム集合を包含していれば、その項を選択する。
より正確には、$obj$を構成するある項$T_i$から重みを省いたアイテム集合を$\alpha_i$、
同じく$zdd1$のアイテム集合を$\beta_j$とすると、
$\alpha_i \supseteq \beta_j$を満たすような$j$が少なくとも一つあれば、$\alpha_i$に対応する項$T_i$を$obj$から選択する。
ちなみに、条件式$\alpha_i \supseteq \beta_j$を$\alpha_i \subseteq \beta_j$に変えれば
\hyperref[sect:permit]{permit}関数となる。

\subsection*{例}
\subsubsection*{例1: 基本例}



\begin{Verbatim}[baselinestretch=0.7,frame=single]
> require 'zdd'
> a=ZDD::itemset("a")
> b=ZDD::itemset("b")
> c=ZDD::itemset("c")
> x=5*a + b*c + 3*b + 2
> x.show
 5 a + b c + 3 b + 2

# xに含まれる4つのアイテム集合a,bc,b,Φ(重み2の項で空のアイテム集合)のうち、
# yの2つのアイテム集合a,bのいずれかを包含するアイテム集合は、a,bcである。
# よってxからa,bcの項が選択される。
> x.restrict(a+c).show
 5 a + b c

# xに含まれる4つのアイテム集合a,bc,b,Φのうち、
# zのアイテム集合bcを包含するアイテム集合は、bcのみ。
# よってxからbcの項が選択される。
> x.restrict(b*c).show
 b c

# xに含まれる4つのアイテム集合a,bc,b,Φのうち、
# アイテム集合Φ(重み1の空アイテム集合)を含むアイテム集合は全てのアイテム集合。
> x.restrict(1).show
 5 a + b c + 3 b + 2

# xに含まれる4つのアイテム集合a,bc,b,Φのうち、アイテム集合abcを含むアイテム集合はない。
> x.restrict(a*b*c).show
 0
\end{Verbatim}


\subsection*{関連}
\hyperref[sect:permit]{permit} : 部分集合の選択
