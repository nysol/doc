
\section{show : ZDDの表示\label{sect:show}}
\subsection*{書式}
$obj$.show([$switch$])

~~$switch$ : string

\subsection*{説明}
ZDDオブジェクト$obj$を多様な形式で標準出力に出力する。
出力形式は、表\ref{tbl:zdd_show_switch}に示された値(文字列)を与えることによって切り替える。
$switch$を省略すればアイテムの積和形式で表示する。


\begin{table}[htbp]
\begin{center}
\caption{ZDDオブジェクトの表示形式のスイッチ一覧\label{tbl:zdd_show_switch}}
{\small
\begin{tabular}{l|l}
\hline
$switch$の値  & 機能 \\
\hline
 (スイッチ無し)	& アイテムの積和形での表示 \\
  bit					& 重みの(-2)進数の各桁別アイテム集合の表示 \\
  hex					& 整数値を 16 進数で表現する積和形表示 \\
  map					& カルノー図で表示。アイテム変数6個まで表示できる \\
  rmap				& カルノー図で表示。冗長なアイテム変数は省いて表示  \\
  case				& 整数値ごとに場合分けして積和形表示 \\
%  /size				& 計算結果のBDD節点数(および処理系全体の節点数)を表示 \\
%  /count				& 式に現れる(0 以外の値を持つ)組合せの個数を表示 \\
%  /density		  & 集合の濃度(0 以外の値を持つ組合せの比率)を表示 \\
%  /value				& シンボル変数にすべて数値を代入したときの式の値を表示 \\
%  /maxcover		& 式に含まれる(0 以外の値を持つ)コスト最大の組合せを1つ表示 \\
%  /maxcost		  & コスト最大組合せのコスト値を表示 \\
%  /mincover		& 式に含まれる(0 以外の値を持つ)コスト最小の組合せを1つ表示 \\
%  /mincost		  & コスト最小組合せのコスト値を表示 \\
%  /plot				& BDDの形を図示する。(使用不可) \\
  decomp			& 単純直交分解形式での出力 \\
%  /export			& BDDの形を図示する(否定枝不使用)。$fileName$に指定があれば、指定されたファイルに出力 \\
%              & 指定されたファイルに出力 \\
\hline
\end{tabular}
}
\end{center}
\end{table}

\if0
\subsubsection*{bit}
ZDDでは重みをマイナス2進数で表現し、各桁ごとにZDDを生成している。
このメソッドにより各桁別に、そのZDDに登録されているアイテム集合を標準出力に出力できる。
\subsubsection*{hex}
重みを16進数で表現し積和形で標準出力に出力する。
\subsubsection*{map}
カルノー図を標準出力に表示する。
アイテム数は6個まで表示できる。
\subsubsection*{rmap}
カルノー図を標準出力に表示する。
ただし、使われていないアイテムは省いて表示される。
\subsubsection*{case}
重みの値別にアイテム集合を標準出力に出力する。
\subsubsection*{decomp}
$obj$を単純直交分解(simple disjoint decomposition)形式で出力する。
\fi

\subsection*{例}
\subsubsection*{Example 1: Basic Example}



\begin{Verbatim}[baselinestretch=0.7,frame=single]
> require 'zdd'
> a=ZDD::itemset("a")
> b=ZDD::itemset("b")
> c=ZDD::itemset("c")
> f=5*a*b*c - 3*a*b + 2*b*c + c
> f.show
 5 a b c - 3 a b + 2 b c + c
> ZDD::constant(0).show
 0
\end{Verbatim}
\subsubsection*{Example 2: bit}



\begin{Verbatim}[baselinestretch=0.7,frame=single]
> require 'zdd'
> a=ZDD::itemset("a")
> b=ZDD::itemset("b")
> c=ZDD::itemset("c")
> f=5*a*b*c - 3*a*b + 2*b*c + c
> f.show
 5 a b c - 3 a b + 2 b c + c

> f.bit
NoMethodError: undefined method `bit' for 5 a b c +  - 3 a b + 2 b c + c:Module
	from (irb):8
	from /Users/stephane/.rvm/rubies/ruby-1.9.3-p448/bin/irb:16:in `<main>'

# "a b c" has weight of -5, expressed in (-2) base is 101. 
# 1*(-2)^2+0*(-2)^1+1*(-2)^0 = 5
# Therefore 0 digit and 2nd digit of itemset "a b c" is display. 
# "a b" has weight of -3 expressed in (-2) base is 1101. 
# 1*(-2)^3+1*(-2)^2+0*(-2)^1+1*(-2)^0 = -3
# Therefore 0,2nd,3rd digit of itemset "a b" is displayed. 

\end{Verbatim}
\subsubsection*{Example 3: hex}



\begin{Verbatim}[baselinestretch=0.7,frame=single]
> require 'zdd'
> a=ZDD::itemset("a")
> b=ZDD::itemset("b")
> c=ZDD::itemset("c")
> d=ZDD::itemset("d")

> f=a*b+11*b*c+30*d+4
> f.show
 a b + 11 b c + 30 d + 4
> f.hex
NoMethodError: undefined method `hex' for a b + 11 b c + 30 d + 4 :Module
	from (irb):9
	from /Users/stephane/.rvm/rubies/ruby-1.9.3-p448/bin/irb:16:in `<main>'
\end{Verbatim}
\subsubsection*{Example 4: map}



\begin{Verbatim}[baselinestretch=0.7,frame=single]
> require 'zdd'
> a=ZDD::itemset("a")
> b=ZDD::itemset("b")
> c=ZDD::itemset("c")
> d=ZDD::itemset("d")
> f=2*a*b+3*b+4
> f.show
 2 a b + 3 b + 4
> f.map
NoMethodError: undefined method `map' for 2 a b + 3 b + 4 :Module
	from (irb):8
	from /Users/stephane/.rvm/rubies/ruby-1.9.3-p448/bin/irb:16:in `<main>'
# Itemset is displayed with item a as the first sequence in the bit string,
# and item b corresponds to the first row of the bit string.
# Weight is displayed as the cell value. Upper left cell contains 0 in a, and 0 in b,
# and constant term 4 is shown.

# The 4 items are as follows. 
> g=a*b + 2*b*c + 3*d + 4
> g.show
 a b + 2 b c + 3 d + 4
> g.map
NoMethodError: undefined method `map' for a b + 2 b c + 3 d + 4 :Module
	from (irb):17
	from /Users/stephane/.rvm/rubies/ruby-1.9.3-p448/bin/irb:16:in `<main>'
\end{Verbatim}
\subsubsection*{Example 5: rmap}



\begin{Verbatim}[baselinestretch=0.7,frame=single]
> require 'zdd'
# Declare 4 items a,b,c,d
> ZDD::symbol("a")
> ZDD::symbol("b")
> ZDD::symbol("c")
> ZDD::symbol("d")

> f=ZDD::itemset("a b") + 2*ZDD::itemset("b c") + 4
> f.show
 a b + 2 b c + 4

# The following displays as map.
> f.map
NoMethodError: undefined method `map' for a b + 2 b c + 4 :Module
	from (irb):12
	from /Users/stephane/.rvm/rubies/ruby-1.9.3-p448/bin/irb:16:in `<main>'

# d is omitted when displayed as rmap. 
> f.rmap
NoMethodError: undefined method `rmap' for a b + 2 b c + 4 :Module
	from (irb):15
	from /Users/stephane/.rvm/rubies/ruby-1.9.3-p448/bin/irb:16:in `<main>'
\end{Verbatim}
\subsubsection*{Example 6: case}



\begin{Verbatim}[baselinestretch=0.7,frame=single]
> require 'zdd'
> a=ZDD::itemset("a")
> b=ZDD::itemset("b")
> c=ZDD::itemset("c")
> f=5*a*b*c - 3*a*b + 2*b*c + 5*c
> f.show
 5 a b c - 3 a b + 2 b c + 5 c

> f.case
NoMethodError: undefined method `case' for 5 a b c +  - 3 a b + 2 b c + 5 c:Module
	from (irb):8
	from /Users/stephane/.rvm/rubies/ruby-1.9.3-p448/bin/irb:16:in `<main>'
\end{Verbatim}
\subsubsection*{Example 7: decomp}



\begin{Verbatim}[baselinestretch=0.7,frame=single]
> require 'zdd'
> a=ZDD::itemset("a")
> b=ZDD::itemset("b")
> c=ZDD::itemset("c")

> f1=(a*b*c)
> f1.show
 a b c
> f1.decomp
NoMethodError: undefined method `decomp' for a b c:Module
	from (irb):8
	from /Users/stephane/.rvm/rubies/ruby-1.9.3-p448/bin/irb:16:in `<main>'
# AND of a,b,c is expressed as a*b*c=a b c

> f2=((a*b*c)+(a*b))
> f2.show
 a b c + a b
> f2.decomp
NoMethodError: undefined method `decomp' for a b c + a b:Module
	from (irb):13
	from /Users/stephane/.rvm/rubies/ruby-1.9.3-p448/bin/irb:16:in `<main>'
# OR of c,1 is expressed as (c+1), in addition, AND of a b is expressed as (a b),
# the complete expression becomes (a b)*(c+1)=a b c + a b

> f3=((a*b*c)+(a*b)+(b*c))
> f3.show
 a b c + a b + b c
> f3.decomp
NoMethodError: undefined method `decomp' for a b c + a b + b c:Module
	from (irb):19
	from /Users/stephane/.rvm/rubies/ruby-1.9.3-p448/bin/irb:16:in `<main>'
# [ a c ] enumerates all combinations from a and c as (a c + a + c).
# AND of b and the above expression becomes b*(a c + a + c) = a b c + a b + b c

> f4=((a*b*c)+(a*b)+(b*c)+(c*a))
> f4.show
 a b c + a b + a c + b c
> f4.decomp
NoMethodError: undefined method `decomp' for a b c + a b + a c + b c:Module
	from (irb):25
	from /Users/stephane/.rvm/rubies/ruby-1.9.3-p448/bin/irb:16:in `<main>'
# [ a b c ] enumerates all combinations from a,b,c as (a b c + a b + b c + c a)

\end{Verbatim}


\subsection*{関連}
\hyperref[sect:export]{export} : ZDDの構造をそのままフィアイルに出力する。
