
\section{概要\label{sect:abstract}}
本パッケージは、ZDD(Zero-suppressed Binary Decision Diagrams: ゼロサプレス型二分決定グラフ)を利用し、
重み付きのアイテムの組み合わせ集合をコンパクトに格納することを可能とするVSOP
(Valued-Sum-Of-Products calculator)\cite{minato2005}をruby拡張ライブラリとして実装したものである。
ZDDの生みの親である湊真一教授が開発/コーディングされたパッケージをruby言語において利用できるように拡張したもので、
\href{http://www-erato.ist.hokudai.ac.jp/index.php?language=jp}{ERATO湊離散構造処理系プロジェクト}
において開発されたものである。

ZDDを使えば、膨大な組み合せ集合を非常にコンパクトに表現することが可能となる。
例えば、スーパーマーケットにおける購入履歴から、ある一定の頻度以上購入された商品の組み合せ(頻出アイテム集合)を列挙すると、
その数は膨大になることが多いが、ZDDを使えば、それら全ての組み合せを非常にコンパクトに格納することが可能となる。
しかも、コンパクトに格納されたZDDオブジェクトに対して、各種演算を直接適用することが可能で、
大規模なアイテム集合を非常に効率よく処理することが可能となる。
例えば、列挙された数億件の頻出アイテム集合から「納豆」を含むパターンのみを選択したり、
男性の頻出アイテム集合と女性の頻出アイテム集合との差異を計算したりすることが、それらのZDDのサイズにほぼ比例して行うことが可能なのである。
理論的な詳細は、巻末の\hyperref[sect:bib]{参考文献}を参照されたい。

本パッケージにおいて、ZDDはrubyオブジェクト(「{\bf ZDDオブジェクト}」と呼ぶ)として扱われる。
そしてZDDオブジェクトに対して定義された各種関数はクラスメソッドとして利用でき、
また、ZDDオブジェクトに対する各種演算子(\verb|+,-,==|など)も、ZDDに対する演算子としてオーバーロードされており、
ZDDとrubyの機能をシームレスに組み合わせて利用することを可能としている。
さらに、自動的な型変換もサポートしており、よりストレスなくプログラミングができるように工夫している。

\section{インストール\label{sect:install}}
ZDDパッケージは、nysolのminingパッケージの一部として配布されている。
ソースからのコンパイル、およびrubygemによるインストールを選ぶことができる。
インストールの詳細は、\href{http://www.nysol.jp}{NYSOL}のページのminingパッケージのマニュアルを参照されたい。

\section{最大節点数の変更}
本パッケージにおけるZDDの最大節点数はデフォルトで4000万である。
この値を超える節点を使おうとするとエラー終了する。
最大節点数は環境変数\verb|ZDDLimitNode|を設定することで変更可能である。
例えば、bashシェルにおいては、以下のように設定すれば、最大節点数を1億に拡張できる。
\begin{Verbatim}[baselinestretch=0.7,frame=single]
$ export ZDDLimitNode=100000000
\end{Verbatim}

1節点あたりのメモリ消費量は、32bit OSでは21~25バイト程度で、
デフォルトの最大節点数4000万節点を使いきれば、主記憶1GB程度を消費する。
64bit OSでは、単純に実装すると32bit OSの2倍になるが、
様々な工夫により3割増し程度に抑えられており、
1節点あたり28~32バイト程度である。
デフォルトの最大節点数4000万節点を使いきれば、主記憶1.3GB程度を消費する。
利用する環境に応じて最大節点数を変更することで、より大規模なZDDを構築することができる。

\section{用語\label{sect:terminology}}
以下では、本マニュアルで利用する専門用語について解説する。
巻末の\hyperref[sect:bib]{参考文献}で使われている用語と異なることもあるので注意されたい。

\subsubsection*{アイテム、アイテム集合、項、重み付き積和集合、式}
本パッケージでは、集合の要素を「{\bf アイテム}」と呼び、
アイテムを要素に持つ集合を「{\bf アイテム集合}」と呼ぶ。
スーパーマーケットにおける商品をアイテム、商品の組み合せをアイテム集合と考えれば分かりやすい。
そしてアイテム集合に重みを与えた「{\bf 項}」を要素としてもつ集合を「{\bf 重み付き積和集合}」と呼ぶ。
例えば、3つのアイテムa,b,cについての重み付き積和集合は「\verb|abc+3ab+4bc+7c|」のように
表記する(この表記形式を「重み付き積和形式」と呼ぶ)。
これは、\verb|abc,3ab,4bc,7c|の4つの項から成り立ち、\verb|3ab|は、重みが\verb|3|のアイテム集合\{a,b\}であることを意味する。
スーパーマーケットで言えば、3つの商品\verb|a,b,c|を同時に購入した顧客が1人いて、
\verb|a,b|を同時に購入した顧客が3人いて、といった意味付けをすると分かりやすいであろう。

\subsubsection*{空アイテム集合、ZDD定数オブジェクト}
要素のないアイテム集合のことを「{\bf 空アイテム集合}」と呼ぶ。
重み付き積和集合「\verb|abc+3ab+4bc+7c+3|」について考えると、
最後の項\verb|3|は空アイテム集合の重みが3であることを示している。
スーパーマーケットの例で言えば、何も買わなかった人が3人いたと考えればよい。
また空アイテム集合のZDDオブジェクトのことを特に「{\bf ZDD定数オブジェクト}」と呼ぶ。

\subsubsection*{アイテム順序表}
ZDDは2分決定木を縮約した2分決定グラフと呼ばれる構造を持つが、
その2分決定木のレベル(深さ)がアイテム対応している。
そしてそのレベル、すなわち根から葉に向かうアイテムの順序は「{\bf アイテム順序表}」と呼ばれる表によって管理されている。
このような管理が必要になるのは、アイテムの順序によってZDDのサイズ(節点数)が大きく影響を受けるからである。
ZDDのサイズが大きくなるということは、それに対する演算速度も低下することを意味する。
アイテム順序表は\hyperref[sect:symbol]{symbol}関数によって随時登録されることになる。
組み合わせ数が極端に大きくなる場合には、登録順序を考慮する必要があるかもしれない。


\section{表記\label{sect:notation}}

\subsubsection*{アイテム集合の表記}
rubyの実行結果として表示されるアイテム集合は、アイテムおよび重みがスペースで区切られて出力される。
ただし、本文やコメントの中では、簡単のために、アイテムを全てアルファベット一文字で表し、スペース区切りを省略している。
例えば、重み3の\{a,b,c\}のアイテム集合は、実際の実行結果としては「\verb|3 a b c|」と表示されるが、
本文中では「\verb|3abc|」と表記している。

\subsubsection*{実行例}

本マニュアルでは多数の実行例を掲載している。そこで使われている記号の意味は以下の通りである。
\begin{itemize}
\item \verb|>| : rubyのメソッド入力を表す。
\item \verb|$| : シェルのコマンドライン入力を表す。
\item \verb|#| : コメントを表す。
\item 記号なし : 実行結果を表す。
\end{itemize}

また実行例は、基本的にはrubyのコマンドライン実行ツールであるirbで実行した結果のログを貼り付けたものである。

