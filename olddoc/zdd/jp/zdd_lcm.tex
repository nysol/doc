
\section{lcm : LCM over ZDD\label{sect:lcm}}
\subsection*{書式}

$obj$.lcm($type,transaction,minsup[,order,ub]$) $\rightarrow$ $zdd$

~~$transaction$ : string

~~$type$ : string

~~$minsup$ : integer

~~$order$ : string

~~$ub$ : integer

\subsection*{説明}
$transaction$で指定されたトランザクションファイルから、LCM over zdd アルゴリズムを利用し、
指定された最小サポート$minsup$以上の頻出パターンを列挙しそのZDDオブジェクト$zdd$を返す。
$order$でZDDのアイテムオーダファイルを指定し、$ub$で列挙するアイテム集合のサイズの上限を与える。

列挙する頻出パターンの種別は$type$で与え、"F"(頻出アイテム集合)
,"M"(極大アイテム集合),"C"(飽和アイテム集合)の三種類が指定できる。
また"FQ"のように、"Q"を付けると、列挙された各頻出アイテム集合に頻度を重みとして出力する。
"Q"をつけなければ、最小サポート条件を満たす頻出アイテム集合のみ出力され、頻度情報は省かれる。

トランザクションファイルは、以下に示すようなテキストファイルで、
一つの行が一つのトランザクションに対応しており、
アイテムは1から始まる連番で指定し、アイテムの区切りには半角スペースを用いる。
アイテムとしてアルファベットを利用することはできない。

\begin{Verbatim}[baselinestretch=0.7,frame=single]
1 2 3 6
4 5 6
1 2 4 6
2 4 6
1 2 4 5
\end{Verbatim}

$order$ファイルは、ZDDのアイテムオーダ表に登録するアイテムの順序を示したテキストファイルである。
通常は、以下のように、トランザクションデータに含まれる全アイテムを番号順に並べて与える。
またトランザクションのアイテム番号に欠番があった場合でも、その欠番を含めて指定する必要があることに注意する。
\begin{Verbatim}[baselinestretch=0.7,frame=single]
1 2 3 4 5 6
\end{Verbatim}

$order$ファイルは省略する(もしくはnilを指定する)ことができるが、その場合、
効率性のためにアイテムオーダはLCMの内部アルゴリズムによって決まる。
しかし、この方法を利用すると、そのオーダに応じてアイテム番号が再び採番されるため、
出力されるZDDの頻出アイテム集合におけるアイテム番号は、元のトランザクションの番号とは異なったものとなる。
アイテムの内容に関係のない解析をするのであれば、$order$を省略することで計算効率は高まるが、
逆に、アイテムの内容についての意味を解析する目的があるのであれば、上記に示した連番としてのオーダファイルを指定する。

$ub$には列挙される頻出アイテム集合のサイズの上限を指定する。
指定を省略するか、nilを与えると上限なしに列挙する。

\subsection*{例}
\subsubsection*{Example 1: Basic Example}



\begin{Verbatim}[baselinestretch=0.7,frame=single]
> require 'zdd'
# Contents of tra.txt
# 1 2 3 6
# 4 5 6
# 1 2 4 6
# 2 4 6
# 1 2 4 5
# Contents of order.txt
# 1 2 3 4 5 6
> p1=ZDD::lcm("FQ","tra.txt",3,"order.txt")
> p1.show
 3 x6 x4 + 3 x6 x2 + 4 x6 + 3 x4 x2 + 4 x4 + 3 x2 x1 + 4 x2 + 3 x1 + 5

# If order file is not specified, the resulting frequent itemset will be the same
# Note that the item number in the results is different from the item number
# in the transaction file.
> p2=ZDD::lcm("FQ","tra.txt",3)
> p2.show
 3 x4 x1 + 3 x4 + 3 x3 x2 + 3 x3 x1 + 4 x3 + 3 x2 x1 + 4 x2 + 4 x1 + 5
\end{Verbatim}


\subsection*{関連}
\hyperref[sect:freqpatA]{freqpatA} : 頻出アイテム集合の列挙

\hyperref[sect:freqpatM]{freqpatM} : 頻出極大アイテム集合の列挙

\hyperref[sect:freqpatC]{freqpatC} : 頻出飽和アイテム集合の列挙


