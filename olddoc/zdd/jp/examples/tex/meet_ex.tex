\subsubsection*{例1: 基本例}



\begin{Verbatim}[baselinestretch=0.7,frame=single]
> require 'zdd'
> a=ZDD::itemset("a")
> b=ZDD::itemset("b")
> c=ZDD::itemset("c")
> f=a+2*a*b+3*b

# a + 2ab + 3b の各アイテム集合と引き数で指定されたアイテム集合 a との共通集合を求めると、
# a + 2a + 3 = 3 a + 3となる。
> f.meet(a).show
 3 a + 3

# アイテム集合bとの共通集合を求めると 1 + 2b + 3b = 5b + 1 となる。
> f.meet(b).show
 5 b + 1

# アイテム集合 ab との共通集合を求めると a + 2ab + 3b となる。
> f.meet(a*b).show
 2 a b + a + 3 b

# 定数1は空のアイテム集合なので、それとの共通集合を求めると係数だけが残り 1 + 2 + 3 = 6 となる。
> f.meet(1).show
 6

# abc + 2ab + bc + 1 の各アイテム集合と引き数で指定された 2ab + a の各アイテム集合との共通集合を
# 求めると、以下の通りとなる(アイテム間のスペースは省略)。
# abc ∩ 2ab = 2ab
# 2ab ∩ 2ab = 4ab
# bc  ∩ 2ab = 2b
# 1   ∩ 2ab = 2
# abc ∩ a   = a
# 2ab ∩ a   = 2a
# bc  ∩ a   = 1
# 1   ∩ a   = 1
# 結果をまとめると 6ab + 3a + 2b + 4 となる。
#
> f=((a*b*c)+2*(a*b)+(b*c)+1)
> g=2*a*b + a
> f.show
 a b c + 2 a b + b c + 1
> g.show
 2 a b + a
> f.meet(g).show
 6 a b + 3 a + 2 b + 4
\end{Verbatim}
