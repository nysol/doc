\subsubsection*{例1: 基本例}



\begin{Verbatim}[baselinestretch=0.7,frame=single]
> require 'zdd'
> c=ZDD::constant(10)
> c.show
 10
# ZDD重みオブジェクトとruby文字列との演算では、
# ruby文字列はアイテム集合と見なされ自動でZDDオブジェクトにキャストされる。
> (c*"a").show
 10 a

# ZDD重みオブジェクトとruby整数との演算では、ruby整数はZDD重みオブジェクトと見なされる。
> (0*c).show
 0

# ZDD重みオブジェクトをruby整数に変換し、ruby整数として演算する。
> puts c.to_i*10
100

# 以下のように、0の重みを定義しておくと、そのオブジェクトとの演算においては、
# RubyStringを自動的にキャストしてくれるので便利である。
> a=ZDD::constant(0)
> a+="x"
> a+="x z"
> a+="z"
> a.show
 x z + x + z
\end{Verbatim}
