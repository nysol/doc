\subsubsection*{例1: 基本例}

以下では、アイテム\verb|a,b,c|、およびアイテム集合\verb|2ab + a + 3b|、
\verb|abc + 2ab + bc + 1|、\verb|ab+a|に関する排他的論理和を求める。


\begin{Verbatim}[baselinestretch=0.7,frame=single]
> require 'zdd'
# まずはアイテム集合の定義
> a=ZDD::itemset("a")
> b=ZDD::itemset("b")
> c=ZDD::itemset("c")
> f=2*a*b+a+3*b
> f.show
 2 a b + a + 3 b
\end{Verbatim}

 \verb|(2ab + a + 3b)|$\oplus$\verb|a| = \verb|3ab + 2b + 1|
\begin{Verbatim}[baselinestretch=0.7,frame=single]
> f.delta(a).show
 3 a b + 2 b + 1
\end{Verbatim}

 \verb|(2ab + a + 3b)|$\oplus$\verb|b| = \verb|ab + 2a + 3|
\begin{Verbatim}[baselinestretch=0.7,frame=single]
> f.delta(b).show
 a b + 2 a + 3
\end{Verbatim}

 \verb|(2ab + a + 3b)|$\oplus$\verb|ab| = \verb|3a + b + 2|
\begin{Verbatim}[baselinestretch=0.7,frame=single]
> f.delta(a*b).show
 3 a + b + 2
\end{Verbatim}

 \verb|(2ab + a + 3b)|$\oplus$\verb|1| = \verb|2ab+a+3b|
 定数1は空のアイテム集合なので、それとの排他的論理和を求めると元の集合のままとなる。
\begin{Verbatim}[baselinestretch=0.7,frame=single]
> f.delta(1).show
 2 a b + a + 3 b
\end{Verbatim}

 \verb|(abc + 2ab + bc + 1)|$\oplus$\verb|(2ab + a)|の項別の演算結果は
 以下の通りとなる。
 \begin{itemize}
 \item \verb|abc |$\oplus$\verb| 2ab = 2c|
 \item \verb|2ab |$\oplus$\verb| 2ab = 4|
 \item \verb|bc  |$\oplus$\verb| 2ab = 2ac|
 \item \verb|1   |$\oplus$\verb| 2ab = 2ab|
 \item \verb|abc |$\oplus$\verb| a   = bc|
 \item \verb|2ab |$\oplus$\verb| a   = 2b|
 \item \verb|bc  |$\oplus$\verb| a   = abc|
 \item \verb|1   |$\oplus$\verb| a   = a|
 \end{itemize}
 
 結果をまとめると\verb|a b c + 2 a b + 2 a c + a + b c + 2 b + 2 c + 4|となる。
\begin{Verbatim}[baselinestretch=0.7,frame=single]
> g=((a*b*c)+2*(a*b)+(b*c)+1)
> h=2*a*b + a
> g.show
 a b c + 2 a b + b c + 1
> h.show
 2 a b + a
> g.delta(h).show
 a b c + 2 a b + 2 a c + a + b c + 2 b + 2 c + 4
\end{Verbatim}
